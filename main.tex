\PassOptionsToPackage{paper=a5paper,inner=24mm,outer=17mm,top=18mm,bottom=24mm,includehead,includefoot,heightrounded,headheight=13pt,headsep=6mm,footskip=10mm}{geometry}
\documentclass[openright]{bookest}

\usepackage[many]{tcolorbox}
\newtcolorbox{personal}[1][]{
  breakable,
  title=#1,
  colback=white,
  colbacktitle=white,
  coltitle=black,
  fonttitle=\itshape,
  bottomrule=0pt,
  toprule=0pt,
  leftrule=3pt,
  rightrule=3pt,
  titlerule=0pt,
  arc=0pt,
  outer arc=0pt,
  colframe=black,
}

\usepackage{array}
\usepackage{longtable}
\usepackage{wrapfig}

\usepackage{fontspec}
\usepackage{polyglossia}
\usepackage[fontsize=9pt]{fontsize}
\setdefaultlanguage{english}
\setotherlanguage{russian}
\setmainfont{Linux Libertine O}
\setsansfont{Linux Biolinum O}
\setmonofont{Linux Libertine Mono O}
\newfontfamily\russianfont{Linux Libertine O}

\usepackage{microtype}
\microtypesetup{protrusion=true,expansion=true}

% Graphics, colors, hyperlinks
\usepackage{graphicx,xcolor}
\definecolor{linkcolor}{HTML}{204A87}

% typesetting of numeric quantities and SI units
\usepackage{siunitx}
\sisetup{
    mode = match,  % math mode numbers look different than text mode, let's use the current mode instead of forcing math mode
    group-digits,
    group-separator = {,},
    per-mode = symbol,
    range-units = single,
    range-exponents = combine-bracket,
    list-units = single,
    retain-explicit-plus,
}
\DeclareSIUnit{\Wd}{Wd}    % \watt\day, but without the space betheen W and d
\DeclareSIUnit{\We}{We}    % watt electrical
\DeclareSIUnit{\Wth}{Wth}  % watt thermal
\DeclareSIUnit{\rem}{rem}  % Roentgen equivalent man
\DeclareSIUnit{\Day}{day}  % spelled out day (as opposed to \day with is just 'd')
\DeclareSIUnit{\rpm}{rpm}  % revolutions per minute
\DeclareSIUnit{\roentgen}{R}
\DeclareSIUnit{\atmosphere}{atm}
\DeclareSIUnit{\kilovolt}{kV}
\DeclareSIUnit{\ton}{t}
\DeclareSIUnit{\force}{f}
\DeclareSIUnit{\betaeff}{\ensuremath{\beta_\mathrm{eff}}}

% Headers and footers
\usepackage{emptypage}
\usepackage{fancyhdr}
\pagestyle{fancy}
\fancyhf{}% clear
\fancyhead[LE]{\thepage\quad\nouppercase{\leftmark}}
\fancyhead[RO]{\nouppercase{\rightmark}\quad\thepage}
\renewcommand{\headrulewidth}{0.4pt}
\renewcommand{\footrulewidth}{0pt}

% Chapter style tweaks
\usepackage{titlesec}
\titleformat{\chapter}[display]
  {\filcenter\bfseries}
  {\large\MakeUppercase{\chaptername}~\thechapter}
  {0.5ex}
  {\huge}
  [\vspace{1ex}\titlerule]
\titlespacing*{\chapter}{0pt}{*3}{*2}

\newcommand{\varthreestars}{\begin{center}\begin{tabular}{c}$\ast$\\[-0.8ex] $\ast\enspace \ast $\end{tabular}\end{center}}

% Front matter meta
\title{Chernobyl -- How It Was}
\author{A. S. Dyatlov}
\date{1995}

% PDF metadata
\hypersetup{
  pdftitle={Chernobyl -- How It Was},
  pdfauthor={A. S. Dyatlov}
}

\begin{document}

\thispagestyle{empty}
\begin{titlepage}
  \centering
  \vspace*{2in}
  {\Huge\bfseries Chernobyl -- How It Was\\}
  \vfill
  {\large \textsc{A. S. Dyatlov}}\\[2ex]
  {\large 1995}
  \vspace*{1in}
\end{titlepage}
\clearpage

\thispagestyle{empty}
\noindent\textcopyright\ A. S. Dyatlov
\clearpage

% Front matter
\thispagestyle{empty}
\frontmatter
\tableofcontents
\clearpage

\chapter*{About the Author}
\addcontentsline{toc}{chapter}{About the Author}
\markboth{About the Author}{}

\begin{wrapfigure}{l}{0.3\textwidth}
  \vspace{-\baselineskip} % align top of image with first line of text
  \includegraphics[width=0.28\textwidth]{portrait.jpeg}
  \vspace{-\baselineskip} % align bottom of image with last line of text
\end{wrapfigure}

Anatoly Stepanovich Dyatlov (\textit{Анатолий Степанович Дятлов}) was born on the 3rd of March, 1931, in the village of Atamanovo, within Krasnoyarsk Krai. His father, a disabled veteran of the Great War, maintained the beacon on the Yenisei River; his mother devoted herself to the care of the household. At the age of fourteen, Dyatlov ran away from home.

Having completed seven years of schooling in 1945, the young Dyatlov entered the Norilsk Mining and Metallurgical Technical College, from whose electrotechnical department he graduated with honors in 1950. For the following three years he labored in Norilsk, at one of the enterprises of the Ministry of Medium Machine Building, acquiring the discipline and precision of industrial practice.

From 1953 until 1959 he pursued higher studies at the Moscow Engineering Physics Institute, again graduating with distinction, and obtaining the degree of engineer-physicist with specialization in automation and electronics.

Upon his assignment, he entered the Lenin Komsomol Shipbuilding Plant at Komsomolsk-on-Amur, where he successively held the positions of senior engineer, head of the physical laboratory, and commissioning mechanic of the main power installation for nuclear submarines. In 1973, compelled by family circumstances, he was transferred to the Chernobyl Nuclear Power Station, then in its formative stage. There he advanced from deputy head of the reactor department to deputy chief engineer for operations. His diligence was recognized by the conferment of the Orders of the Badge of Honor and of the Red Banner of Labor.

\varthreestars

During the catastrophe of 26~April~1986, Dyatlov sustained an exposure of no less than \qty{550}{\rem}. The Supreme Court of the USSR, by its verdict, declared him among those responsible for the disaster and condemned him to ten years' imprisonment in a general-regime labor colony. He served his sentence in the settlement of Kryukovo, Poltava Region.

After repeated pleas from scientific institutions, friends, and from A.~D.~Sakharov himself -- later, following Sakharov's death, from E.~G.~Bonner--Dyatlov was granted early release on 1~October~1990, under Article~220 on grounds of illness. The progress of his radiation sickness was swift and inexorable. Despite the attentive care of German physicians -- he was twice yearly a patient, from 1991 onward, in the burn department of the University Clinic of Munich -- Anatoly Stepanovich Dyatlov passed away on 13~December~1995.

\chapter*{Note from the Translator}
\addcontentsline{toc}{chapter}{Note from the Translator}
\markboth{Note from the Translator}{}
It has come to the translator's attention that but two incomplete renderings of this work are presently in circulation. Neither has sought to render the book accessible to readers unacquainted with the technical phraseology or administrative machinery of the Soviet era. They retain, moreover, the original Russian abbreviations and refer to sources scarce or altogether inaccessible. The present translation intends to remedy these defects by supplying such explanations and contextual remarks as may assist the general reader, while maintaining, so far as may be, fidelity to the author's text and manner.

The footnotes are of mixed character: part consist of the translator's own observations, part of supplementary information designed to make clear expressions or analogies which might otherwise remain unclear. Some reproduce or explain Russian terms and abbreviations; others furnish the historical or technical background requisite for understanding.

The original flow of the book (that is, starting from the first chapter until the beginning of the appendices) has been preserved. The translator, however, wished to compile more documents related to the contents of the book, as well as preserve certain rare materials cited within the text.

The translation has been completed. Additional proof-reading and editing has been completed up to and including the Section \textit{Aftermath}. Final editing (unification of abbreviations, writing style, etc.) has not yet commenced.

Criticism of this translation, as well as any suggestions toward its improvement, will be gratefully received and may be addressed to \textit{k@iczelia.net}.

The translator dedicates the derived part of the work to the public domain by waiving all of her rights to the work worldwide under copyright law, including all related and neighboring rights, to the extent allowed by law.

\begin{flushright}\small{
-- Kamila Szewczyk
}\end{flushright}

\chapter*{Foreword}
\addcontentsline{toc}{chapter}{Foreword}
\markboth{Foreword}{}

The catastrophe which befell the Chernobyl Nuclear Power Plant on 26~April, 1986 remains -- by its magnitude, complexity, and enduring consequences -- the most grievous disaster in the history of the nuclear industry.

Vast amounts of radioactive particles were discharged into the atmosphere from the exposed core of the reactor, dispersing chiefly along three great plumes that traversed Belarus, Ukraine, and the western districts of Russia. The territory contaminated by cesium-137 in concentrations exceeding one curie per square kilometre extends to nearly \num{130000}~square~kilometres.

The configuration of these radioactive zones was determined alike by the nature of the emissions and the prevailing meteorological conditions. During the period of most violent release (26~April--5~May~1986), both the composition and the intensity of the radionuclides varied from day to day. In the first three days, under northerly winds, the radioactive clouds moved toward Belarus; by 30~April the wind had veered to the south and east, and the trace upon the ground shifted accordingly. The lighter radioactive gases and aerosols ascended into the higher atmosphere, where they circled the globe for months -- some for a year -- before gradually settling, while the heavier particles, remaining in the lower strata of air, descended more swiftly to earth.

In the earliest weeks, the radiological environment was governed by short-lived isotopes, chief among them iodine-131. As these decayed, the hazard was assumed by nuclides of longer endurance -- cesium-137, strontium-90, and several isotopic forms of plutonium.

The consequences were most dire for the Republic of Belarus: nearly one-quarter of its territory lay within the zone of contamination, embracing \num{3678}~inhabited~localities and about one-fifth of the population.

Within the European portion of the former USSR, the lands polluted by cesium-137 in excess of one curie per square kilometre comprised \num{3.2}~percent of the total area, while contamination exceeding \num{0.2}~curie extended over \num{23}~percent. In the Russian Federation, twenty-nine administrative regions were affected, most severely the Bryansk, Tula, Kaluga, Ryazan, and Penza districts, together containing some \num{2.3}~million inhabitants within the tainted zone.

Even fifteen years after the event, the causes of the accident and the measure of human responsibility remain the subjects of divergent judgment. Among the investigators, the scientist A.~L.~Ilyin, in his work \textit{Realities and Myths of Chernobyl} (p.~79), traced the sequence of occurrences that culminated in the disaster and ascribed the ultimate cause to defects inherent in the design of the RBMK reactor, particularly to its positive void -- or steam -- coefficient of reactivity.

\begin{flushright}\small{
-- V.V. Lomakin

\textit{Deputy Director for Safety Assessment, State Scientific and Technical Committee for Nuclear and Radiation Safety (Kyiv, Ukraine)}
}\end{flushright}

\begin{flushright}\small{
-- T.G. Samkharadze

\textit{Editor-in-Chief of the journal Ecological Systems and Devices, Professor, Doctor of Technical Sciences, Academy of Engineering Sciences of Russia}
}\end{flushright}

\chapter*{Foreword of V.~A.~Chugunov}
\addcontentsline{toc}{chapter}{Foreword of V.~A.~Chugunov}
\markboth{Foreword of V.~A.~Chugunov}{}

Dear Respected Reader.

The book now placed before you has been written by an engineer who was himself a direct participant in the events of the greatest man-made disaster -- namely, the accident at the Fourth Nuclear Power Unit of the Chernobyl Nuclear Power Plant.

This book will help the unprejudiced reader -- and the prejudiced as well -- to form, or to correct, an opinion concerning the causes of the catastrophe and the legitimacy of the accusations brought against the operating personnel of the Chernobyl plant, who were officially declared the sole culprits of the disaster.

Addressing you, dear reader, I ask you to summon patience and read this book rather than lay it aside on the sole ground that you work in another field and lack sufficient knowledge of reactor theory or the operation of nuclear power installations. The matter is of greater import: it concerns the moral dimension of judging this event, the honour and dignity of those who, by the workings of fate, were drawn into its unfolding. Many of the direct participants are no longer among us. But for those who survive today, for the families of the dead, for the collective of the Chernobyl station, the judgment of society is not indifferent.

It is impossible today to gauge the extent of the harm inflicted upon the personnel of the entire nuclear-energy sector by the official declaration that the staff and technical leadership of the Chernobyl plant were the sole culprits. Left ``off-screen'' was the fact that this personnel did not scatter in order to save their own lives but remained at their posts, performing their prescribed functions and doing all that was possible -- and impossible -- to mitigate the consequences of the accident. This same personnel, in the course of restoration, carried out the modernization of the safety systems whose deficiencies became the very causes of the catastrophe, and thereafter operated those units until the most recent times. The harm done by the still-prevailing ``high official assessment'' of the personnel's actions lies in this: that in any future extreme situation at a nuclear facility, the staff will face a choice -- whether to abandon their posts, to renounce the struggle, or to continue working at the risk of their health and their lives, and be accused thereafter of all ``grave sins.''

By fate's design I had to work under the direction of Anatoly Stepanovich Dyatlov, both at the shipbuilding plant in Komsomolsk-on-Amur and at the Chernobyl station. I recall with gratitude that time, the time of our joint labour. There is no need to characterize him; in reading this book you will perceive for yourself the integrity of character and the uncommon civic courage of this Man.

And even today the Chernobyl station is staffed by personnel who passed through ``Dyatlov's school'' and remember it with gratitude.

On 13~December 1995 A.~S.~Dyatlov passed away, and on 15~December 2000, by decision of the government of Ukraine, the operation of the third and last functioning power unit was prematurely terminated. A noisy campaign concerning this event was organized in the press and on television. Yet this event was no holiday for the personnel. The duty shift at the block control room donned mourning armbands in protest. History has yet to render a truthful and impartial judgment upon all the events at that enterprise and upon its staff.

And so, what is the Chernobyl personnel doing today? An unusual stillness reigns in the turbine hall; it is cool and quiet in the central hall; the staff are trying to comprehend their place in light of the wording of OPB-88: ``The concluding stage of a nuclear station's operation is the stage of its decommissioning\ldots''

Dear reader, in concluding this introduction I wish to note the following. In one of the central-television reports devoted to the decommissioning of the Chernobyl station, the ``official version'' again resounded. It was precisely this that compelled me to take up the pen. I do not know what more must be written or done for this utterly disingenuous ``official version'' to pass into oblivion.

There is no doubt that the book before you is a weighty stone cast upon the pseudo-scientific heap of pseudo-facts and facts, of falsehood and injustice, erected as a kind of barrier by learned gentlemen and state agencies. This ``constructed edifice'' aimed to shift the entire measure of responsibility onto the Chernobyl personnel, onto the author of the book, onto the engineers who dared to hold opinions that did not coincide with those of doctors and scientists more concerned for the honour of their uniform.

In principle, the expectation that a verdict on this matter would be rendered by the Prosecutors General of Russia and Ukraine has not been justified. They have no time for this today. But perhaps the National Academies of Sciences of Russia and Ukraine could find within their ranks men of courage capable of speaking out on this matter, given that no fundamental research is required. Yet no one wishes to look back and consider the engineering practice of that time -- or perhaps something else. But there is no doubt that a verdict must be rendered which will allow this tragedy to be laid at last in the drawer of history. I should like to see such a document in my lifetime. Let us not lose hope.

\begin{flushright}\small{
-- V.A. Chugunov

\textit{Engineer at the Chernobyl Nuclear Power Plant, former supervisor of reactor hall No.~1}
}\end{flushright}

% Main matter
\mainmatter

\chapter{Five Years Later}

At 1:23:40~a.m. on 26~April~1986, Aleksandr Akimov, the shift supervisor of Unit~4 at the Chernobyl Nuclear Power Plant (\textit{ChAES}), gave the command to shut down the reactor upon completion of the pre-maintenance test. The order was issued in a calm and orderly atmosphere; the central monitoring system registered neither emergency nor pre-alarm signals, nor any indication of deviation in the parameters of the reactor or its servicing systems

The reactor operator, Leonid Toptunov, removed the protective cover from the AZ\footnote{\textit{Avariynaya Zashchita}, or Emergency Protection -- the rough equivalent of the SCRAM procedure employed in Western reactor designs.} button -- intended to prevent inadvertent activation -- and pressed it. In response, \num{187}~control rods of the reactor's Control and Protection System began to descend into the core. Indicator lamps upon the mnemonic display\footnote{A lamp matrix corresponding to each channel of the reactor, affixed on a wall in the control room.} lit, and the position gauges of the rods moved accordingly. Akimov, standing half-turned toward the reactor control panel, watched the readings and observed that the balance indicators of the automatic regulator \textit{``swung to the left,''} as was proper, denoting a reduction in reactor power. He then turned toward the safety panel to continue his supervision of the test.

% --

But then occurred that which no imagination, however unrestrained, could have foreseen. After a momentary decline, the reactor's power began suddenly to rise -- and with ever-increasing velocity. Alarm signals flashed forth. Toptunov cried out that the power was mounting abnormally, yet nothing more could be done. All measures within his control had already been taken -- he continued to press the AZ button, and the control rods of the system still descended into the core. No further means remained, either to him or to any other.

Akimov exclaimed sharply, \textit{``Shut down the reactor!''} and ran to the panel in order to cut power from the electromagnetic clutches of the control-rod drive mechanisms. The action was correct, yet futile. The logical circuits of the control-and-protection system -- all its logical elements -- had performed their functions properly; the rods were indeed moving into the core. It is now manifest that once the AZ button had been pressed, no further correct action could alter the event's course. It was a different level of causality that had failed.

Two violent explosions followed in swift succession. The rods ceased their movement before reaching half their intended depth. There remained no path open to them.

At 1:23:47~a.m. the reactor was annihilated by a thermal explosion resulting from a runaway power excursion on prompt neutrons. It was the ultimate catastrophe -- the utmost accident conceivable within a nuclear reactor. Never had such an occurrence been imagined; no preparation had been made for it, nor had any technical provision for its containment been contrived, either within the unit or at the plant as a whole. Neither were there organizational measures adequate to such a circumstance.

% --

Confusion, disbelief, and a complete incapacity to comprehend what had occurred held us but for a moment. Urgent tasks, the completion of which drove all other thoughts from our heads, pressed upon us, demanding instant action and leaving no room for reflection. Looking back now, I cannot say whether the event seems distant -- for more than five years have elapsed -- or recent, for the images remain as vivid as ever. Yet of one thing I am absolutely certain: in those circumstances we did all that was humanly possible. Nothing more effectual could have been accomplished.

There was no panic, no hysteria. None departed the control room of his own accord; all withdrew only upon command. Every one of us came forth grievously injured -- many unto death. It must be said: these were men of their craft, fully conscious of the perils inseparable from their duty. They did not falter. Their skilful, courageous, and well-nigh self-sacrificing labor in those first moments deserves the highest acknowledgment.

I do not purpose to trace the origins of such conduct, nor to analyse the psychology of men placed in extremity -- that belongs to the province of literature. My intention is more limited: to explain how it came to pass that people were found in circumstances demanding the utmost exertion of moral and spiritual strength. Was this an inevitable consequence of the employment of nuclear power, or did other causes prevail?

I shall speak only of the past and adhere strictly to verified fact. Every statement herein can be confirmed, and I can indicate the locations wherein the supporting documents are preserved. The matter is too grave -- it touches innumerable lives and generations yet unborn -- for conjecture or embellishment. I lay no claim to literary art, nor would I have undertaken this task; but five years have passed, and still no true and faithful account of the events, or of their causes, has been given. A duty must be discharged -- to those colleagues who perished (or, more truthfully, were killed).

% --

\begin{personal}[From the ruling of the prosecutor's office:]
\par \textit{"Criminal prosecution is terminated in regard to A.~F.~Akimov, L.~F.~Toptunov, and V.~I.~Perevozchenko on the basis of Article~6, Clause~8 of the Criminal Procedure Code of the Ukrainian S.S.R., 28~November~1986."}
\end{personal}

They too would doubtless have been tried and imprisoned, had they survived. Now they can no longer speak in their own defence. As though their families had not already suffered enough, the prosecutor's office reminds them: \textit{Your son, your father, your husband was a criminal -- remember that.} A true death-grip; but they seized the wrong men.

No, I have not been silent during these five years. Refusing to acknowledge guilt -- either personal or on behalf of the staff -- for the reactor's destruction, I prepared detailed technical arguments in vindication of our actions. To whom did I send them? It were easier to say to whom I did not. All proved in vain. Only R.~P.~Sergienko, in his film, and the Ukrainian newspaper \textit{Komsomolskoye~Znamya}, afforded me opportunity to speak a little. Yet within the compass of a film or a newspaper column it was impossible to expound a matter so complex. I write now and cannot but wonder -- will this ever be published?

There is a peculiar circumstance in our blessed land: once certain persons gain access to the press, the door is closed to all others. Perhaps it is ordained so -- for why should differing opinions be permitted upon the same affair? Truth, after all, is held to be single. When I was in Germany, they deemed it worth while to devote nearly half an hour of television airtime, and to print a newspaper article -- without the least solicitation on my part.

In October~1990 I read the report prepared by the IAEA (\textit{International Atomic Energy Agency}) team of specialists, published in 1986 following the presentation by Soviet experts in Vienna concerning the causes of the Chernobyl catastrophe. Since the Soviet delegation, led by V.~A.~Legasov, neither sought the truth nor spared the operating staff from reproach, but rather defamed them by distortion and omission, the IAEA report inevitably abounded in serious inaccuracies. I forwarded my comments upon that report to the Director-General, Hans~Blix.

Then a curious event occurred: my notes, by some means, came into the hands of the editor of the journal \textit{Nuclear Engineering International}. He invited me by letter to contribute an article, which indeed appeared in the issue of November~1991. Sensible men desired to learn from the errors of others; we, however, seem unwilling even to learn from our own -- each must bruise his own forehead against the same wall.

% --

I once read in \textit{Ogonyok}\footnote{\textit{Ogonyok} -- one of the oldest weekly illustrated magazines in Russia.} an interview with the scientist A.~P.~Aleksandrov -- guerilla-like, partisan in spirit, in that it clung obstinately to a single thesis which laid the entire blame upon the operating staff. I composed a reply and brought it to the editorial office. I did not ask them to take my word for granted; I indicated precisely where every assertion I made might be verified. I consented to any editorial alteration, provided the meaning were preserved. They did not publish it. They wished one side to be heard, and not the other. I understand -- space in \textit{Ogonyok} is limited. Yet thereafter, space was somehow found for the slanderous inventions of Kevrolev and Asmolov concerning the personnel of the station. I say it plainly: slanderous. And this was in 1991!

It would be unjust to maintain that nothing has changed. In spite of the opposition of powerful circles of professors and scientists, the steadfast labours of certain individual investigators -- V.~P.~Volkov, A.~A.~Yadrikhinsky, B.~G.~Dubovsky -- and now of several research collectives, are gradually disclosing the true causes of the catastrophe. Considering the resistance they have encountered, it could scarcely have been otherwise. Yet that, too, is not quite exact: the causes are not now being discovered; they have long been known, and to the reactor's designers they were evident immediately after the accident. What is new is only this -- that they are at last being committed to writing, a thing once strictly forbidden. Even now, such information circulates within but a narrow circle, and obstacles yet remain. A strange community of scientists we gathered -- doctors and professors who for years have looked directly upon the obvious and perceived it not.

Still, I believe the truth will at length be made public -- and I believe it will come to light not in fifty years, but sooner.

% --

The official version of the causes of the catastrophe of 26~April~1986 -- still unaltered to this day -- placed, without qualification, the entire responsibility upon the operating staff. Clarifications appeared only later. Why this occurred is difficult to determine with certainty, yet I shall set forth how I comprehend it. I have pondered the matter often; some points are clear, others remain in obscurity.

As to the conclusions of the official commissions, the situation was plain enough. At that time no other verdict could well have been rendered, for the investigation was entrusted -- most unnaturally -- to the very designers of the reactor, that is, to the possible culprits themselves. Not one commission included a person who might have had any interest in naming the reactor, or its structural features, as the source of the accident. On the contrary, it was convenient for all -- directly, indirectly, or by mere professional solidarity -- to lay the blame upon the operators.

Above all, it was easy and familiar. The inquiry followed a well-worn Soviet course: in our country, no accident could be attributed to a defect of design or to a systemic fault -- always to the negligence or incompetence of those at the controls. Even had the commissions arrived at conclusions nearer the truth (and that cannot be ruled out), such findings would have been suppressed for ``political'' considerations, and the official version would nevertheless have been affirmed. No other result was possible.

As for the press -- why did our vigilant and conscientious journalists accept all that they were told so readily, so uncritically? Why did none inquire into the partial and prejudiced composition of the commissions? To be sure, those bodies appeared impressive, authoritative, beyond reproach. Yet there were dissenters -- men whose opinions stood in open contradiction to the official line. They were disregarded.

% --

The press devoted itself wholly to the vilification of the operating staff -- from every quarter and with varying degrees of zeal. Apart from two articles in \textit{Literaturnaya~Gazeta}, which explained what the \textit{RBMK} (High-Power Channel-Type Reactor) is and described certain of its structural features, scarcely anything appeared that offered a differing view. One journalist, M.~Odints, even reproached A.~Dyatlov for presuming to defend himself in court. In our unquestionably righteous Soviet tribunals, it would seem, one was not expected to defend oneself at all.

Yet open hostility was at least sincere, unlike those accusations disguised beneath a veil of sympathy. Thus, in an interview with \textit{Argumenty~i~Fakty}, the head of the Chernobyl press office, Kovalenko, confidently asserted: \textit{``All textbooks and manuals state that a reactor cannot explode under any circumstances.''} He added further: \textit{``That was how it seemed at the time. They lived by the knowledge and understanding of their era. Back then they were sure that, whatever one did to the reactor, an explosion was impossible.''}

I have never encountered a single textbook or manual that declared a reactor explosion impossible under all circumstances. By 1986 I already knew of at least five such explosions within our own country. Every reactor operator -- and most certainly every RBMK operator -- knows precisely what may and may not be done with a reactor. Whether or not it explodes, a grave accident is inevitable if the rules of operation are violated.

They portrayed us as morons -- as though nothing better might have been expected. It is true that Dr.~O.~Kazachkovsky, contrariwise, termed us \textit{``professionals''} -- a balm to the soul, though not without its admixture of tar. Many took occasion to speak at our expense. According to them, the staff committed violations so extraordinary as to border upon the fantastic. But that is the privilege of the scientist: he is inventive. And the press dutifully transmitted these inventions to the public.

In the end, the conclusions regarding the causes of the disaster which gained currency among both the technical community and the general public were those adopted by the Interdepartmental Scientific and Technical Council (\textit{МВТС}), presided over by the President of the USSR Academy of Sciences, A.~P.~Aleksandrov. Yet the press somehow failed to remark that this same president was himself the inventor and scientific overseer of the RBMK project. How such a circumstance accords with ethics is a question worth posing; as for legality, there is, indeed, nothing to discuss.

% --

The first to question the official explanation of the causes of the catastrophe were the operating personnel of those nuclear power stations equipped with RBMK reactors. This was natural: it sufficed for them to observe and to analyse the technical modifications introduced at the remaining plants to discern the true condition of such reactors as of April~1986. They knew well what they had relied upon for years. Yet theirs was a narrow circle -- technically competent, but necessarily limited in number.

The authorities, however, regarded the matter as concluded: the culprits had been identified, named, and imprisoned; therefore, all was deemed in order. Society reacted otherwise -- indeed, and most logically, in the opposite direction. The disaster had entailed grave consequences, rendering vast territories unfit for habitation for decades. The public was informed that this was the result of human error on the part of the operators. Could such errors be excluded in future? Assuredly not. No rational person could honestly answer \textit{``yes.''} However well trained the operators might be, errorless work can never be guaranteed. There are thousands of operators; the logical conclusion, therefore, was that the very use of nuclear energy was impermissible.

We know to what that reasoning led -- and we have yet to see to what further consequences it may give rise. Could such a reaction have been foreseen? Certainly; it was the most natural response of human reason and emotion alike. Yet the capacity to foresee the social consequences of decisions has never been the strength of Soviet officials. That capacity never even developed, because it was never required. Once the propaganda apparatus was set in motion, black became white, and, when expedient, coercive measures were ever at hand. Thought was unnecessary.

The Soviet engineering corps -- beyond question skilled and able -- had no means of forming an independent judgment, being held in complete informational isolation. And yet, what was there to conceal, if the entire matter had already been laid before the IAEA?

The position adopted by the IAEA experts merits separate consideration. In conformity with the new political course of the Soviet Union, a group of specialists prepared a report for the international community upon the Chernobyl disaster. It consisted of two parts: a shorter volume addressing the causes of the accident, and a larger one concerning its radiological and medical consequences. The latter need not detain us here; the former -- how the causes of the accident were presented to the world -- will be examined in detail hereafter.

For the present, let us note only this: the reactor exploded under conditions entirely ordinary and routine:

\begin{enumerate}
  \item There were no natural calamities -- no floods, no earthquakes, no meteorite impacts.
  \item There were no acts of sabotage.
  \item There were no acts of terrorism.
\end{enumerate}

Although the IAEA experts possessed incomplete information, they were nevertheless supplied with the essential data of the accident, together with the graphs of the principal parameters. Yet, under such circumstances, they in effect accepted the Soviet account and likewise laid the blame upon the operators.

This prompts a question: do these experts truly believe that a reactor -- constructed in full conformity with all regulatory standards -- can explode solely through the fault of its operators? If so, their advocacy of nuclear power is ethically indefensible. Operators in the West, too, are liable to error.

% --

After the catastrophe, I examined numerous regulatory documents governing the design of nuclear reactors within the Soviet Union and found no circumstance in which a reactor constructed in accordance with the \textit{Nuclear Safety Rules}\footnote{\textit{Правила ядерной безопасности} (Nuclear Safety Rules).} and the \textit{General Safety Provisions}\footnote{\textit{Общие положения безопасности} (OPB). The translated contents of OPB-82, in effect at the time of the catastrophe, are given in the Appendix~E.} could possibly explode. I did not take into account natural anomalies or acts of sabotage. I do not, of course, claim exhaustive completeness -- such a task surpasses the powers of any single investigator. Yet the collective authors of those documents evidently conceived of no such eventuality either; otherwise, they would surely have prescribed appropriate countermeasures.

Given the professional qualifications of the IAEA (\textit{International Atomic Energy Agency}) experts, it should not have been difficult for them -- using the materials provided -- to discern the numerous discrepancies between the RBMK design and the established safety standards, and to conclude that such a reactor was unfit for service. Nor would it have been difficult to perceive that the supposed procedural violations (which, in fact, never took place) could not, under any sound design conforming to those standards, have caused an explosion.

Apparently overwhelmed by the quantity of information -- information which, until then, they had received from the Soviet Union only through strictly controlled channels -- the IAEA experts hastened to publish their report, repeating in substance the position of the Soviet delegation. How else is one to interpret the following passage?

\begin{personal}[From the IAEA report:]
\textit{"This printout shows that too many control rods were withdrawn from the reactor core, and that it did not possess sufficient reactivity margin to meet the requirements for shutdown. At that time, the operator should have shut down the reactor."}
\end{personal}

Let us consider what is wrong here, and why such a statement is unworthy of specialists of their standing:

\begin{enumerate}
\item There existed no such printout, it was reconstructed after the event. Giving benefit of the doubt, let us ascribe that falsehood to the Soviet informants.

\item The document in question corresponds to 1:22:30~a.m. The AZ button was pressed at 1:23:40~a.m. Between these times lies scarcely a minute -- supposedly sufficient for a complete analysis of the positions of \num{211}~control rods. Which is not quick to perform, by the way. The Soviet informants clearly sought to cast blame upon the operators; but why did the IAEA experts not pause to reflect?

\item A logical contradiction follows. They write: \textit{"\ldots that it did not possess sufficient reactivity margin to meet the requirements for shutdown. At that time, the operator should have shut down the reactor\ldots"} As it turns out, \textit{``Execute not pardon''} has no punctuation at all\footnote{The author refers to a children's novella \textit{``In the Land of Unlearned Lessons''} by a Soviet writer Liya Geraskina, as well as its popular animated adaptation. The protagonist was sentenced to death, but since his written sentence lacked any punctuation, he was challenged with resolving the ambiguity in his favor by placing a comma in the right place: \textit{Execute, [do] not pardon} or \textit{Execute not, pardon}. Dyatlov argues that there was no solution to the Chernobyl puzzle: an attempt to \textit{``place a comma''} anywhere leads to the catastrophe.}. Suppose that we had indeed noted a diminished margin: according to the Operating Regulations, upon any deviation of parameters, the protective system must be engaged -- and that is precisely what we did at the end of maintenance. We pressed the protection button, and the explosion ensued.

\item The principal question is this: what does it mean to say that the reactor \textit{``\ldots did not possess sufficient reactivity margin to meet the requirements for shutdown\ldots''}? In reactor physics it is well understood that a reactor must have a reactivity not greater than can be compensated by its control elements. This principle is both physically and logically sound. Yet in no textbook is there mention of any \textit{minimum} reactivity margin. RBMK project documents, instructions and the Operating Regulations have no mention that the AZ button was intended to serve as a power-augmenting device -- a \textit{``universal protection.''} Nevertheless, the post-accident analysis revealed precisely that effect. The design of the control rods was defective. The constructors had, in essence, produced a mechanism capable of amplifying reactivity at the very instant when it ought to suppress it. Even after the explosion, the Soviet representatives at the IAEA maintained that nothing was amiss -- and, astonishingly, they succeeded in alluring the experts.
\end{enumerate}

Incredible. The Soviet side knew perfectly well that such a rod design was unfit for service: immediately after the catastrophe, the range of rod insertion was curtailed, and later the rods were replaced entirely. Their aim was simple -- to convince the world that \textit{``bad operators destroyed a good reactor.''}\footnote{A tongue-in-cheek reference to the phrase \textit{``Good Tsar, bad Boyars,''} used in Russian historiography to describe the tendency to attribute state mismanagement to the incompetence of officials rather than to systemic flaws.} And in this, they succeeded.

Why Legasov was not awarded the title of Hero, as V.~Gubarev once lamented, I cannot comprehend.

With such handling of truth, perhaps next time they will content themselves merely with silence. Thirty words in that passage -- and how much falsehood they conceal. Other statements within the IAEA report are of like character. And that report spread throughout the world, soothing the conscience of the Soviet informants.

Whether the IAEA experts, when revisiting the report in 1991, found the fortitude to produce a version faithful to reality and worthy of their institution -- the near future will show.

Of course, scientists throughout the world continue to study the Chernobyl catastrophe, and they are not disposed to accept assertions uncritically. Yet most of their investigations concern isolated aspects, while the IAEA report -- comprehensive in form -- continues to exert a harmful influence.

% --

It is difficult for any man to withstand the torrent of information that was then unleashed. And this was not mere \textit{``information.''} Reports of government commissions, conclusions of committees, newspapers, journals, and writers -- all spoke with a single voice, all played the same tune. How, indeed, could one fail to believe them?

For why, one might ask, should B.~E.~Shcherbina, Deputy Chairman of the Council of Ministers, utter what was untrue? Why should he, the chief authority over all, forget the guilty and accuse the innocent? Therefore it must have been as he said. Why should another Deputy Chairman, G.~Vedernikov, simply lie by asserting that all \textit{``four stages of foolproof protection''} at Chernobyl had been disabled? He would have had no motive; therefore that, too, must have been true.

The German magazine \textit{Der~Spiegel} (No.~29,~1987), beneath a photograph of the defendants -- Director Bryukhanov, Deputy Chief Engineer Dyatlov, and Chief Engineer Fomin -- published the caption: \textit{``Disorder, negligence, carelessness.''} It is difficult to conceive how such qualities could have altered the physical properties of a reactor. Was it offended, perhaps? Yet the weight of the official accusation pressed -- and still presses -- upon the public mind.

No, people thought, such a disaster could not have occurred of itself. The reactors had operated before, had they not? As though that proved anything. They failed to perceive what was self-evident:

\begin{enumerate}
  \item Why were the materials of the investigation classified? They remain inaccessible even now, though the reactor itself was never secret.
  \item One need not be a specialist to understand that -- whatever errors the operators may have made -- the reactor exploded under routine conditions. Therefore, such a reactor is unfit for service.
  \item Why has not a single accident within the Soviet Union ever been officially attributed to equipment failure? Of course, Soviet equipment was proclaimed to be \textit{``the best in the world,''} yet it was far from flawless.
\end{enumerate}

Perhaps I am mistaken. Therefore I shall attempt to answer these and other questions.

\chapter{The Power Plant}

The Chernobyl Nuclear Power Station stood upon the banks of the Pripyat River, a tributary of the Dnipro. By 1986 it had grown into a vast industrial complex, with an aggregate capacity of four million kilowatts. The first reactor unit entered service on 26~September~1977, the second in December~1978, the third in December~1981, and the fourth in December~1983.

Concurrently with the plant's construction, the operations staff was organized and steadily developed. Recruitment and retention presented little difficulty -- owing, doubtless, to the prospect of housing and to the plant's advantageous location. The reactor department of the first unit was staffed chiefly by specialists already experienced in operating similar industrial reactors; these men became the professional foundation of the enterprise. Later, as that reservoir of trained personnel was depleted, staff could be transferred internally from operating units to those newly commissioned. The usual difficulties attendant upon a growing enterprise were felt, yet they were mitigated by the gradual nature of the plant's expansion.

In general, the plant's operation was satisfactory. Prior to 1986 only one serious incident had occurred -- the rupture of a process channel in Unit~1, in 1982. This necessitated prolonged repairs and caused appreciable radiation exposure among maintenance personnel, though the doses remained within occupational limits.

There was also a single instance of surface contamination within the plant grounds -- several dozen square metres -- caused by a decontamination solution released during the washing of the primary circuit after a minor pipeline leak. The affected topsoil was removed and buried. In sum, the Chernobyl plant experienced fewer incidents than the national average for nuclear power plants.

In the period immediately preceding the disaster, annual electricity generation reached approximately \num{28}~billion~\unit{\kWh}, only slightly less than that of the Leningrad Nuclear Power Station. At Leningrad, however, the workforce was long established, whereas Chernobyl suffered continual turnover and the constant influx of new employees. In 1985--1986, part of the experienced operating staff was transferred to the fifth unit, then under construction. Naturally, only the most capable were chosen, for several reasons:

\begin{enumerate}
  \item The plant formed a single organization; those transferred were not departing, and all understood the difficulties inherent in commissioning a new unit.
  \item Such transfers were ordinarily accompanied by promotions, and it would have been improper to withhold them.
  \item The management of the third sector (Units~5~and~6) consisted of the plant's own personnel, who were well acquainted with the rest of the staff.
\end{enumerate}

At Chernobyl, the promotion of mid-level technical managers customarily took place from within the ranks of the station itself, save in its earliest years. This practice possessed both merits and defects; yet, on the whole, I am convinced that its advantages prevailed.

All unit shift supervisors had served at Chernobyl not less than five years. They were no mere administrative functionaries confined to desks, but were directly engaged in the conduct and supervision of the technological process.

After the catastrophe, the entire operating staff underwent re-examination -- most rigorously, as one may imagine -- and were certified competent for duty. I refer here to the report of the State Committee for Industrial and Nuclear Power Oversight (\textit{Gospromatomenergonadzor}) dated 4~January~1991:

\begin{personal}[From the Gospromatomenergonadzor report:]
\textit{"The research conducted by the sectoral psychological laboratory 'Prognosis' of the USSR Ministry of Atomic Energy and Industry (similar studies were carried out elsewhere -- A.D.) yielded results analysing the personal and socio-psychological characteristics of the Chernobyl staff before and after the accident. These showed that the personal data of the Chernobyl operational personnel did not differ in any significant way from those of staff at other nuclear power plants that could have directly caused the accident. Overall, the Chernobyl staff in 1986 was characterized as an ordinary, mature, established group of qualified specialists -- at a level recognized in the country as satisfactory. The team was neither better nor worse than that of other nuclear power plants."}
\end{personal}

Why then, one must ask, did these ordinary and competent operators suddenly commit \textit{"an extremely improbable combination of procedural and operational violations,"} as Soviet representatives asserted before the IAEA? Was the shift of 26~April in any respect extraordinary? By no means -- it was a routine shift. And is there not, in that official account, an excess of the \textit{"improbable"}? Assuredly something unusual occurred; of that I shall speak later.

One sector\footnote{Очередь -- also queue, phase, or stage. The term is used in the original manuscript, as well as in secondary and tertiary sources, to denote an administrative subdivision of the plant.} of the station comprised two power units, though in practice each unit functioned independently, with little interconnection. The principal equipment of a unit consisted of the reactor, two turbogenerators,\footnote{A turbogenerator converts thermal energy from steam into electrical energy. It comprises a turbine (driven by steam) and a generator.} and a transformer.

\section{The RBMK-1000 Reactor}

A nuclear reactor is a device engineered to convert nuclear energy into heat. In most reactors the fuel employed is low-enriched uranium. Naturally occurring uranium consists of two isotopes: about \num{0.7}~percent with mass number~235, and the remainder with mass number~238. Of these, only uranium-235 serves as fuel. When a nucleus of uranium-235 captures a neutron, it becomes unstable and in an instant divides into two, generally unequal, fragments, releasing a large amounts of energy. Each such fission event yields millions of times more energy than the combustion of a single molecule of oil or gas. In a large reactor such as that of Chernobyl, operating at full power, roughly four kilograms of uranium are \textit{``burned''} each day\footnote{The four \unit{\kilogram\per\day} that Dyatlov refers to is plausible, but it refers to fissile destroyed (mainly Uranium 235 equivalent), not total fuel mass. For a \qty{1000}{\mega\watt} (electrical) unit, like the RBMK-1000, at ~\qty{33}{\percent} efficiency, Thermal power $P_{\mathrm{th}}\approx \qty{3000}{\mega\watt}$ (thermal). Energy per day $E\approx \qty{3000}{\mega\Wd}$. One kilogram of Uranium 235 fission gives $\qty{8.2e13}{\joule}$. $\qty{3000}{MWd} = \qty{3.0e9}{\watt}\times\qty{86400}{\second}=\qty{2.592e14}{\joule\per\Day}$. Fissile burned/day, $\num{2.592e14} / \num{8.2e13} \approx \qty{3.2}{\kilogram\per\Day}$ is commonly rounded to about \qty{4}{\kilogram\per\Day}. In real reactor cores, about \qtyrange{20}{40}{\percent} of fissions come from Pu-239 bred from U-238. So the U-235 share might be \qtyrange[range-open-phrase={\text{between} },range-phrase={ \text{and} }]{2}{3}{\kilogram\per\Day}, with the rest from Pu-239.}.

The energy released in every fission appears chiefly as the kinetic energy of the fission fragments,\footnote{Fission fragments are the freshly formed nuclei immediately following the split, highly energetic and unstable. The term \textit{fission products} is broader, including not only the initial fragments but also their radioactive decay products.} which, in slowing down, impart almost all their energy to the fuel rod\footnote{Heat-releasing element (\textit{ТВЭЛ}).} and its metallic cladding. Any appreciable escape of fission fragments beyond that cladding is impermissible.

From the periodic table it is evident that the fission fragments possess an excess of neutrons and are therefore unstable. Through a sequence of $\beta$-decays they gradually transform into stable isotopes, moving rightward across the periodic table. Each transformation proceeds along its own decay path and half-life, accompanied by the emission of $\beta$-particles and $\gamma$-radiation. These fission fragments constitute the principal source of radioactive contamination released in any accident wherein the fuel rods are ruptured or expelled.

During normal operation, the $\beta$-particles remain confined within the fuel rods, depositing their energy therein, while most of the $\gamma$-radiation is absorbed within the reactor structure. After the chain reaction is halted in shutdown, the residual heat arising from the decay of fission products continues to demand the sustained cooling of the fuel elements.

Each fission of a uranium nucleus releases, on average, about two and a half neutrons. Their kinetic energy is absorbed by the moderator, the fuel, and the structural materials, and is thence transferred to the coolant. These neutrons sustain the chain reaction in uranium-235. If, on average, one neutron from each fission provokes one new fission, the rate of reaction remains constant.

Most neutrons are emitted instantaneously at the moment of fission; these are termed \textit{prompt} neutrons. A small fraction -- about \num{0.7}~percent -- are emitted with a brief delay of several seconds to several tens of seconds. These \textit{delayed} neutrons render possible the control of the fission rate and the regulation of reactor power. Without them, a nuclear reactor could not operate at all -- only an atomic bomb would be conceivable. The remainder of the fission energy appears as prompt $\gamma$-radiation, emitted at the moment of division, and as the energy carried away by neutrinos, which are neither captured nor detected.

Nuclear reactors generally employ uranium enriched in the isotope~235, although uranium-238 remains the predominant constituent and absorbs a considerable number of neutrons. When a nucleus of uranium-238 captures a neutron, it becomes unstable and, through two successive $\beta$-decays, is transformed into plutonium-239, which -- like uranium-235 -- is capable of fission under the impact of thermal neutrons (i.e., neutrons with low kinetic energy). The nuclear properties of plutonium differ from those of uranium, and as it accumulates during prolonged operation, the physical behaviour of the reactor alters accordingly.

Plutonium released in an accident likewise contributes to environmental contamination. Its decay is exceedingly slow -- the half-life of plutonium-239 exceeds \num{24000}~years -- so that only the gradual migration of the element into deeper layers of the earth may diminish its danger. Other isotopes of plutonium are also present.

Key properties of uranium-235 are as follows:

\begin{itemize}
\item It undergoes fission upon absorbing a thermal (low-energy) neutron.
\item Each fission event releases a considerable quantity of energy.
\item Fission produces additional neutrons, thus permitting a self-sustaining chain reaction.
\end{itemize}

Uranium-235 constitutes the fundamental fuel of nuclear reactors.

Almost all nuclear reactors operate with thermal neutrons. The neutrons produced by the fission of uranium or plutonium undergo moderation, diffusion, and eventual capture by the nuclei of the fuel or by structural materials. A portion escapes from the core -- this is termed \textit{leakage}.

Because fission occurs continuously, the reactor maintains a high population of neutrons, known as the \textit{neutron flux} or \textit{neutron field}. The fuel is consumed only gradually; therefore, over a considerable interval, the total fuel mass may be regarded as constant. Under such conditions, the number of neutrons absorbed by the fuel -- and consequently the number of fissions and the quantity of energy generated -- is directly proportional to the neutron flux in the core. In practice, reactor operators observe and regulate the neutron flux to maintain the desired power level.

If we conceptualise the fission neutrons as belonging to successive generations -- recognizing that this is but an analogy, since fissions are not synchronous -- then, if each generation contains the same number of neutrons, the reactor power remains constant. Such a reactor is said to be \textit{critical}, with a neutron multiplication factor $K = 1$. If $K > 1$, the neutron population and the power increase continuously: the reactor is \textit{supercritical}. The greater the value of $K$, the more rapid the exponential rise of power.

In operations it is customary to employ the concept of \textit{reactivity} $\rho$ in place of $K$; for $K$ close to $1$, $\rho \approx K - 1$. In routine operation, positive reactivity does not exceed a few tenths of a percent. Larger values cause the power to rise too swiftly, endangering the integrity of the reactor and its auxiliary systems. All power reactors are equipped with an automatic emergency protection system that shuts down the reactor if the rate of power increase exceeds safe limits. In the RBMK, this system -- designated \textit{AZ} -- was actuated when reactor power doubled within twenty seconds.

A crucial point: only about \num{0.7}~percent of neutrons are \textit{delayed} neutrons. These lengthen the effective neutron generation time. If the positive reactivity attains $\beta$ (the delayed-neutron fraction) or exceeds it, the reactor becomes critical on \textit{prompt} neutrons alone. The generation time then becomes exceedingly short -- determined only by moderation and diffusion -- and the rise of power is rapid. In such a condition, no protective action can intervene; the chain reaction can be terminated only by the physical destruction of the reactor itself. That, precisely, was what occurred on 26~April~1986 at Unit~4 of the Chernobyl Nuclear Power Plant.

In fact, owing to the accumulation of plutonium within the core and to the differences between prompt and delayed neutrons, the effective $\beta$ in the RBMK reactor was not \num{0.7}~percent but approximately \num{0.5}~percent.

\section{Design Overview}

The RBMK-1000 is a channel-type reactor employing graphite as a moderator and ordinary water as a coolant. The fuel assembly (\textit{ТВС}) consists of thirty-six fuel rods (\textit{ТВЭЛ}), each \num{3.5}~metres in length. The rods are spaced by grids mounted upon a central carrier rod and disposed in two concentric circles: six rods in the inner circle and twelve in the outer.

Each assembly is composed of two tiers in height, giving the reactor core an overall height of seven metres. Each fuel rod contains pellets of $\mathrm{UO_2}$, stacked within a hermetically sealed tube of zirconium--niobium alloy. Unlike vessel-type reactors, in which all fuel assemblies are enclosed within a single pressure vessel, the RBMK places each assembly in its own \textit{process channel} -- a tube of \qty{80}{\milli\metre} internal diameter.

The RBMK core, \qty{7}{\metre} high and \qty{11.8}{\metre} in diameter, is formed of \num{1888} graphite columns, each bored centrally to create a channel. Of these, \num{1661} are fuel process channels containing fuel assemblies; the remainder belong to the Control and Protection System (\textit{СУЗ}) and house \num{211} neutron-absorbing control rods together with \num{16} monitoring detectors. These channels are distributed uniformly throughout the core, both radially and azimuthally.

The coolant -- ordinary water maintained under high pressure -- is introduced from below into the process channels to remove heat from the fuel rods. A portion of the water is converted into steam, and the resulting steam--water mixture ascends to the drum separators, where the steam is disengaged and directed to the turbines. The separated water is then pumped by the Main Circulation Pumps (\textit{ГЦН}) back to the channel inlets. After performing work in the turbines, the steam is condensed and returned to the coolant circuit. Thus the circulation loop of the reactor is closed.

Given this configuration of the core, let us consider the fate of the fission neutrons. Some escape beyond the boundaries of the core and are thereby lost. Others are absorbed by the moderator, the coolant, the structural materials, and the fission products -- these constitute non-productive losses. The remainder are absorbed by the fuel itself. To maintain constant power, the number of neutrons absorbed by the fuel must likewise remain constant. Since each fission liberates on average about \num{2.5}~neutrons, up to roughly \num{1.5}~neutrons per fission may be \textit{``spent''} upon leakage and non-fuel absorption in a critical reactor.

Such a reactor cannot operate stably of itself, for fission produces a variety of secondary elements, among them xenon-135 in appreciable quantity, possessing an exceedingly large neutron-absorption cross-section. When the reactor power increases, xenon accumulates and may wholly suppress the chain reaction. This phenomenon first manifested itself in the earliest American reactor. Enrico Fermi, having computed the capture cross-section of xenon, observed that the nucleus appeared \textit{``as large as an orange.''}

To compensate for this and for similar effects, reactors are constructed with an excess of fuel reactivity. With leakage and non-fuel absorption essentially constant, this design increases the proportion of neutrons absorbed by the fuel. To prevent a continuous rise of power under such conditions, reactivity-control elements -- materials that absorb neutrons strongly -- are introduced into the core. The means of such control vary, but here we shall confine attention to the RBMK type.

The channels of the Control and Protection System contain rods composed of a potent neutron absorber, boron. These rods maintain the necessary neutron balance and, consequently, the desired level of power. When it becomes necessary to raise power, certain rods are partially or wholly withdrawn from the core, thereby increasing the fraction of neutrons captured by the fuel; the power then rises, after which the rods are reinserted to stabilize the new level. The new positions of the control rods ordinarily differ from the previous ones, in accordance with the reactor's power coefficient of reactivity.

To decrease power, the control rods are inserted, thereby introducing negative reactivity; the reactor becomes subcritical, power declines, and stability is restored by fine adjustments of rod position. These operations are executed by the Automatic Regulator (\textit{АР}): the operator selects a new power level by pressing a control button, and the regulator performs the necessary movements. In the RBMK, however, this procedure was not wholly automatic; operators were often obliged to intervene to correct the regulator's actions, chiefly to equalize the spatial distribution of power among the various regions of the core.

In a newly constructed reactor, the process channels are at first loaded with fresh fuel assemblies. Were all \num{1661}~channels to be charged simultaneously, the neutron multiplication factor~$K$ would become so large that the available control rods could not bring the reactor to a shutdown state. Therefore, approximately \num{240}~process channels are initially loaded with special neutron-absorbing rods in place of fuel, and several hundred additional absorbers are inserted within the central carrier rods of the fuel assemblies. As the fuel is consumed, these absorbers are gradually withdrawn and replaced with fuel assemblies. When all temporary absorbers have been removed, the requisite reactivity is maintained by the regular replacement of the most depleted assemblies with fresh ones -- a process known as \textit{steady refuelling}.

In the RBMK, fuel assemblies are replaced while the reactor remains at power, by means of a special refuelling machine. Consequently, the core at any given time contains a mixture of fully burned, partially burned, and fresh fuel assemblies. The number of control and protection rods is designed to ensure proper regulation under this mixed-burnup regime.

Each control rod contributes a definite amount of reactivity, depending upon its position and upon the configuration of the neutron field. In RBMK practice, reactivity is expressed in \textit{rods}, the effectiveness of one rod being conventionally taken as~\qty{0.05}{\percent}. As noted earlier, the greater the positive reactivity, the more rapid the increase of power; likewise, the greater the introduced negative reactivity, the more rapid the decline.

Because off-normal conditions and system malfunctions may arise, it must at all times be possible to shut down the reactor rapidly, so as to prevent damage. Hence the total number of control rods must exceed that required to render the reactor sufficiently subcritical. When the reactor is critical (that is, $K = 1$, $\rho = 0$, in the neutral sense, not the \textit{catastrophic}), a certain minimum number of rods must remain withdrawn from the core and ready for immediate insertion to terminate the chain reaction. The greater the number of rods available for prompt insertion, the greater the assurance of achieving a rapid and deep shutdown. This principle holds for all reactors designed in accordance with accepted standards of safety.

In every reactor, a portion of the reactivity-control elements remains within the core to permit power manoeuvring. For instance, following a forced reduction of power, the concentration of xenon temporarily rises (\textit{``xenon poisoning''}); to counteract this additional absorption, some operational control rods must be withdrawn. Otherwise the reactor must be shut down and allowed to stand until the xenon decays.

In the RBMK, during operation at power, certain control rods are kept partially or fully inserted to suppress excess reactivity. This condition gives rise to the notion of the \textit{operational reactivity margin}\footnote{ОЗР (\textit{Оперативный запас реактивности}).}. The operational reactivity margin is defined as the quantity of positive reactivity which the reactor would possess were all control rods to be fully withdrawn.

Like every reactor, the RBMK requires a reactivity margin for power manoeuvring. After the accident of 1975 at the Leningrad Nuclear Power Plant, Unit~1, a minimum margin of fifteen rods was prescribed for RBMK reactors, this figure being determined by the need to regulate the power distribution within the core. After the Chernobyl catastrophe, an astounding and monstrous fact was revealed: at low reactivity margin the emergency protection system -- the \textit{AZ} -- did not shut down the reactor, but instead accelerated it. The smaller the reactivity margin, the greater the nuclear hazard of the RBMK. \textit{``Trust our uniqueness.''}

Reactors endowed with such properties exist nowhere else. One might conceive of an \textit{AZ} system failing to shut down a reactor, yet that it should actually \textit{increase} power -- such a thing could scarcely be imagined even in a nightmare.

In like manner to the operational reactivity margin, this discussion will frequently refer to the \textit{steam} (or \textit{void}) effect of reactivity and to the \textit{power coefficient of reactivity}. Let us now define these terms.

Assume the reactor operates at a given power level under constant coolant flow. In each process channel the water is heated to its boiling point, and steam is formed. Along the channel's length, as heat is removed from the fuel rods, a fraction of the water is converted into steam; thus, under steady conditions, a certain proportion of the coolant within the core exists in the gaseous state.

Now let power be increased. Heat generation rises, and with it the amount of steam in the core. Whether this change augments or diminishes reactivity depends upon the ratio of moderator nuclei to fuel nuclei within the core. Water, like graphite, serves as a neutron moderator, and as the proportion of steam grows, the amount of liquid water -- and hence the moderation -- declines. For reasons seemingly dictated by considerations of economy, the designers of the RBMK selected a moderator-to-fuel ratio such that the complete substitution of steam for water would \textit{increase} reactivity by approximately five to six~$\beta$.

Why is this dangerous? Suppose, for example, that an \qty{800}{\milli\metre} coolant pipe were to rupture: within seconds voiding would occur, and a slow-acting \textit{AZ} system could not compensate for the sudden addition of positive reactivity. The inevitable result -- an explosion, as on 26~April~1986.

Yet this is not the sole cause for concern. As power increases, the temperature of the fuel likewise rises, thereby diminishing reactivity. In the RBMK, two rapid and opposing effects chiefly govern the variation of reactivity with power: a negative fuel-temperature effect and a positive steam (void) effect. Together these determine the \textit{fast power coefficient of reactivity} -- the change in reactivity per unit change in power (whether per~MW or per~kW). Other power-dependent influences, such as those connected with graphite temperature and xenon poisoning, though considerable in magnitude, act slowly and do not affect the immediate dynamics of the reactor.

In a properly designed reactor the power coefficient of reactivity should be negative. In that case, any perturbation increasing reactivity first raises power, but the consequent negative feedback promptly diminishes reactivity and stabilizes the system at a new, higher equilibrium. In the RBMK, however, the power coefficient was \textit{positive} throughout a wide range of operating levels -- in direct contravention of regulatory standards. This property was an immediate and decisive contributor to the catastrophe of 26~April~1986.

\chapter{The Program}

The full title of the Program was \textit{``Work Program for Testing Turbine Generator\footnote{The terms ``Turbine Generator'' and ``Turbogenerator'' are used in the book interchangeably.} No.~8 of the Chernobyl Nuclear Power Plant in Joint Run-Down Modes with Its Own Load Requirements.''}

There was nothing exceptional in this Program -- it was an ordinary, properly drafted test procedure. It became known only because the catastrophe occurred during its execution. There existed no technical connection between the Program and the accident; their concurrence was purely accidental, later distorted by the bad faith of the investigators.

Had any automatic protection system been actuated in the final minutes preceding the test (believe not the commissions nor the amateur writers who have asserted that we \textit{``disabled the protections''} -- they were all in operation for the \qty{200}{\mega\watt} mode), the disaster would nevertheless have taken place. Had the Program itself been the cause, the remedy would have been simple: prohibit such tests elsewhere, and the matter would have been concluded. But that was not the case.

\section{Critics of the program}

Some have said: \textit{``The tests cannot be regarded as purely electrical -- they are complex and concern the entire unit.''}
Who, I ask, ever claimed that they were purely electrical? Did they invent that claim themselves? A single glance at the signatures under the Program resolves the matter: the departments of reactor operations, turbine operations, and thermal automation all endorsed it. If the test had been purely electrical, why should their participation have been required?

The State Committee for Supervision of Work in the Industrial and Atomic Energy Sectors wrote in its 1991 report:

\begin{personal}[From the Gospromatomenergonadzor report:]
\textit{``Such tests should be classified as integrated unit tests, and their Program should have been coordinated with the General Designer, the Chief Constructor, the Scientific Supervisor, and the State Inspectorate. However, the regulations in force prior to the accident -- ПБЯ-04-74 (Nuclear Safety Rules) and ОПБ-82 (General Safety Provisions) -- did not oblige nuclear plant management to coordinate such Programs with those organizations.''}
\end{personal}

For the sake of formality, I proposed to the chief engineer that we seek external approval. Coordination with outside institutions was the responsibility of the plant's Technical Department and the chief engineer. I was satisfied with the signatures already affixed.

They said: a nuclear accident occurred, yet the Program had not been coordinated with the plant's Nuclear Safety Department. True; but the introduction of excess reactivity was not caused by the Program. The same commission wrote:

\begin{personal}[From the Gospromatomenergonadzor report:]
\textit{``A specific thermohydraulic feature of the planned regime was a higher-than-normal initial coolant flow through the reactor. Steam content was minimal, with only slight sub-cooling of the coolant at the reactor core inlet. Both factors proved directly relevant to the scale of effects observed during the tests.''}
\end{personal}

In other words, the commission implied that, while the Program might not have caused the accident, it somehow contributed to it. Not so. When coolant flow exceeded nominal, the reactor behaved entirely as designed. Indeed, all design and operational documentation -- including the Operating Regulations -- explicitly stipulate a flow \textit{``not less than''} nominal; nowhere do they prescribe \textit{``not more than.''}

Upon reviewing the full record, the commission found no parameter deviating from the norm up to the instant the Emergency Protection button was pressed. At that moment, coolant flow was already nominal. Sub-cooling was not under operator control and remained constant. Thus the commission's conclusion lacks foundation. And as for the so-called ``difference in reactivity effects'' -- say, if six pumps had been operating instead of eight -- it is as though one were to say: a man drowned at a depth of \num{100}~metres, but had it been \num{90}\ldots

This example shows that even those who partially distanced themselves from the false accusations directed against the operating staff -- by acknowledging, in their own report, the reactor's full non-compliance with the Nuclear Safety Rules and the General Safety Provisions -- could not relinquish the ingrained habit of ascribing fault to personnel. Such insinuations recur throughout their document.

\section{Safety Measures}

The critics' favourite theme. They claim: \textit{``Safety was neglected.''} Yet what, if not safety, does the entire second section of the Program set forth? It enumerates the reserve power sources, the switching sequence, and the list of mechanisms thereby maintained -- sufficient not merely for residual heat removal, but, had it been desired, for continued operation of the reactor at power. One needed only to read; blindness alone could have missed it.

No reactivity effects other than those inseparable from routine operation were either anticipated or obtained during the test. All parameters remained within the limits of the Operating Regulations. The staff acted strictly in accordance with the established documentation; there was nothing to \textit{``invent.''}

\section{Power Level}

The Program set a power range of \qtyrange{700}{1000}{\mega\watt}. Immediately before its execution the reactor stood at \qty{200}{\mega\watt}. The cause of that reduction will be explained later. We threw a bone in the mouths of our accusers they still gnaw on. Even the Soviet informants to the IAEA let themselves be misled: eager to discredit the staff, they, Legasov at their head, declared to the world that operation below \qty{700}{\mega\watt} was forbidden by the Operating Regulations.

Why? Because after the accident it became evident that low power in an RBMK-1000 is the most hazardous regime. And when blame was to be assigned, professors and doctors of science were put forward -- who would suspect such men of deliberate lies?

There are tests for which power level is indeed crucial: the main safety valves, for example, cannot be checked at low power, since their opening swiftly reduces primary-loop pressure and trips the Main Circulation Pumps. For a turbogenerator coast-down, however, power is irrelevant; the reactor was to be shut down at the start of the experiment (see clause 2.12 of the Program)\footnote{A translation of the Program is given in the appendix.}.

Plant instructions on drafting test programs require that a power be stated. When the Program was written it was unknown what operating mode would immediately precede the test, so \qtyrange{700}{1000}{\mega\watt} was indicated as an upper bound, not a lower one. After the subsequent drop during regulator transitions, there was no technical reason to restore power. For a normal reactor -- one complying with the Nuclear Safety Rules and the General Safety Provisions -- this possessed no safety significance. We violated nothing, contrary to all commission statements and informants' claims.

\section{Disabling the Emergency Core Cooling System}

This subject has long since been exhausted. As early as 1986, the G.~A.~Shasharin Commission determined that there existed no connection between this action and either the origin or the development of the accident. Only the scientist A.~P.~Aleksandrov continues to pursue it -- let us wish him success in that endeavour.

The Soviet representatives to the IAEA asserted that the disabling of the Emergency Core Cooling System\footnote{\textit{САОР}, typically referred to as ECCS in Western literature.} had deprived the plant of the means to mitigate the scale of the catastrophe. Without further comment, I cite a passage from the report of the N.~A.~Shteinberg Commission:

\begin{personal}[From the Shteinberg Commission report:]
\textit{``Thus, the possibility of mitigating the accident's scale through the disabling of the ECCS was not lost, but was absent in principle under the specific conditions of 26~April~1986.''}
\end{personal}

\section{Operation with Eight Main Circulation Pumps}

We violated nothing. Such operating regimes are fully permitted by the instructions. There exist no technical impediments to the parallel operation of pumps running at constant speed together with others decelerating, powered by the coasting generator. Once the head of any pump falls below the prescribed limit, its protective system disconnects it\footnote{When the pressure produced by a pump (its head, measured as the height of the water column it can sustain) drops below a specified safety threshold, the system automatically shuts that pump off.} -- precisely as in an ordinary shutdown.

Other criticisms of the Program will be considered later in this text. Even with the knowledge acquired after the catastrophe, were one to compose that Program anew today, nothing essential would be added or removed -- perhaps only a few citations from the Operating Regulations or auxiliary instructions, but nothing of substance beyond that.

\chapter{How it Happened}

A disastrous day. For many it divided life into \textit{``before''} and \textit{``after.''} As for my own, it cleft existence by a deep chasm into two wholly dissimilar parts.

I had been practically healthy, requiring sick leave but three or four days in a year; thereafter I became physically disabled. I had been a reliable, law-abiding citizen; I became a criminal. Finally, I had been a free man; I became what is now officially termed a \textit{convicted citizen.} Who, indeed, conceived so grotesque a combination of words?

Yu.~Feofanov wrote in \textit{Izvestia}\footnote{\textit{Izvestia} -- a daily broadsheet newspaper of Russia, founded in February~1917, devoted to foreign and domestic affairs, and serving as the organ of the Supreme Soviet of the USSR, disseminating official state policy.} that, after analysing recent laws \textit{``protecting human rights,''} one must confess: \textit{``So far, alas, the word 'citizen' in our country is still closer to the word 'come along.'''} Then how much of the \textit{citizen} remains in a \textit{convicted citizen}?\footnote{A bitterly ironic remark of Feofanov, here extended by Dyatlov. In Russian, when the police detained someone, they would often say, \textit{``Grazhdanin, prosim proyti''} -- literally, \textit{``Citizen, please come along,''} meaning in effect, \textit{``You are under arrest.''} Feofanov thus implied that the official term \textit{``citizen,''} intended to denote a person vested with rights, in practice more nearly resembles a command of arrest. Dyatlov, adopting the irony, asks: if even an ordinary \textit{citizen} is so treated, what trace of that dignity or status can remain in one already condemned by the State?}

% --

To complete this division of life into two parts, my memory itself has effaced nearly all recollection of 25~April. Only faint impressions persist, while every event connected with the catastrophe remains vivid and continuous -- each detail confirmed either by witnesses or by instrument readings.

I do not even recall walking to the plant that evening. I always went on foot, both to and from work -- four kilometres each way. That made about two hundred kilometres a month. Add another hundred from regular jogging, sufficient to keep the body in form. Most of all, walking preserved my composure. When anxious thoughts arose, I merely quickened my pace. That habit later proved of inestimable value. Under such working conditions, walking and running were indispensable.

In our system, steady and orderly labour scarcely existed -- least of all at a construction site. I took part in the assembly, commissioning, and operation of all four units of the Chernobyl plant -- as deputy head of department, head of the Reactor Department, and deputy chief engineer. A ten-hour day was the minimum; Saturdays were invariably working days, and during critical periods we worked Sundays as well. Yet fatigue came not from the long hours, but from irrational organization, unreasonable demands, and schedules impossible of fulfilment. Six months after the startup of Unit 4, everything had settled down and regular work had resumed. However, I still didn't leave work before six o'clock in the evening. It gave me the opportunity to update or supplement technical data at work, something I believe an engineer cannot do without.

% --

Much has been written concerning the alleged \textit{``poor quality of construction''} at Chernobyl, said to have arisen from the pressure for premature completion. I do not share that view. I arrived at the power plant in September~1973. Upon the cafeteria building was painted a slogan promising the start-up of the first unit in~1975. When that year passed, the figure \textit{``5''} was painted over with a \textit{``6.''} In reality, the first unit was commissioned on 26~September~1977. The second followed in December~1978 -- its delay doubtless a consequence of the first's lateness. The next two proceeded in similar fashion. Thus any talk of \textit{``early completion''} is without foundation.

What was forbidden -- at least until 31~December -- was to utter aloud that the unit would not begin operation within the year. Then a government representative would appear, and new, equally impossible schedules would be devised. Once signed, the same sequence ensued: nervous strain, harsh progress meetings, night-time summons, and inevitable postponements. When it became plain to all that the schedule could not be met, the pressure relaxed and work proceeded normally -- until the next inspection.

I never comprehended the logic of these \textit{``motivational injections.''} In my opinion, they produced only harm. When deadlines are impossible, conscientious men still make the attempt at first; later all acknowledge the futility, and the indolent take it as licence for idleness. I observed this many times. Such campaigns were no more useful than the perennial slogan: \textit{``We shall give all our strength to commission Unit No.~\_\_\_ by \_\_\_ date.''} A rational man may well smile: \textit{All my strength? And thereafter -- what remains?}

% --

I believe that men such as V.~T.~Kizima, and the fitters N.~K.~Antoshchuk, A.~I.~Zayats, and V.~P.~Tokarenko did not take those \textit{``motivational campaigns''} seriously, though they concealed the fact with tact. They could tell anyone, at any time, precisely how and when the work would in truth be completed. In fact, I suppose that the installation workers are not susceptible to AIDS. They had enough to develop immunity against any outer taint -- of biological or psychological origin -- for otherwise no one could long endure those conditions of labour.

By Soviet standards, assembly at Chernobyl was good,\footnote{See Appendices~A through~D for internal KGB reports on this matter. Dyatlov's favourable judgment of the quality of assembly may arise from his lack of awareness of certain structural defects; yet it is also possible that, being aware, he regarded them as insignificant contributors to the catastrophe.} despite the immense number of welded joints in the primary-circuit piping. I recall but one cracked weld in a major pipeline, and that likely owing to rigidity of the structure and insufficient allowance for thermal expansion. Installation work -- and the workers themselves -- had no relation whatever to the accident of 26~April.

I came to Chernobyl from a shipyard, where I had taken part in the delivery of nuclear submarines. Conditions there were also far from easy -- night shifts, continuous twenty-four-hour labour. I remember one incident, almost comic. In the shipyard hostel we were playing cards when mechanic V.~Buyansky entered and addressed the military inspector:

\begin{quote}
-- ``We've tuned the system. You must accept it.''\\
-- ``Impossible, Viktor. I'm ill,'' replied the inspector.\\
-- ``I must have it accepted -- the bonus depends on it. I'll bring you by transport.''
\end{quote}

A few minutes later the inspector reappeared, confessing that his \textit{``illness''} was an aggravated haemorrhoid. Buyansky had proposed a \textit{``comfortable ride''} on the rear seat of a motorcycle. In the end they took the bus. Yet even amid such absurdities, matters there were better organized.

% --

At the plant I could never comprehend why, as an operations engineer, I was required to know how many valves or how many metres of pipe had been installed each day. What I required was a completed pipeline -- with supports, hangers, and every component in place -- so that adjustment work might begin. Such information only distracted from the real tasks which no one else could perform. Let the installation crews account for their own fittings -- that is their concern.

A cult of minutiae -- a cult of \textit{``control over every detail''} -- had been elevated undeservedly high, usurping the place of genuine professional competence. Save for reports to higher authorities, such data were entirely useless. One manager would seize a sheet of paper and devise a new reporting form; another would demand a different one. Even with so-called \textit{``computerization,''} these reports consumed an immense quantity of time.

% --

Then there were the schedules. They were drawn up for every conceivable occasion, without the slightest regard for the available manpower, materials, or equipment -- only for the deadline proclaimed by some visiting official. Naturally, such schedules were never fulfilled. Apart from Unit~2, each unit had no fewer than ten distinct handover schedules.

A supervisor from Glavatomenergo\footnote{Chief Administration of Nuclear Power Units.} named Nevsky would arrive and present yet another schedule for the completion of the pipeline systems. In June a new plan would appear; by August -- according to the nominal start-up date -- the flushing of the Multiple Forced Circulation Circuit (\textit{КМПЦ}) of the primary loop was allegedly in progress. The pipelines, \qty{800}{\milli\metre} in diameter, required skilled welders -- of whom only a few were certified -- and each weld, according to procedure, required seven days. Yet Nevsky, himself once an installer, could not have failed to perceive the impossibility. Then another emissary from the Central Committee, Maryin -- formerly an electrician -- would appear, and soon a new schedule was drafted for \textit{``electrified valve adjustments.''} And so it continued.

Even the authors of these schemes did not believe in them. Construction proceeded at its own pace; builders and fitters had long since adapted to such improvisations. To me, coming from naval engineering, it was incomprehensible. In the shipbuilding industry matters stood otherwise. We had studied the reports of Admiral~Rickover -- the founder of the American nuclear submarine fleet -- who faced similar difficulties. Yet after the first few submarines, all subsequent schedules were realistic and were met; major system tests might shift by a week, but never by months.

At Chernobyl, the gulf between plans and physical reality was immense. Let two examples suffice -- one from \textit{``there,''} and one from \textit{``here.''}

When, in an almost completed submarine, the reactors were contaminated by ion-exchange resins from a shore-based filter, it became necessary to replace the reactor fuel. The director arrived, reviewed the situation, and extended the delivery date by one quarter. There was no talk of \textit{``You caused it, therefore make it up yourselves.''}

Meanwhile, after the Chernobyl explosion, the first head of the Government Commission, B.~E.~Shcherbina, is said to have demanded a schedule for restoring Unit~4 by the autumn of~1986. Pure phantasmagoria.

% --

As for the commissioning schedules at Chernobyl, to those of us at my level the situation was never entirely clear. The construction periods did not differ greatly from international norms. None of the units was completed within the originally prescribed time, yet, it seems, all received the customary \textit{``on-time''} bonuses. How this came about I do not know -- perhaps it was a peculiarity of accounting practice, perhaps the result of some undisclosed \textit{``internal schedule''} known only to the upper management. Either explanation is equally plausible.

The greatest difficulties during construction arose from the so-called Technical Design Changes\footnote{Технические решения -- modifications to project conditions introduced for various reasons.}. After the Leningrad Nuclear Power Plant, equipped with RBMK units, had entered the construction stage, the government resolved to erect analogous plants -- Kursk and Chernobyl -- without awaiting the results of start-up and pilot operation.

The RBMK reactor is not so much a \textit{``piece of equipment''} as an entire structure. Its principal metallic structures cannot be transported by ordinary means; they are assembled and welded on site from factory-fabricated elements. Even so, for large assemblies, the production of ten or more sets already constituted a \textit{``series.''} The smaller components -- for the reactor itself and its auxiliary systems -- were classed as non-standard equipment: supports and hangers for reactor piping, transport mechanisms for moving fuel and radioactive assemblies, fuel-handling devices, and the like.

Such equipment demanded the highest technical precision, particularly that intended for the handling of spent fuel, which is intensely radioactive. Fresh fuel, by contrast, is only weakly active and poses no great danger, though it must be treated with care lest the slightest damage lead later to serious operational complications.

The Leningrad plant, under the Ministry of Medium Machine Building (\textit{MinSredMash}; a government body supervising the Soviet nuclear industry), was both designed and supplied by organizations belonging to that ministry, and its factories were of the most modern kind. The Kursk and Chernobyl plants, however, fell under the Ministry of Energy and Electrification. A government decree stipulated that the non-standard equipment for the first four units of these plants should be manufactured by the same factories as for Leningrad. Yet for MinSredMash, even in those days when ministries still half-heeded the government, a decree carried little authority. Their answer was: \textit{``You have your own factories -- make it yourselves; we shall send the drawings.''}

I visited several of MinEnergo's\footnote{Ministry of Energy, responsible for the energy sector of the Soviet Union.} auxiliary-equipment plants. Their tooling was scarcely better than that of a mediocre workshop. To entrust them with the manufacture of Reactor Department equipment was as though one should bid a carpenter to perform the work of a cabinetmaker. Thus we laboured, constructing each unit piece by piece: some components were produced, others never materialized.

Characteristic of that period of stagnation, MinEnergo, throughout all those years, did not succeed in modernizing even a single enterprise to manufacture such comparatively simple apparatus.

\section{Construction Realities and Staff Formation}

During the construction of the first unit, difficulties arose concerning the manufacture of supports and hangers for the reactor piping systems. The factory of the Ministry of Medium Machine Building declined to produce them for us. I do not know who conceived the notion of having them made at the Ministry of Energy's plants by means of direct contracts. I would not say that I opposed the idea, yet I placed no faith in building reactor systems \textit{``by local means,''} and therefore displayed none of my usual persistence.

I was, in truth, relieved when the chief engineer, V.~P.~Akinfiev, informed me that I was being freed from that responsibility and need not concern myself with it further. Some time later Akinfiev said to me, with mild reproach: \textit{``You stopped dealing with it -- and things began to move.''} They had managed to produce a few simple components. I replied: \textit{``Well, let us hope our calf succeeds in eating the wolf.''}\footnote{An ironic Russian proverb reversing the natural order of things. Normally, a wolf eats a calf. Dyatlov expresses scepticism or resigned hope about an unequal struggle -- when a weak or defenceless side is set against a powerful adversary. In context it means: \textit{The odds are hopeless, but perhaps a miracle will occur.}} As expected, the undertaking led nowhere until the work was placed on a proper footing.

At that time, the plant director, V.~P.~Bryukhanov, knew little of reactors and was unaccustomed to their operational discipline. For a long while he continued to believe that a reactor was far simpler than a turbine. There were even proposals -- which he supported -- to rename the reactor operator as \textit{reactor control engineer} (and to reduce his salary accordingly), while retaining the title \textit{senior turbine control engineer} for the turbine operator.

I remarked, not without irony: of course -- the turbine makes three thousand revolutions per minute, whereas the reactor makes but one revolution per day, together with the Earth. In truth, neither turbine nor reactor operators are \textit{senior engineers}; they have no subordinates. Yet since it was impossible to remunerate genuinely complex work adequately, inflated titles had to be invented.

Gradually Bryukhanov -- an educated engineer -- came to perceive that a reactor is no mere mass of metal, no inert mechanism. He was, I think, most deeply impressed by the accident of 1982 in Unit~1, when a process channel ruptured and a fuel assembly was ejected into the graphite stack.

Chief engineer Vyacheslav Pavlovich Akinfiev, who before his transfer to Chernobyl had worked on similar reactors, knew the RBMK better than anyone at the plant. Why he sanctioned such an improvised approach is difficult to explain. Perhaps the habits of his former ministry influenced him: at the Ministry of Medium Machine Building such methods were feasible -- if something could not be produced in-house, it could readily be arranged elsewhere. That ministry possessed both the means and the authority.

Once, the factory supplying equipment for the Leningrad power plant reported a shortage of X-ray film for weld inspection. The ministry dispatched an entire \textit{Ikarus}\footnote{Bus manufacturer based in Budapest.} bus filled with film -- and even left the bus at the factory's disposal.

The Ministry of Energy, by contrast -- in 1981, when Chernobyl was already generating billions of kilowatt-hours -- did not possess even a serviceable minibus. I recall meeting Yugoslav specialists at Kyiv airport in a wretched vehicle that rattled, leaked, and whistled with both northern and southern winds. A disgrace.

\section{Technical Difficulties and Personnel}

There were likewise many difficulties with piping arising from missing components. One thought ten times before drafting or signing a Technical Decision\footnote{Original: Техническое решение.} that departed from the design documents. Such departures almost invariably entailed some degradation of quality and left an unpleasant aftertaste. Thus it went, from unit to unit, for a decade.

I felt relief only after Unit~4 was completed and my sole concern became operation. True, the construction of Units~5 and~6 still made itself felt -- some operating personnel were transferred there -- but that was natural and provoked no resentment. With four units already in service, the staffing of one new unit presented no grave difficulty. There was a steady inflow of new personnel and sufficient time for their training.

From long observation I concluded: after one year an operator becomes fully competent in his post; after two years he may be trusted without reserve. How long a man should remain in one position thereafter depends upon the individual. Some never seek change and discharge their duties conscientiously. The principal danger is not to overlook indifference -- for loss of interest is perilous.

Most employees sought advancement. Such ambition was natural and worthy of encouragement. Commonly these workers proved excellent: diligent, methodical, constantly broadening their technical knowledge.

Danger lay in the ambitious man without sound technical foundation -- one who felt \textit{``squeezed out''} or \textit{``undervalued,''} who bore resentment toward all, and who acted rashly. Such persons were unfit for operational service -- indeed, for any work of responsibility.

By 1986 the third and fourth units possessed a strong core of operating personnel, though about twenty percent had held their posts less than a year, owing to transfers to the fifth unit. Even so, such a staff was fully adequate for the operation of the plant.

\section{Work and Responsibilities}

It may appear strange, yet the post of Deputy Chief Engineer for Operations -- and earlier, that of Head of the Reactor Department -- demanded no small measure of physical exertion.

The equipment of a nuclear power plant is housed within vast halls; one cannot comprehend it at a glance -- one must walk. I recall resolving, after the start-up of the first unit, to inspect all equipment daily. It proved physically impossible to accomplish even within half a day; nor could more time be spared, for there were personnel to direct and documents to attend. Reluctantly, I established an inspection schedule.

Even so, there was much walking: checking for leaks, investigating vibration in a pump, overseeing equipment under repair, conversing with operators at their posts.

I took satisfaction in the work. I was always present at start-ups and shutdowns from beginning to end -- twenty-four hours or more. The gratification came only when the task was complete. My health then allowed me to labour thirty hours without interruption, if necessity required. I was not an operator; operators were forbidden such excess.

Without false modesty, I may say that I knew my trade. I knew the reactor and its service systems thoroughly, having personally examined every part more than once. Others knew less, but they knew enough.

The foundation of that knowledge lay in the general technical disciplines acquired at the Moscow Engineering Physics Institute -- mathematics, physics, mechanics, thermodynamics, and electrical engineering. Upon that base, one could comprehend nearly all systems and processes of the plant.

I knew virtually all operating procedures and schematics -- not the inner wiring of every instrument (a lifetime would scarcely suffice for that), but the plant systems as a whole. The process was this: contractor organizations, and at times our own departmental staff, prepared draft procedures and submitted them to me. After review and annotation, I approved them for typing, or, if the comments were numerous, returned them for revision. Then followed the final review. From that alone, one could not but learn greatly.

At this point, a question may arise: how could I possess such confidence, when the reactor itself violated the elementary principles of nuclear safety? That discussion will follow. For now, let it suffice that I went to my duties calmly, assured of the reliability of the equipment.

Two more aspects deserve attention.

\subsection{Myth of Unreliable Operators}

G.~Medvedev writes in \textit{``Chernobyl Notebook''} that \textit{``we, the experienced operators''} (let us, for the moment, accept his self-description) always felt, as he puts it, the sharp edge separating us from catastrophe. I cannot conceive how one could go to work each day in fear. That would be a form of technical masochism. No normal person can perform his duties while sustained for hours by constant dread; no nervous system could endure such strain. It would require, not nerves, but ropes.

\subsection{On Reactor Reliability}

Another accusation asserts that, because the personnel deemed the reactor reliable, they treated it carelessly -- \textit{``like a cupboard.''}

Indeed, we regarded the reactor as reliable, and we trusted the Emergency Protection system. Who would work otherwise?

That the RBMK was a complex and demanding apparatus, requiring utmost concentration and vigilance, was perfectly evident to every young Reactor Control Engineer,\footnote{СИУР; note that the ``С'' denotes \textit{``senior,''} a purely formal element of the title and not an indication of supervisory authority.} not to mention the remaining technical staff. Operators were well aware, from training, of numerous potential reactor conditions that might lead to serious accidents -- though none comparable to what befell us on the 26th of April. To us, that occurrence was no accident, but a catastrophe.

No operator ever imagined treating the reactor \textit{``freely.''} For the ordinary public, the word \textit{accident} denotes what happened that night. For the operator, however, an accident signifies merely a reactor trip -- a shutdown without any damage to the reactor or even to auxiliary systems. Should an operator cause damage to equipment (even the most minor, internal kind, of which the public would neither hear nor care), his professional career is at an end.

Newspapers report, for example, \textit{``how many reactor shutdowns or power reductions occurred.''} Such figures are meaningless to the layman. If the reactor is stopped by the Emergency Protection system because a parameter has exceeded its limit or a mechanism has failed, this is a normal occurrence, preventing damage to equipment. For the plant, of course, it entails economic loss -- less power produced, possible penalties for supply deficits. The public, however, should be informed only of genuine accidents -- those involving destruction, release of radioactivity beyond containment, or contamination of areas not designed to be radioactive. There exist, moreover, intermediate cases: no shutdown, no power reduction, yet contamination arises within or even beyond the station grounds.

Only such events merit public attention. Frequent trips will inevitably result in staff replacement, managerial change, or, under private enterprise, outright bankruptcy. Order will be restored, one way or another.

\section{Premonition}

People often ask whether I had any premonition of the disaster. No -- none whatsoever. To speak frankly, I place little faith in premonitions. The printed tales that seem to confirm them never persuade me; one must know the individuals concerned. If a man acts always by habit, never hesitates, and one day, for no discernible reason, breaks his routine -- and that act saves him -- then perhaps one may speak of intuition. But if a person is forever uncertain, plagued by doubts, altering his decisions repeatedly, and happens by chance to miss a trip or a flight -- what of it?

Two acquaintances from Komsomolsk-na-Amure, Ilya~Leva and Anatoly~Volodya, once went on leave. During a layover in Khabarovsk before their flight to Moscow, they went to a restaurant. Volodya got drunk; they missed the flight -- and that very aircraft crashed near Irkutsk. Was that a premonition? No; for Volodya, such behaviour was perfectly ordinary. Had Leva been the one who got drunk, then, perhaps, it would have been noteworthy.

\section{The Last Evening Before the Explosion}

That night I acted precisely as usual. I came to my office, telephoned the unit to verify operating conditions, smoked a cigarette, changed clothes, and, as always, first went to the control room of Unit~3 to see how matters stood there. Only afterward did I proceed to Unit~4.

According to prior agreement with the power grid, Unit~4 was scheduled for shutdown on 25~April~1986 for preventive maintenance. By midday the reactor power had been reduced to fifty percent, and one of the two turbogenerators was stopped. Then the grid dispatcher forbade any further reduction until after the evening peak-load period and permitted the shutdown at 11:00 p.m. Nothing unusual occurred at that time. Routine pre-shutdown tests and inspections were being conducted under standard programs.

Perhaps only one circumstance of note took place that day. After the power reduction, xenon poisoning began: fission products accumulated within the core, thereby decreasing the operational reactivity margin. There are other influences upon reactivity, but xenon poisoning is the principal one.

The minimum margin recorded by the unit's computer was 13.2 control rods, below the regulatory limit of fifteen rods prescribed by the Operating Regulations. However, owing to a computational error, the program failed to include the reactivity compensated by twelve automatic regulators positioned at intermediate depths within the core. These effectively accounted for the missing \num{1.8} rods. Later, as the reactor began to de-poison, the margin increased; by 11:00 p.m. on 25~April it had reached twenty-six rods, with the reactor operating at fifty percent power and the turbogenerator No.~8 (only one connected) functioning normally.\footnote{This paragraph is important from the perspective of the disaster: as of 2025, a fault is ascribed to the personnel, accusing them of not shutting down the reactor due to xenon poisoning, and instead continuing the test. For reasons that should be clear, the book does not address this. However, it is clarified that the reactor began de-poisoning long before the test, increasing the operational reactivity margin.}

All parameters lay within the prescribed limits.

\section{The Sequence of Events Begins}

To present a coherent picture of events at the unit, I will recount the occurrences and conversations without, for the present, analysing the physical processes or the motives of the personnel. Everything shall be rendered faithfully, without omission or addition, in the order carefully verified against instrument recordings and operational logs by the team of the Scientific-Technical State Committee for Industrial and Nuclear Power Oversight, in their report of 1~January~1991 entitled \textit{``On the Causes and Circumstances of the Accident at Unit~4 of the Chernobyl Nuclear Power Plant on 26~April~1986.''} These data do not contradict earlier technical reports; they are simply more detailed. A complete chronological list of events is provided in an appendix; here I present only the principal sequence.

At 11:10 p.m. on 25~April, after permission from the power-grid dispatcher, a further reduction of reactor power and a corresponding decrease in the load on the operating turbine were initiated. At midnight, now 26~April, when the shift was handed over, the reactor conditions were as follows:

\begin{itemize}
\item Reactor power: \qty{750}{\mega\watt} (thermal);
\item Reactivity margin: \num{24} control rods;
\item All parameters within Operating Regulations.
\end{itemize}

Before the handover I spoke with the outgoing shift supervisor, Yury~Tregub, and the incoming one, Alexander~Akimov. Only two tasks remained: measurement of turbine vibration in coasting mode (idle generator) and the execution of the experiment under the Turbogenerator Run-Down Test Program. No questions arose. Vibration measurement accompanies every maintenance shutdown -- nothing exceptional. Akimov had examined the preparations for the experiment the preceding day.

Thereafter I left the main control room of Unit~4 to inspect several areas before shutdown, as was my custom. Defects reveal themselves most readily during regime transitions, and during power reduction one may examine the high-radiation rooms more thoroughly. I was never afraid to work in radiation fields, yet I did not seek unnecessary dose. One must not reach the annual limit prematurely, lest one be barred from zone work.

I returned to the control room at 00:35 a.m. -- time later established from reactor power logs. From the doorway I observed several men leaning over the reactor control panel: the operator Leonid~Toptunov, the shift supervisor Alexander~Akimov, and the trainees V.~Proskuryakov and A.~Kudryavtsev -- perhaps others. I glanced at the instruments: reactor power \qtyrange[range-open-phrase={\text{between} },range-phrase={ \text{and} }]{50}{70}{\mega\watt}. Akimov explained that during the transfer from LAR (local automatic regulation) to AR (automatic regulation) by the lateral ionisation chambers, the power had fallen to \qty{30}{\mega\watt} and was now being restored. This neither alarmed nor surprised me; such incidents were not uncommon. I authorised the power increase and withdrew from the panel.

\subsection{Preparation for the Turbogenerator Run-Down Test}

Together with G.~P.~Metlenko, I reviewed the preparations under the \textit{``Turbogenerator Run-Down Test Program''} and marked the completion points in his copy of the document. A.~Akimov approached and proposed that we refrain from raising the reactor power to \qty{700}{\mega\watt}, as prescribed in the Program, and instead limit it to \qty{200}{\mega\watt}. I concurred.

The Deputy Head of the Turbine Department, R.~Davletbayev, reported that the pressure in the first circuit was falling and that he might have to shut down the turbine. I told him that reactor power was already increasing and that the pressure would soon stabilise. Davletbayev also conveyed a request from A.~F.~Kabanov, a representative of the Kharkov Turbine Plant, to measure turbine vibration during free run-down -- that is, as the turbine coasted without generator load. Such a procedure, however, would have delayed the experiment, and I declined, saying:

\begin{personal}[Dyatlov to Davletbayev:]
\par \textit{``During the experiment we shall be shutting down the reactor -- try to catch the revolutions then (around \qty{2000}{\rpm}); there will still be sufficient steam.''}
\end{personal}

\subsection{Final Preparations}

At 00:43~a.m., the reactor emergency shutdown signal -- triggered by the stop of both turbogenerators -- was blocked. A little earlier, the Emergency Protection setpoint for the turbine-trip signal on pressure drop in the drum separators had been reduced from \num{55}~atmospheres to \num{50}.

At 01:03~a.m. and 01:07~a.m., the seventh and eighth Main Circulation Pumps were started in accordance with the Program. A.~Akimov reported readiness for the final experiment.

I gathered all participants to review their duties: who was to monitor which parameters, and what actions were to be taken in case of irregularities -- except for the reactor operator, who was not to leave his post under any circumstances during this regime. Thereafter, everyone dispersed to his assigned position.

In addition to the on-duty operators, the control room at that moment contained:

\begin{itemize}
\item Electrical Department workers (Suryadny, Lysyuk, Orlenko);
\item Representatives of the commissioning enterprise (Palamarchuk);
\item The Deputy Head of the Turbine Department (Davletbayev);
\item From the previous shift, Yury~Tregub and Sergei~Razin, who remained to observe;
\item The Chief of the Reactor Department's shift (V.~Perevozchenko);
\item Trainees Proskuryakov and Kudryavtsev.
\end{itemize}

\subsection{Initial State of the Block}

Operating mode:

\begin{itemize}
\item Reactor power: \qty{200}{\mega\watt} (thermal);
\item From Turbogenerator~No.~8, electrical power was supplied to the feedwater pumps and four of the eight Main Circulation Pumps;
\item All other electrical equipment operated from standby sources;
\item All parameters within normal limits.
\end{itemize}

The control system registered no warning or alarm signals from either the reactor or the auxiliary systems.

\subsection{Test Command Sequence}

To record the electrical parameters, a loop-type oscilloscope was installed outside the control room, to be activated by the telephone command \textit{``Oscillograph -- start.''} Upon this command, three operations were to be performed simultaneously:

\begin{enumerate}
\item Steam supply to the turbine was to be cut off;
\item The MPA (\textit{МПА}) button -- an auxiliary switch for the generator field rundown mode -- was to be pressed;
\item The AZ-5 (SCRAM) button was to be pressed to shut down the reactor.
\end{enumerate}

Akimov gave the command to Toptunov.

\subsection{01:23:04 a.m. -- The Test Begins}

At 01:23:04~a.m., the monitoring system recorded the closure of the main steam valves feeding the turbine. The Turbogenerator Run-Down Experiment had commenced.

As steam flow to the turbine ceased, the generator revolutions began to decrease, reducing both the frequency of the electrical current and the speed of the circulation pumps powered by that generator. Flow through the remaining four pumps increased slightly, yet the total coolant flow fell by \numrange{10}{15}~percent within forty seconds. This reduction introduced positive reactivity into the reactor. The automatic regulator held reactor power steady, compensating for that reactivity.

Until 01:23:40~a.m., no deviation in parameters was observed. The run-down proceeded calmly. The control room was quiet; no one spoke.

Hearing a brief exchange, I turned and saw the reactor operator, Leonid~Toptunov, speaking with Alexander~Akimov. I was about ten metres away and did not hear Toptunov's words. Akimov gave the command for shutdown and gestured -- press the button. He turned back toward the safety panel, which he was observing.

There was nothing alarming in their behaviour: calm conversation, calm command. This is confirmed by G.~P.~Metlenko and by A.~Kukhar, a meister from the Electrical Department who had just entered the unit control room.

Why Akimov delayed issuing the shutdown command can no longer be determined. In the first days after the accident we were still in communication -- before being scattered among different hospital wards -- and one might have asked. But then, and even more so now, I attached no importance to it: the explosion would simply have occurred thirty-six seconds earlier, that is all.

At 01:23:40~a.m., the system registered actuation of the AZ (Emergency Protection) button to shut the reactor down after the test. This button is employed both for emergency shutdowns and for normal operations. One hundred eighty-seven Control and Protection System rods began to enter the core and, by every canon, should have terminated the chain reaction.

But at 01:23:43~a.m., alarm signals appeared indicating power overshoot and a shortened period -- that is, a rapid power increase. Upon such signals, the AZ rods should also have driven in, but they were already descending due to the earlier AZ-5 command. Other alarms followed: rising power, rising pressure in the first circuit\ldots

At 01:23:47~a.m. -- an explosion that shook the entire building, and, one to two seconds later by my own sense, an even stronger explosion. The AZ rods halted, having traversed less than half their path. That was the end.

\textbf{In that routine, businesslike setting, the RBMK-1000 reactor of Unit~4 was destroyed by the Emergency Protection button.} Further on I shall endeavour to show that no special conditions were required for that reactor to explode. If I fail, it will be only because I cannot explain it with sufficient clarity; there are no other reasons. By now, everything that happened is well understood.

\section{Aftermath}

Having presented the events of April~26,~1986 as they were perceived by eyewitnesses, explanations are needed: what occurred within the reactor and its systems; why the personnel acted as they did; what rules they allegedly violated, and for what purpose. According to the officially proclaimed version, the personnel were at fault. The reactor -- if not elegant -- was supposedly \textit{``good.''} Only an incredible concatenation of procedural and operational violations, they said, made it explode. A collective of some twenty authors, adorned with every conceivable academic title, wrote in the journal \textit{Atomnaya~Energiya}\footnote{\textit{Atomnaya~Energiya} -- a Russian peer-reviewed journal devoted to the peaceful uses of atomic energy.} that the operating staff had committed unforeseeable violations. Our scientists (scientists in what?) said much against the staff, yet no one allowed the staff a word in reply. Even five years later, no central newspaper or journal printed a single line of what I wrote. And I wrote only in response to new fabrications and slanders, indicating precisely where my statements could be verified -- knowing well they would not believe a convict. A doctor they would believe; a professor all the more. Only the Kyiv paper \textit{Komsomolskoye~Znamya} printed something -- thanks to them. Our vaunted \textit{glasnost}\footnote{A concept in Russian and Soviet politics generally relating to government openness and transparency.} proved very one-sided indeed.

The shift that took over Unit~4 on the night of April~26 had little to do: disconnect the generator from the grid, measure turbine vibration at idle, and conduct the Turbogenerator Run-Down Program. When I left the control room, apparently owing to some miscoordination between shift supervisor B.~Rogozhkin and Akimov, instead of merely unloading the generator and leaving the reactor at \qty{420}{\mega\watt}, they began to reduce power. The reactor was then under LAR (Local Automatic Regulation) using in-core detectors. That regulator simplified the operator's task at relatively high power, but performed poorly at low power. Hence they decided to switch to Automatic Regulation using four out-of-core ionization chambers. There are two equivalent AR systems plus a low-power regulator. During the transfer from LAR to AR -- the AR turned out to be defective -- the power fell to \qty{30}{\mega\watt}.

Two violations are attributed to the staff here: the restoration of power after the dip, and its restoration to \qty{200}{\mega\watt}.

A power decrease for one reason or another is not uncommon; probably no reactor operator has escaped such an occurrence. Could power be raised afterward, and upon what should the operators rely? Upon instrument readings and the Operating Regulations.

% --; reviewed until here, page 32 first sentence in the original

According to the Regulations, any manual or automatic reduction of power to any level not below the minimally controlled level is considered a partial power reduction. The minimally controlled level is that at which the low-power regulator can be placed in automatic mode, i.e., \qtyrange{8}{100}{\mega\watt}. Without discussing technicalities, I refer to the reactor operator's log entry, which records that he lowered the power setpoint, balanced the regulator, and placed it in automatic mode. There is no reason to doubt that entry; at the time he made it, he could not have known \textit{``how to lie correctly,''} and there is no cause to suspect deceit in any case.

One further point. The Regulations state that if the Operational Reactivity Margin falls below fifteen rods, the reactor under manual control must be shut down. At \qty{30}{\mega\watt}, the margin cannot be measured -- the measuring instrument is inapplicable at that level. Only an approximate estimate can be made, based on the then-known effects of xenon poisoning and the power coefficient of reactivity. By that estimate, the margin at the time of the power dip exceeded fifteen rods. Therefore, the staff violated nothing. More on this later.

Now, concerning the power level. There is not the faintest hint, in any operational, design, or directive document pertaining to the RBMK, of any prohibition on operating at a particular power. Reactors are not defined in that way. The Regulations explicitly state that time spent at minimally controlled power is not limited. The same Regulations recommend, when disconnecting a unit from the power grid, reducing reactor power to a level sufficient to supply plant auxiliaries -- approximately \qty{200}{\mega\watt} -- the very level for which we are now reproached.

Hence, there was no violation when the staff began reducing power -- whoever ordered it, and for whatever reason. I concurred with Akimov's proposal to restore power to \qty{200}{\mega\watt} after the dip for a very simple reason: to raise power to \qty{700}{\mega\watt} under the Regulations would have required at least thirty minutes, and we had only about half an hour of work ahead; such power was unnecessary either for vibration measurements or for the run-down test -- for the latter, the reactor was to be shut down. On submarines, we continually had to calculate the start-up positions of reactivity-control elements if some time had elapsed since an AZ trip, taking xenon poisoning and other reactivity effects into account. In the RBMK, such precision is impossible, but a rough estimate suffices. By my estimate, until about 01:30~a.m., the reactivity margin could not have fallen below fifteen rods. I remain confident of that.

I did not anticipate a trap from the plant's Nuclear Safety Department. Under the standards, that department periodically measured reactor characteristics, including the steam (void) reactivity effect ($\beta_v$, or as they termed it $\alpha_v$) and the fast power coefficient of reactivity ($\alpha_N$). The latest data, issued to the operating staff for guidance, were: $\alpha_v = \qty{+1.29}{\beta}$ and $\alpha_N = \qty{-1.7e-4}{\beta\per\mega\watt}$.

After the accident, measurements on other units revealed that the steam effect was not \qty{+1.29}{\beta}, but approximately \qty{+5}{\beta}. A vast discrepancy, and consequently a large difference in its influence upon the reactivity margin when the seventh and eighth main circulation pumps were started and when feedwater flow increased -- both of which tend to reduce that margin.

The power coefficient had been determined by the Nuclear Safety Department at near-nominal power. Perhaps at that regime it truly was as reported. Yet, as established after the accident, at low power (from what level exactly the Scientific Supervisor and Chief Designer have still not specified) the reactor possessed not a negative but a positive power coefficient -- and to this day its precise value remains unknown. Thus, instead of the predicted \textit{increase} in margin by one ``rod,'' a \textit{decrease} of indeterminate magnitude occurred as power was reduced. The forecast was false.

Whether the Scientific Supervisor and Chief Designer knew that, over a sufficiently broad power range, the RBMK exhibited a positive power coefficient, I cannot say. But it was not taken into account in practice. The plant's Nuclear Safety Department operated under their methodological direction and was obliged to measure the reactor's characteristics in the most unfavourable regions. Therefore, either such direction was never given, or what was given was, to put it mildly, substandard. After all, that same department measured a steam effect of \qty{+1.29}{\beta} by their method, when the actual value was about \qty{+5}{\beta}.

The designers clearly understood the detrimental influence of a large positive steam effect upon reactor dynamics. Here is what the Chief Designer of the RBMK, Scientist N.~A.~Dollezhal, wrote in a note to the investigator:

\begin{personal}[Dollezhal's note:]
\par \textit{"At the very beginning of building RBMK reactors, based on the knowledge level of the time (mid-1960s), the core was designed for uranium enriched to \qty{1.8}{\percent} in U-235. After some period of operating the first reactor, it became evident that raising enrichment to \qty{2.0}{\percent} was advisable, which would to some extent reduce the negative impact of the steam coefficient of reactivity. Further study of all operating parameters led to the conclusion that raising enrichment to \qty{2.4}{\percent} was justified. Such fuel with active elements has been manufactured and is undergoing satisfactory representative tests on operating channel reactors at NPPs."}
\par \textit{"With cores built at this enrichment level, the impact of the steam coefficient is localized. Before that -- i.e., at \qty{2.0}{\percent} -- its effect is controlled by installing special absorbers in channels, as strictly prescribed by operating instructions. Deviation from them is inadmissible, since it makes the reactor uncontrollable."}
\end{personal}

No further clarification of the word \textit{``uncontrollable''} is necessary. Unit~4's RBMK-1000 had \qty{2.0}{\percent} enrichment, and there were no additional absorbers in its core. By the Chief Designer's own definition, the reactor was uncontrollable. The operating instructions contained no such warnings -- and could not -- for the Chief Designer's project materials included none. In a NIKIET\footnote{NIKIET (\textit{НИКИЭТ}) JSC -- one of the largest nuclear design and research centres in Russia, specializing in reactor technologies.} report titled \textit{``Nuclear Safety of Second-Stage RBMK Reactors. Neutron-Physics Parameters,''} the steam coefficient is given as $\leq \qty{1}{\beta}$, and the power coefficient as negative. Very well -- those are calculations. Reality intervened. Core-loading patterns for RBMK reactors were based upon NIKIET calculations. They did not disclose this in the design documentation. They knew that, configured thus, the reactor was uncontrollable -- and still they built it.

It was precisely the excessively large positive steam (void) effect that rendered the power coefficient positive. Why is that dangerous?

For a critical reactor, power is held constant. If positive reactivity is introduced by any means -- variation of coolant flow, feedwater flow, or first-circuit pressure -- power begins to increase. In a properly designed reactor, this rise in power introduces negative reactivity (a negative power coefficient), cancelling the added reactivity and stabilizing power at a higher level. This is self-regulation. In the RBMK -- at least at low power -- the power coefficient proved \textit{positive}. Now, as power increases, additional positive reactivity is introduced; the reactor accelerates its power rise, which in turn introduces yet more positive reactivity, creating conditions for runaway. One cannot say that such a reactor cannot operate at all; an automatic regulator or operator can restrain it -- for a time. But once the excess reactivity reaches $\beta$ (the delayed-neutron fraction), the reactor becomes prompt-critical, power surges on prompt neutrons, and nothing can avert destruction. Exotic research reactors are not under discussion here.

The General Safety Provisions require:

\begin{personal}[General Safety Provisions (ОПБ-82); Article 2.2.2:]
\textit{"As a rule, the fast power coefficient of reactivity must not be positive for any operating mode of a nuclear power plant and for any condition of the first-circuit heat-removal system. If the fast power coefficient is positive in any operating modes, the safety of the reactor must be ensured and justified in steady-state, transient, and accident regimes."}
\end{personal}

With an AZ actuation time of \numrange{18}{20}~seconds (a champion in slowness), even with normal control rod design, one cannot speak of \textit{justified safety} in the presence of a positive power coefficient. The same follows from the Nuclear Safety Rules\footnote{Document referred to: \textit{ПБЯ-04-74}.}. We may thus state: the Chief Designer knew how to construct a safe reactor; the normative documents required it; yet the opposite was built.

At 00:43 a.m., shortly after the power dip, Shift Supervisor A.~Akimov blocked the reactor emergency shutdown on the stopping of both turbogenerators. The simple reply is this: the Regulations permit disabling that protection below \qty{100}{\mega\We}, and we had \qty{40}{\mega\We}. Hence, there was no violation. Yet this alleged breach attained international notoriety, so I shall elaborate. This emergency shutdown condition was commonly inhibited in advance \textit{during} shutdowns, since continued reactor operation was required for some time to complete routine checks. The Regulations further prescribe: reduce power under automatic regulation, then press AZ-5 to shut down the reactor. This is standard and, importantly, normal procedure. The purpose of the Turbogenerator-stop condition is to prevent a sharp rise in first-circuit pressure, since halted turbines no longer consume steam. At low power, turbine steam consumption is negligible; once stopped, there is simply nothing from which to \textit{``protect''} the reactor.

How many pages I have written about this -- I have lost count. Even if it had been removed in contravention of procedure, the question remains: had it not been blocked, would there have been no explosion? The answer is no. Disabling the emergency shutdown upon the stopping of both turbogenerators bore no relation to it.

After my arrest, when the charge was laid, I indicated to the investigator the precise clause proving that no violation occurred in blocking that emergency shutdown condition. One might think that would settle the matter. Not so. The point appeared in both the Indictment and the Verdict. The judge asked witness M.~Yelshin, who had been on shift that night, who, in his opinion, had inhibited the emergency shutdown condition. He replied that, under operational discipline, he doubted Akimov could have done it \textit{``himself.''} The implied next step followed naturally: Dyatlov ordered it. The judge brightened and instructed the clerk, \textit{``Make sure to record that,''} as though courts did not record everything. A curious phenomenon -- how people fidget before investigators, prosecutors, judges. Especially before investigators; in court, at least there are others present. Do not believe those who boast that they \textit{``talked back''} boldly to an investigator. Even witnesses, facing no danger, seldom retain composure. Wake the same Yelshin at night under normal circumstances and pose the same question, and his answer would differ: the emergency shutdown condition was blocked in accordance with the Regulations, and in that case the shift supervisor requires no one's permission. Paradoxically, Yelshin's answer -- and similar ones -- gave me a certain satisfaction. Though it seemed to accuse me, it showed that I had trained the staff correctly.

That time, Akimov did not ask me; had he asked, I would have approved. It had to be done. After the power dip at 00:28 a.m., first-circuit pressure began to fall. To prevent a deep pressure sag, it might have been necessary to cut steam to the turbine -- which would have triggered emergency protection.

For the same reason, the setpoint for the turbine-trip on drum-separator pressure drop (first circuit) was changed from \numrange{55}{50}~atmospheres. Operators select this setpoint at their discretion; the keys are located on the operating panel. The protection itself was not removed. For forensic \textit{``experts''} and others, this became \textit{``blocking AZ on first-circuit pressure.''} There is indeed such a protection -- on overpressure -- which shuts down the reactor. It remained fully operational. Thus, the staff's supposed \textit{``criminal''} removal of AZ functions was in full accordance with operating documentation, dictated by technical necessity, and bore no relation to the accident.

Another AZ cited as ``blocked'' was that on low level in the drum-separators, below \qty{-600}{\milli\metre}. Its logic is as follows: at high power (above \qty{60}{\percent} of nominal), a level drop triggers an automatic reduction to \qty{60}{\percent}; at low power, it SCRAMs the reactor. The mode switch is performed manually, by key, by the operators. After power reduction, we did not switch it. Why is the change not automatic? The designer explains: after power reductions (for example, by pressing the AZ-2 button to \qty{50}{\percent}), the drum level often falls below \qty{-600}{\milli\metre}, and an automatic switch would shut down the reactor completely. Therefore, one must wait for parameters to stabilize before changing the mode. At low power, the feedwater regulators perform poorly. On April~26, after the power reduction, the separator level dropped to \qty{-600}{\milli\metre}. Would the reactor have been SCRAMmed had that protection been in the low-power mode? Unknown -- for it is difficult to determine precisely when that protection became inoperative. Even if we knew with certainty that, had the mode been switched, a deviation at 01:00~a.m. would have SCRAMmed the reactor, it proves nothing. One cannot base reactor operation upon \textit{``ifs.''} The accident did not occur because of a level deviation, but for entirely different reasons. Moreover, the protection on \textit{very} low level -- down to \qty{-1100}{\milli\metre} -- remained active.

Therefore, the reactor's emergency protections were fully functional for this regime, except that the drum-level SCRAM threshold stood at \qty{-1100}{\milli\metre} rather than \qty{-600}{\milli\metre}.

\section{Including all eight Main Circulation Pumps}

There were no limits on the maximum \textit{total} coolant flow, only on the flow through an individual process channel, based on fuel-assembly vibration. We were far from that limit; not a single alarm ever indicated excess channel flow. The entire logic of the Operating Regulations and related documents is directed toward ensuring a \textit{minimum} coolant flow, so as to avoid a meltdown. Yes, six pumps are usually in service (three per loop). That is reasonable -- one keeps a reserve when three suffice. There is no technical ground forbidding the operation of four pumps per loop. The reactor operating manual, approved by the scientific organizations, explicitly describes such modes: when exchanging one pump for another, the fourth is started first and only then is the one scheduled for outage stopped; likewise for testing a repaired pump. No improvisation -- everything according to the documentation. The pumps were started in accordance with the \textit{Turbogenerator Run-Down Program}, so that when the generator coasted down and one set of four pumps tripped, the other set of four -- supplied from the standby network -- remained in operation.

A curious claim has circulated through documents for five years: that at high total coolant flow, the inlet temperature to the core approaches the saturation temperature at which water boils, and therefore the core becomes thermohydraulically unstable. False. That is true only for the suction of a Main Circulation Pump, not for the core inlet. If any instability existed, it was intrinsic to the core design, not caused by the operators.

After the power dip, hydraulic resistance decreased, and flow through two or three Main Circulation Pumps exceeded the level permissible for the available feedwater rate. Those pumps could have broken suction -- i.e., ceased to deliver coolant. The middle-panel operator, B.~Stolyarchuk, was occupied with regulating level in the steam drum-separators and had no time to adjust pump flow. Even if three of eight pumps had broken suction, the remaining five would have been sufficient to remove heat at that power. The objective monitoring system recorded normal operation of all pumps, with no signs of suction break or cavitation, right up to the explosion.

Many judges of the operating staff claim that personnel violated the Operating Regulations and operating instructions to meet a production target. I have described exactly what occurred in the main control room on 26~April~1986. In practice there were almost no violations. The emergency protection -- contrary to countless reports -- were configured exactly as prescribed for this regime; so were the parameters. There was no reason we could not complete the task. Of course, we intended to perform it -- it was a production task, not a youth-club resolution. But no one intended to do it \textit{``at any price.''} The staff had no incentive: no reward for completion, no penalty for delay. Nor was I careless. On this same Unit~4, during a previous outage, at the first stage of a planned power-level test (near nominal power), the emergency protection falsely actuated on primary-circuit overpressure. We traced the cause at once and corrected it. Yet the Operating Regulations require that, before resetting emergency protection and raising power again, the operational reactivity margin be at least fifty control rods. It was not, and I ordered a cool-down without completing the program. 

Here, by contrast, everything had been done except one item. We could have carried it out forty days later after the outage. There was no reason to strain beyond limits. Our actions must be judged not by what is known now about the reactor, but by the documentation then in force and the knowledge then available to the operating personnel. As noted, before the \textit{Turbogenerator Run-Down Program} began, reactor parameters were normal; there were no pre-alarm or alarm signals anywhere on the unit.

And yet the bomb was already fully armed. Had we, for any reason, declined to perform the final test and -- as the Operating Regulations recommend -- pressed AZ-5 to shut down the reactor, we would have obtained the same explosion. Likewise, if any emergency protection had actuated on any signal. In retrospect, RBMK reactors had reached such a state more than once; earlier, they had been separated from catastrophe by only a thin margin. It turns out that, like all reactors, an RBMK is hazardous at high reactivity margin -- but unlike others, it becomes even more hazardous at low margin. Reactor textbooks do not describe this. The RBMK's creators, having produced this inversion, remained silent from modesty or from shame. Had they admitted it openly, few would have agreed to operate such a reactor.

The Chief Designer, Scientist N.~A.~Dollezhal, writes in the already cited document:

\begin{personal}[Continued passage from Dollezhal's note:]
\textit{``The constant drive of reactor designers toward maximum economy is tied, in particular, to the need to remove from the core as many elements as possible that uselessly and parasitically absorb neutrons. Among others, one such element is water that remains in the lower part of a channel occupied by a power-control rod. To avoid this influence, a certain lower portion of the control rod -- of a strictly calculated size (A.D.) -- is made of non-absorbing material, thus displacing the corresponding amount of water in that channel, which previously had served, to an appropriate degree, as an absorber.''}
\end{personal}

In brief, the designers' choice was this: to the boron-carbide absorber rod -- a strong neutron poison -- they attached a graphite displacer \qty{4.5}{\metre} long. With the absorber withdrawn, the displacer lies symmetrically along the height of the active zone, leaving water columns of \qty{1.25}{\metre} each at the top and bottom of the channel. One might suppose the displacer should span the full \qty{7}{\metre} of the core for greater gain. Yet a symmetric full-height displacer would require either lengthening the channel -- impossible within the building -- or complicating the rod's construction; and since, in operation, the neutron flux near the very top and bottom is relatively weak, the gain in neutrons would be slight. They settled on \qty{4.5}{\metre}.

Here it becomes evident that the scientific phrase \textit{``of strictly calculated size''} is mere bravado. Whether it was ever truly calculated, I do not know; that it was not \textit{strict} is obvious.

As the rod moves downward from its upper limit, the absorber enters the upper part of the core and introduces negative reactivity, while in the lower part the graphite displacer expels water and thus introduces positive reactivity. With the neutron field shifted downward, the net reactivity during the first three seconds of motion proves positive. This is intolerable. It was observed at the Ignalina plant, and at Chernobyl during Unit~4's physics startup; yet the scientific staff gave it no proper weight. Nor do the \num{4.5}-metre tricks end there.

The RBMK is large -- geometrically and, more importantly, physically. Individual regions may behave almost as autonomous reactors. When the Control and Protection System trips the Emergency Protection and many rods descend simultaneously, the lower part of the core can be driven into a local critical mass.

\begin{personal}[Rules on Nuclear Safety (ПБЯ); Clause 3.3.28]
\textit{``The number, placement, effectiveness, and insertion speed of Emergency Protection actuators must be defined and justified in the reactor design, which must demonstrate that, under any accident conditions and with the single most effective actuator assumed inoperable, the actuators shall: provide a power rundown rate sufficient to prevent fuel-element damage beyond allowable limits; bring the reactor to, and maintain it in, a subcritical state; and prevent the formation of local critical masses.''}
\end{personal}

The Control and Protection System rods not only failed to prevent such a local critical assembly -- they themselves created it at the bottom of the core. Deputy Director of NIKIET, I.~Ya.~Emelyanov, under whose direction the control-rod design was produced, testified with cool pedantry: \textit{``Actuators that affect reactivity must be designed so that, while moving in one direction, the sign of the reactivity they introduce does not change.''} As though rods with the opposite property had not been created under his supervision.

When a control rod stands at an intermediate position, the water in the lower part of its channel has already been displaced; upon motion it immediately introduces negative reactivity. With a large operational reactivity margin, some rods stand at intermediate heights, and the Emergency Protection may, to some extent, accomplish its purpose.

With a small operational reactivity margin, most rods are withdrawn; when the Emergency Protection actuates -- by signal or by pressing the AZ-5 button -- it can introduce positive reactivity, which post-accident calculations put as high as \qty{+1}{\beta} (one effective delayed-neutron fraction). Only after \numrange{5}{6}~seconds -- a lifetime in an accident -- does the protection begin to add negative reactivity.

\begin{personal}[Rules on Nuclear Safety (ПБЯ); Clause 3.3.5]
\textit{``At least one of the provided reactivity-control systems shall be capable of rendering the reactor subcritical and maintaining it subcritical under any normal and accident conditions, assuming failure of the single most effective reactivity-control actuator.''}
\end{personal}

On 26~April~1986 the Emergency Protection, once the button was pressed, alas (I do not misspeak), functioned in full -- and blew up the reactor. Had it failed, the accident might not have occurred. A paradox, yet such was this protection.

\section{Operational Reactivity Margin}

The operational reactivity margin exists to permit power manoeuvre. One cannot construct a reactor with zero power coefficient and perfect neutrality; some margin is ever required as regimes change. For both economy and safety it ought to be minimal. In the original RBMK design no lower bound was set. In 1975, at the Leningrad Nuclear Power Plant Unit~1, during power ascent after an Emergency Protection trip, a process excursion ruptured a channel from local overheating. Power could not be reduced locally -- rods could not be inserted there and withdrawn elsewhere -- because xenon poisoning left no margin. That was the first tolling of the bell. The plant stood upon the brink of catastrophe; one channel unsealed, and under such conditions there might have been several, which -- as we now know -- would have led to a Chernobyl-type disaster. The Kurchatov Institute\footnote{Russia's leading research and development institution in the field of nuclear energy, named after Igor Kurchatov and located at 1~Kurchatov Square, Moscow.} and NIKIET commission inspected the reactor and, in 1976, issued recommendations to improve the RBMK's characteristics -- recommendations not embodied until 1986, after the catastrophe. Ten years, it seems, were insufficient.

Hence the Operating Regulations entry forbidding operation with operational reactivity margin below fifteen control-rod units. All at Chernobyl, as at other RBMK stations, understood its necessity for shaping the power distribution: to suppress \textit{``hot''} regions and reinforce \textit{``cold''} ones. That, at low operational reactivity margin, the Emergency Protection becomes its antithesis -- a power-surging device -- owing to a fundamentally perverse control-rod design, the reactor's creators did not disclose. Did they know? From the documents now extant, they should have known. The Kurchatov Institute and NIKIET maintained RBMK working groups. For their leaders such posts had long degenerated into sinecures; proposals (from the Leningrad accident commissions, from staff V.~P.~Volkov and V.~Ivanov) were treated as affronts to their repose. \textit{``The reactors run; why pry further?''} -- such was the philosophy\footnote{This suspicion is confirmed by the contents of Appendices~A through D, which include translated excerpts from relevant, previously classified KGB files. The author of the manuscript -- at the time -- could not have seen these files.}.

That the designers did not tie operational reactivity margin directly to Emergency Protection operability is plain from a clause in the Operating Rules copied from their standard template:

\begin{personal}[Operating Rules; Article 2.12.6]
\textit{``If the reactor cannot be brought critical within fifteen minutes, although all control rods (except short absorber rods) are withdrawn from the core, then shut down with all rods to their lower limits.''}
\end{personal}

After 26~April the scientists from the Kurchatov Institute and NIKIET instantly grasped the true causes. I am certain of it. I myself, in my first memorandum, listed several possible versions and rejected all but one -- faulty AZ action due to a control-rod end effect -- which was the correct conclusion, though not the full story. With the operating data they possessed, reaching that conclusion required little effort. Yet they obscured matters, and still do. NIKIET in particular. From their pens -- save a hedged note by Scientist N.~A.~Dollezhal -- I have not seen a single candid document. Whether the falsehood originates in their doctoral prelims or is polished afterward, they wield it professionally.

Their first diversion, a main circulation pump \textit{``suction break,''} failed through blatant falsification of fact. Then followed other inventions -- alleged operator violations, spurious calculations, inverted conclusions -- with NIKIET again the most aggressive. We were accused thus: because the operational reactivity margin was small, Emergency Protection \textit{``lost functionality.''} Not, they said, because of a pathological rod design, but because the margin was low. Let us, for a moment, concede their premise and open the \textit{Rules on Nuclear Safety (ПБЯ-04-74)}, in force since 1974: \textit{``The Rules are mandatory for all enterprises and organizations engaged in the design, construction, and operation of nuclear power plants.''}

If the operational reactivity margin parameter can disable Emergency Protection -- what could be worse -- why did the design violate:

\begin{personal}[Rules on Nuclear Safety; Article 3.1.8]
\textit{``The reactor-plant signalling shall give alarms (visual and audible) upon reaching Emergency Protection setpoints and upon emergency process deviations, and pre-alarms when approaching Emergency Protection thresholds, radiation rises above limits, or normal equipment function is impaired.''}
\end{personal}

\begin{personal}[Rules on Nuclear Safety; Article 3.3.21]
\textit{``The Control and Protection System shall include a fast Emergency Protection actuation ensuring shutdown when an emergency arises; Emergency Protection signals and setpoints shall be justified in the design.''}
\end{personal}

% Unhappy about this.
None of this existed -- in defiance of binding rules -- with respect to operational reactivity margins. After the accident we were told this is the \textit{``most important parameter.''} Permit disbelief. Every small leakage sump on site had level alarms and often automatic pumping; yet the \textit{``most important parameter,''} deviation of which blows up the reactor, had nothing -- not even a continuous-reading instrument. That is rhetoric; the law quoted above is substance. The operating staff was not furnished with the control means and automation required by law. A lone line in the Operating Regulations did not suffice. One of the Soviet specialists who informed the international community at the IAEA told me in August 1986 at Hospital No. 6 in Moscow that foreign specialists had commented on this about the excessive workload on the operator. But our specialists began to talk about the difficulty of separating functions between humans and machines. It takes a fair amount of cynicism to even discuss separation under these conditions. Parameter deviations lead to global catastrophe, and the personnel don't even see or hear.

What measuring tools existed will be noted shortly. If a deviation trips the unit without damage, well and good -- dock my bonus, issue a reprimand. But why must staff pay with life, health, and liberty for others' sins, and not the sinners themselves?

Though we did not know that low operational reactivity margin could invert Emergency Protection into a power-surging device, and though, lacking proper instruments, we could not deem the operational reactivity margin \textit{``most important,''} we never intended to violate it. A violation is to knowingly ignore an indication; on 26~April no one saw the operational reactivity margin below fifteen. The unit process computer -- a small machine -- computed it by the \textit{PRIZMA} program in five-minute cycles on demand. One cannot track a parameter that may change by three to four \textit{``rod-units''} in seconds when feedwater flow shifts. In steady regimes this sufficed; in transients it did not. For power-shape control, together with the physical monitoring system -- its \num{130}~radial and twelve seven-zone axial detectors -- this operational reactivity margin method sufficed. For the now-claimed new role as guarantor of Emergency Protection operability, it did not; and the staff had never been told so. A reactor operator already performs over a thousand manipulations and monitors over a thousand parameters.

We may, perchance, have overlooked a transient fall of the Reactivity Margin; the arrangement of the experiment rendered such omission almost inevitable. The Nuclear Safety Department supplied the control room with wrong data, whereby all rational prediction was made vain. ``To measure was impossible; to forecast, equally so.'' Could it be that the Scientific Supervisor and the Chief Designer grasped not the hazard? If they did not, the case is grievous; if they did, it is graver still. In either case, the obligation of knowledge was theirs. For it is the office of designers to deliberate and determine beforehand, that operators, constrained by the urgencies of time, be not driven into extremity. Every avenue leading toward catastrophe must, by design, be foreclosed. The vast forces now confined within modern installations permit no reliance upon traditions born of a lesser age.

The RBMK reactor is rated for \qty{3.2}{\giga\watt} of thermal power; yet, at the moment of calamity, the estimates rise from twentyfold to a hundredfold of that figure. It would have mounted indefinitely until the reactor exploded. Against a \textit{prompt-critical} excursion upon instantaneous neutrons, no safeguard avails; only prevention is conceivable. It is impermissible that the advent of such a state should depend upon operator error. The code of safety itself declares:

\begin{personal}[General Provisions for Safety; Article 2.7.1]
\textit{``Protective systems must fulfil their safety functions even in the presence of failures independent of the initiating event, as defined in §1.2.4. Operator error constitutes an initiating event.''}
\end{personal}

No liberty of discretion remains -- no room for \textit{``if I wish.''} It is an imperative. Had the reactor been endowed with signalization and automatic protection governed by the operational reactivity margin, no hazardous dip could have arisen.

Furthermore, the prescription of a fifteen-rod minimum in the \textit{Operational Rules} was never a guarantee of safety; it derived from considerations wholly extraneous. The post-accident \textit{``calculation''} which pretends to show no positive reactivity overshoot for a certain arrangement of fifteen rods is an appeal to credulity. First, a single value of operational reactivity margin may correspond to multiple rod configurations; the most adverse must be assumed. Second, the assertion of \textit{``no positive overshoot''} tells nothing of the efficacy of the Emergency Protection; what is demanded is a sufficiency in the \textit{rate of negative reactivity insertion from the very first instant.}

After the catastrophe, the remaining RBMK units adopted a thirty-rod minimum operational reactivity margin, even with redesigned control rods and some eighty additional fixed absorbers\footnote{\textit{ДП} -- ``supplementary absorbers.''} that notably diminished the steam-void coefficient. Yet it remains an unnatural circumstance that this reactor is most perilous when its excess reactivity is least.

When the stop-valves closed at 01:23:04~a.m., movement of steam to the turbogenerator ceased and revolutions declined. The test should have concluded near \qty{2000}{\rpm}. I cannot say why, but I recall the figure \qty{2370}{\rpm} -- whether at the moment of explosion or when I glanced again at the dials after A.~Akimov spoke with Senior Reactor Operator L.~Toptunov. All proceeded calmly, without deviation from the expected. In the first moments after pressing AZ-5, none nearby betrayed concern. The centralized monitoring systems -- specifically the DREG\footnote{Generic abbreviation for the diagnostic parameter recording program.} -- registered no parameter variations up to 01:23:40~a.m., the moment the button was pressed, that might have justified a manual Emergency Protection. The commission of the State Committee for Industrial and Nuclear Power Oversight, chaired by V.~O.~Brunsh, gathered and analysed extensive material and, as its report records, could not determine any credible cause for the Emergency Protection actuation. There was none to find: the reactor was being shut down at the end of work.

I have already stated what I witnessed. There is likewise the testimony of G.~P.~Metlenko, seated at the shift-superintendent's desk near Akimov and Toptunov; his words are preserved on tape. In a letter to me he wrote of that command: \textit{``The process was proceeding calmly, and he gave the command in a calm voice, turning half-round and making a slight gesture; thereafter there was the impression of a rolling water-hammer.''}\footnote{A \textit{water hammer} is a shock wave in pipes caused by a sudden change in water flow -- producing a loud, heavy thudding or rumbling noise, often felt as vibration.}

There is also A.~Kukhar's account; he entered the main control room just before Akimov's order to shut down the reactor. It is now broadly conceded that Emergency Protection was actuated without any technical cause, and that Emergency Protection itself initiated the power surge -- save among those who resolved, from the outset, to prove the staff guilty.

Here is an example of how a technically transparent and natural phenomenon may be distorted. As a result of prior actions of the Automatic Regulator, it remained with a negative imbalance (in fact, the reactor power stood below the set value) within the regulator's insensitivity range, as shown by the \textit{``SHK up''} signal recorded by the diagnostic parameter program. When the generator coasted down, coolant flow diminished, which in turn increased reactivity, and the reactor power began to rise, entering the positive imbalance range, whereupon the regulator commenced downward motion. The \textit{``SHK down''} signal was removed at 01:23:30~a.m.

Forensic specialists photographed the power trace with \num{17}$\times$ and \num{30}$\times$ magnification and \textit{``saw''} a rise twenty seconds before the explosion. This entered the indictment. Magnify by seventeen and one observes a faint uptick; we had no microscopes. A visible small rise is not extraordinary; the regulator responds only once the deviation surpasses its deadband.

G.~Medvedev adds colour: \textit{``Senior Reactor Operator Toptunov first sounded the alarm: `We must drop Emergency Protection (AZ), Aleksandr Fyodorovich, we're running up.' Akimov quickly looked at a printout from the computer. The process was developing slowly\ldots''} Nonsense. The computer produces no printouts whereby one may \textit{``watch a process develop,''} and no printout could reach the control room in less than two minutes. There existed no printout for 01:22:30~a.m. until after the catastrophe. Moreover, the \textit{``process''} began only after the button had been pressed; and that button was pressed by Toptunov, not by Akimov. If one writes a \textit{``documentary,''} one ought to cleave to fact.

I did not hear what Toptunov said to Akimov; only Metlenko could have heard, and he has never reported it. B.~Stolyarchuk, engaged at the middle panel, did not listen either. The others were distant; quiet and composed speech would not have reached them. Judging by their demeanour and by the signal log, one may safely infer that as the Automatic Regulator rods neared the bottom of the core, Toptunov inquired what should be done, and Akimov -- in accordance with the pre-test briefing -- ordered a shutdown.

Then all began. After a trifling dip in power at the very onset of rod motion -- unsurprising, since the axial flux was double-humped, with a top maximum and a mid-height depression, as after any power reduction -- the rods' end-effect created a local prompt-supercritical region at the core's base; neutron flux and heat release surged there, while the upper region declined. The resultant reactivity from the rods became positive, and power increased chiefly in the lower part of the core.

At 01:23:40~a.m., when the button was pressed, power could scarcely have exceeded \qty{200}{\mega\watt} (thermal) by much, else the large deviation would have forced the regulator from automatic control. Yet by 01:23:43~a.m., the over-power emergency signal\footnote{\textit{АЗМ}} and the signal on excessive power-rise rate had both actuated.

Such signals must not appear while the Emergency Protection rods are descending -- with properly designed rods. Some now recall earlier cases when Emergency Protection had tripped on various signals (drum level, and others), and these same over-power indications flashed. Neither the operators nor the control-rod designers could explain them. They were judged spurious and ascribed to electronic malfunctions. We now see that they bore witness to faulty Emergency Protection behaviour.

There were genuine, rapid power spikes, but imperfect instruments obscured and deceived. Even on 26~April, when power had soared to many times nominal, the power meter still read below unity owing to inertia. Those earlier spikes were lesser and briefer, yet a fragile boundary stood between operation and catastrophe.

Reactor Operator L.~Toptunov cried out about an emergency power rise. Akimov shouted, \textit{``Shut it down!''} and lunged toward the reactor controls; that second command all present heard. Likely it followed the first explosion, for Akimov told me in hospital that he de-energized the clutches of the control-rod drives, and the computer log records that at 01:23:49~a.m. The second order could alter nothing; the button had already been pressed; the Emergency Protection rods were in motion, until they could move no more.

Experts and investigators strained to prove that the reactor began to fail before AZ-5. Upon what evidence? By the time of the indictment, the unit's parameter plots were already in the file; they yield no grounds for such a claim. Yet once a version is fixed, facts are bent to fit it.

As for the control rod position printout: the unit computer periodically recorded parameters, including rod heights, and computed the operational reactivity margin. After 01:22:30~a.m. it completed no fresh calculation before the catastrophe; one was later executed at the Smolensk Nuclear Power Plant and yielded an operational reactivity margin of six to eight \textit{``rods''} -- a breach of the Operating Regulations. That moment coincided with maximum coolant flow and, to maintain drum level, the operator had raised feedwater -- which, at this low power, collapsed steam in the core. A minute later the margin was near twelve rods, perhaps more.

The accusers say: \textit{``The staff knew, ignored it, and went on.''} Suppose, then, that a 01:22:30~a.m. rod-position printout existed. It must be torn off, logged, carried fifty metres, and handed over. No one ran. Such a printout, lacking the computer's calculated operational reactivity margin, conveyed nothing to the reactor operator; we never used raw position lists to gauge the margin -- we used the computed value.

Emergency Protection -- by name and in fact -- exists to terminate the chain reaction without damage, under both accident and normal conditions, as the safety codes prescribe and the RBMK Operating Regulations affirm. On 26~April~1986 we pressed AZ-5 with normal parameters, in a stable regime, with no alarms or pre-alarms -- and obtained an explosion.

The \textit{Rules on Nuclear Safety}, Article~3.3.26, require Emergency Protection to ensure automatic, rapid, and reliable chain-reaction termination:

\begin{itemize}
\item Upon reaching the emergency power setpoint;
\item Upon reaching the emergency setpoint for power-rise rate;
\item Upon pressing the Emergency Protection (AZ) buttons.
\end{itemize}

We know what occurred. Which of these requirements did the Emergency Protection button fulfil? At 01:23~a.m. -- and for an unknown interval preceding -- the reactor stood in the state of an atomic bomb, and there was neither alarm nor pre-alarm. The staff discerned no danger on any instrument, not from blindness, but for lack of means. What standard, then, did the monitoring system satisfy?

\chapter{After the Explosion}

I heard -- or rather, saw -- Akimov conversing with Toptunov, and then turned again to the instruments. I knew the frequency at which the generator turns on, and converted it into revolutions per minute upon the digital indicator, since that measure was easier to observe. I accomplished nothing more -- there came a blow. Fragments of the suspended fibreboard ceiling fell down. I looked upward -- and at that moment a second, still more violent explosion shook the whole structure. The lights were extinguished, and then returned. A multitude of alarms began flashing.

My first thought was that some mishap had occurred with the deaerators -- those vast vessels, partly filled with hot water and steam, situated in the chamber above the control board. Though a steel-grated floor lay between, the shock was such as might well have torn it asunder, and the scalding torrent have descended upon the main control room beneath. I ordered all to withdraw to the backup panel. Yet all soon grew quiet; there followed neither leak of water nor outburst of steam, nor any flame. I countermanded the order.

I advanced along the instrument panels toward the reactor desk. Even before reaching it, a single glance -- for indeed I looked at nothing else -- revealed all: both primary-circuit pressure and coolant circulation stood at zero. From those two alone I knew this was no ``accident'' in the accustomed sense. Loss of circulation because the main pumps (MCPs) had ceased is not yet fatal if pressure remains; at so low an initial power, natural convection suffices for heat removal. But without pressure, the fuel rods perish in the very first minute. Still, the discipline of many years -- \textit{ensure the cooling of the core} -- asserted itself. I instructed Akimov to start the Emergency Core Cooling System pumps from the automatically engaged diesel generators, and bade Valery Perevozchenko open the valves to the loop. Yet even then I understood that this could not preserve the fuel assemblies; not knowing the nature of the damage, I assumed the rods were already melting from overheating, that fuel would pass into the water systems, and, eating through the pipes, reach the rooms. I regarded the reactor as shut down.

At the reactor desk my eyes were struck with amazement. The control and protection system rods, meant to descend via AZ-5, were fixed midway; they would not descend, though the servo-clutches were de-energized\footnote{The control rods move through servomotors. De-energising the servomotors engages simply cutting their power supply, which under normal circumstances would have resulted in the control rods dropping into the core via gravity.}; the reactivity meter showed positive reactivity. The operators stood confounded -- I suppose I appeared no less so. I immediately dispatched A.~Kudryavtsev and V.~Proskuryakov to the central hall, with the hall operators, to lower the rods by hand. The lads departed at once. I instantly perceived the absurdity of my command -- if the rods would not descend when the clutches were de-energized, neither would they move by handwheels. The reactivity indication, indeed, signified nothing. I darted into the corridor, but they had already vanished. Since that day I have reviewed my actions of the 26th of April, 1986, many, many times -- almost daily, and I still do -- and only that one order was in error. I should like to see the man who could keep a clear head in such a moment. It sufficed that this was my first and last folly. Then came calm -- not stupor, but calm -- and a plain reflection: what may yet be done?

The corridor was filled with dust and smoke. I returned to the control room and ordered the smoke-extraction fans set in motion. I myself passed out by another way to the turbine hall.

There the spectacle was worthy of Dante. Part of the roof had fallen in -- how much I cannot say -- perhaps three or four hundred square metres. The slabs had crashed down and torn the oil and feed-water lines. Rubbish lay on every hand.

From the \qty{+12}{\metre} elevation I looked down through a gap -- at \qty{+5}{\metre} the feed-pumps are placed. Jets of hot water burst from broken pipes in every direction, striking the electrical apparatus. Steam filled the air. Sharp, gunshot-like cracks of short circuits resounded.

In the region of the seventh turbine-generator oil escaping from the ruptured lines had caught fire; operators were hurrying thither with extinguishers and unrolling fire-hoses.

Through openings in the roof I saw tongues of flame.

% --

I returned to the control room and directed Akimov to summon the fire brigade -- \textit{with full reinforcement}, as I phrased it. It appeared afterward that the station's own fire engines had already set forth, one of their number having been outside at the instant of the explosion.

Ambulances were likewise called.

Akimov informed the shift supervisor of the entire plant, B.~Rogozhkin, who in accordance with procedure notified both Moscow and Kyiv. The station personnel are warned automatically by recorded telephone messages, according to the class of accident; in this case a \textit{General Accident} -- the most grave -- was declared. The manner in which the higher authorities were roused has been related elsewhere: by morning they were already arriving from Kyiv and Moscow. Somewhere the plant's recording system failed, and the operator, to make certain, summoned staff by name from the list.

All, save the chief engineer N.~M.~Fomin, reached the plant soon after the event. None of the higher officials I saw upon the unit before my departure. Valery Perevozchenko returned to the control room after an unsuccessful attempt to reach the cylinder chamber, where the valves lie that unite the Emergency Core Cooling System water with the primary loop. The entrance was blocked; passage was impossible.

Sasha Kudryavtsev and Vitya Proskuryakov returned likewise. I habitually called most operators by their short forenames; full name and patronymic -- or surname -- only in moments of tension; and those were rare and brief. Sasha and Viktor too had failed to reach the central hall, hindered by debris. In fact, on the twenty-sixth of April, at least until five in the morning, no one entered the hall.

Yet he who would act must first \textit{know}. The control-room indicators portrayed a dreadful scene, yet yielded no data suggesting any definite course of action.

I quit the control room, intending to reach the reactor hall, where the cover of the reactor opens. I did not arrive. I encountered the gas-circuit operators I.~Simonenko and V.~Semikopov, and the central-hall operators O.~Genrikh and A.~Kurguz. Poor Tolya Kurguz was terribly burned; the skin of his face and hands hung in shreds, and what lay beneath his garments could not be seen. I bade them hasten to the first-aid post, where by now an ambulance should have been present. Igor Simonenko reported that the reactor-building structure was destroyed. I advanced a few metres farther along the \qty{+10}{\metre} corridor, looked from a window -- and saw, or rather did \textit{not} see -- there was no wall. From \qty{+70}{\metre} down to \qty{+12}{\metre} the outer wall had collapsed throughout its height. Of what else had fallen one could discern nothing in the darkness. I proceeded along the corridor, descended the stair, and passed outside. Slowly I walked around the reactor buildings of Units~4 and~3, looking upward. There was much to behold, though, as the saying goes, may my eyes never again behold the like. Despite the night and the meagre illumination, enough was visible. The roof and two walls of the building were gone. Through the voids where walls should have been, one saw jets of water, flashes of electrical discharge upon the apparatus, and several points of fire. The gas-cylinder room was wrecked; the cylinders stood awry. Access to the valves was out of the question -- Perevozchenko had judged rightly. Upon the roofs of Unit~3 and the chemical workshop several small fires were visible. Evidently ignition had come from large fragments of fuel hurled from the core by the explosion; perhaps also from graphite, though at two hundred megawatts thermal the graphite temperature would have risen no higher than three hundred and fifty degrees Celsius and, flying through air, should have cooled. Yet fuel-particles, lodging in the graphite, might have rekindled it after ejection -- a doubtful supposition. I saw no glowing pieces of graphite upon the ground, nor dark ones; though I walked about both units later, I did not look down -- no large fragments met my foot, nor did I stumble.

Near the chamber of the standby control panel for Unit~3 stood the fire engines. I inquired of one driver who was in command; he indicated a man approaching. It was Lieutenant V.~Pravik -- his face I remembered. I told Pravik they must reach the header of the fire-water riser leading to the roof; a hydrant for connection lay near. The engines began to manoeuvre, and I ascended to the control room of Unit~3.

I inquired of the Unit~3 shift supervisor, Y.~Bagdasarov, whether anything impeded their operations. He replied, ``Not yet; we have inspected what we could reach.'' The operators' faces silently demanded, \textit{What?!} -- but not a single question was uttered. Iodine-prophylaxis tablets were distributed. I took one, and, saying nothing, withdrew.

What indeed could I have said? I was not then pondering the causes. Reflection began only later, when we had been taken to the hospital. Before that there was no leisure for analysis; there was work to be done. As I walked the perimeter of the buildings, the picture began to take form -- I perceived that the reactor was lost. In imagination I reconstructed it thus: the process channels had ruptured, pressure had mounted within the reactor cavity, and had torn away the two-thousand-ton upper structure; steam had rushed into the hall, wrecking the building; then the upper structure had ``settled'' back into place. That it had been flung upright, to stand on edge, I did not conceive; yet the difference was of no account.

From that moment the reactor of Unit~4 existed for me only as a source of peril to the remaining units.

Upon returning to Unit~4 control, I ordered Akimov to stop the pumps which had been started after the explosion -- no water could now be supplied to the reactor, for the valve manifold was destroyed, and half an hour after the blast there was no longer any sense in the attempt. Whatever harm the absence of cooling could inflict had already been wrought. We took no further measures in that respect.

P.~Palamarchuk, a man of great stature, brought in and placed upon a chair Vladimir Shashenok, a commissioning engineer. He had been monitoring non-standard instruments at the \qty{+24}{\metre} level and had been scalded by water and steam. Now he sat motionless save for the faint movement of his eyes -- neither cry nor groan escaped him. Plainly the agony had surpassed endurance and numbed the mind. I had noticed a stretcher in the corridor; I told them where to find it and to bear him to the infirmary. Palamarchuk and N.~Gorbachenko carried him out. By morning Shashenok was dead -- the second fatality. Palamarchuk, in seeking him, had taken a great dose, and when he bore him his back was soaked; the water was radioactive, and even five years later the burns upon his back had not wholly healed.

Shift supervisor V.~Perevozchenko reported that the main-circulation-pump operator Valery Khodemchuk and two central-hall operators were missing. I gave a brief command: ``Search!'' The representative of the Kharkiv Turbine Plant, A.~F.~Kabanov, approached with two of his colleagues. I ordered them to leave the unit. Kabanov began to explain that a vibration-measurement laboratory remained in the turbine hall -- a most excellent West-German apparatus which recorded all bearing vibrations simultaneously and whose computer produced admirable plots. He was loath to lose it. That was the only time on the twenty-sixth of April that I raised my voice -- I swore at him: \textit{``To hell with that machine -- leave the unit immediately!''}

I must record this: on that day, the 26th of April 1986, everyone upon the unit obeyed at the first word; there were no excuses. I never had to repeat an order. Whatever they could do, and saw was needed, they did of their own accord. Ignorance of what should be done -- that occurred; who could have known? No one prepares for such a calamity; nor, in my opinion, should. Such a calamity must not exist; it must never be permitted. Yet the will to act was present in all; even those not of the operations staff laboured, though we soon sent them away from the unit. Only the plant shift supervisor, Rogozhkin, I think, failed in his duty. In practice he had naught to do with the operation of Unit~4, though by office it was he who should have commanded in emergency. He alone ``could not'' reach the control room ``because of debris.'' Others went. Indeed, it was no scene for timid eyes. Was it fearful? Yes. But what must a shift supervisor do when confronted with debris? Take men and clear it. Along the \qty{+10}{\metre} corridor, the ``debris'' consisted of aluminium ceiling-plates of scarcely a kilogram apiece. On the other hand, he escaped a high dose. Be well, Boris Vasilievich.

Together with the dosimetrist Samoylenko we measured the situation within the control room. His instrument ranged to \qty{1000}{\micro\roentgen\per\second}, that is, \qty{3.6}{\roentgen\per\hour}. In the left and middle portions it registered between \num{500} and \qty{800}{\micro\roentgen\per\second}; on the right it exceeded the scale. As I expected no strong source there, I assumed no more than \qty{5}{\roentgen\per\hour} on that side -- I had no other choice. We then measured the dose rate at the standby control panel; there the reading was beyond range, and removal thither was therefore precluded.

I instructed Akimov to send the reactor operator L.~Toptunov and the turbine operator I.~Kirshenbaum to the control room of Unit~3. They could be of no further use here, and the conditions had become most unfavourable. Remaining at the desk were Akimov and Stolyarchuk.

We then set ourselves to the chief and most serviceable task which the operating staff of Unit~4 accomplished, at peril to life and health. Owing to the numerous ruptures of piping and breaches in the structure, continuous electrical short circuits were taking place -- the germ of new fires. As I came from Unit~3, I met the deputy head of the Electrical Department, A.~G.~Lelechenko, and took him with me. Now I joined Lelechenko with Akimov and ordered that the mechanisms be de-energized and the electrical circuits opened, that power might be cut from the greatest possible number of cables and networks.

I further commanded that the turbine oil be drained into the emergency tanks and the hydrogen purged from the generators. All this was executed by the personnel of the Electrical and Turbine Departments. Executed -- and some died; some were grievously wounded. The deputies R.~I.~Davletbaev (Turbine Department) and A.~G.~Lelechenko (Electrical Department) rendered excellent assistance to the shift crews. A most extraordinary man, Aleksandr Grigorievich -- no large fellow, yet he somehow found strength, even after the twenty-sixth of April, to return for two or three days more into the same field of radiation. When at last they bore him to a hospital in Kyiv, he did not live long.

It pained me, in later years, to learn of the neglect and profanation of the graves of those dead operators at Mitino Cemetery in Moscow, when compared with the honoured resting places of the fallen firemen. I shall speak of the firemen soon; now I must speak of the operating staff. Had they not done what they did, fresh fires would inevitably have arisen, and with the scant number of people on duty they could have been discovered only when already far advanced. The fires kindled by the explosion -- some extinguished by the staff themselves -- had disabled two entire fire stations, those of the plant and of the city of Pripyat. Who then would have quenched the next, and at what cost? I hold that the staff acted rightly, with uncommon self-devotion, and accomplished all that could be accomplished under those conditions. Nothing more was possible. I have explained \textit{how it was}; you may now judge for yourselves.

I would not speak in condemnation of any man. Yet to profane the graves of the dead -- whosoever they be -- is sheer barbarism. None is obliged to bring flowers; but to cast aside those which others have laid -- no. The falsehood invented by the State, which laid all blame upon the slain, denies them rest even in the earth. Men say that lies cannot run far; yet this one, it seems, both leaps and endures.

Akimov set himself to work. I went again outside -- the roof-fires had not yet been wholly subdued -- and at Unit~3 I ordered the reactor to be scrammed and cooled down with all speed. The plant shift supervisor, Rogozhkin, present at the Unit~3 desk, told me to discuss this with Director V.~P.~Bryukhanov. I replied: \textit{``Scram now, while conditions are still in some degree normal.''} Of course, nothing was in truth \textit{``normal''} at Unit~3 either; only in a technological sense was operation not yet obstructed.

Lately there has been much idle talk concerning the firemen -- that their actions were mistaken, unjustified by the circumstances. A reporter from \textit{``Komsomolskoye Znamya''} inquired of me whether the fire brigades had violated instructions. I do not know; perhaps they did. Yet nothing thereby would have been altered. Had they donned the prescribed dosimetric apparel, it would not have availed them. Their regular equipment was of coarse fabric; the boots shield against beta radiation, but against gamma radiation there exists no protection -- no such garment has ever been devised. Only automatic fire-suppression, requiring no men upon the roofs of the reactor and chemical buildings, could have preserved them. There was none. There existed but a ring-main around the perimeter, with outlets for coupling hoses kept in nearby boxes. Without men, nothing could be done.

Still more incomprehensible was the statement of Director V.~P.~Bryukhanov, in his interview, that there had been no fire at all, and that the firemen had been sent to death in vain -- to thrust aside red-hot graphite. What? Did I dream the flames? It was precisely because of those flames that I commanded Unit~3 to be shut down. True, there was no raging conflagration -- that alone was wanting -- only scattered seats of fire. And what, then, was Lieutenant Pravik to do? Wait until they joined into one vast sheet of flame? Then the spread to the remaining three units would have been inevitable, with consequences utterly incalculable. Or perhaps he should have waited until it extinguished itself? Commonly it ``goes out of itself'' only when all that can burn has burned. And the reporter's questions to me -- he had not invented them -- and Bryukhanov's interview, likewise inspired by a reporter, are but links in the same obscure chain. Are honours for perilous labour objects of envy? Then let them repeat that labour; our world affords occasions in abundance.

As I left the control room of Unit~3, I met V.~Chugunov and A.~Sitnikov in the corridor, already dressed with some consideration of the dosimetric conditions. I wore the usual plant garb and half-boots. Overshoes would have spared my feet the dreadful burns that even now have not healed; yet how could one move in them? Nor did I give the matter thought. I carried a respirator in my pocket; I put it on once, entered a cloud of steam, could not breathe, tore it off, and never put it on again. They told me that Bryukhanov, sitting in the civil-defence shelter, had sent them to inspect Unit~4. I had no leisure for conversation; I said there was nothing to see and returned to Unit~4. There appeared the deputy head of industrial safety, G.~Krasnozhen. He is of small stature; in his haste he had caught up clothing of the wrong size, and his head was wrapped, like a turban, in a waffle-towel so that only the eyes were visible. He said nothing of the dosimetric situation, but his aspect made me laugh. Silently, to myself, I laughed heartily, despite the tragic scene and my own misery. Nausea came in waves; there remained nothing to vomit but my own entrails. There is nothing to describe -- it has been written of many times by those who\ldots never endured it.

% --

At the desk V.~Perevozchenko reported that the central-hall operators had been found; only V.~Khodemchuk was missing. The operators had not left their posts. When Perevozchenko had earlier told me they were missing, he had not given names; had it been Kurguz and Genrikh, I would have known. Three of us went forth, taking S.~Yuvchenko and a dosimetrist. The dosimeter, as before, ranged only to \qty{1000}{\micro\roentgen\per\second}; sometimes it gave a reading, sometimes it pegged at the limit. At the entrance to the main pump hall the floor-slab had fallen in. We sent the dosimetrist back -- useless with that instrument. Sasha Yuvchenko and I remained by the collapse while Perevozchenko crept along a girder toward the operators' room, where -- though unlikely -- Valery Khodemchuk might yet be.

The door to the room was pinned by a crane. Climbing was perilous; water poured down from above. A thought flashed: \textit{do not do it}. Another thought drove it out: \textit{and will you be able to live with yourself, if he lies there and, by then, has not yet received a lethal dose?} Khodemchuk was not there; his body was never recovered. He lies entombed beneath concrete and steel. But Valery Perevozchenko, it seems, took his death there; drenched with water, he perished not from a great whole-body dose, but from radiation burns of the skin.

Then my strength gave way -- a complete apathy. It was due as much to the physical condition as to the absence of any definite immediate task. I saw nothing more that could be done. We had done all that was possible, and done it rightly. Of ventilation I was uncertain -- even now I know not what would have been best. Then I ordered Unit~4's ventilation shut off, and in the turbine hall of Unit~3 turned on all supply air, to prevent the spread of contaminated atmosphere from Unit~4. The air outside was polluted in any case. Let wiser heads judge. In court an ``expert'' -- some civil-defence or public-health man -- accused me of wrong measures concerning ventilation. The same worthy accused me of violating the duties of the plant shift supervisor. How could I have violated them, seeing I was not the shift supervisor? By that logic, one might accuse me likewise of violating Chiang Kai-shek's\footnote{General and former President of the Republic of China between 1928 and 1975.} instructions.

They called me to the telephone. V.~P.~Bryukhanov. I do not remember what was said -- perhaps nothing. He said only, ``Come to the civil-defence command post.'' I took with me three chart-rolls: two showing reactor power, one the primary pressure. I showered -- according to rule, first in cool water, then hot.

There were many people in the bunker -- plant personnel and strangers alike. I saw Volodya Babichev, a unit shift supervisor. Around three o'clock I had told Akimov to summon his relief; he had done so. I asked Babichev, ``Why are you here?'' He answered, ``They will not let me in.'' ``Let us go,'' said I. And Babichev went to relieve Akimov. Unfortunately, Sasha remained at the unit after the relief.

I entered the adjoining room of the bunker. Director Bryukhanov -- never a talkative man -- sat silent. He asked me nothing. I sat down, unrolled the charts, and showed the power spikes and the pressure trace. I said only: ``There is something amiss in the Control and Protection System response.'' That was all. Bryukhanov sat crushed and wordless.

% --

A colonel of some department approached the table and began to question the Director concerning the extent of damage for his report -- how many square metres of roof, and the like. My words -- ``Write it down: Unit~4 is destroyed'' -- the colonel loftily ignored.

A wave of nausea overcame me; I rushed from the bunker and up the stairs, where I.~N.~Tsarenko helped me into an ambulance. And then -- the hospital: half a year.

One further operation at Unit~4 was undertaken on the twenty-sixth of April, so to speak, upon impulse; all that followed thereafter was done according to plan. The chief engineer, N.~M.~Fomin, arrived at the plant later than the rest -- about 4 o'clock or 5 o'clock -- and it would have been well had it been some hours later still. He resolved to organize the injection of water into the reactor. Why, so long after the explosion? I do not know what Bryukhanov discussed with Akimov, or whether they spoke at all; the Director gave me no orders. And, indeed, what orders were there to give? The situation was manifest -- I knew the reactor better than he, and since I was upon the unit, I would do whatever could be done. I did not see Fomin that day, nor speak with him by telephone; they began the injection after my departure -- otherwise I should have said at once that the idea was futile. The operation was useless, even harmful, and costly. That Moscow should inquire whether the reactor was being cooled is natural -- for men of reactors, that is the cornerstone in any accident. But Moscow could not see the facts before our eyes.

That the operation was useless I believe I have made sufficiently clear -- to specialists no further explanation is needed.

That it was harmful became evident after several hours of water injection. Because the piping was ruptured, the water did not reach the reactor (and, in truth, there was no reactor), but spread through the rooms of Unit~4 and into other sections, carrying contamination. They stopped, of course.

Yet that operation cost several men grievous injury, and cost L.~Toptunov, A.~Akimov, and A.~Sitnikov their lives. After inspecting the unit -- where he, naturally, received a large dose, though not a lethal one -- A.~Sitnikov understood at once that the reactor was destroyed, and reported accordingly. He had not been upon the roof and did not look down upon the reactor from above. They had attempted to reach the roof, but the metal door was locked. They could not. Otherwise A.~Kovalenko and V.~Chugunov would have met the same bitter end. I cannot comprehend why Sitnikov, already knowing that the reactor was gone, took part. There he received a wholly needless addition. Others, still unaware of the true state of things, participated in good faith. Tolya was a disciplined man; for him the saying, \textit{``An order from a superior is law for a subordinate,''} was unquestionable.

L.~Toptunov had been sent from the unit together with I.~Kirshenbaum; had he not returned, he would have received but a minimal dose, with no lasting consequence. When, after my second round outside, I came again into the control room, I saw Toptunov. Sharply I asked, ``Why are you here?'' He made no answer, only showed the notebook under his arm. I thought he had come for the logbook. As it proved, he stayed.

Akimov, of course, took a larger dose, for he went into the plant rooms; V.~Babichev arrived about five o'clock. Even so, Akimov's dose would have remained within two hundred rems.

Both remained and took part in pumping water toward the reactor. There they received their fatal exposure. Of anyone's ``guilt'' there was no word -- neither on the 26th nor in the first days thereafter, at least not in my hearing. Men sought only to understand \textit{why} it had happened; all discourse was upon that point. D.~P.~Kovalenko, chief of the Reactor Department, has said that he heard Akimov in the hospital murmur: ``Our chief mistake was pressing the emergency protection too late.'' Sasha was mistaken. That was not the cause. Nor was it the operating staff who committed the fatal errors. It is a grievous pity that the lads died under a false sense of guilt.

% --

At the Pripyat hospital a dosimetrist measured my exposure; I undressed, washed, changed clothing, and went to a ward. Utterly exhausted, I lay down at once to sleep. No such luck. A nurse entered with an IV. I pleaded, \textit{``Let me sleep -- then do what you will.''} In vain. And strangely, after the infusion -- what it contained I never learned -- sleep would not come; rather, I revived somewhat and went out of the ward. The others were in the same state. In the smoking-room there were animated conversations, all upon the same subject, repeated endlessly. \textit{Cause, cause, cause?}

I said: ``We consider even the most foolish hypotheses; none is dismissed outright.'' And the discussion continued thus, until our later separation into individual rooms at the 6th Hospital in Moscow, a few days after our arrival there.

My wife came. She brought cigarettes, a razor, and toiletries. She asked whether vodka was needed. A rumour had already spread that vodka was beneficial at high radiation doses. I refused. In vain -- not that the accursed stuff is of such great use, but because, as it turned out, I ended by refusing it for four and a half years. A small loss, to be sure -- if voluntary. We did drink on the 26th -- I do not remember to whom they brought it. The first group departed for Moscow that evening. We left by bus on the 27th, shortly before noon. They announced the boarding, and the women who came to see us off began to wail. I said: ``Women, you are burying us too soon.'' By every sign I knew how grave our condition was; yet, frankly, I thought -- we shall live. My optimism did not prove true for all.

Rumour travels marvellously. Immediately upon leaving Chernobyl we passed through the village of Zalesye; women stood along the road, hands to their cheeks, pity in their eyes. There was a small incident: Viktor Smagin felt unwell, and the doctor was in another bus, so we had to stop. At once a crowd of women gathered about the bus, lamenting, gazing upon us in our hospital garb. Yes, our people are responsive and warm-hearted -- why, then, do they suffer both Chernobyl and all the other calamities that drive them to bitterness?

We reached the airfield without mishap and went straight to the aircraft. In Moscow, buses drew up to the plane as well -- and straight to the hospital.

There they cleared several wards for us -- some patients were discharged home, others transferred to different hospitals. At first I found myself in the gynaecological ward, but since I failed to give birth to anyone in alloted time, they transferred me elsewhere. Only six months later, on the 4th of November, was I discharged.

It has now become customary to revile our medicine -- and not only medicine. There is such a torrent of vilification of all things that it seems a man, even when alone with himself, will persist in that noble pastime. Other words appear forgotten. In an eight-page paper one can scarcely find a neutral line. The \textit{intelligence} -- the nation's soul, its writers -- behave like spiders in a jar. They have forgotten their vocation: \textit{to write.}

They even contrived to cast scandal upon a good deed -- the disinterested aid of the American physician, Dr.~Gale. It is evident that one man cannot do much. Equally evident -- whatever he did accomplish -- is that normal people ought to feel nothing but the deepest gratitude.

No, I will not reproach the staff of the Sixth Hospital. They drew me, and many others, back from the bony old woman. The line was slender. My thoughts were dim, but still thoughts. I believed it was the end when they could not stop the nosebleed and merely kept replacing the gauze packs. Alas, I knew that routine. I cannot say how long it lasted; I think I did not lose consciousness, though evidently I hovered in twilight. Suddenly I realized that my legs were mine, and my body was mine. From that moment I perceived myself as whole. Out of such a state the physicians dragged us. My thanks, first and foremost, to Sergei Filippovich Severin -- he was there in the worst hours. Thanks to Sergei Pavlovich Khalezov, Lyudmila Georgievna Seleznyova, Aleksandra Fedorovna Shamardina, and other doctors.

And what feeling can one have toward the nurses who gently, yet insistently, persuaded me to eat at least a little? Perhaps then they even annoyed me by it. But without food one cannot live. They fed me with a spoon. No, those girls were not merely doing a job; they \textit{nursed} their patients. To them, my gratitude.

Soon after discharge I received another summons to the Sixth Hospital, where I remained about three weeks; six months later I returned again. Both times they mended me. Elena Mikhailovna Dorofeeva especially soothed my throat; for a year thereafter I lived without the constant dry torment. Now, it seems, I shall be treated in Kyiv when need arises -- for travel is arduous, though the distance is not great.

\chapter{Who led the commissions?}

The immediate formation of a commission to investigate the causes of the catastrophe was wholly natural. Chronologically, the first commission consisted of officials from the Ministry of Medium Machine Building and the Ministry of Power and Electrification -- namely the Deputy Ministers A.~G.~Meshkov and G.~A.~Shasharin -- together with organizations subordinate to those ministries: the Kurchatov Institute of Atomic Energy (IAE) and NIKIET, creators of the RBMK reactor; the Institute \textit{Hydroproject}, the general designer of the plant; and VNIIAES, representing the operating organization. Who was formally appointed chairman I do not known -- I never saw the order establishing the body. In the report, no one desired to call himself chairman of a commission or subgroup. In almost all early documents concerning the Chernobyl catastrophe, chairmen vanish. For brevity I shall call this body the \textit{Meshkov commission}, after the senior official involved, since G.~A.~Shasharin, though likewise a deputy minister, did not sign the report.

In principle, the appointment of these officials was logical and provoked no protest. They knew the reactor, the station, and the personnel best; who else should inquire into the causes? The difficulty lay in this: all these men were interested in one and the same thing, though in differing degrees.

The accident was so grave that responsibility for it threatened not merely loss of a bonus, nor a break in one's career, but one's very freedom. In such circumstances there is no place for scruples. Honour and conscience -- if ever present -- depart at once; who has need of them? Above all, the designers of the reactor were interested in concealing the true causes, should they lie within the reactor itself. Others had the same stake. What, after all, had the numerous staff of VNIIAES been doing -- what was their function, and ha

% --

Were there other competent and comparatively neutral persons? Certainly. RBMKs were not the only reactors in the country. There were other organizations designing nuclear reactors and conducting reactor studies. Even without an exact knowledge of the RBMK construction, such specialists could have understood the situation -- especially when joined with power-plant personnel from Chernobyl or from other RBMK reactors (Kursk, Smolensk). For example, the stations' shift supervisors -- two or three competent men might have been selected and included. These alone would have had no interest in unjustly blaming operating staff, since they themselves might one day stand in that position. Their practical knowledge of plant and reactor, united with the deeper theoretical knowledge of neutral research and design personnel, might have formed a counterweight -- particularly against staff from IAE and NIKIET, the most unscrupulous and relentless. My remark concerns not those institutes as a whole but the individuals taking part in the investigation -- and not even all of them.

But, as is customary in our country (and I speak not from Chernobyl alone), investigations are entrusted to those who are potential -- and most often actual -- culprits.

\section{The Meshkov Commission}

\begin{personal}[Principal conclusion of the Meshkov Commission:]
\textit{``The most probable cause of the explosion was boiling-up of the reactor core with rapid drying-out of the process channels as a result of a cavitation regime in the main circulation pumps (GCPs).''}
\end{personal}

The explosion occurred \qty{42.5}{\second} after closure of the turbine steam-valves, that is, at 01:23:46.5 a.m. Everything else in the report serves only to bolster this version.

The men of the commission were competent; they knew the unit first-hand, had taken part in earlier accident inquiries, and were familiar with reactor and system calculations. Yet something prevented them from perceiving the plain absurdity of their conclusion.

Boiling-up and rapid drying-out? The authors do not explain when this began, nor what ``rapid'' signifies. If immediately before the explosion, then by that time the Control and Protection rods had entered at least \qty{2.5}{\metre}; why, then, did the emergency protection (AZ-5) fail to shut down the reactor? If at the moment of pressing AZ-5 -- and one must suppose that this was precisely why it was pressed -- then how did we detect a need for such action, since only three seconds later came signals of overpower and rapid power increase? Nor was there any signal of pump-trip. Why, indeed, should the pumps have failed at all, when earlier they had run without trouble under less favourable conditions? And by what mystical cause could pumps driven by a coasting generator have failed? For them, conditions were thoroughly normal; even with the loss of four, four operating pumps were sufficient to cool the core at \qty{200}{\mega\watt}.

Why did the commission ignore the data recorded by the SKALA\footnote{The name of the computer used for centralized monitoring of the Chernobyl plant's parameters.} centralized monitoring system showing the flows of all eight pumps? The readings at 01:23:04 a.m., the start of the turbine rundown, were normal: the four ``coasting'' pumps showed the expected decrease in flow; the four supplied from the reserve network showed the expected slight increase. At 01:23:40 a.m. AZ-5 was pressed; at 01:23:43 a.m. came the alarms of overpower and rapid power rise; and still the pumps continued to operate normally. Are the instruments lying? Hardly can eight independent sensors lie at once -- four in one direction, four in another -- and all in harmony with a preconceived theory. Only when reactor power leapt to some unknown magnitude did the pumps naturally reduce flow.

But why should the commission consult that soulless stone, the SKALA recorder? It showed not what they wished.

Even NIKIET appears to have forgotten how to calculate. The report claims that at \qty{200}{\mega\watt} with four pumps per side the steam-content of the coolant would be \qty{2}{\percent}, whereas in fact it was under \qty{1}{\percent}. Their numbers betrayed them. To prove pump cavitation they cite a pressure drop of \qty{8}{\metre} water head at \qty{21000}{\metre\cubed\per\hour}, while in another memorandum they give \qty{4}{\metre} at a higher flow. All things are possible -- when necessity compels them.

Why then did the commission \textit{choose} pump failure? I say \textit{choose}, for I have no doubt that the real causes were evident from the beginning to most of them. The men of IAE and NIKIET knew, and so did Meshkov -- formerly head of the main directorate for RBMK reactors, acquainted with all documentation on the Leningrad and other stations, with accident reports and operational measurements. The commission was not searching for causes but for the most suitable appearance, and pump failure was the most convenient. After the power reduction the pump flows increased, and in two or three of the eight exceeded the permissible limit for that regime. Operator B.~Stolyarchuk overlooked this -- or saw it but had no time, being occupied elsewhere. Thus a regulation violation by personnel appears. The rest is a matter of routine. Could such a violation trip those pumps? Possibly. Did it? Irrelevant. The operating staff are to blame!

Here I must add: even had the pumps truly failed, the explosion would not have been the staff's fault. Trip of two or even all the main pumps is entirely possible. For example, if the main relief valves open and fail to close -- especially at low power -- the primary-circuit pressure falls sharply and pumps lose suction. Improper work of the feedwater controller may trip pumps on one side, enough for an explosion in a reactor of that design. Therefore the reactor must be designed to survive loss of all main pumps. That is the business of science and of the designers. All such conditions should have been analyzed in the design stage, with every necessary safety measure provided.

Did the Meshkov commission understand this? Certainly. But their logic was simple: by the time anyone sorts it out, time will have passed, and deficiencies will quietly be corrected. And will it ever be sorted out? Anything labelled Secret, and even much that was not, remained inaccessible. Whoever had access and understanding would have his mouth closed by threat of dismissal or, worse, loss of clearance -- without which a nuclear specialist can do little.

In the end, because manipulation of facts grew too blatant, the pump-failure version was quietly abandoned. Only NIKIET still clings to it, faintly, forgetting even its own report No.~05-075-933, where it states that complete drying-out of a hot RBMK core always produces negative reactivity. (That report, however, was itself mistaken.)

\section{The Shasharin Commission}

Deputy Minister G.~A.~Shasharin refused to sign the Meshkov report, and a group from Hydroproject and VNIIAES, with participation of the All-Union Thermal Engineering Institute and the pump designers, conducted its own inquiry and issued a document entitled \textit{Supplement to the Investigation Report}.

Already in May 1986 this document correctly reflected the essence of what occurred at the unit -- indeed, it might have served as the foundation of an objective study. It demonstrated convincingly:

\begin{itemize}
\item the untenability of the pump-failure version;
\item that the turbine-rundown experiment was irrelevant to the accident;
\item that even had the reactor been automatically shut down at 01:23:04 a.m., the explosion would have occurred \qty{36}{\second} earlier;
\item that no rupture of \qty{300}{\milli\metre} or larger primary pipes had occurred.
\end{itemize}

I shall quote the document with some abridgment. It merits citation not because it introduces anything unknown today, but because already in May 1986 the true causes of the accident were in effect established -- had one but approached the matter without bias. The numbering of its clauses is retained.

\begin{personal}[Findings of the Shasharin Commission:]
\textit{According to DREG printouts, decoded oscillograms of parameter changes during the experiment on turbine rundown with internal loads (Appendix~2), chart-recorder diagrams, written explanations of the operating staff, and information from the pump-design organization (Appendix~3), there was no loss of circulation in the main loops up to the moment of uncontrolled reactor power surge and pressure rise. The coolant flow through each pump and through the circuit as a whole remained stable up to 01:23:45 a.m.; no signs of flow breakdown were observed.}

\textit{The unit operated on the normal scheme with Turbine~8 in service, generating \qty{40}{\mega\watt} (electrical) at a reactor thermal power of about \qty{200}{\mega\watt}. Power was held by automatic control. All reactor parameters before the accident, up to pressing AZ-5, were normal and stable. No emergency alarms were recorded.}

\textit{After turbine trip the regime changed only gradually over \qtyrange{30}{40}{\second}: total flow through the reactor decreased by \qty{20}{\percent}, the circuit shifted from eight to four pumps while reactor power remained about \qtyrange{6}{7}{\percent} of nominal. The four pumps on normal power supply increased individual flow, reducing the margin to boiling, but no signs of head loss, decreased performance, or changes in reactivity or power were observed.}

\textit{Differences from ordinary operation:}
\begin{enumerate}
\item \textit{For the experiment simulating loss of external power, all eight pumps were run from the coasting turbine, which was not prohibited.}
\item \textit{Operating reactivity margin before the accident was about eight rods, whereas regulations required at least fifteen.}
\end{enumerate}

\textit{Violations by the operating staff:}
\begin{enumerate}
\setcounter{enumi}{4}
\item \textit{Flow through some pumps exceeded the \qty{7000}{\meter\cubed\per\hour} limit when feedwater flow is below \qty{500}{\tonne\per\hour}.}
\item \textit{During the \num{12}-minute transient the thermal power fell to \qtyrange[range-phrase=\text{--}]{40}{60}{\mega\watt}, operating reactivity margin dropped below permissible minimum, reaching eight rods one minute before the accident; reactor power at \qty{200}{\mega\watt} deviated from program.}
\end{enumerate}
\end{personal}

The expert group noted that to determine operating reactivity margin an operator must request a PRIZMA calculation -- requiring \numrange{7}{10} minutes, during which conditions may change. Another method is to sum \num{211} rod-position indicators -- also slow. Neither design materials nor regulations provided a safety justification for the minimal operating reactivity margin, nor explained the consequences of low margin, nor guided optimal rod distribution under xenon poisoning, nor warned of special dangers at low power. All emphasized only the danger of overpower at high levels. Thus personnel were unprepared, technically and psychologically, for the fact that low-power operation might be equally -- or more -- dangerous.

\begin{personal}[Findings of the Shasharin Commission (contd.):]
\textit{Causes of the accident.}  
\textit{According to VNIIAES calculations, the chief cause of the uncontrolled surge was pressing AZ--5 under specific conditions: operating reactivity margin of eight rods and small inlet subcooling. The runaway resulted from the combined effect of:}

\begin{enumerate}
\setcounter{enumi}{10}
\item \textit{Fundamentally incorrect control-rod design introducing positive reactivity at initial insertion.}
\item \textit{Positive void coefficient.}
\item \textit{Existence of positive prompt-power coefficient (contrary to prior assertions).}
\item \textit{Pump operation at low power with high total flow and small feedwater flow (not prohibited).}
\item \textit{Unintentional breach of rules on operating reactivity margin and test-program conditions.}
\item \textit{Insufficient technical protections and operator information; absence of warnings about these dangers.}
\end{enumerate}

\textit{These facts show that the reactor design did not satisfy essential safety requirements of paragraphs 2.2.2 and 2.3.7 of the General Nuclear Safety Regulations (OPB).}
\end{personal}

This was the only commission that explicitly noted the reactor's non-compliance with safety standards. True, something -- perhaps lack of time -- kept it from enumerating all violations later revealed; but had this document been officially adopted, every deficiency of the reactor in meeting safety requirements would have revealed itself of its own accord.

\section{The Volkov Commission}

An employee of the Kurchatov Institute of Atomic Energy, V.~P.~Volkov, long before the accident had drawn attention to the unsatisfactory characteristics of the RBMK reactor core and of its control and protection system. He, singly and in collaboration with others, put forth definite proposals for modernization. In particular, he suggested a variant of a fast-acting Emergency Protection. I do not know the precise content of these proposals and therefore cannot pass judgment upon them; but the phenomena at which they were directed ought to have been eliminated -- whether by adopting Volkov's suggestions or in some other way -- for it was precisely owing to those phenomena that the accident occurred. For a number of years his immediate superiors, V.~I.~Osipuk and V.~V.~Kichko, took no steps to put the proposals into practice. Volkov then addressed a memorandum to the director of the Institute, the scientific head of the RBMK programme, Scientist A.~P.~Aleksandrov. No ordinary person. His resolution upon the memorandum was: ``Comrade Kichko, convene a meeting in my office at once.''

But either some flourish in the signature hinted, ``pay no attention,'' or for some other reason the meeting never took place before the accident. Volkov had nowhere higher to appeal: Aleksandrov was at the same time President of the Academy of Sciences.

They waited for an accident. After it occurred, Volkov turned his documents over to the procuracy, since he was convinced -- and quite rightly -- that the reactor had exploded because of its unsatisfactory quality and by no means through the fault of the operating staff. And here Aleksandrov's reaction was instantaneous: Volkov was no longer admitted to the Institute.

This was not the sort of machinery where one could wait ten or twelve years. Here someone had begun to undermine his own reputation, and the response was not slow.

Colonel Skalozub once said: ``To Prince Grigory, and to you, I will assign a sergeant as your Voltaire.''\footnote{Colonel Skalozub, speaking sarcastically, promises to appoint a sergeant to serve as the listener's ``Voltaire'' -- a supposed man of enlightenment. The joke lies in the contrast: Voltaire signifies high intellect and refined satire, whereas a sergeant represents the most prosaic, unlettered military type. Skalozub is mocking the pretensions of ``Prince Grigory'' and the addressee, suggesting that their philosophy deserves no better an interlocutor than an ordinary drill-sergeant.}

Our President needed no sergeants as assistants.

But Volkov -- stubborn man that he was -- wrote to M.~S.~Gorbachev himself. On the basis of his letter a group was formed under the chairmanship of the Deputy Chairman of the State Committee for Supervision of Nuclear Power Safety (Gosatomenergonadzor), V.~A.~Sidorenko. In essence it confirmed the reactor's unfitness for operation. The group appended an interesting remark to its covering letter: that Volkov underestimated the prescriptions of the Operating Regulations. What they had in mind, of course, was the Operating Regulations' requirement of a minimum operating reactivity margin of fifteen rods. This means that the supervisory authority was defending the designers' decision to substitute, by a mere clause in the Regulations, those safety devices demanded by law: alarms upon deviation of a parameter, automatic shutdown when it exceeds permissible limits, and, at the least, a proper system for measuring it.

And this is a \textit{supervisory} body! The very body which should ensure that reactors are built in conformity with the safety standards. Yet there is no cause for surprise. It was V.~A.~Sidorenko who bore responsibility in Gosatomenergonadzor for the nuclear safety of reactors. The same story over again.

After I had been convicted, I sent a complaint to the Central Committee; there were people there sufficiently competent to understand my arguments concerning the groundlessness of the conviction. For example, V.~I.~Grechny had once served as Deputy Chief Engineer of the plant for scientific work, dealing specifically with nuclear-safety questions. And what befell? My letter in the Central Committee was passed to the Deputy Procurator-General, O.~V.~Soroka, the man who had approved my indictment. You can guess the answer. Precisely so it was.

Here is how Volkov speaks of the causes of the catastrophe in one of his reports:

\begin{personal}[Volkov's report:]
\textit{``Analysis of the Chernobyl accident has shown: a large effect of the displacement blocks, a large steam (void) effect of reactivity, and the development during the accident of an excessively great volumetric non-uniformity of power release in the core. The latter circumstance is among the most important and is determined by the large dimensions of the core (\qtyproduct{7 x 12}{\metre}), the low speed (\qty{0.4}{\metre\per\second}) of motion of the heterogeneous rods (containing absorbers, displacers, and water columns), and the large steam effect of reactivity -- \qty{5}{\betaeff}. All this predetermined the scale of the catastrophe.}

\textit{Thus the scale of the accident at the Chernobyl nuclear power plant was determined not by the actions of the operating staff, but by the failure -- primarily on the part of the scientific leadership -- to understand the influence of steam content on the reactivity of the RBMK core. This led to an incorrect reliability analysis of operation; to ignoring the repeated manifestations in operation of the large value of the steam effect of reactivity; to a false confidence in the sufficient effectiveness of the control rods, which in reality could not cope either with the accident that occurred or with many others, including design-basis accidents; and, naturally, to the drafting of incorrect Operating Regulations.}

\textit{Such scientific and technical leadership is explained, among other things, by the extremely low level of scientific and technical work devoted to substantiating the neutron-physics processes occurring in the cores of RBMK power reactors; by ignoring discrepancies between results obtained by different calculation methods; by the absence of experimental studies under conditions closest to full-scale; by the lack of analysis of the specialized literature and, ultimately, by providing the chief designer with incorrect methods for calculating neutron-physics processes and by a failure to fulfil one's own functions -- namely, substantiating the processes occurring in the core and substantiating the safety of RBMK nuclear power plants.}

\textit{An important circumstance is also that the Ministry of Power, for a long time, passively operated RBMK nuclear power plants whose cores had neutron-physical instability, paid no due attention to the frequent occurrence of AZM and AES signals when Emergency Protection was actuated, and did not demand thorough analysis of accident situations\ldots One must acknowledge that an accident like Chernobyl's was inevitable.''}
\end{personal}

So it is: if the scientific workers -- who have at their disposal the computational and experimental apparatus -- do not know, how are the operators to know? If science cannot make sense of the experimental data from the plants, how are men on shift, occupied with their watch duties, to do so? In fairness it must be said -- and Volkov is an example of this -- that not everyone sat contentedly. People saw the defects and made proposals for their elimination. But they encountered a solid wall of leadership.

On 2 and 17 June 1986 meetings of the Interdepartmental Scientific and Technical Council (MVTS) were held under the chairmanship of A.~P.~Aleksandrov. The calculations of VNIIAES and the conclusions of the Shasharin group were ignored. Nor, naturally, were Volkov's arguments heeded. The President and thrice Hero overwhelmed all with the authority of his position. As a result, the causes of the accident were reduced exclusively to errors and improper actions by the operating staff. The MVTS decision opened the way to the disinformation of specialists and the public; thereafter all drew upon this decision, with slight variations.

It was entirely ``logical'' in its illogic that the conclusion of the Government Commission, chaired by Deputy Chairman of the Council of Ministers B.~E.~Shcherbina, should be what it was.

The commission established that the RBMK reactor possessed a positive prompt power coefficient of reactivity; it immediately declared that such a coefficient should not be positive. One must suppose they consulted some regulatory document, though they cited none. If this is an impermissible property of the reactor, the commission then established something still more paradoxical and beyond ordinary comprehension: that in the first few seconds after actuation, Emergency Protection introduced \textit{positive} reactivity.

And what was the conclusion of this high commission?

The operating staff are to blame!

Logic indeed\ldots

Did the staff design the reactor core, did they calculate it so that it would have a positive power coefficient? No.

Did the staff invent the monstrous construction of the control rods? No. Then why are the staff, killed and maimed by this mastodon, to blame? For what, precisely, are Shift Supervisor Aleksandr Fedorovich Akimov -- who ordered the emergency protection to be actuated when there were no emergency or even warning signals -- and Reactor Operator Leonid Fedorovich Toptunov, who executed that order, supposed to be guilty? For nothing, of course. How can one accuse an operator who, whether from necessity or not, presses the AZ button? In any case -- given that AZ is now admitted to have blown up the reactor -- common human logic, as distinct from ``government'' logic, demands that all charges against the operator be withdrawn.

Equipment is produced in this country without compliance with standards; they have even ceased to write about it. Now the Government Commission, by its conclusion, legalizes this practice. Decisions are taken as those in power deem needful -- for the benefit, naturally, of ``the people.'' There is, it seems, no time to think of justice.

The MVTS decision on the causes of the catastrophe, dictated by the reactor's creators, is intelligible; they were defending their own interests. Less obvious, perhaps, is why that decision met with such enthusiastic endorsement in the Government Commission and higher up. Yet this too is not hard to grasp. One has only to imagine what the acknowledgement and publication of the fact that the reactor was unfit for service would have entailed:

\begin{itemize}
\item Both domestic and foreign opinion would have demanded immediate shutdown of the remaining RBMKs -- \num{13}~million kilowatts of electric capacity, Ignalina not included. Restrictions on industry would have been unavoidable.
\item The country's two leading nuclear-industry institutes, IAE and NIKIET, would have been exposed as incompetent.
\item Gosatomenergonadzor, whose duty it is to prevent unfit reactors from entering service, had, to put it mildly, slept through the situation. That would have cast doubt not only on the RBMK but on other reactors as well -- in other words, on the entire nuclear-power programme.
\item What a blow to the prestige of the Land of Soviets\ldots
\end{itemize}

All this is true; so it stands in reality. But it could not be admitted.

It is much simpler to blame the staff -- a mere handful of obscure men.

Thus the Conclusion of the Government Commission, and the subsequent Decision of the Politburo, opened the way for all who wished to accuse the staff and, conversely, closed the path to an objective investigation. From that time on, commissions, authors of articles in journals and newspapers, and writers of fictional and ``documentary'' works all knew what and how they must write about the Chernobyl accident.

According to the new state policy, the USSR had embarked upon the path of \textit{glasnost} and openness. Therefore specialists headed by Scientist V.~A.~Legasov were sent to the International Atomic Energy Agency (IAEA) to inform ``the international community,'' as one of the informants, Dr.~A.~S.~Shulenkov, expressed it.

\section{Soviet informers to the IAEA}

By the time this commission set to work, the version of a trip of the main circulation pumps had already been abandoned, though the tendency to place the blame upon the operating staff was carefully preserved. I shall not reproduce and dissect in detail the material of the Report to the IAEA; it is, in any case, highly specialized. The Report, though issued in a limited number of copies, can be obtained and read. I shall dwell only on what is necessary.

There are no calculations in the Report, nor any references to them. It offers qualitative explanations, and these are fairly arbitrary. The beginning of the reactor power surge is explained there as follows:

\begin{personal}[From the Report of the Soviet informants to IAEA:]
\textit{``By the beginning of the test, namely at 01:23 a.m., the reactor parameters were closest to stable. Closure of steam to the turbine led to a slow rise of pressure in the drum-separators at a rate of about \qty{6}{\kilo\pascal\per\second}. At the same time the flow of coolant through the reactor began to decrease, owing to the rundown of four of the eight main circulation pumps (MCPs). A minute earlier the operator had reduced the feed-water flow. The reduction of coolant flow through the reactor and of feed-water flow, despite the fact that the associated increase of pressure counteracted these factors in terms of steam generation, ultimately led to a rise of reactor power, since the reactor possesses positive feedback between power and steam formation. Under the conditions of the experiment, immediately before the start of the turbine-generator rundown there was an insignificant amount of steam in the core, and its increase was many times greater than under normal operation at rated power.}

\textit{It was precisely this rise of power that could have prompted the staff to press the emergency protection button AZ-5. Since, in violation of the operating regulations, the staff had withdrawn from the core more absorbing rods of the manual control system (RR) than allowed, the effectiveness of the emergency protection rods proved insufficient, and the total positive reactivity continued to grow.''}
\end{personal}

Thus, in the commission's view, the power surge of the reactor began as a result of the combined action of the reduction of coolant and feed-water flow and could not be suppressed by the emergency protection. I shall attempt to set this out in a form clear even to a non-specialist.

Adding reactivity to a critical reactor leads to an increase of its power. The RBMK reactor possessed a positive steam (void) reactivity effect: as the volume of steam in the core rose, the reactivity rose. A decrease of coolant and feed-water flow leads to an increase of steam volume. A rise of pressure in the primary circuit, on the contrary, leads to a decrease of steam volume.

What, then, will be the result of their combined action? First of all, we are dealing with no extraordinary phenomenon, but with a situation entirely possible under real operating conditions, differing only in that in practice the reduction of coolant flow is usually caused not by rundown of the pumps but by their trip. Therefore the reactor and its control and protection system rods are obliged to cope with it.

In the case at hand, a change of feed-water flow led to a rapid and comparatively large change of reactivity, and up to 01:23 a.m. there were several such episodes; the automatic power regulator of the reactor coped with them quite satisfactorily -- no power spikes appear on the chart-recording.

A change of feed-water flow, a normal occurrence when regulating the water level in the drum-separators, took place one minute before the start of the rundown. A simple calculation shows that by 01:23 a.m. the disturbance from this cause had already been worked off by the regulator. And had it not been worked off, for example owing to malfunction, there would have been alarms. There are none.

What remains is the smoother change of reactivity caused by the change of coolant flow, and this is normally compensated by the Automatic Regulator; on the reactor-power chart an increase of five megawatts over the last four minutes is recorded. For a RBMK-1000 this is negligible.

The phrase ``since the reactor possesses positive feedback between power and steam formation'' is some sort of revelation on the Commission's part. And what, indeed, should the feedback be in a boiling reactor? It is obvious that the higher the power, the more steam is formed. That is precisely how it ought to be.

If the commission wished to say that there existed positive feedback between power and \textit{reactivity}, that is another matter. Such feedback can be of either sign: negative in a properly designed reactor, positive -- as in the RBMK -- which is impermissible.

The phrase about the increase of steam being many times greater than during ``normal operation at rated power'' is likewise obscure. It should be read thus: at low power (always, and not only under the conditions of the experiment) the increase of volumetric steam content in the coolant per unit increase of power is several times greater than the same increase at rated power. It was precisely for this reason that at low power the RBMK reactor possessed positive feedback between power and reactivity. And it was precisely this fact, of primary importance for safety, which the scientific workers failed to take into account when they recommended that stations measure the power coefficient only at powers close to rated. Had the recommendations been competent, the positive power coefficient would long since have been discovered by the Nuclear Safety Departments of the stations.

And what is meant by \textit{``normal operation at rated power''}? Before the accident, \qty{200}{\mega\watt} was also a normal, regulated power level.

Further: \textit{``since, in violation of the operating regulations, the staff had withdrawn from the core more absorbing RR rods than allowed, the effectiveness of the AZ rods proved insufficient \ldots''}

There is not a single clause in the Operating Regulations that states how many rods may be withdrawn from the core. There is only an instruction in the ``RBMK Operating Manual'' and in the ``RBMK Control and Protection System Operating Manual'' which says that at low reactivity margin and with the neutron-flux maximum in the lower part of the core, the number of rods withdrawn fully from the core must be limited to \num{150} in total, and the remaining rods must be inserted by not less than \qty{0.5}{\metre}. On 26~April our neutron-flux profile had its maximum in the upper part. Therefore we violated no instruction. These addenda to the manuals were agreed with the Scientific Supervisor and the Chief Designer and were introduced on the basis of a letter from the Deputy Director of NIKIET, Yu.~M.~Cherkashov, which, in particular, states:

\begin{personal}[Letter of Yuri Cherkashov:]
\textit{``Let us emphasize once more that a positive reactivity overshoot will be observed when rods move only from the extreme upper position and only when the neutron field is skewed towards the bottom.''}
\end{personal}

The effectiveness of the AZ rods proved not merely low, but of opposite sign. Such it was because of the incorrect design of the rods. Why invert concepts? The effectiveness of the rods depends not on how many of them are raised to the upper limit switch, but on their design and on the configuration of the neutron field.

With a neutron field neutral in the lower part of the core, a rod starting from the upper limit switch always introduces positive reactivity at first -- which contradicts every canon of design. True, at a large reactivity margin part of the rods, when Emergency Protection is actuated, start from intermediate positions along the core height and immediately begin to introduce negative reactivity. But in any case some rods at first introduce positive reactivity, and what the sum will be -- only God knows. Can a conscientious man defend such a situation?

In their zeal to vilify the staff before the \textit{``international community,''} the Soviet informers slid into the most vulgar lying. Well, something must be sacrificed, once it has proved impossible both to preserve virtue and to acquire capital.

As to the number of rods withdrawn from the core, we already know the truth.

Foreign specialists remarked that an excessive burden had been placed upon the operator: he was required to maintain a minimum reactivity margin essentially without means of monitoring and information under transient conditions. In reply our informers said that the Operating Regulations specified a minimum margin of thirty rods. It is true that such a figure is given there -- \textbf{but only \textit{after} the accident}.

It proved impossible to conceal the positive power coefficient at low reactor powers; and from the theory of automatic control it is known that operation under such conditions is extremely dangerous, and that sooner or later that coefficient must manifest itself. For our informers nothing is easier than to say: operation at power below \qty{700}{\mega\watt} was forbidden by the Regulations. Yes, it was forbidden by the Regulations\ldots \textit{after} the accident.

By August~1986, when the informers went to Vienna, there already existed calculations showing that AZ could introduce positive reactivity up to a magnitude of $\beta_{\mathrm{eff}}$. This they passed over in silence: the phenomenon was too scandalous. One might just manage to admit that Emergency Protection had failed to cope; but that Emergency Protection itself had blown up the reactor -- such a thing is intolerable. How could they confess to that?

The Report contains a table of violations committed by the staff on 26 April 1986. It consists of three columns: violation, motivation, consequences. The second column does not interest us. I shall cite the entries in order, omitting it. Below each row of the report, I shall give comments.

\small\begin{center}
  \begin{longtable}{|p{4em}|p{13.5em}|p{13.5em}|}
    \hline
    & \textit{Violation} & \textit{Consequences} \\
    \hline
    \textit{Report} & \textit{Reduction of the operational reactivity margin (ORM) substantially below the permissible value.} & \textit{Emergency protection of the reactor proved ineffective.} \\
    \hline
    Comment & Apparently true. At the moment of pressing the AZ-5 button the margin was of the order of \numrange{12}{13} control rods instead of the permissible \num{15}. & Yes -- but \textit{at} low margin, and not \textit{because} of the low margin. \\
    \hline
    \textit{Report} & \textit{Power drop below the level stipulated by the test programme.} & \textit{The reactor ended up in a hard-to-control state.} \\
    \hline
    Comment & There is no violation either of the Operating Regulations or of any instructions. The AR was maintaining reactor power satisfactorily. On the power recorder there are no oscillations, no spikes. & Yes, as is now clear, all the reactor's negative features showed themselves most sharply at such a power level. But what has that to do with the staff? \\
    \hline
    \textit{Report} & \textit{Connecting all the main circulation pumps (MCPs) to the reactor, with exceedance of the individual MCP flow limits set by the Regulations.} & \textit{The temperature of the coolant in the primary circuit became close to the saturation temperature.} \\
    \hline
    Comment & There is no violation in connecting all the MCPs. Such operating regimes are envisaged in the manuals. & For the inlet to the core (and that is what the informers have in mind), this statement is simply untrue. \\
    \hline
    \textit{Report} & \textit{Blocking of reactor protection system relying on shutdown signal from two turbogenerators.} & \textit{Loss of the capability for automatic shutdown of the reactor.} \\ 
    \hline
    Comment & No violation. It was done in accordance with the Operating Regulations. & This had absolutely no significance. The accident would have occurred \num{36} seconds earlier. \\
    \hline
    \textit{Report} & \textit{Blocking protections on water level and steam pressure in the drum-separators.} & \textit{Reactor protection on thermal parameters was completely disabled.} \\
    \hline
    Comment & For water level, protection was active from another group of instruments. For overpressure, the protection remained in service. For low pressure, the turbine protection was in service; only the setpoint was changed, which lies within the operator's rights. There is no violation. & Reactor protection was in force both on physical and on thermal parameters, in accordance with the Regulations for that mode of operation. In the 1991 report of the commission headed by V.~O.~Brunsch (at long last! -- A.D.) it is stated on this very point: ``\ldots the statement made in the Report that `reactor protection on thermal parameters was completely disabled' does not correspond to reality.'' \\
    \hline
    \textit{Report} & \textit{Disconnection of the protection system against the maximum design-basis accident (disconnection of the ECCS).} & \textit{Loss of the capability to limit the scale of the accident.} \\
    \hline
    Comment & It was disconnected in accordance with a programme approved by the chief engineer. Under the documents in force at that time the chief engineer was fully entitled to authorize temporary disconnection. & By August the commission knew the system and the nature of the destruction perfectly well, and clearly understood that the capability to limit the accident had not been ``lost'' but had been absent in principle. \\
    \hline
  \end{longtable}
\end{center}\normalsize

Such is the state of affairs with the violations shoveled onto the staff. I do not know whether the international community believed the Soviet informers upon their word, or demanded that they produce documents. It may well have believed upon their word, for it is hardly evident to that community why one would wish to slander one's own staff at all, and how such a thing is possible.

The conclusions of the commission were the natural continuation of such an inquiry.

\begin{personal}[Conclusions of the Soviet informers to IAEA:]
\textit{``The designers of the reactor installation did not envisage the creation of safety protection systems capable of preventing an accident under the combination of deliberate disconnections of technical safety devices and violations of operating regulations that actually occurred, since they considered such a combination of events impossible.''}
\end{personal}

You see, they lightly chide and commiserate with the poor designers: how could they possibly foresee violations and intrigues by the evil-intentioned staff, striving at any cost to spite the designers and blow up a perfectly serviceable reactor. And these very designers, knowing the supposed maliciousness of the staff, recognized the utter hopelessness of taking any measures against its tricks and spat upon fulfilling the lawful, written requirements of the safety standards. About this, however -- silence. Read the informers' Report to the IAEA carefully, and you will not find a single word about the RBMK reactor's failure to comply with the Nuclear Safety Rules (NSR) and the General Safety Provisions (OPB).

\begin{personal}[Conclusions of the Soviet informers to IAEA (contd.):]
\textit{``Thus, the root cause of the accident was an extremely improbable combination of violations of operating order and regime committed by the unit staff.''}
\end{personal}

The commission first invents violations and then cries out to the whole world: \textit{but this is incredible!}

\begin{personal}[Conclusions of the Soviet informers to IAEA (contd.):]
\textit{``The catastrophic scale of the accident resulted from the fact that the reactor had been brought by the staff into such a non-regulated state in which the influence of the positive reactivity coefficient on the power rise was significantly enhanced.''}
\end{personal}

I was much taken with the expression \textit{``such a non-regulated state of the reactor''}; many have since adopted it. And what sort of state is that, precisely? Why then did the monitoring system not respond to it with a single alarm? Either the state lies within the regulations, or the monitoring system is blind. And in either case, what claim can there be against the staff? Were we to determine it organoleptically, or how?

And the power coefficient is \textit{smaller} at high primary-circuit flow, since for one and the same change of power the change in steam content, and hence in reactivity, will be smaller. With their conclusions, the gentlemen informers have disturbed the very first General of the Order of the Jesuits, Ignatius of Loyola: even he would hardly have conceived of such a construction.\footnote{Ignatius of Loyola is famed for rigorous doctrinal reasoning and casuistic argument. The jab is that their reasoning is so tortured that not even the founder of the Jesuits, a master of intricate constructions, would have imagined anything so contrived.}

While I was still in hospital, my wife told me that her niece had read in the German magazine \textit{Der Spiegel} an abusive article about the staff. As if our hair had not fallen out and our tails had not dropped off, and yet still -- we were running a reactor. I swore, not aloud, at ``those damned foreigners.'' It turned out that our own compatriots had done them the favour\footnote{Dyatlov's first reaction was anger at ``those foreigners.'' But he then discovers that the defamatory material did not originate in Germany; it was supplied by his own compatriots. That is the bitter twist of the irony here.}. With great success, dear compatriots, upon the international stage in the era of \textit{glasnost}.

They mixed us, the staff, with tar; but they did not cleanse the reactor. People abroad know how to count. Later they understood; at first the informers rode in in triumph. The Procuracy calls them leading nuclear physicists. Indeed, \textit{leading}. But where?

\section{The IAEA experts}

They too became victims of the (mis-)informers. Shortly after the statement of the Soviet specialists, and in close collaboration with them, the International Nuclear Safety Advisory Group at the IAEA issued a report on the Chernobyl accident. Its causes are set forth in two sections: the first for the ordinary reader; the second virtually repeats the first, but with technical detail for specialists.

Let us glance at the first section.

\begin{personal}[Quote from the INSAG-1 report:]
\textit{``After delays by the grid dispatcher, further preparation of the unit for the test resumed during the night of 25~April, including reduction to the intended level of \qtyrange{700}{1000}{\mega\watt} (thermal). This proved difficult because of incorrect handling of the control system by the operator. As a result the reactor power dropped to too low a level.''}
\end{personal}

I have already described how the power reduction proceeded. The drop occurred because of a regulator malfunction. The ordinary reader will suppose that ``too low a level'' is in itself something criminal, impermissible for reactors. In reality, for normal reactors, engineered in accordance with accepted safety standards, it presents no danger whatever. It merely meant that the power was too low to continue the tests and had to be raised to \qty{200}{\mega\watt}, a level sufficient for conducting the remaining work.

\begin{personal}[Quote from the INSAG-1 report:]
\textit{``Power was again raised. With some difficulty a level of \qty{200}{\mega\watt} was reached, and this required the withdrawal of many control rods. It should be noted that prolonged operation at a level below \qty{700}{\mega\watt} thermal is prohibited by normal safety procedures because of problems of thermohydraulic instability.}

\textit{Two more circulation pumps were brought into operation in order to ensure that, after the tests, the reactor could continue operation with the required number of pumps. The high water flow caused by switching on these additional pumps represented a violation of normal operating procedures, since it exceeded the approved values both for the core and for some individual pumps and, more importantly, made it more difficult to control the main coolant systems.''}
\end{personal}

Everything in this ample paragraph is wrong. Why follow men who are pursuing a quite definite aim? Yes, in the Report the informers say that the operators evidently were unable to raise the power higher. But any reactor engineer ought to understand that if the reactor has been brought into a critical state and the power raised to \qty{200}{\mega\watt}, then with a positive prompt power coefficient of reactivity -- and such it was -- nothing prevents further increase of power.

There were no restrictions whatever upon the time for which the reactor might operate at powers \qtyrange[range-open-phrase={\text{from} }]{8}{3200}{\mega\watt}. There were no restrictions upon the maximum flow through the core, nor upon the connection of all eight pumps.

Let us suppose that in questions of power below \qty{700}{\mega\watt}, pump connection, and flow through the core, they were misled by the informers. But why did the experts so readily assent to the assertion that control was made more difficult at high coolant flow? At high flow the amount of steam within the core and the steam-water piping is less than at low flow for the same power level. Therefore, for instance, when feed-water flow is increased to maintain level in the drum-separators, steam collapse may occur; and since there is less steam at high flow, the effect of this collapse upon core reactivity and upon drum level will be smaller. Nor is the subcooling at the core inlet less at high flow; at the very least it is not smaller. This is a purely technical question and requires no acquaintance with the special instructions of the Chernobyl plant.

\begin{personal}[Quote from the INSAG-1 report:]
\textit{``One important consequence was that the operators blocked the automatic shutdown of the reactor on parameters such as steam pressure and water level in the drum-separators, so that their instability would not cause such a shutdown of the reactor and interrupt the tests: again, a serious violation of normal operating procedure.''}
\end{personal}

They did not block them. I have already written of this.

\begin{personal}[Quote from the INSAG-1 report:]
\textit{``Just before this the computerized centralized monitoring system had supplied the operator with information on the state of the reactor, including the positions at that moment of all the control rods. This was a clear warning, since it indicated that there was no remaining compensating capability in the control rods to provide protection against an emergency situation. Immediate shutdown of the reactor was required. Nevertheless the operator proceeded to the electrical tests, although the state of the unit, as is evident and as will be discussed below, was highly unstable.''}
\end{personal}

We already know that there was, in fact, no print-out at the staff's disposal. And it could not have been available by the start of the tests, even had it been produced. Let that rest upon the informers' conscience.

I should like to say something else here. The IAEA experts are not ordinary readers. From the print-out of rod positions (I remind you, it was made after the accident) it is clear that most of them were withdrawn from the core -- that is, they stood in a position in which their compensating capability, i.e. their capability to suppress reactivity, is maximal. For normal control and protection elements this is a generally accepted rule. Yet the experts speak of an absence of compensating capability in the control rods. On an emergency signal they act not as regulating rods, but as shut-down rods. This statement by the experts must be taken as assent to a design of rods such that, when they start to move, they first introduce positive reactivity. That is what it amounts to. But this is absolutely inadmissible, and the flaw of the rod design is now generally recognized (the design has since been changed). It is wholly incomprehensible to me how the experts allowed themselves to be persuaded on this point.

\begin{personal}[Quote from the INSAG-1 report:]
\textit{``From the beginning of the test the rundown of the turbine-generator started. Here a serious violation of procedure should be noted. Earlier the automatic shutdown of the reactor on disconnection of both turbine-generators had been blocked, so that the reactor would remain at power and the tests could be repeated if necessary. It should be explained that the tests could and should have been carried out in such a way that the reactor would have tripped at the start of the tests.''}\footnote{The translator of this work believes that the original author has mistakenly translated original, English-language document. In Dyatlov's translation, it incorrectly refers to the operation of ECCS being mandatory upon the trip of both turbogenerators: \textit{Следует пояснить, что испытания могли и должны были проводиться таким образом, чтобы сработала САОР при начале испытаний.}.}
\end{personal}

There was no violation whatever of procedures or manuals in blocking protection on trip of two turbine-generators. No one intended to repeat the tests; that is an invention of the informers. The assertion of the informers, readily echoed by the IAEA experts, that the tests could and should have been carried out with automatic reactor scram, rests upon nothing. It cannot be supported either by documents or by technical reasoning. Of course they \textit{could} have been carried out with the protection in service, and then the accident would merely have occurred \num{36} seconds earlier.

The description in the report of how and why the reactor surged in power differs in no respect from the version set out in the Soviet informers' Report and cited above. It does not correspond to reality.

The staff actions as presented by the experts are, for all practical purposes, the entire course of events of 26~April. And everywhere -- violations, violations, violations. Did this not arouse the IAEA experts' suspicions? What then -- did we study the manuals and regulations and take examinations only in order to know them and then do the contrary? Did they not ask the informers: why do you employ such staff at all? It cannot be that the staff strictly followed the manuals and suddenly, on 26~April~1986, went berserk and began doing everything wrong. That would mean that they had been acting in that way earlier too, but for some reason were kept on. Very well, let us suppose not all, but that Dyatlov alone, from a competent and -- almost by general acknowledgment -- thoughtful engineer, turned into a sort of hussar and began bawling left and right: \textit{``Block it! Disable it!''} It was never easy for us to compel an operator to violate the Regulations or a manual. Either he would refuse outright, or he would say: \textit{``Write it in the log -- I shall do it.''} Such are the questions that arise when one reads the IAEA experts' report.

On certain issues the experts were misled by the Soviet informers' references to provisions that did not exist or were introduced into the manuals after the accident. But they followed the informers willingly also in technical matters where they had simply to think.

I am far from desiring to repeat after V.~Yavorivsky that the IAEA is a mafia living off propaganda for nuclear power. I am sure that these are people convinced of the necessity and permissibility of nuclear plants, who believe in their acceptable safety today and discern in prospect the possibility of improving it.

Even so, their pliancy in the Chernobyl affair is incomprehensible to me. I therefore regard them as voluntary victims. I first read the Advisory Group's report in October~1990. I was surprised that, one would think, experienced sparrows had been caught with chaff. Without undertaking a full analysis of the report, I wrote comments on the second section and sent a letter to the Director of the IAEA, Mr.~H.~Blix. In that letter I posed questions whose self-evident answers showed the untenability of the version of the accident presented in the report, and also pointed out that the report permits the most shaggy-haired operator mug-shots to be made out quite distinctly.

In his reply Mr.~H.~Blix acknowledges that we too possess human qualities. As he writes:

\begin{personal}[Quote from the letter of H.~Blix:]
\textit{``The courage and dedication of your group during and after the accident are certainly not the qualities of ignorant `ghouls' and `troglodytes', and this is recognized throughout the world.''}
\end{personal}

Thank you, Mr.~Blix.

And yet it would be better, and far more useful, if the people of your Agency were able to assess their own positions critically, taking account of new information that has become known to them. Unfortunately I discern neither such desire nor such ability in the reply of the Chairman of the International Nuclear Safety Advisory Group, Mr.~H.~J.~C.~Kouts.

In my letter I indicated that, in the arrangement of the rods, we had violated nothing. Mr.~Kouts responds thus:

\begin{personal}[Quote from the letter of Mr.~Kouts:]
\textit{``Regardless of whether the configuration of the rods played an important role in the accident and whether it was at that time permissible in accordance with the rules, the actions that led to this situation were mis-judged.''}
\end{personal}

So that is how matters stand! Our comrades and gentlemen scientists, somewhere in their offices with their computers, somehow contrived to get to the bottom of it, but the operator was supposed to work it out in passing. It turns out that not only by Soviet standards, but also by Western criteria (or only by those of Mr.~Kouts?) it is all the same whether you acted according to the rules or not -- you are guilty in any case. Professor B.~G.~Dubovsky made the following comment on this:

\begin{personal}[Comment from Professor B.~G.~Dubovsky:]
\textit{``They -- that is, the operating personnel -- could have avoided the accident only if they had known more about the reactor than the Scientific Supervisor did.''}
\end{personal}

That is plain enough. Yet Mr.~Kouts continues:

\begin{personal}[Quote from the letter of Mr.~Kouts:]
\textit{``In our view, Mr.~Dyatlov's position is that he attributes the accident entirely to the insertion of positive reactivity by the AZ rods.''}
\end{personal}

Although the very fact that AZ inserts positive reactivity is monstrous in itself and, under other circumstances, by itself could have led to catastrophe, on 26~April it acted in concert with the positive power coefficient. I stated this in my letter in perfectly unambiguous terms.

It is the same with the other questions touched upon in my letter. The conclusion of Mr.~Kouts' reply is in a form very familiar to me -- from the replies of the Procuracy and the court:

\begin{personal}[Conclusion from the letter of Mr.~Kouts:]
\textit{``We have no grounds for changing our point of view. The accident occurred as a result of unsatisfactory operation regardless of what underlying causes there were, combined with the specific poor characteristics of the reactor design. The fact that at present additional important details concerning these characteristics have emerged does not radically change this point of view.''}
\end{personal}

The informers slandered the staff, and Mr.~Kouts' group has repeated it after them to the whole world. In my simplicity it seems to me that, if one group of people has calumniated the staff, this gives no right to another group to do the same. I pointed out those respects in which the authors of the report accuse us unjustly. I do not ask that they take my word; let them verify it. Are they answerable for their words or not? I am no professor, no international expert; I am a former convict and therefore employ only such words as I can support by documents or by generally recognized technical facts. I stand behind my words.

From the measures that began to be implemented immediately after the accident upon the remaining RBMK reactors, it is clear to any specialist that the destroyed reactor should never have been in operation. Later there appeared other documents as well. Yet Mr.~Kouts still speaks of unsatisfactory operation, regardless of what underlying causes there were. As a scientist, he has every right dispassionately to state what actions were taken and to classify them: these were correct, these were incorrect. Only let him not ascribe to the staff breaches they did not commit. In that case he truly need not concern himself with ``what underlying causes there were.''

I should also like to hear from the International Nuclear Safety Advisory Group -- I believe this lies within its remit -- not merely a statement about ``poor reactor characteristics,'' but a clear opinion on the admissibility or inadmissibility of designing and putting into operation a reactor with the following properties:

\begin{itemize}
\item A positive power coefficient of reactivity, with a full reactivity excursion due to this effect of several $\beta_{\mathrm{eff}}$.
\item Reactivity-control elements that change the sign of the reactivity they introduce while moving in one and the same direction, and, as a consequence, an emergency protection which, when actuated in various situations, can introduce positive reactivity.
\item Reactivity-control elements that do not prevent, but themselves create, a local critical mass.
\end{itemize}

The Group's report, issued in 1986, in effect repeats the statement of the Soviet specialists to the IAEA. It does not serve the cause of seeking the truth. The revelations of Scientist V.~Legasov and Dr.~A.~Shulenkov pleased the IAEA experts so much that they decided not to think for themselves. Of this they wrote in the report:

\begin{personal}[Quote from the INSAG-1 report:]
\textit{``On the basis of this information we have a reliable explanation of the sequence of events at Unit 4 of the Chernobyl Nuclear Power Plant, and we did not attempt to find an alternative explanation.''}
\end{personal}

\section{The forensic technical expert commission.}

A rabid crew. In some matters incompetent, in the main tendentious, and in any case devoid of objectivity. Frankly, I would rather not speak of this commission at all. I beheld it at the most critical moments of my life, which I should be better off forgetting -- yet cannot. The representative of NIKIET, O.~Shorokh, appeared particularly \textit{``engaging''}: insolent and unscrupulous. Little better than he was V.~A.~Trifonov of Gosatomenergonadzor.

The other commissions I did not see at the time, and therefore they present themselves to me in a fairly abstract fashion, not as living persons. Yet I know most of them by sight -- for example, behind the assertion of the forensic technical commission, offered as proof of the reliability of the RBMK reactor, that by the time of the accident they had accumulated roughly one hundred \textit{``reactor years''} of operation, one may distinctly discern the smirking features of the NIKIET employee V.~N.~Vasilevsky.

% --

Let us begin with that. In reality, by 26~April the RBMK reactors -- there were fourteen of them -- had operated in aggregate for \num{87}~reactor--years. Let us, for the sake of argument, agree with the experts that this is a criterion of reliability. Then, with the remaining thirteen RBMK units, we divide \num{87} by \num{13} -- and every six to seven years we shall have one more Chernobyl. I do not think such a prospect will inspire anyone. Consequently, this is not a criterion of reliability, but of hopelessness. And that is not all. After construction of the reactor, not all technological channels are loaded with fresh fuel, because there are not enough control rods in the Control and Protection System to suppress the excess reactivity. Approximately \num{240}~channels are occupied by flow meters (DP), and several hundred more absorbers are placed in special cartridges. In this manner the large steam effect is suppressed. These absorbers are withdrawn as the fuel burns up, roughly over a period of two years, when the reactor enters the so -- called regime of ``stationary refuelling.'' Thus \num{28}~years must be subtracted from the \num{87}.

The RBMK was especially dangerous at low power levels and with a small reactivity margin. It operated in such regimes during shutdowns. That amounts to all of \numrange{10}{20}~hours per year per reactor. Thus all the reactors were in such a regime for no more than two or three months in total. But with the development of nuclear power plants, it would have become necessary at night to unload the units with these reactors as well, and then there would have been far more dangerous regimes. Even by 1985 the power system was already forcing us to reduce power at night, though not yet greatly. Only operation in base--load mode, at steady full power, permitted those reactors to stretch out those \num{87}~reactor years.

In their effort to prove that, by disabling the SCRAM on the signal of trip of two turbine--generators, the staff violated the Operating Regulations, the commission arbitrarily interprets a clause of those Regulations. How can one derive from the phrase: \textit{``The switching on and off of protections, automatic devices and interlocks shall be performed in accordance with the operating manuals and operating charts''} the conclusion that \textit{`blocking is permitted only during shutdown and cooldown of the reactor, and not simply during steady operation with electric output below \qty{100}{\mega\watt}''}? Incidentally, steady operation at a \qty{100}{\mega\watt} load on the turbine is prohibited; no such mode existed on 26~April.

Here are the violations of operating documents which the commission discovers in the \textit{``Turbogenerator Run-Down Program''}:

\begin{itemize}
\item It is unclear why the commission decided that, in our view, the tests concerned only electrical systems and not the entire unit. Why then did we involve the reactor, turbine, and instrumentation and control departments in drawing up the programme? Their signatures are upon the programme. The electrical staff wrote the programme text itself; that is entirely natural.

\item They object to connecting four main circulation pumps (MCPs) to the run--down generator, since in that case ``either the non-return valves at the discharge of the running-down pumps will close, and ultimately their operation will be disrupted, or these valves will alternately close and open, causing oscillations of flows through all the pumps, which is exactly what was observed during the tests of 26~April.''

Either the experts do not know, or they do not wish to know, that each pump is equipped with an emergency protection (trip) system that actuates on low flow, and when the valve closes it is tripped by that system.

They are likewise unwilling to see that on 26~April, right up to the explosion of the reactor, the monitoring system recorded flows of not less than \qty{5000}{\meter\cubed\per\hour}. At such a flow there can be no question of a valve closing.

\item For the withdrawal of the ECCS they refer to clause~5.4 of the Nuclear Safety Rules (NSR), which deals with the \textit{``emergency cooldown system of the reactor,''} whereas the ECCS is the \textit{``emergency core cooling system of the reactor.''} These are two different systems.

There was no documentation for installation of the non-standard button for the maximum design-basis accident (MDA). Firstly, the programme indicates the terminal numbers for connecting the button. Nowhere is it stated that documentation must exist only in graphic form. Secondly, what sense is there in talking about a button when the ECCS itself is disconnected?

\item The \textit{``RBMK Operating Manual''} stipulates that transitions on MCPs (that is, replacement of one pump in operation by another) shall be carried out in the presence of a representative of the plant's Nuclear Safety Department. There were no transitions on MCPs. The experts did not read the clause to the end, where it states: \textit{``until such time as an order has been written.''} Such an order had indeed been written. Evidently the experts were in haste.

\item \textit{``The programme does not indicate where the excess steam is to be directed.''} The unit has automatic and remotely operated steam--dump devices; this is stated in the Regulations and other manuals. There is no sense in copying them into the programme.
\end{itemize}

Such, then, are the \textit{``violations.''}

In the opinion of the commission, the RBMK reactor, in the form it possessed in 1986, was quite fit for operation. For the court such a conclusion is quite suitable; for life -- it is not. Therefore, modernization began at once, after the accident, upon the remaining RBMK reactors.

% --

The masterpiece of the forensic technical commission is the following formulation: \textit{``The displacement of water in the lower parts of the Control and Protection System channels may have introduced an additional positive reactivity, which was envisaged in the design.''} What virtuosi! Not a blunder of the designers, but their \textit{foresight}! Straight out of the satirical magazine's rubric: ``You could not invent it.'' Only I do not find it amusing. All this passed, one to one, into the Bill of Indictment and thereafter into the verdict. And try then to prove the contrary. It was ``envisaged'' -- and that is an end of the matter.

The article by a large group of authors, \textit{``The Accident at the Chernobyl Nuclear Power Plant: One Year Later,''} in the journal \textit{Atomic Energy}, 1988, vol.~64, no.~1, January, states:

\begin{personal}[Quote from the article ``The Accident at the Chernobyl Nuclear Power Plant: One Year Later'':]
\textit{``The lesson of Chernobyl forced recognition of the fact that violations of the Operating Regulations may be utterly unpredictable. Therefore, first and foremost it was necessary to exclude the possibility of an uncontrolled reactor power surge in the event of violations of the technological operating regulations. From this standpoint, the most important factors to be considered are, firstly, the positive steam reactivity effect $\Delta \rho_{\text{пар}}$ and the corresponding positive reactivity effect under de-watering of the core, and secondly, the insufficient speed of action of the emergency protection under violations of the operating regulations regarding the minimum reactivity margin in transient and steady-state regimes'' (p.~180).}
\end{personal}

The acrobatic group act of these scientists could well be envied by the performers of the most celebrated circus.

We already know in what manner violations of the Regulations were imputed to the staff. What remains is the parameter \textit{``reactivity margin.''} Just how \textit{``unpredictable''} is its violation -- or, more precisely, the staff's failure to notice its reduction? The conversation between the scientists and the operator might well have run thus:

\begin{quote}
--- Really now, how could you have failed to notice the reduction in the reactivity margin? \\
--- Yet we have a magnificent device for measuring that parameter. The operator needs only five minutes (!) to obtain a reading. \\
--- But the parameter may change by three or four rods in half a minute -- for example, when the feed--water flow changes. \\
--- Then repeat the measurement. \\
--- The operator has no such time. In one minute he performs \numrange{20}{30} manipulations with the reactor--control devices and monitors more than \num{4000} parameters. \\
--- Oh, do go away. It is \textit{``unpredictable''} in any case.
\end{quote}

It is, you see, \textit{solely} for the sake of the \textit{``rotten operating staff''} that they are obliged to reduce the positive steam effect. There are no other reasons. The safety standards, as it were, do not exist. The chief designer of the reactor acknowledges it to be uncontrollable with such an effect. And what of that? The GKNT commission says that it exploded under a maximum design-basis accident. The Meshkov commission -- and in practice IAE and NIKIET -- assert that it exploded under loss of MCPs. And what of it? It is solely in deference to that \textit{``rotten operating staff''} that they are obliged to alter the design of the excellent, one might almost say exotic, emergency protection -- a world champion for slowness of action, and with the capability of turning from a protective into a power surging device! They achieved such a combination of functions -- and now they must change it.

You perceive, Reader, under what difficult conditions our scientists and reactor designers are compelled to work. And here is what is truly alarming: had the accident not occurred on 26~April, nothing would have been altered. Thus the reactor was moving inexorably towards explosion. As has become known since the accident, practically all of this had been perfectly clear to our learned gentlemen long before 26~April. No, Reader, I am no stool-pigeon\footnote{Dyatlov recognises that disclosures might look like insider tattling, and he rejects that idea.}. I learned this from the investigative materials, and they obviously came via the Procuracy. I assume there are yet more materials. I have grounds for that assumption.

Let us read further in the article:

\begin{personal}[Quote from the article ``The Accident at the Chernobyl Nuclear Power Plant: One Year Later'':]
\textit{``Calculations with different models yield similar results; for example, they show a rapid shutdown of the reactor with a regulatory reactivity margin (\num{15} rods) at the moment of dropping the emergency protection rods'' (p.~18).}
\end{personal}

Permit me not to believe it. Without resorting to other arguments, let us turn to the same article, p.~18:

\begin{personal}[Quote from the article ``The Accident at the Chernobyl Nuclear Power Plant: One Year Later'':]
\textit{``It is envisaged that the reactor will be shut down automatically when the reactivity margin falls to \num{30} control rods.''}
\end{personal}

Mark well: \num{30}~rods and \num{15}. And this now -- \num{30} -- under a steam effect reduced severalfold and thus reduced rate and magnitude of reactivity insertion; with the control rod design altered; with shortened absorber rods being inserted from below into the core on an emergency signal, that is, under far more favourable conditions. One cannot but recall the non-commissioned officer's widow who flogged herself -- not in Gogol's sense, but in the literal.\footnote{The line alludes to a well-known joke from Gogol's \textit{``The Tale of How Ivan Ivanovich Quarreled with Ivan Nikiforovich.''} In that story, an NCO's widow falsely accuses someone of beating her, only for it to be revealed that she beat herself in order to make the accusation stick. Here Dyatlov says he is not invoking Gogol's figurative or comic case of self-inflicted injury used to blame another. He means it literally: someone (or some institution) is effectively punishing itself by its own actions. The point is that the authorities, by exposing or amplifying damaging stories about their own people, have behaved like a person who whips himself and then cries out about being wronged -- a self-destructive, self-accusing act.}

In the Soviet Union the basic regulatory documents defining the nuclear safety of reactors are the Nuclear Safety Rules (NSR). They set forth the requirements a reactor must satisfy in order that accidents do not become catastrophes. Accident situations may arise both from technical malfunctions and from operator error.

Clause~2.7.1 of the \textit{``General Safety Provisions''} (OPB) directly obliges designers to think through possible operator errors that may lead to serious consequences, and to neutralize or preclude them. Thus the authors of the article in vain pose as benefactors of the operating staff. All this is their direct duty -- and not after an accident, but before it.

All the more because, apart from automatic protection upon reduction of the reactivity margin below \num{30}~rods, everything else has nothing whatever to do with the staff and ought to be implemented in accordance with general engineering rules.

Let us consider the technical measures implemented after the accident upon the remaining RBMK reactors in the light of their consistency with the requirements of NSR and OPB:

\begin{itemize}
\item Installation of additional absorbers (DP) in the core to reduce the steam reactivity effect. -- Brings the reactor into compliance with clause~2.2.2 of OPB.
\item Modification of the control rod design, insertion of shortened absorber rods into the core upon an emergency signal, increase of the operational reactivity margin, and improvement of the speed of action of the emergency protection. -- Brings the reactor into compliance with clauses~3.3.5, 3.3.26 and 3.3.28 of NSR.
\item Installation of alarms for deviation of the reactivity margin. -- Brings the reactor into compliance with clause~3.1.8 of NSR.
\item Automatic shutdown of the reactor when the reactivity margin falls to \num{30}~manual control rods. -- Brings the reactor into compliance with clause~3.3.21 of NSR and clause~2.7.1 of OPB.
\end{itemize}

As we see, the measures implemented after the accident upon the reactors merely bring them into agreement with the requirements of the mandatory regulatory documents -- and nothing more. Consequently, previously the reactor did not meet those requirements and was being operated illegally. To say that these measures are aimed at increasing the reliability of the reactors is impermissible. This is not an increase of reliability; it is the elimination of \textit{hopelessness}.

The authors of the article in the same journal say precisely this (p.~27):

\begin{personal}[Quote from the article ``The Accident at the Chernobyl Nuclear Power Plant: One Year Later'':]
\textit{``The analysis performed has shown that the priority measures already implemented guarantee the safety of the RBMK.''}
\end{personal}

Consequently, earlier its safety\ldots well, that is now clear to everyone. Yet the admission of the authors is valuable in that they -- the creators of the reactor -- continued, even after the accident, to cry from every street corner what a fine reactor the RBMK was. Here too, in this article, they persist in their unseemly business. Only in that one phrase did their \textit{``ears stick out,''} and there is no place where one might tuck them in. According to their calculations, the reactor does not explode if the main circulation pumps did not stop; and it does explode when the main circulation pumps have stopped. What can one say to that?

\begin{itemize}
\item Apart from the system's recorded evidence of normal pump operation (all eight pumps, not merely one, which might be in error and give grounds for doubt), this was acknowledged by all other investigators -- Kurchatov institute, VNIIAES, the main circulation pump designer, and others.
\item At the trial of the ``Chernobyl criminals'' the witness Orlenko, shift supervisor of the Electrical Department, testified that he suppressed the generator's excitation field -- that is, he shut it down -- after the explosion of the reactor. At the moment of the explosion he was thrown back from the panel under a heavy beam of the column, and then, mastering his fear, he again approached and shut down the generator as had been stipulated in the briefing in case some disturbance should arise.
\item And one more trifle -- the reactor actually did explode.
\end{itemize}

The authors of the article resemble the zoo visitor who, looking at a giraffe, says: \textit{``Such a long neck cannot exist.''}

If the authors' calculation is not deliberately deceitful, then here is the very moment, the ideal opportunity (Heaven forbid such an opportunity again) to refine coefficients and computational programmes, rather than persist in obstinacy. NIKIET submitted many such computations to the court record.

Here is Candidate of Technical Sciences Gavrilov. According to his calculation, as soon as the speed of the main circulation pumps powered from the running-down generator falls to \num{0.9} of nominal, the pumps, according to their ``head-flow'' characteristic, move onto the left side of the ``hump-shaped'' curve -- in other words, cease to pump water.

And what concern has the Candidate for the following facts:

\begin{itemize}
\item the recorded normal operation of the pumps at speeds down to at least \num{0.75} of nominal;
\item that the characteristic of the pump-throttle-valve complex is not ``hump-shaped'' at all, but falling;
\item that, had the flow fallen, the pump would have been tripped by its individual protection -- and no such trip was recorded.
\end{itemize}

Now a group of authors. I do not know whether there are Doctors among them, but Candidates there certainly are. Once again they prove main circulation pump rundown -- this time under the fall of pressure in the primary circuit. And they somehow fail to see upon their own composite graph that the drop in pressure is the consequence of increased feedwater flow, and that under those conditions the main circulation pumps can perfectly well operate.

Here is Candidate of Technical Sciences K.~K.~Polushkin. As a witness (it was his department that designed the exotic control rods) he testified in court that the staff had a printout of control rod positions, indicating a low reactivity margin, yet continued to work. This printout, obtained after the accident, is merely the last state that remained in the computer's memory. Let us suppose K.~K.~Polushkin does not know that the printout appeared after the accident. But he knows very well the layout of the control room and of the computer room. Let us compare the times. The printout is for 01:22:30. After the printout emerges, it must be torn off the teleprinter, registered in the log, and carried to the control room -- about forty metres away. Clearly, no one was running. And the rundown test began at 01:23:04 a.m. Could the printout have appeared on the control room desk in thirty four seconds? Of course not.

And what a computation of the reactivity-margin curve (with poisoning) NIKIET presented! From the aforementioned control rod position printout from Smolensk they computed a reactivity margin of \numrange{6}{8} rods at 01:22:30 a.m. NIKIET issued a curve beginning at 23:10, when power was \qty{50}{\percent} and the margin \num{26} rods. But the recorded \num{24} rod margin in the midnight entry of the operating log was ignored -- an ``ugly'' break in the curve -- so they drew it through \num{19} rods. And\ldots miraculously it lands exactly at the point -- margin \num{7} rods. As they say, spot on. Yet at 01:23:30 a.m. the margin was smaller by \numrange{3}{4} rods because of the large feedwater flow. How did the machine contrive to compute it so ``nicely''? Indeed, say what you will, but NIKIET must have clever programmes and clever machines? Or\ldots more likely the latter.

Dear Reader, I hope that in some degree I have explained why the accident materials were classified. The people who needed this were, and remain, influential. And there are, it appears, many. A.~I.~Solzhenitsyn in \textit{The Gulag Archipelago} says that for decades they selected -- who would live, who would die. And it would seem that their success in the ``selection of man'' is unquestionable. The NKVD\footnote{People's Commissariat for Internal Affairs, the Soviet secret police and security agency.}, it appears, is our only ``breeder'' who achieved results. We want Dutch cows, English pigs; but in men, for commissions, there was never any shortage. Commissions ready to formulate and sign whatever was desired. Yes, these are people with unblemished conscience -- by virtue of having none at all.

And since classification of materials can occur only with the agreement and approval of high officials, it is therefore done, it seems, in the interests of the State. And what sort of people are we, that the State must hide everything from us? No, surely we are a bad lot. Some radical or ``non--conformist'' -- there are many of them now -- might even entertain the treasonous thought: are some in power confusing their private and group interests with those of the State?

\textit{Molodaya Gvardiya} no.~8 (1990) printed a letter from Donetsk region mine rescue workers to the Chairman of the Council of Ministers, the Chairman of the All-Union Central Council of Trade Unions, and the Prosecutor General of the USSR It states:

\begin{personal}[Quote from the letter printed in \textit{Molodaya Gvardiya} no.~8 (1990):]
\textit{``In our country, the rate of severe and fatal injuries is an order of magnitude higher than in any other developed country. Data on miners' occupational illnesses are kept strictly secret.}

\textit{In our deep conviction, the greatest damage is caused not so much by the specific nature of underground work as by the irresponsibility of those who organize production. A system of collective irresponsibility and double standards is created\ldots{}''}

\textit{``\ldots{} A system of collective irresponsibility and double morality is created. `Safety days' are held, headquarters to fight violators convene, and beside an entire army of supervisory personnel there is also an army of public inspectors. But in parallel with this there operate harsh, unwritten rules of the `game,' according to which supervisory personnel in the mine must deliver the production plan -- must fulfil the shift assignment at any cost. Those who do not agree with these rules are mercilessly pushed out and replaced by others. The victims of this `game' will find no support anywhere.}

\textit{A system has been created and operates with precision for shielding from responsibility the principal `organizers' and `inspirers' of the outrages. The monopoly itself investigates accidents, itself drafts and adopts measures, itself supervises their implementation. Trade unions and Gosgorpromnadzor bodies, due to numerous reasons and `telephone law,' are dependent on the monopoly. Most often, during accident investigations, it is `overlooked' that the accident had been pre-programmed in the scheme and procedure of the work in advance.''}
\end{personal}

The rescue workers spoke well -- spoke rightly. Everything, or almost everything, occurred in the same way at Chernobyl. We have seen to whom the investigation of the accident was entrusted. And from the outset it was impossible to expect objective conclusions. Much the same can be said of the supervisory bodies. In 1983, during the physical startup of Unit~4 at the Chernobyl Nuclear Power Plant, an unacceptable phenomenon was discovered: the control rods, at the beginning of their motion into the core, introduced positive reactivity. The inspector of Gosatomnadzor noted this phenomenon -- and nevertheless allowed the reactor into operation. Gosatomnadzor formed part of the Ministry of Medium Machine Building. The organizations that had created the reactor likewise belonged to that ministry. The dependence of the supervisory body is obvious. Its successor, Gosatomenergonadzor, later became a formally independent committee -- yet it, too, did nothing. And still, it must be said that precisely the supervisory body, now Gosatomenergonadzor, was the first among the organizations which, after five years, began an objective investigation. At the very least, it refused to join in the false accusation of the staff.

In the Soviet Union, accidents do not occur because of poor equipment. Open any newspaper report about accidents in any field. The guilty party is the foreman, the dispatcher. In major accidents, in catastrophes, the guilty party is the captain, the director. This is not like ``over there, beyond the border.'' There it is different.

Read of the tragedy in Bhopal -- an explosion at a chemical plant. The guilty party is the firm that supplied substandard equipment. A closer example: the accident at the American TMI (Three Mile Island) nuclear plant. Scientist Aleksandrov says in \textit{Pravda}: \textit{``The TMI accident could only occur in the capitalist world, where safety is replaced by profit.''} According to our newspapers: in their cases accidents occur only because of bad equipment; in ours -- always because of bad operators. Neither is true.

I have long suspected that equipment is a non-party, and once, in its creation -- as A.~P.~Aleksandrov did in creating the RBMK -- one neglects elementary natural laws, the equipment simply refuses to work.

Powerful ideological conditioning by the mass media has always showered blame upon operators -- either wholly innocent, or guilty to a degree far smaller than that imputed to them. By operators I mean the operating personnel regardless of job title.

A correspondent of \textit{Literaturnaya Gazeta}, Bocharov, writing about an aircraft incident where pilots long failed to control the plane owing to loss of consciousness, says that in our \textit{``fast-paced time''} nearly all accidents occur through operator error. Whence such certainty? For understandable reasons, in recent years I have taken an interest in accidents and have reached a different conclusion. Accidents occur, in the majority of cases, because of the approaches of designers and planners, founded upon centuries-old traditions; because of senior officials who force facilities into service unfinished, even when the design itself is sound. Given the concentration of energy in modern equipment -- energy which humanity, by wisdom or folly, has confined within it -- the technical solutions must likewise be modern.

The RBMK reactor can develop power wholly unknown, unimaginable in magnitude. But even that is not the worst; far more terrible is the accumulated radioactive filth. To bridle such an uncontrollable reactor is scarcely a feasible task. One can only prevent the runaway. And that is the direct task and obligation of the designers -- an obligation they did not fulfil, although it stands written in the normative documents.

Now consider the explosion in Sverdlovsk\footnote{In 1988, a runaway freight train carrying tons of high explosives slammed into a coal train at a switching yard in Siberia and exploded Tuesday, killing four people and injuring \num{280} others. The blast in the city of Sverdlovsk left a bomb crater nearly \num{200} feet across and \num{32} feet deep, the newspaper Izvestia said, and triggered a fire and a second explosion at an oil storage tank in the yard. More than \num{30} homes and other buildings were damaged by the twin blasts. Izvestia said the entire town was engulfed in smoke and that \num{12} apartment houses and \num{22} other buildings were badly damaged. More than \num{850} families were evacuated from damaged buildings to emergency shelters. Izvestia said the rail yard dispatcher ``had made a serious mistake,'' and it blamed her for the fatal collision.}. And consider the questions:

\begin{itemize}
\item What sort of track arrangement is this, when a train on one line can be crossed by another?
\item Why are tens of tons of explosive carried through a densely populated city in an ordinary wagon by an ordinary train?
\item Why is it transported at all in an explosive state, when by moistening it can be rendered safe?
\end{itemize}

The issue is not whether the dispatcher \textit{``has the right''} to make a mistake. They have erred and will err. This is not a station of the nineteenth century where one went out and shifted the switch, pondered, checked again. How many commands does a dispatcher give in a shift? And in a month, a year? How many dispatchers are there? Therefore mistakes that can lead to grave consequences must be blocked by the designers.

The tragedy of the steamship \textit{Admiral Nakhimov}\footnote{An accident between two passenger ships in the Black Sea in 1986 that resulted in the deaths of over \num{500} people.}. No one removes guilt from the captains for the collision. But the number of those killed is not upon their conscience. Yes, in a collision there might have been casualties -- but not half a thousand. The shore was near, the sea warm and calm, help at hand. With such favourable factors and yet such a tragedy! Ships of that type are forbidden to operate. Why was the vessel pushed out to sea? The prosecutor says that the Register permitted it. And on that the matter ends. Why did it permit it -- and in whose interest?

And the explosion near Ufa\footnote{Deadliest train accident of the Soviet Union. On June 4, 1989, 01:15 a.m., two passenger trains of the Kuybyshev Railway carrying approximately \num{1300} vacationers to and from Novosibirsk and a resort in Adler on the Black Sea exploded, \num{11} kilometres from the town of Asha, Chelyabinsk Oblast. Without anyone knowing, a faulty gas pipeline \num{900} metres (\num{3000} feet) from the line had leaked natural gas liquids (mainly propane and butane), and weather conditions allowed the gas to accumulate across the lowlands, creating a flammable cloud along part of the Kuybyshev Railway. The explosion occurred after sparks from the overhead wiring feeding the locomotives of the two passenger trains, or wheel sparks ignited this flammable cloud. Estimates of the size of the explosion have ranged from \num{250} to \num{300} tons TNT equivalent to up to \num{10} kilotons TNT equivalent.} -- whether of trains or of the pipeline, I do not know how best to say it. An entire pipeline of the most explosive product -- coupled with practices dating back to the nineteenth century. What are valves every five kilometres worth without automation?

The chairman of the commission that investigated the Ufa explosion was G.~Vedernikov, likewise a Deputy Chairman of the Council of Ministers, as B.~E.~Shcherbina was. According to G.~Vedernikov, correspondent N.~Krivomazov writes in \textit{Pravda} (09 June, 1989):

\begin{personal}[Quote from Pravda (1989):]
\textit{``At various times I have had to head more than one such commission, and I have seen that despite the differences among our misfortunes and accidents, they have one indistinguishable similarity. Let us recall that at Chernobyl there existed four whole `idiot-proofing' systems -- and all four somehow managed to be disabled\ldots''}
\end{personal}

Of course, everything allegedly proceeded exactly so: we disabled one system, then the second, and thought -- well, we shall not explode. Let us disable everything! Only there was not a single \textit{``idiot-proofing''} system at the Chernobyl reactor -- though there ought to have been.

Twice I addressed letters to \textit{Pravda}, and -- as the saying goes --received not so much as a word in return. I wrote where my words could be checked. But in our country one may freely smear people and then assume an air of innocence. Here they speak of \textit{``idiot-proofing,''} while Doctor of Physical and Mathematical Sciences O.~Kazachkovskiy, again in \textit{Pravda} (15 October, 1989), says:

\begin{personal}[Quote from Pravda (1989):]
\textit{``\ldots the Chernobyl reactor at the moment of the accident turned out to be insufficiently `professional-proof,' and `professionals' were found who set about such a risky experiment with great self-confidence.''}
\end{personal}

It is doubtful that in 1989 O.~Kazachkovskiy did not know that the experiment and the accident were unrelated. In the same article he says:

\begin{personal}[Quote from Pravda (1989):]
\textit{``Existing reactors possess, to one degree or another, internal stability, ensured by negative feedbacks in reactivity. These feedbacks can be improved by perfecting the reactor physics.''}
\end{personal}

Golden words, although of course not new. And again, it is doubtful that in 1989 O.~Kazachkovskiy did not know of the existence in the RBMK reactor of a positive fast (and full) power coefficient of reactivity. And yet he does not mention this in the article. Apparently he does not wish to. Many other newspaper pieces likewise appeared. And all strove to pour additional mud upon the operating staff of the Chernobyl Nuclear Power Plant.

Here are the \textit{Izvestiya} of 11 February, 1990: again a Doctor of Physical and Mathematical Sciences, Deputy Chairman of Gosatomenergonadzor, V.~A.~Sidorenko, writes:

\begin{personal}[Quote from \textit{Izvestiya} (1990):]
\textit{``Indeed, many IAEA experts believe that hardly any reactor in the world could withstand such incompetent operation as was carried out at Unit~4 in Chernobyl.''}
\end{personal}

If O.~Kazachkovskiy might not have known all circumstances exactly, then V.~A.~Sidorenko knows well both the RBMK reactor and the circumstances and causes of the accident, and how foreign specialists were \textit{``informed.''} He himself, even before the accident, wrote to the creators of the reactor what a \textit{``fine''} reactor it was -- only lacked the principle and courage to halt its operation, though he had both grounds and authority. But already there were reports and studies with evident doubts as to the correctness of the official version -- or wholly rejecting it -- of which V.~A.~Sidorenko knew. Newspapers were already carrying notes about a reconsideration of the causes of the Chernobyl catastrophe. A barefaced accusation of the staff no longer passed. Hence the need arose for the authority of foreign scholars. See how many there were -- commissions and individual accusers -- and all blew in one direction. Titles beginning with ``Scientist, Doctor,'' positions beginning with ``Deputy Chairman of the Council of Ministers.'' How could one not believe? And indeed they convinced. Even the operating personnel themselves believed, for a time. And no wonder: the accident materials were sealed to all. And only gradually, as operators grasped the measures being implemented on the remaining RBMK units, did they begin to understand upon what powder keg they had been sitting -- or, more precisely, upon which they had been held for a long time.

When at trial the witness, shift supervisor I.~Kazachkov, said that on Units~1 and~2 at the station the reactor modernization had been carried out, and not yet on Unit~3 -- the judge told him: ``Well, you see, no modernization is being done.'' To which Kazachkov replied: ``Unit 3 is not yet in operation. And if by the time of startup it has not been done, then it will start up without me.''

I am one hundred percent certain that if today the reactors were to be returned to their former state, not a single operator would come to work tomorrow.

When the trouble lands on your own doorstep, you start to see things differently. Earlier I regarded the appointment of a Government Commission to investigate the causes of an accident with satisfaction -- until I encountered it myself. And then I understood: in our conditions, at least at that time, the appointment of a Government Commission is a direct road to concealing the truth.

% --

The rule of inserting politics into everything -- even where it had never lain -- turns all things upside down. A high ranking leader of the commission, in the rank, for example, of Deputy Chairman of the Council of Ministers, is practically responsible neither before the people nor before the law. Whether he acts rightly or wrongly -- no punishment will follow. And if there is no threat of punishment to the chairman, there is none to the members either: sheltered by his broad back, they, too, will sign whatever is required without fear. Since the commission head is, in one way or another, implicated if the accident was caused by equipment, then equipment will not appear in the findings -- or, at any rate, will not be made public. How could B.~E.~Shcherbina be ``interested'' in concealing the true causes? Simple: he oversaw that breach. Would he wish to hear reproaches, perhaps be removed from his post -- even if only to some other position not so pleasant? And thus the secrecy unfolds -- for both causes and consequences of the accident.

In general I am persuaded that a commission ought not to have a single dominant person -- neither with the authority of a specialist nor with the authority of power. In the first commission there were two deputy ministers -- and two documents appeared: the investigation act signed by A.~G.~Meshkov, and the ``Supplement'' to the act, which in fact constituted an independent act signed by G.~A.~Shasharin. From the very beginning everything might have proceeded in a normal fashion, but other forces and alien considerations intervened.

Nor can I count myself among \textit{``carp-idealists''}\footnote{In Russian, the verb \textit{карпить} means to nag, carp, or fuss; a \textit{карпящий идеалист} is someone who frets idealistically, insisting that truth and free ideas will inevitably prevail if only one keeps scolding the world long enough. Thus the satire means: \textit{I am not one of those fussy idealists who believe that truth will automatically win out; I abandoned faith in the universal victory of free ideas long ago.}}; I lost faith in the universal triumph of free ideas long ago. In the years of my formation, from fourteen to twenty two, I lived in the Arctic city of Norilsk. Those who lived there at that time know that the people were of every sort -- some of high moral quality, others the vilest scoundrels. In such an environment it is difficult to preserve illusions.

% --

For example, acquaintance with political prisoners sufficed to dispel -- by the age of twenty -- any belief in the sanctity of Comrade Stalin\footnote{As a young man, Dyatlov met people imprisoned for political reasons (in Norilsk, one of the major GULAG complexes). Hearing their stories was enough to destroy any belief in Stalin's infallibility or moral authority.}. And it is worth noting that those northern acquaintances proved very steadfast. After the accident, when I lay in the hospital, many of them came to see me, even those with whom I had almost lost contact after leaving Norilsk in 1953: M.~I.~Medvedkov, the Korneichuks, and others.

Having lost my illusions, I did not become either a nihilist or a cynic. I learned to defend my own opinion and human dignity with firmness. I accepted people as they are -- with their virtues and their failings. I cannot endure lies -- I regard them as the gravest of masculine vices. And it was precisely with falsehood that I had to deal in abundance after the accident. This was the greatest shock of all, especially when old men indulge in lying. For such men, like those stricken with cancer, standing already upon the threshold of the \textit{afterlife,} ought to speak only the truth, and nothing but the truth.

But our old men are not of that kind. Do they think to live twice? Scientist Petrosyants spent some fifteen minutes on All-Union television discoursing on how rotten the personnel at the Chernobyl plant were -- how they had shut down the emergency core cooling system, otherwise\ldots It is bad enough when a correspondent broadcasts falsehoods from secondhand; the man has been deceived, trusted the wrong source. But Scientist Petrosyants has a book on the RBMK -- whether he wrote it himself or had it written for him, he certainly read it. That is to say, he knew that this system could not have helped. Although he may be understood in one respect: the scientist was defending his position. Chairman of the State Committee for the Utilization of Atomic Energy is a ministerial post, with corresponding salary and rations. And no responsibility -- only honour. A sinecure. Contacts abroad. As Vysotsky\footnote{Vladimir Vysotsky, Soviet song-writer, singer and poet.} said:
\begin{quote}
``Perhaps they'll tell you: drink and eat---\\
Or perhaps: not a thing.''\\[4pt]
But here, look:\\[2pt]
``Constantly drink and eat---\\
And they won't say a thing.''
\end{quote}
For such a post one would slander even one's own mother. Though in his case it did not succeed. Well, after all\ldots

Here is an example of a different sort. Scientist L.~A.~Buldakov writes in \textit{Smena} no.~24 for 1989:
\begin{personal}[From the Statement of L.~A.~Buldakov:]
\textit{``First of all, there was no belated evacuation--rather, there was a timely evacuation from the city of Pripyat.''}
\end{personal}

% --

I draw attention to the date: this is written at the end of 1989. The evacuation of the inhabitants of Pripyat began at 14:00 on 27~April. By 10:00 a.m. on 26 April it was already clear, from dosimetric measurements, that long-term residence in the city was impossible. Even if one disregards the possibility of an increase in dose rate in the following days -- though there were no grounds for such hope, since releases continued -- the necessity of evacuation was obvious. It ought to have begun a full 24 hours earlier. Did the ten or so rem\footnote{CGS unit of equivalent dose, effective dose, and committed dose, which are dose measures used to estimate potential health effects of low levels of ionizing radiation on the human body.} received by every inhabitant during those 24 hours do them any good? Not being a specialist in radiation medicine, I shall not dissect the entire article. But the scientist's assertion seems to me quite arbitrary:

\begin{personal}[From the Statement of L.~A.~Buldakov:]
\textit{``An analysis of global experience regarding the influence of radiation on the human organism shows that the minimally significant dose, under prolonged exposure, is \qtyrange{100}{250}{\rem}.''}
\end{personal}

Radiation safety norms state only that \qty{25}{\rem} is a dose at which modern medicine does not detect changes in the organism. They say nothing of its harmlessness. Here, I confess, I have no firm conviction. But in the matter of evacuation all is clear: it was delayed by a day. And, I believe, everyone will now assent to this. I doubt that a scientist of the Academy of Medical Sciences would have doubts; he, better than others, knows the principles of radiation protection of the population -- of minimizing individual and collective doses.

And once more mark this: L.~A.~Buldakov had no relation whatever to the occurrence of radioactive contamination; he bore not the slightest responsibility for it. Why, then, compromise his conscience? To whose service -- what cause -- does he yield the authority of a scholar? And he is not alone. Science has been violated -- turned into a servant. What, then, is to be said of men personally guilty in the birth of the catastrophe?

Dear Reader, I propose next to examine another Report -- with an attempt to separate it from the others -- because it was compiled significantly later than the rest, and because more organizations took part in it. We shall not concern ourselves with the measures aimed at improving the reliability of RBMK reactors: they are described correctly (unsurprising, given the competence of the authors), well justified, and in truth have greatly improved the physical characteristics of the core and the Emergency Protection (AZ) of the reactor.

One must not be surprised that the same knowledgeable specialists -- even after five years -- continue to assert that the RBMK-1000 reactor, in the state it had in 1986, was sound, that the control and protection system (CPS) met the requirements imposed upon it. They make this assertion in defiance of the facts, in defiance of their own statements -- including those contained in this very Report. Evidently, without such declarations the leadership would not have signed it.

\chapter{Enviable Resilience}

% --

In 1991 there appeared a report entitled \textit{``Causes and Circumstances of the Accident at Unit~4 of the Chernobyl Nuclear Power Plant. Measures to Improve the Safety of Nuclear Power Plants with RBMK Reactors,''} bearing the signatures of the director of the Kurchatov Institute, E.~P.~Velikhov; the general director of NPO ``Energia'' (VNIIAES), A.~A.~Abagyan; the director of NIKIET, E.~O.~Adamov; the director of the Institute for Problems of the Safe Development of Atomic Energy of the USSR Academy of Sciences, L.~A.~Bolshov; the chief specialist of the The State Committee for Science and Technology, E.~I.~Chukardin; and the director of the Scientific and Technical Center of Gospromatomenergonadzor, V.~A.~Petrov. The report was drawn up by staff of these organizations; and since these comprise, essentially, all the institutions concerned with RBMK reactors, this document is evidently to be regarded as the final word. Nothing further, it seems, may be expected from them.

Naturally, the compilers of the report have laid no blame upon themselves. It would be in vain to seek in their pages any mention of violations, in the design, of the requirements of the regulatory documents. In their view there are none; no deviations. And the documents themselves ``do not exist''; they ``do not know'' them. In their eyes, RBMK reactor installations possess merely certain ``features'':

\begin{itemize}
\item \textit{``insufficient automatic protection of the reactor installation from being brought into a non-regulatory state.''}

That is to say, the reactor may be in an explosion-hazardous condition, and the control and automation system does not so much as bat an eye. There is neither warning signal nor automatic actuation of AZ to prevent the reactor from entering a dangerous state -- including with regard to \textit{``an important physical characteristic from the standpoint of control and safety of the reactor, called the operational reactivity margin.''} Such is the \textit{feature}. I beg you to note the cynicism of the formulation: \textit{``\ldots from being brought into a non-regulatory state.''} The operating personnel, it would appear, do nothing but dream of how to bring the reactor into an explosion-hazardous condition; they have no other occupation. In the selfsame report the compilers, for reasons not altogether clear, quote information from the Regulations: that at nominal power in steady-state operation the reactivity margin must be \numrange{26}{30} rods, and that when the margin falls to \num{15} rods the reactor must be shut down immediately. Thus, a mere \num{15} rods separate the reactor in a normal state from the reactor as atomic bomb, although the reactivity effects in regime changes may amount to tens of rods. Hence it is the reactor itself that turns into a bomb, rather than being somehow \textit{``brought''} into one.

Only a clear system of measurement, signalling, and automation could have prevented such a transformation, as required by the OPB. But the most essential thing the compilers leave unsaid: the reactor must not become \textit{nuclear-dangerous} when the reactivity margin is reduced. Reactors with such a property do not exist; it was solely the bungling of the physicists and of the designers of the control rod system that produced it here.

\item \textit{``the nature of the variation of the steam-void reactivity coefficient and the dehydration effect depending on the decrease of coolant density in the core.''}

Of course, the compilers are all scientific workers and under no obligation to express themselves clearly, but even with that indulgence the phrase is remarkably ``particular.'' Its sense, however, is the same: that the reactor, with the core composition it had, possessed an impermissibly large positive steam reactivity effect. We have already spoken of this.

\item \textit{``insufficient speed of response of AZ and the possibility of introducing positive reactivity.''}

What a \textit{``feature''}! Emergency Protection which, upon actuation, introduces positive reactivity -- that is, \textit{accelerates} the reactor. With such \textit{``features''} it became quite natural that the RBMK reactor acquired yet another feature: that of exploding from time to time.
\end{itemize}

The compilers of the report discern no violations of the requirements of the national regulatory documents in the reactor design, but they do remark -- \textit{``The reactor control and protection system is based on the movement of \num{211} solid absorber rods in specially allocated channels cooled by water of an autonomous circuit. In regulatory regimes and in the conditions of a design-basis accident, the system ensured \ldots''} and follows a list of the usual requirements imposed upon such a system, including shutdown of the reactor. Further: \textit{``These characteristics of the reactor installation together with the safety systems -- protective, localization, and support systems -- ensured reliable and effective operation of the RBMK in all regulatory regimes and safety for the entire range of design-basis accidents in accordance with the approved design documentation.''}

What follows from these two excerpts? That the Control and Protection system and the other systems are normal, workable, and that all modes of reactor operation, including accident modes, were ensured in exemplary fashion. The RBMK reactor \textit{``did not explode''} -- \textit{``Churchill invented all that in eighteen''}\footnote{The construction \textit{``Churchill invented all that in eighteen''} alludes to a stock Soviet propaganda trope: when something was politically inconvenient, it was dismissed as a Western invention, sometimes specifically attributed to Churchill, supposedly dreamed up ``back in eighteen-something.'' The exact date is intentionally vague and absurd; the point is the reflexive blaming of the West for any unpleasant truth.}; there was no explosion.

% --

That the reactor exploded under a pump rundown follows from the act of the Meshkov commission, upon which NIKIET and the IAE -- the reactor's creators -- set their stamp. Later they abandoned this version, not because an explosion was impossible, but because no pump rundown occurred; and that, according to the Kurchatov Institute's own report, the reactor could explode in the event of AR failure -- of this the compilers are silent. Yet such situations are many; these alone would suffice. Perhaps they do not classify such accidents as ``design-basis''? They ought not. Pump rundown is entirely conceivable, and in various circumstances; AR failure all the more so, and is examined in every textbook on automatic reactor control. There is also the conclusion of the GKNT commission that with such a steam reactivity effect the reactor explodes under a maximum design-basis accident. This last cannot be classed as ``non-design-basis'' by any stretch.

Nevertheless, the compilers of the report do not think what they write. Here the old policy continues: defense of the honour of the uniform, contempt for people. After analyzing and enumerating the measures undertaken at the remaining reactors, they write:

\begin{personal}[From the 1991 Report:]
\textit{``Implementation of the scheduled measures to improve the neutron-physical characteristics of the reactor and the sharp increase in the effectiveness of AZ made it possible to eliminate uncontrolled power growth in accidents with loss of coolant and to limit the consequences of all design-basis accidents to acceptable levels of radiation impact on personnel, the population, and the environment.''}
\end{personal}

Whether this assertion is indeed indisputable -- whether it truly ensures the safety of the RBMK reactor -- is not altogether obvious. But with this phrase the authors neatly strike out the two preceding ones I have cited. There is no doubt that the technical measures carried out after the accident increased the reliability of the reactors; more accurately, they made it possible at last to pose the question of the reliability of RBMK reactors. What existed before 1986, in view of the numerous departures from the requirements of the regulatory documents for reactor design, cannot be called a reactor at all, still less can one speak of its \textit{``reliability.''} If, by force of habit, one continues to call the 1986 RBMK a \textit{``reactor,''} then the RBMK-1000 of today is something entirely different. Yes, there will be no accidents for the reason that obtained in 1986, nor for many other reasons. Yet the inventors and designers of the reactor so deeply violated, at its foundation, the very concept of safety that to attain the level already realized in other designs is scarcely possible by any modernization. It is unpleasant to say so, but most likely it is the case: only the grave will straighten the hunchback.

% --

Through the means of an example let us take the case of rupture of a technological channel. Almost all reactors (the exception being Units~3 and~4 of the Smolensk Nuclear Power Plant) are designed for the simultaneous rupture of two channels, no more. Rupture of more than two channels leads, if not to another Chernobyl, then to a quite comparable accident. According to NIKIET's computations, the simultaneous rupture of two channels is possible with a probability of \num{e-8} events per reactor per year. This is a small probability, and rupture of three or more channels is even less likely -- one may call it hypothetical. Only there is one ``but.'' The computation must not be deceitful. I do not question the conscientiousness of the calculator; one must assume he performed it using all available knowledge and mathematical apparatus, in accordance with the assignment. But the assignment itself is the point at issue. For, beyond channel ruptures conditioned by manufacturing technology, inspection, and operating conditions (environment, temperature, pressure, cycling), there exist other, far less tractable events (for instance, local overheating of the core, disturbance of circulation). Have all these been included?

In the three excerpts from the report I underlined the words \textit{``design-basis accident,''} not because they bear any special sense in context, but for another reason: it is not altogether clear why the authors press this term with such persistence. Chance is out of the question. Do they wish to suggest that the accident of 26~April was ``non-design-basis'' and that therefore ``we bear no responsibility''? Yes, the accident was non-design-basis. In my view, one ought not even to conceive such an accident in design, save under purely hypothetical conditions. It must be excluded by the reactor design and by the plant design; and our regulatory documents correspond to this requirement. The reactor, by virtue of its design, did not meet them; thus the accident occurred. And have the reactor's creators and the supervisory organization nothing to do with it? The accident is \textit{``non-design-basis''}? No: it is conditioned \textit{precisely} by the design.

This encroachment by the reactor's creators threatens serious consequences for the future, if permitted to take root. Through the back door the compilers insinuate that operation of the reactor at low power was prohibited. They give:

\begin{itemize}
\item RBMK-1000 power units operate in base-load mode (at constant power);
\item The prompt power coefficient at the operating point -- and they give a negative value.
\end{itemize}

They strive by every means to create the impression that the reactor was \textit{``good.''} One might ignore this, since in the report there are also sound and correct notions (indeed, all ought so to be: the compilers -- \num{23} persons -- are fully competent, and the signatories outright luminaries). But something fetters them, and even after five years continues to fetter them. And this cannot simply be brushed aside, for it is precisely upon these people -- if not upon them personally, then upon their institutions -- that the future of nuclear power depends.

Quite in keeping with the authors' double-minded position is the conclusion of the report:

\begin{personal}[From the 1991 Report:]
\textit{``The accident occurred as a result of the superposition of the following main factors: the physical characteristics of the reactor, features of the design of the control rods, and bringing the reactor into a non-regulatory state.''}
\end{personal}

We have, I think, already worn our teeth down upon these \textit{``features.''} As for the \textit{``non-regulatory state,''} the accusing finger is once more directed at the operating personnel. But any deviation of any parameter beyond its norm is a non-regulatory state; on 26~April it was the reactivity margin. And what then -- is the reactor to explode whenever a parameter deviates? In that case all reactors ought to explode, for there is none in which parameters never depart from nominal values. Yet their design and protection are such that the chain reaction is terminated without unacceptable infringements.

\begin{personal}[From the 1991 Report:]
\textit{``The appearance of new modern codes, the use of powerful computing equipment, and also experimental study of dehydration in the RBMK made it possible to refine the main physical parameters of the reactor and, consequently, to develop new requirements for systems increasing its safety.''}
\end{personal}

This assertion is perhaps five percent true -- in the trivial sense that study and refinement must in any case continue so long as the reactor remains in operation. It has nothing to do with the accident; everything essential was known to them long beforehand: the steam--void effect (\textit{``This is not so. The steam effect was not known,''} notes Dr.\ Tech.\ Sci.\ Ya.~V.~Shevelev (Kurchatov Institute, 1992)), and Emergency Protection, and the rod design. Nor is there a single ``new requirement'' in the modernization measures adopted. That the steam-void effect must not exceed a given value -- this was decided already in 1976, and the way to achieve it was indicated; these very paths were followed after the accident. A fast-acting shutdown system with film cooling was developed no later than 1973. That one must not design reactivity-control rods which change the sign of the introduced reactivity while moving in a single direction -- these are the basics. In truth, the accident revealed not one unknown factor, not a single new requirement; all that has been done since is directed solely to fulfilling the OPB and NSR, adopted and put into force more than ten years before the catastrophe.

% --

Such, then, is the final report of the organizations responsible for the RBMK reactor, together with the supervisory body. Highly instructive is the position of the representative of the supervisory body, V.~Petrov: in this report he has signed under statements that the RBMK reactor with all its systems, including the Control and Protection system, ``ensured'' safe operation; and at practically the same time he signed another report in which it is indicated, and quite rightly, that this reactor did not comply with fifteen articles of the OPB and NSR directly influencing the occurrence of the accident of 26~April, 1986. It follows that, in the view of the State controller Mr.~Petrov, a reactor endowed with a bouquet of non-compliances with regulatory requirements is nevertheless sound and quite suitable for operation. Perhaps Mr.~Petrov, confronted with so many deviations in the 1986 RBMK design from the reactor-design norms, no longer considers it a reactor (and that would be fair), and is guided by some other documents (which?), and by intuition? But in the report the 1986 RBMK is called a reactor, and therefore ought to satisfy the OPB and NSR.

There \textit{is} a shift after five years. They have squeezed out of themselves the admission that the accident occurred because of the physical characteristics of the reactor, the features of the design of the control rods, and the bringing of the reactor into a non-regulatory state. Formerly these same people acknowledged as cause only an improbable combination of violations of instructions and a non-regulatory state. And since it is evident that on 26~April, 1986 no accident would have occurred even:
\begin{itemize}
\item with a steam reactivity effect equal to \qty{6}{\beta};
\item with a positive prompt power coefficient of reactivity over a wide power range of the reactor;
\item even if Emergency Protection had failed to meet the requirements imposed upon it, provided only that it did not introduce positive reactivity.
\end{itemize}

One may well ask how many years these people will need for unconditional recognition that the cause of the accident lies \textit{exclusively} in the properties of the reactor. I write this only because these people remain in active nuclear power, they are steering it, and, as you see, their candour is not to be relied upon.

\textit{Note.} Later V.~Petrov withdrew his signature.

\chapter{Dramatis Personae}

\section{Scientist A.~P.~Alexandrov}

Beyond question, he deserves separate consideration. In 1986 he was the President of the USSR Academy of Sciences, director of the Kurchatov Institute, Scientific Supervisor of the RBMK programme, the inventor of the RBMK reactor\footnote{Strictly speaking, the inventor of the RBMK was not A.~P.~Alexandrov, but S.~M.~Feinberg.}. What other offices and dignities he then held I do not know; but in those I have named he stood in direct relation to the catastrophe of 26~April 1986.

It may be that I shall show myself biased toward him, for I am persuaded that it was precisely through the agency of Scientist A.~P.~Alexandrov that I lost my health and became a convict. As inventor of the reactor and as Scientific Supervisor he did not ensure the necessary quality; as President, as director of the institute, as presiding officer at the sessions of the Interdepartmental Technical Council, he directed the investigation upon a false track. Even so, I shall not allow myself inventions: I shall set forth only the facts known to me, and my understanding of them.

Some twenty years ago, while still working in Komsomolsk-on-Amur, I came to the Kurchatov Institute on a business trip. Staff of the institute, E.~Alikin and others, told me that the director had a rule: any paper must first \textit{``lie a while,''} and if it roused scandalous notoriety, then it was needed and the time had come to give it course. I did not then imagine that this habit of A.~P.~Alexandrov would roll like a heavy road-roller over the fates of many people, mine among them.

Apparently following that rule of his, A.~P.~Alexandrov did not give progress to the proposals of the commission after the accident at Unit~1 of the Leningrad Nuclear Power Plant -- the proposals of V.~P.~Volkov and V.~L.~Ivanov. These proposals, and the decisions of the Scientific and Technical Council of the Ministry of Medium Machine Building to reduce the steam reactivity effect to \qty{1}{\beta}, lay, like others, under the blotter throughout the 1970s. Were they waiting for a \textit{``scandal''}? They waited\ldots

Tell me, who would have wished to prevent a scientist from putting all this into practice, and who could have done so? There are no such people. He himself need not even have ``done'' anything: only to permit, to order. And just as before 1986 the world had heard nothing of Chernobyl, so it would have heard nothing of it to this very day.

I shall quote from the report of A.~A.~Yadrikhinsky:

\begin{personal}[From the Statement of A.~A.~Yadrikhinsky:]
\textit{``In the IAE itself since 1965 there were employees I.~F.~Zhezherun, V.~P.~Volkov, and V.~L.~Ivanov, who pointed to the nuclear hazard of the proposed and subsequently implemented RBMK design. Their actions were successfully blocked by scientist A.~P.~Alexandrov, and their memoranda were put `under the blotter'.}

\textit{The startup and operation of Unit~1 of the Leningrad Nuclear Power Plant in 1975 had already in practice confirmed the nuclear hazard of RBMK reactors. If before the startup of Unit~1 of the Leningrad Nuclear Power Plant the violations of the Rules in the RBMK design could be regarded as mistakes, then after startup and their experimental confirmation by the operating experience of Unit~1 of the Leningrad Nuclear Power Plant, and their subsequent replication in all newly commissioned RBMK units, they can only be called crimes.''}
\end{personal}

Later, in operation, there were other manifestations of the reactor's unsatisfactory, dangerous qualities, of which A.~P.~Alexandrov was informed.

% --

There can be no doubt that, knowing all this even before the catastrophe of 26~April, he clearly understood, once it had occurred, that the accident was, in pure form, the result of scientific and design miscalculations. And then, in order to save his own reputation (who would have judged him -- a thrice Hero of the Soviet Union, with eight Orders of Lenin?), he set in motion all levers to shift the blame exclusively onto the operating staff. This was not in the least difficult for him, since the investigation lay in the hands of the Ministry of Medium Machine Building, the Kurchatov Institute, and NIKIET. And such a conclusion poured balm into all hearts, up to the very top.

And mark with what tenacity the scientist defends his line! After five years there is not the slightest deviation. In the partisans he would have been priceless. In \textit{Ogonyok} no.~35 for 1990 his interview was printed -- highly characteristic. To the correspondent, A.~P.~Alexandrov answers that he did not belong to the commission investigating the causes of the accident. Formally, yes: he is not the sort of scientist to thrust himself into affairs from which one may expect nothing but rebuffs. In fact he constantly followed and directed matters wherever he found it needful.

Further, his words:

\begin{personal}[From the Interview with A.~P.~Alexandrov:]
\textit{``Understand, the reactor has shortcomings. It was created long ago by Scientist Dollezhal, taking into account the knowledge of that time. Now these shortcomings are reduced, compensated. The matter is not in the design. You drive a car, you turn the steering wheel the wrong way -- an accident! Is the engine to blame? Or the car's designer? Everyone will answer: `The unqualified driver is at fault.'\,'' }
\end{personal}

A very characteristic pronouncement. It is astonishing how much falsehood can be packed into some two dozen words. Let us sort out what is what.

First. The reactor \textit{``has shortcomings.''} No: it had impermissible defects, which excluded its operation -- direct violations of the regulatory documents adopted in the country. Of these the professor does not speak; like the commissions, he stubbornly refuses to see those documents. In the text above I have shown how Emergency Protection was \textit{``coordinated''} with the requirements of the Rules. Here is how Professor B.~G.~Dubovsky, who until 1973 headed the nuclear safety service in the USSR, says in an abstract:

\begin{personal}[From the Statement of B.~G.~Dubovsky:]
\textit{``It is beyond comprehension how the managers of the Control and Protection system designers, as well as of Gosatomenergonadzor of the USSR, could have allowed such major, and in some cases devoid of elementary logic, miscalculations.}

\textit{For in essence the RBMK-86 reactors (he has in mind the RBMK-1000 in the state they were in as of 1986 -- A.~D.) had no normal protection. They had no Emergency Protection at all! Neither from the bottom of the core, nor from the top.''}
\end{personal}

All this B.~G.~Dubovsky says after analyzing the protection system. Several years before the accident he had likewise made proposals for improving it. Their fate was the same: into the wastebasket. The Chief Designer himself, M.~A.~Dollezhal, admitted that a reactor with such a large positive reactivity effect is uncontrollable. He did not find it possible, at the end of his life, to lie. Can the creator of a reactor not know that it must meet standards? Does that accord with common sense?

Second. A.~P.~Alexandrov very deftly, one might say masterfully, shifts the arrow to M.~A.~Dollezhal. Of course the Chief Designer bears responsibility; but A.~P.~Alexandrov ought not to slip away from his share. It was not Dollezhal who received the money for the invention, but Alexandrov. Twice the patent applications were rejected by the Union Bureau (see \textit{Literaturnaya gazeta}, no.~20, 1989); then they pushed them through under departmental secrecy. In my view, the Union Bureau was wrong to reject them. There were obvious signs of an \textit{``invention''} here:

\begin{itemize}
\item all reactors are nuclear-hazardous only at large operational reactivity margin; the RBMK is hazardous both at large margin and at small;
\item a universal Emergency Protection: it both shuts the reactor down and accelerates it.
\end{itemize}

Moreover, A.~P.~Alexandrov held the official post of Scientific Supervisor of the RBMK topic. What the reactor was at its inception, and what it remained thereafter up to the accident, is thus directly his contribution.

Third. A.~P.~Alexandrov says that now these shortcomings have been reduced, compensated. That is correct. But he keeps silent that all these \textit{``shortcomings''} were known to him long before the Chernobyl accident. More on the professor's selective memory below.

Fourth. The professor lies by deception when he speaks of the car, the designer, and the driver. According to the objective testimony of the control system we pressed the AZ button in the absence of any emergency signals whatever. Were we entitled to expect a normal shutdown of the reactor? Without doubt. The protection system is obliged to perform this even \textit{in the presence} of emergency signals -- that is what Emergency Protection is for. Actuation of reactor protection by the operator can in no way be regarded as a violation of nuclear safety. Therefore we did not \textit{``turn the steering wheel the wrong way.''} A more plausible comparison with a car would be this: \textit{``You are driving a car, you press the brake. Instead of braking, the car accelerates. An accident! Is the driver to blame? Or the designer, Comrade Professor?''}

True to his academic nature, A.~P.~Alexandrov says:

\begin{personal}[From the Interview with A.~P.~Alexandrov:]
\textit{``Think about why the accident happened at Chernobyl, and not at Leningrad.''}
\end{personal}

That was all we lacked there!

The Leningrad Nuclear Power Plant in 1975 escaped a catastrophe only by chance; its immediate causes would have been different, but the scale comparable with Chernobyl. As a result of local overheating of the core, a technological channel broke open. In those circumstances three or four channels might well have failed simultaneously; in fact some twenty were replaced in the repair. As is now clear, simultaneous rupture of three or four channels led straight to a Chernobyl.

The scientist's memory works in a strictly selective mode. He remembers the turbine, the valves; but as to the reactor -- though this lies squarely in the subject of the conversation -- A.~P.~Alexandrov ``remembers'' no accident. His memory equally fails him when he speaks of the fleet, of submarines:

\begin{personal}[From the Interview with A.~P.~Alexandrov:]
\textit{``In naval installations we had no mishaps, although in 1957 industry was less developed; but still, we managed.''}
\end{personal}

At Chernobyl industry also ``managed,'' yes\ldots

I too do not recall, prior to 1973, accidents on shipboard pressurised-water reactors due to manufacturing defects in the equipment. But on account of scientific support -- or rather, the lack of it -- to which the scientist stood in direct relation, there were such accidents.

If memory serves, in 1962 on a nuclear submarine an impulse tube of \qty{10}{\milli\metre} diameter broke off a primary-circuit pipeline. The tube was small, and the primary circuit itself small in volume. The crew tried to take measures so that the core should not be left without cooling. The men understood that they were being over-irradiated, and they acted. Why? Because they feared that:

\begin{itemize}
\item the highly enriched fuel, after melting, would gather into a compact mass and a nuclear explosion would occur;
\item if an explosion did not occur, the bottom of the vessel and the hull of the submarine might melt through.
\end{itemize}

After the accident the institute carried out a calculation which showed that neither an explosion nor melting-through of the hull would have occurred. The calculation was made \textit{after} the accident, not before. But the dead will not be brought back. Had the crew possessed this information, they would have sealed the compartment and returned to base; the reactor, in any case, was doomed. And there would have been no casualties.

I knew the crews of many submarines. I can say nothing about those of today, but then the officers who serviced the nuclear plant were, in the overwhelming majority, competent specialists; and the chiefs of propulsion (commanders of Combat Section~Five) without exception. But a crew is not an institute collective; its possibilities and tasks are different. The crew fulfilled its duty with excess; the Scientific Supervisor, A.~P.~Alexandrov, fulfilled his belatedly.

% --

Another case. At one of the shipyards a temporary plug was left on a nozzle of the reactor head; during hydraulic testing it was torn off. Through this nozzle passed the guide rod of the grid that suppressed reactivity. When the water poured out, the flow lifted the grid and an explosion took place. The pressure raised the reactor head (it tore out the studs), flung water into the casing, and the reaction ceased. I do not recall whether there were human casualties; perhaps there were none.

They imprisoned the head of the plant's physics laboratory. Just like that. Two institutes -- the scientists and the designers -- had not played through the emergency situations; but the plant worker, it seems, was supposed to guess that, if the plug were torn off, the grid might be lifted.

How like Chernobyl this is. And again the main protagonist is the same. An accident occurred -- and immediately they computed that this Emergency Protection, in the first seconds, can introduce positive reactivity up to \qty{1}{\beta}. Not before the accident, but after!

There is no need to speak afresh of the removal of the ECCS and of the test programme, to refute inventions over and over. I do, however, wish to comment on the scientist's tirade intended to depict his horror and indignation:

% --

\begin{personal}[From the Interview with A.~P.~Alexandrov:]
\textit{``So then -- believe it or not! -- at the very beginning of the Regulation for that experiment was written: `Shut off the reactor emergency cooling system -- the ECCS system.' Yet it is precisely this system that automatically switches on the emergency protection system. What is more, all the valves were closed so that it would be impossible to switch on the protection system. Twelve times (!) the experiment's Regulation violates the operating instructions for the Nuclear Power Plant. One would not see such a thing even in a nightmare. For eleven hours the Nuclear Power Plant operated with the ECCS disconnected! As though the devil himself were directing and preparing this explosion.''}
\end{personal}

What verve, what pathos! And yet all this is play for the gallery. The professor knows perfectly well that removal of the ECCS did not in any way affect the occurrence of the accident. This is admitted by everyone, including his pupil V.~A.~Legasov, and even by the extremely tendentious judicial-technical commission. We disconnected the ECCS because, under the documents in force at that time, the chief engineer was permitted to do so, although one may agree that a categorical prohibition on removing the system would be proper, however small the probability that an MPA (Minimal Projected Accident - it was not an MPA on 26~April) would occur at that very moment.

The phrase, \textit{``Yet it is precisely this system that automatically switches on the emergency protection system,''} lies beyond my non-academic understanding. According to the OPB, the ECCS itself is a protection system and does not \textit{``switch anything on.''}

We already know how the \textit{``violations''} are wound up. Yes, the ECCS on Unit~4 was disconnected for eleven hours. A nightmare. Dear Anatoly Petrovich, forgive me a naive question: do you not dream nightmares about the fact that two units at each of the Leningrad, Kursk, and Chernobyl Nuclear Power Plants had been operating for years without ECCS? For what stands there now only very approximately meets the requirements.

And the Christian meekness of Professor A.~P.~Alexandrov -- \textit{``I am nobody's judge''} -- is a hypocritical posture that suits him no better than a sheepskin suits a wolf. What then is he doing in this interview? Making confession? That is not in evidence. As unjustly as he accused the staff before, so he continues. As they say, whichever way you fling him, he lands on his back. So it is with A.~P.~Alexandrov: whatever the article, it is a lie. In an issue of \textit{Izvestiya} from 14~October 1989 we read:

\begin{personal}[From the Interview with A.~P.~Alexandrov:]
\textit{``Earlier still the nuclear power era began -- the first nuclear power plant in the world was built (Chernobyl is the result of the `period of stagnation,' that is, of a period of universal irresponsibility).''}
\end{personal}

And here again he throws it upon the period of stagnation, upon the \textit{``system.''} What has the period of stagnation to do with it? Who hindered him, who could have eliminated the impermissible defects of the reactor? Above all, the irresponsibility was his own. If there was irresponsibility, then it was far from universal. Proposals existed, and necessary ones, long before the accident. Over the RBMK reactor there was no one above A.~P.~Alexandrov; all that he prescribed would have been carried out\footnote{A.~D.: Here I am mistaken in thinking that Alexandrov could do absolutely everything: what I.~Ya.~Emelyanov did not wish to do, Alexandrov could not compel.}. Another question is whether it is possible truly to manage when one holds a dozen--plus posts and positions.

The newspaper \textit{Pravda} slipped the professor a bitter pill by printing his speech in the Central Committee at the mass retirement of elderly members. There he said:

\begin{personal}[From the Speech of A.~P.~Alexandrov:]
\textit{``To direct such an institute as the Kurchatov Institute, the largest institute, with the most complex work, and at the same time take upon oneself care of the Academy -- it must be said that this was extraordinarily hard. In the end it ended sadly. And when the Chernobyl accident occurred, I consider that from that time my life began to end, and my creative life as well.''}
\end{personal}

Before that he recounted that he had been forced into the presidency of the Academy. Perhaps so. One ``could not'' refuse. But could he have resigned as institute director? Could he have relinquished his other posts? What can one do fruitfully in ten or fifteen fields at once? Only reap where one has neither ploughed nor sown.

\section{Scientist V.~A.~Legasov}

I agree that the rule, \textit{``of the dead speak well or not at all,''} should in general be observed. Yet I do not see here any great sin in departing from it, since the scientist himself did not trouble to observe it. The operators whom he accused by signing the conclusion of the Government Commission were already dead by that time. He did the same as head of the Soviet specialists -- the informers to the IAEA.

I do not wish to discuss V.~A.~Legasov's work in eliminating the consequences of the accident. Lately statements have appeared about the erroneousness of some technical decisions then taken. There has never been a shortage, nor is there now, of people strong in hindsight. Sitting in a comfortable office, over several years, one may indeed think out many useful things. But under extreme conditions, with very little time, decisions must be taken first and only then tested. Besides, why should incorrect decisions be placed to Legasov's account? Was he alone there? Velikhov was there. Or is it dangerous to criticise Velikhov? He is alive and in power.

By speciality, V.~A.~Legasov was not a reactor engineer; he did not know power units in detail and, I believe, in that whirlwind might not have worked everything out. By character he trusted others. But that in no way excuses or explains his signature under the conclusion of the Government Commission, nor his activity at the IAEA.

A man of broad erudition, he concerned himself with safety questions of production in general. For all the specificity of chemical, oil, or nuclear plants, safety issues have much in common.

No, Scientist V.~A.~Legasov could not fail to understand that to accuse the staff in such an explosion was unlawful. He could not fail to understand that if the reactor exploded under the most ordinary conditions, without any natural cataclysms, then it had no right to exist.

Reactors cannot, must not, explode with the release of enormous quantities of radioactive substances into the environment. With this understanding -- and he simply could not lack it -- he ought naturally to have come to the question: why, after all, did the explosion occur? Even without sifting, and not sifting at all, the errors of the staff.

From that question there ran a straight path to the following: did the reactor comply with the adopted nuclear safety norms? If it did, then did those norms themselves sufficiently satisfy the criterion of safety? These questions could not but arise. Any accident investigation is conducted with reference to operating and design documentation, to equipment passports. There is nothing novel in this. The very first steps in that direction would have revealed obvious non-compliances of the reactor with the NSR and OPB. There was no need to seek them anywhere; they stand already in the conclusion of the Government Commission itself, only without reference to the \textit{``Rules''}: they are termed \textit{``shortcomings.''}

I am therefore certain that the omission of the regulatory documents from the conclusion was deliberate. And the man first and foremost answerable for this is Scientist Legasov, together with the chairman of the commission, B.~E.~Shcherbina. It was not the Minister of Internal Affairs who was answerable in the commission for the technology.

V.~A.~Legasov bore no personal guilt for the RBMK reactor; he had had no relation whatever to its existence prior to the accident. With his signatures he covered the sins of others -- covered them consciously. It is therefore not so unexpected to read the letter of a senior research worker of the institute, cited by V.~Gubarev in \textit{Pravda}: \textit{``Legasov is a vivid representative of that scientific mafia whose politicking instead of directing science led to the Chernobyl accident\ldots''} Nor is it so unexpected that he was not elected to the institute's Scientific and Technical Council: \num{100} votes in favour, \num{129} against.

What did he stake his scientific authority upon, who pressured him? We shall not learn that. No, it was not world fame that his report in Vienna at the IAEA conference brought V.~A.~Legasov. And evidently he understood this. I should like to believe that the scientist erred through ignorance, that he did not comprehend the causes of the catastrophe, for it is too grievous to think the contrary.

Yet to force myself to accept this I cannot; it will not do. Elementary logic will not permit it. And V.~A.~Legasov's departure from life precisely on the anniversary of Chernobyl speaks the same language. I do not, however, rank him among the \textit{``mafia.''} This man had a conscience. Under certain circumstances he went to a cruel compromise with that conscience and could not bear it. The \textit{``selection''} continues. By one means or another, the last people who yet retain human qualities are being weeded out.

\section{Doctor A.~A.~Abagyan}

Director of VNIIAES -- his other posts do not concern us here.

He too belonged to the first commission investigating the causes of the Chernobyl accident. Together with Deputy Minister G.~A.~Shasharin he refused to sign the act, and as part of a group of Minenergo employees took part in drafting a much more realistic supplement to the investigation act. Once included in the group of Soviet specialists informing the \textit{``world community''} at the IAEA, he sharply changed his position. For what reason he did so -- whether because Shasharin had by then been removed from his post and was therefore no longer his superior, or because he had \textit{``seen the light''} and joined the majority -- I do not know. What I do not know, I do not assert.

I merely note that the opinion of Doctor A.~A.~Abagyan, in the space of two months, turned into its opposite without the appearance of any new investigative material. In that time there appeared only the report of the Government Commission, which contained no technical information that was new to A.~A.~Abagyan, and the decision of the Politburo, which of course contained no technical information at all.

I shall not compare the two documents in full; I shall cite only one concrete example.

\begin{itemize}
\item From the supplement to the investigation act, p.~8: \textit{``The removal from service of Emergency Protection on the signal of shutdown of two turbogenerators does not contradict the Technical Regulation and the Instructions, and actuation of this protection could not have prevented the accident; it would have occurred \qty{35}{\second} earlier.''}

\item From the report to the IAEA (table of violations committed by operating personnel): \textit{``Violation. Blocking of reactor protection on the signal of shutdown of two turbogenerators. Consequences. Loss of the possibility of automatic shutdown of the reactor.''}
\end{itemize}

Both documents bear the signature of A.~A.~Abagyan: the first in May, the second in July 1986. Let us suppose his technical opinion on this question changed, and he drew different conclusions. One might, with some effort, understand that. But how, Doctor Abagyan, are we to understand your two opinions on whether the blocking of the protection system was or was not a violation of the instructions? In the Regulation it is clearly stated when it is removed from service; there can be no double interpretation.

In a journal -- I think it was \textit{Nash sovremennik} -- the doctor reports that they answered questions from specialists and correspondents five hours a day. Little by little it becomes clear how they answered, how they \textit{``tried to present the staff to the world community''} (Abagyan's expression) -- \textit{``beautifully,'' ``objectively''} they presented it. Thank you indeed.

As to the protection system, we have just seen. Now as to reactor power:

\begin{itemize}
\item in the supplement to the act it is shown in detail that in no pre-accident document is there even a hint of limitations on reactor operation at any power level, \qty{200}{\mega\watt} included;
\item to the world community they reported, succinctly and clearly, that operation at power below \qty{700}{\mega\watt}W was prohibited by the Regulations.
\end{itemize}

A lie -- and what of it? Such \textit{``principled''} people were, and remain, engaged in the investigation. After all, a man is master of his word: he gives it, and he takes it back.

\chapter{Works of Fiction}

For a long time I read nothing at all about the accident: neither journals nor newspapers. In hospital, once I was able to read, Volodya Pchelin supplied me with the classics, and Pyotr Vyrodov with detective novels. It seems I read nothing else there; I practically did not watch television, although there was one in the ward. For a time V.~S.~Konviz was in the hospital; he offered me V.~Gubarev's \textit{Sarcophagus}, but I refused it.

The fact is that, in hospital, the investigator questioned me several times -- then still as a witness -- and I had already discerned the general direction of the investigation, the way in which it was rolling. Therefore I did not expect to encounter realistic evaluations of events in newspaper and journal publications.

I was discharged from hospital on 4~November 1986, and the next day my wife and I came to Kiev. I lived a month in freedom and, with the help of twice-daily walks, began to restore my coordination of movement and, in general, to come to myself little by little. All this came with difficulty. During the illness I had \textit{``eaten away''} fifteen kilograms and have not regained them to this day. I \textit{``ate away''} muscle -- there never was any fat on me -- and such loss cannot be restored by food.

Then, on 4 December, I was transferred to a cell. In a pr--trial detention centre one's ability to follow events is limited. And after the trial, shattered as I was, for a long time I could not read, not only periodicals, but even normal books. Yes, to speak frankly, even now I have no particular desire to read about the catastrophe, although I do read. By now there is no longer indiscriminate vilification of the staff, but every so often -- even well-wishers -- slip something in that makes one flinch.

For example: it is said that the staff regarded the reactor as simple and reliable as a cupboard; they were not warned, and therefore the staff violated the instructions. Gentlemen, we have no need of such \textit{``defence''} any more than we have need of such accusations. You might first have spoken with operators on RBMKs and on other reactors. No, they do not expect an explosion, such as at Chernobyl; that belongs to pathology. No, they do not expect Emergency Protection to introduce positive reactivity -- that cannot even be classified in any way. As a Ukrainian would say: \textit{bezgluzdya, nisenitnytsia} (nonsense, absurdity).

That is, operators do not expect dirty tricks, or traps, from designers. But even a normal reactor, executed in accordance with standards, threatens trouble if the operating rules are not observed, even though such trouble is in no wise comparable with Chernobyl. In any article you may stumble upon a slap in the face. One would think there had been so many that a man ought to have grown used to them. But no -- it always hurts.

Purely fictional works we shall leave aside; that is a matter for literary critics. We shall examine only two documentary narratives, from the standpoint of how documentary they truly are.

\section{G.~Medvedev. \textit{The Chernobyl Notebook}}

There would be no need to dwell upon this work had the author declared it a work of fiction. Instead, he presented it as documentary and retained the real surnames of the participants in the events. As I understand the genre, its laws permit only a slight subordination of the presentation of events and of persons' actions to the author's will or fancy. Accuracy must never be sacrificed for vividness. If one desires to write beautifully, artistically, without binding oneself to moral obligations toward the living and the dead, then one ought to write precisely so -- as a free artist -- but without employing actual names.

In his tale G.~Medvedev adopts a mentor's tone, or rather a prosecutor's. By its content and the peremptoriness of its judgments, the work may be regarded as an Indictment, fit to be submitted for review with the aim of issuing us (me and the operating staff) an even harsher sentence, for in Medvedev's narrative he adduces new \textit{``crimes''} not noted by the investigation.

Well then: presumably he has the moral right to pass sentence if he conscientiously understood the circumstances of the accident and, where he does not cite the opinion of knowledgeable persons, relies upon his own great experience as an operator -- of which he repeatedly reminds the reader. He reminds us of episodes that never occurred. Upon verification it emerges that G.~Medvedev did not work a single day at any operating nuclear power plant. He worked in Melekess from 1964 until 1972 on the VK-50, an experimental reactor and by no means a nuclear power plant. At the Chernobyl Nuclear Power Plant he was present from 1972 until 1974, when it was far from operation: Unit~1 was commissioned only on 16 September 1977. Since 1974 he has lived in Moscow, where, as far as I know, nuclear power plants have never existed. Nor has he been connected, even in an office capacity, with Nuclear Power Plant operation; he worked on the supply of equipment to stations. Such, then, is the reality behind his ``when I worked in nuclear-plant operation.''

From the same realm is his assertion:

\begin{personal}[From G.~Medvedev's \textit{The Chernobyl Notebook}:]
\textit{``I arrived at the construction site of the Nuclear Power Plant in the settlement of Pripyat straight from a Moscow clinic, where I had been treated for radiation sickness. I still felt unwell, but I could walk, and decided that by working I would return to normal faster.''}
\end{personal}

How he walked -- poorly or well -- I do not know; but according to information from the Sixth Hospital (A.~K.~Guskova and A.~F.~Shamardin), G.~Medvedev had no radiation sickness, and his dose was minimal.

Given these circumstances, Medvedev's knowledge of the technical side of the Chernobyl Nuclear Power Plant and its technological systems (in 1974, when he left, even the schematics did not yet exist) is exceedingly approximate. He therefore could not have understood the causes of the accident on his own. He did not resort to knowledgeable assistance and, judging from the text, his desk reference was the report of the Soviet specialists to the IAEA -- whose erroneous assertions he further augmented by his own Medvedev-style \textit{``interpretations.''} As for reactor physics, and RBMK physics in particular, Medvedev appears, in his own estimation, a great \textit{``expert''}; he consults no one and missteps at every turn.

It is astonishing how one can achieve such a nearly one-hundred-percent discrepancy between the technical narrative and events on the one hand, and reality on the other. To criticise this portion of the tale is pointless: one would have to rewrite every paragraph. Therefore, simply to illustrate the \textit{``documentary''} quality of the work, I make several remarks. After each quotation I indicate the page in \textit{Novyi Mir} No.~6 for 1989; the text is taken sequentially, and just as sequentially is inaccurate.

Medvedev writes:

\begin{personal}[From G.~Medvedev's \textit{The Chernobyl Notebook}:]
\textit{``During the shutdown of the unit, according to the program approved by Chief Engineer N.~M.~Fomin, it was planned to conduct tests with the reactor protections disabled, in a mode of complete blackout of the Nuclear Power Plant equipment. To generate electricity, it was planned to use the mechanical energy of run-down of the Turbogenerator rotor (rotation by inertia).''} (p.~16)
\end{personal}

Here before me lies the program; it is included in an appendix. In the Program of Turbogenerator Run-Down there is not a single word about disabling the reactor protection. Either Medvedev never laid eyes upon the program, or he understood nothing in it.

No complete blackout of the unit equipment was envisaged. On the contrary: according to the program, all mechanisms of the unit were transferred to reserve power, and only those mechanisms required for the test were powered from the running-down the Turbogenerator. This arrangement ensured the normal cool-down of the unit after the Turbogenerator frequency fell and the mechanisms were disconnected from it. In particular, four of the eight Main Circulation Pumps were supplied from the reserve. So also the remaining auxiliary mechanisms and all mechanisms of reliable power supply.

Medvedev continues:

\begin{personal}[From G.~Medvedev's \textit{The Chernobyl Notebook}:]
\textit{``What is the essence of the experiment and why was it needed? The fact is that if a nuclear power plant suddenly becomes de-energised, then naturally all mechanisms stop, including the pumps that force water through the reactor core. As a result, melting of the reactor core occurs, which is equivalent to an design basis accident. The use of any possible sources of electricity in such cases is what the experiment with the run-down of the Turbogenerator rotor is intended to provide. For as long as the Turbogenerator rotor spins, electricity is generated. It can and must be used in critical cases. The run-down mode is one of the subsystems in a design basis accident.''} (p.~16)
\end{personal}

When the mechanisms of plant auxiliaries are de-energised (without a design basis accident), cooling of the core is ensured by the Main Circulation Pumps owing to the kinetic energy of the flywheel inertia built into each pump, and thereafter by natural circulation. The remaining mechanisms are powered by emergency diesel generators and accumulator batteries. The ECCS does not participate in this.

Melting of the core is by no means equivalent to a design basis accident. If the core melts, an RBMK -- and indeed the entire unit -- may be considered lost. Contamination of the building and, likely, the station territory cannot be avoided. In a design basis accident none of this should occur, though the accident is serious.

And all the subsequent arguments in the tale on this subject are equally groundless -- purely speculative constructions which unfailingly excite the imagination of readers unfamiliar with the unit and the reactor. Such readers constitute the overwhelming majority, even among specialists. General nuclear knowledge is of no use here; what is needed is specific, concrete knowledge of the RBMK and its systems.

Here is one sentence from that realm, simultaneously revealing the author's grasp of physics -- very revealing indeed:

\begin{personal}[From G.~Medvedev's \textit{The Chernobyl Notebook}:]
\textit{``Yet these \qty{350}{\metre\cubed} of emergency water from the ECCS tanks, when the run-up on prompt neutrons began (emphasis mine -- A.\,D.), when the Main Circulation Pumps were torn off and the reactor was left without cooling, could perhaps have saved the situation and quenched the steam effect of reactivity, the most significant of all\ldots''} (p.~22).
\end{personal}

No -- if a power excursion on prompt neutrons has begun, then there is no salvation. Only the destruction of the reactor itself will stop it; in nuclear power reactors there exist no other means. On page~10 of the tale Medvedev lists Soviet reactor accidents: 

\begin{personal}[From G.~Medvedev's \textit{The Chernobyl Notebook}:]
\textit{``7 May 1966, a power surge on prompt neutrons at the Nuclear Power Plant with a boiling reactor in the town of Melekess. A dosimetrist and the shift supervisor were irradiated. The reactor was shut down by dumping two sacks of boric acid into it.''}
\end{personal}

People were unimaginably lucky -- the \textit{``prompt neutrons''} in question must have been exceptionally lethargic. They had time to run for sacks of acid and extinguish the reaction. With an ordinary power surge on prompt neutrons a human being has no time even to form a thought. Such magnificent nonsense!

Now consider the following:

\begin{personal}[From G.~Medvedev's \textit{The Chernobyl Notebook}:]
\textit{``At the same time, the introduction of such a number of rods into the core gives, at the first moment, a positive spike of reactivity, since the graphite tips (\qty{5}{\metre} long) and the hollow sections a meter long enter the core first. A spike of reactivity in a stable, controlled reactor is not frightening, but when unfavorable factors coincide this addition may prove fatal\ldots Did the operators know this\ldots?''} (p.~27).
\end{personal}

Medvedev presents the control rod construction and its mechanism of introducing positive reactivity in a distorted manner. But that is not the main point.

The main point is this: Medvedev considers a reactivity spike during the fall of AZ \textit{``not frightening.''} To say such a thing one must invert every fundamental concept of safety. Emergency protection is intended to shut the reactor down not only in normal conditions but above all under emergency ones. And here -- for example -- the signal of a shortening power-doubling period arrives, when the reactor already has excess emergency reactivity and further reactivity continues to be introduced for technological reasons -- and the Emergency Protection adds still more.

This is precisely what occurred on 26~April 1986. True, at the moment the AZ-5 button was pressed the reactor was only weakly supercritical; it entered an emergency state three seconds later. Is that \textit{``not frightening''}? It is monstrous.

Thus the bewilderment of the shift supervisor, Sasha Akimov, was wholly justified:

\begin{personal}[From G.~Medvedev's \textit{The Chernobyl Notebook}:]
\textit{``I did everything right. I do not understand why this happened.''} (p.~27)
\end{personal}

And how could he understand such a thing? In a normal situation, without any emergency indication whatsoever, the shutdown button is pressed -- and the result is an explosion.

Only with a complete perversion of moral and legal norms in the country, with total neglect of the law, can one blame the operating personnel -- as was done, and is still done.

Neither Toptunov nor Akimov, nor any operator at any RBMK unit knew about this property. And had we known, would we have agreed to work? Our ignorance was not \textit{``blessed.''} It is the direct achievement of the Scientific Director A.\,P.~Alexandrov and the Chief Designer N.\,A.~Dollezhal. Their staff were obliged to know, and Alexandrov himself undoubtedly knew -- there are document -- of the unacceptable properties of the reactor, and to take measures to eliminate them.

Above all, they were obliged to inform the operating personnel in the project documentation. And they did inform -- falsely. A NIKIET report asserts that the power coefficient of reactivity is always negative -- whereas in fact it was positive. Now there are clever fellows -- including Medvedev -- who reproachfully say: \textit{``The operating personnel did not know.''} Such an \textit{``elementary''} thing. But it is not elementary at all. Medvedev himself does not understand it even now.

Kurchatov Institute and NIKIET possessed -- already before the accident -- entirely sufficient operational data from the plants to form a complete and deep understanding of RBMK behaviour. They did not think it through and did not act. Nor did Gosatomenergonadzor, which likewise had the data. I still believe they did not think it through; I cannot imagine that they knew, and were silent, and did nothing.

Another excerpt:

\begin{personal}[From G.~Medvedev's \textit{The Chernobyl Notebook}:]
\textit{``Here another short explanation is needed. A nuclear reactor can be controlled only thanks to the fraction of delayed neutrons, designated by $\beta$. According to the Nuclear Safety Regulations, the rate of increase of reactor power must not exceed \qty{0.0065}{\beta} over \num{60} seconds. If the fraction of delayed neutrons is at \qty{-0.5}{\beta}, a power surge on prompt neutrons begins. Violation of the regulations and the reactor protections\ldots threatened a release of reactivity of at least \qty{5}{\beta}, which meant a fatal explosive power surge.}

\textit{Did Bryukhanov, Fomin, Dyatlov, Akimov, Toptunov understand this entire chain?''} (p.~27)
\end{personal}

How did \textit{you} understand it, Reader? Not at all? Quite natural. It is impossible to understand such a thing.

I also do not understand why G.~Medvedev measures the rate of power increase in units of reactivity. One wonders whether he has not attempted to measure it with a spoon or a shot glass -- just as incorrect, but at least more familiar implements.

\textit{``If the fraction of delayed neutrons is \qty{-0.5}{\beta}.''} How can a quantity be equal to half of itself, since $\beta$ \textit{is} the fraction of delayed neutrons? In every textbook on reactor physics it is stated unambiguously that a power surge on prompt neutrons occurs at a \textit{positive} reactivity of \qty{1}{\beta}, not \qty{0.5}{\beta}. A discovery has been made. Has it been registered yet? Only a fevered imagination could assemble the ``chain'' he proposes.

This, I believe, is sufficient to assess the competence of the author of \textit{The Chernobyl Notebook} in matters of power plant design, the circumstances of the accident, and reactor physics. And to see clearly how impossible it is, from such a vantage, to explain either the causes of the catastrophe or the actions of the operators. Difficult -- extraordinarily difficult. But G.~Medvedev fears no difficulties. He must press forward -- toward the honorarium. That the men he maligns were already slandered, and many already dead -- why should that restrain him? He must make his living.

Even this seems insufficient for him. The slanders of various commissions do not suffice; he must exhume the maximum design-basis accident button. On page~26 he lists \textit{``the grossest violations, both embedded in the program and committed in preparing and conducting the experiment.''} I shall not repeat them; they come directly from the Soviet specialists' report to the IAEA -- except for one novelty. See item~5: there the consequences are misstated, and Medvedev further distorts them, writing: \textit{``they blocked the protection on the shutdown of the two turbines, on the water level and steam pressure in the drum-separators, on the thermal parameters.''} He could not even copy it without error.

Too feeble? He invents more:

\begin{personal}[From G.~Medvedev's \textit{The Chernobyl Notebook}:]
\textit{``Finally they blocked both diesel generators, as well as the working and start-reserve transformers, disconnecting the unit from the power sources and from the power system. Striving to conduct a 'pure experiment,' they in fact completed the chain of preconditions for an extreme nuclear catastrophe.''} (p.~26)
\end{personal}

Is he lying? Only a little; others lie more. Medvedev adds but a brushstroke. Yet what a tableau results:

\begin{itemize}
\item Protections -- all blocked.  
\item Power supply -- everything disconnected.  
\item Personnel -- troglodytes, or lately descended from the trees.
\end{itemize}

Thus, dear Reader, are we surrounded on all sides -- and still are -- with the help of Medvedevs. His narration of the pre-accident events is incorrect, and cannot be correct, given his knowledge. One may write truthfully even without prior knowledge -- by listening to competent people. But Medvedev's swagger, his self-advertising as an \textit{``experienced operator,''} prevents him from consulting anyone. Let us pause at several further points.

\begin{personal}[From G.~Medvedev's \textit{The Chernobyl Notebook}:]
\textit{``At 01:07 a.m., to the six operating Main Circulation Pumps, one more was additionally switched on, with the calculation that after the experiment ended four Main Circulation Pumps would remain in the circulation loop for reliable cooling of the core''} (p.~30).
\end{personal}

Correct. But on p.~34: \textit{``The total flow through the reactor began to fall because all eight Main Circulation Pumps were operating from the rundown turbine generator.''}

Thus within four pages he has forgotten his own statement -- or did not understand it. How, indeed, could all eight Main Circulation Pumps be powered from the rundown Turbogenerator, and four remain afterward for reliable cooling? It is simple: four Main Circulation Pumps, like most mechanisms, were powered from the reserve supply. All his arguments about the absence of reserve power are bluff. It was present.

One need only examine the program, or the Shasharin commission's findings based on SKALA's objective recordings -- not on speculation.

Hence comes his direct accusation -- first of all directed to me -- of bungling:

\begin{personal}[From G.~Medvedev's \textit{The Chernobyl Notebook}:]
\textit{``One asks: was it possible in this situation to avoid the catastrophe? It was possible. One only needed to refuse the experiment, connect the ECCS, reserve the power supply, and by hand, in steps, lower the power all the way to shut-down, in no case performing a SCRAM\ldots But this chance was not used.''} (p.~30)
\end{personal}

He has offered his counsel. The matter was simpler. One needed only to lower six control rods into the core, then another six, until shutdown, and then trigger AZ-5. One could have begun with the regulators. All this is clear only \textit{now}, after the reactor's unnatural Emergency Protection property has become known. ECCS and power manipulations were unnecessary.

% --
I didn't understand it at the time. And if I had possessed such prophetic insight, I might as well have performed for Kashpirovsky and Chumak\footnote{Soviet perestroika-era television healers and supposed miracle-workers.} and claimed a chestful of medals, more so than Brezhnev. In other words: only a psychic could have pieced it together then.
% --

No, I do not reproach myself for not foreseeing the danger of enabling the Emergency Protection. I reject all reproach. It could not be deduced; it had to be \textit{known}. And had I known, I would not have worked a single day on such a reactor. On 26 April we ourselves actuated AZ -- but it could have actuated \textit{automatically} at any moment. What then?

Medvedev therefore paints a fantastical scene. Why? There was no need to invent: reality was phantasmagoric enough.

\begin{personal}[From G.~Medvedev's \textit{The Chernobyl Notebook}:]
\textit{``And suddenly Perevozchenko flinched. Strong and frequent water-hammer blows began; \num{350}-kilogram cubes -- 'assembly eleven' -- jumped on the channel heads as though seventeen hundred men tossed their caps in the air\ldots''} (p.~33)
\end{personal}

% --
The \textit{``cubes''} weigh fifty kilograms, not \num{350} -- but let that pass. It is beautifully written. He commands the men to jump and hurl their caps. Rich imagination. Only: technically impossible. He's got the gist, but none of the sense. He is describing hydrogen.
% --

And the timing is impossible. 01:23:40 a.m., \qty{200}{\mega\watt}, stable parameters. SKALA records it. Nothing can be \textit{``happening.''} Three seconds later: overpower and short-period signals appear. Still nothing visible; the power is only \qty{520}{\mega\watt}. Even if we accept this as the start of the \textit{``dance,''} then -- at 01:23:47 a.m., the explosion occurs. In four seconds no one can get from the \qty{+50}{\metre} balcony into the central hall -- there is no spiral staircase there; he confuses Sector~1 and~2\footnote{\textit{Очередь}, previously explained in the \textit{The Power Plant} chapter; essentially - a collection of two units.}.

When still in prison I wrote to eyewitnesses. Sasha Yuvchenko, senior mechanical engineer of the Reactor Department, replied:

\begin{personal}[Yuvchenko to Dyatlov:]
\textit{``From the beginning of the shift until the explosion, Perevozchenko and I were together\ldots He was urgently called to the Unit~4 control room; he left; \numrange{1}{2} minutes later the first blow sounded, then the explosion. He could not have visited the hall. He never told such a story.''}
\end{personal}

I confirm: Perevozchenko came to the Unit~4 control room immediately before the rundown began. No one ever heard such a story from him. Perhaps Medvedev, by inspiration from above, offers a \textit{``documentary picture.''}

\begin{personal}[From G.~Medvedev's \textit{The Chernobyl Notebook}:]
\textit{``Thus, if one believes the machine, in the upper third of the core there formed a flattened sphere of high energy release\ldots It was this part that was thrown out by the explosion\ldots''}
\end{personal}

One should believe the machine; it is not subject to expediency. But destruction began from the \textit{lower} part of the core -- this is generally acknowledged. When the rods begin to enter, absorbers depress the neutron field in the upper region; in the lower region the water columns in the Control and Protection System channels are displaced by the displacers, which absorb less than water, introducing positive reactivity. It was there that the rapid rise of power began, and there that the core first failed.

It is astonishing how bold people have grown: irresponsible pronouncements on matters in which they \textit{``have neither ear nor snout''} now appear in droves. Until a reliable picture of the reactor explosion has been constructed, one may speak of the quantity of ejected fuel only on the basis of measurements of environmental contamination and dosimetric surveys inside the unit buildings. And, naturally, this is possible only where science is free of ideological pressure.

In any case, references in print to Medvedev's tale as an authority on the quantity of released products are improper. His notions concerning the course of the explosion are manifestly distorted, and from false premises, as you understand, no correct conclusions can be derived. Now as to Dyatlov's actions -- that is, my own -- on 26~April and not only then. That I appeared to Medvedev cross-eyed and bow-legged is no great disaster. Perhaps I am indeed such. We all appear handsome in our own eyes; could we see ourselves from without, then\ldots\,But that is beside the point.

Who am I, and how did I come to occupy the post of Deputy Chief Engineer for Operations?

After technical school I worked three years at a plant. My diploma, with distinction, entitled me to enter an institute without preliminary work, but I decided first to reinforce my knowledge with practice. After graduating from the Moscow Engineering Physics Institute in 1959, I was sent to the Far East. Soon I was appointed head of the physics laboratory. The work there seemed to me too slight, the pay too small. A family, two children. I had energy, knowledge, conscientiousness, and a desire to work. I requested transfer to the training centre, where I was trained as an operator of the main power plant of nuclear submarines. Remaining head of the laboratory, I also sailed in the delivery team on sea trials. Later all the operators were placed under my authority. The work and the salary now satisfied me. It only seems, to the uninitiated, that all that ``iron'' is the same; in reality each submarine, even of the same project, has its own peculiarities. The work went well; relations with subordinates and superiors were normal. My chiefs did not love me -- for I was obstinate -- but they respected me as a worker. I never sought love, either from superiors or from subordinates. For normal production relations it suffices, in my view, to be competent and fair. In any case, in all my time there, none of my subordinates left because he could not work with me. Perhaps I was a little harsh, but hardly more than that. I was demanding, yes.

It is difficult for me to judge what sort of manager I was, whether I ``possessed the art of dealing with people.'' Still, I do not think I was the worst. When I resigned from the plant and went to Chernobyl, several men who had been my subordinates also came there to work under me. I had invited none of them. Of course, when they arrived, knowing their qualities, I recommended to the director that they be hired. I am far from thinking that they came in order to work under my authority; no, they wished to leave Komsomolsk and were not afraid that I would once again be their chief.

Nor is there any need to speak of them with the haughty disdainful epithet ``pals-mates,'' Mr. Medvedev. All of them -- A.~A.~Sitnikov, V.~A.~Chugunov, V.~A.~Orlov, V.~V.~Grishchenko, A.~V.~Kryat -- proved themselves good workers at the station. And why bury Slava Orlov in advance, Mr. Medvedev? I saw him yesterday. And it was he who came to Poltava to collect me after my release from prison. The time will come -- we shall all die.

Slava Orlov, Tolya Kryat, and Valera Lomakin came to see me three times in Poltava, in prison, on visits. My enormous thanks to them. One cannot put a price to that. One must be there to understand what it means.

How did I relate to people in general? As each deserved. At work, only a man's qualities as a worker mattered to me. I was fully aware that it is impossible to gather two hundred people, all agreeable in every respect. There was no one to whom I gave indulgence, and no one toward whom I was captious. When the trouble came, the investigation persistently sought to create an impression of the guilt of the personnel, and especially of Dyatlov. And yet it could obtain no incriminating testimony about me, save from two ``authors'' whom the court did not even summon to the hearing. From Komsomolsk, to which the investigator sent an inquiry, all eight references were favourable. Thank you, esteemed colleagues.

Nor did Dyatlov, nor any of the other managers, strive to create unbearable working conditions at the station. There simply was no such aim. During my time there, twice -- by which organizations I no longer recall -- social-psychological studies were carried out. No deviations, either in the psychological qualities of the operators or in the social relations, were found in comparison with other nuclear power plants. This, assuredly, was not the cause of the accident of 26~April.

I should have gone on working in Komsomolsk, had the path not turned otherwise. In addition, the long business trips, the harsh conditions at sea, when on board a submarine there are the fleet crew, the delivery team, and the acceptance commission, inclined me to change.

While on leave, I stopped in Pripyat and agreed with Director V.~P.~Bryukhanov about the post of deputy head of the Reactor Department. We did not speak of the shop head's post; that post was already filled. When I spoke with Bryukhanov no one else was present, and Medvedev has nothing to do with it. In general, he clearly exaggerates his role in staffing the station; he left in 1974, when no money whatever had yet been allocated for recruiting personnel. Incidentally, this is the surest indicator of whether the Ministry believes in a scheduled start-up date: if no money is allotted for recruiting, there will be no start-up. It is possible that Medvedev did speak with Bryukhanov about people in general, but only in principle; concrete individuals could not yet have been under discussion.

I arrived at the station after receiving a summons in September 1973, after fourteen years at the plant. Submarine reactors are, of course, different in design and much smaller in size. But they are no toys -- real power reactors. At Chernobyl I served as deputy, then as head of the Reactor Department. And since February 1983 I had experience of work with reactors.

The accusations levelled at Dyatlov by Medvedev have no foundation.

\begin{personal}[From G.~Medvedev's \textit{The Chernobyl Notebook}:]
\textit{``So then, was Dyatlov capable of an instantaneous, uniquely correct assessment of the situation at the moment of its transition into an accident? I think not. Moreover, he apparently did not possess to a sufficient degree the necessary caution and sense of danger, so needed by the leader of nuclear operators. What he did have in abundance was disrespect toward the operators and the technological regulations\ldots It was precisely these qualities that unfolded in Dyatlov in full force when, after the shutdown of the Local Automatic Regulator, Senior Reactor Operator Leonid Toptunov was unable to hold the reactor at a power of \qty{1500}{\mega\watt} and let it drop to \qty{30}{\mega\watt} (thermal)''} (pp.~24-25).
\end{personal}

Though conjectural, Medvedev nevertheless denies Dyatlov the capacity to assess the situation. On what grounds? On none. He had never seen me in operational work. And on 26~April there were no dramatic decisions to be taken. We acted in strict accordance with the operating documents in force at that time. The tragedy lies in the fact that the catastrophe occurred in the most routine setting. Our actions must be judged by the provisions in effect on 26~April, not from today's belfry. I have already said: I could not have invented such an absurdity as a prohibition on pressing the AZ-5. And the printout of the rod positions at 01:22:30 a.m., of which Medvedev speaks on p.~31, did not exist; I have discussed this earlier.

Those who worked with me say the opposite, that I am cautious. \textit{``Disrespect toward technological regulations\ldots in abundance''}? On what basis? This same phrase appeared in the Bill of Indictment. I did not personally perform any switching; all was done through the unit or plant shift supervisors. I therefore could not have concealed anything. At trial I put specific questions to these witnesses; all answered in the negative. In the end the judge remarked that I was putting strange questions, as though I had been on duty at the station for the sake of violating regulations. There was no such testimony even at the preliminary stage. Nor was there anything of the sort on 26~April. Thus both the prosecutor and the home-grown prosecutor Medvedev are mistaken.

\textit{``Disrespect toward operators, in abundance!''} What is the basis of this statement? I shall explain, for Medvedev's benefit. There is no violation whatever in raising reactor power after Toptunov's power drop. According to Item~6.7 of the Standard Regulations, a drop to such a level is a partial unloading of the unit; to raise power afterwards it is not required to have a minimum reactivity margin of thirty rods, as for a short-term shutdown. It is sufficient to have fifteen rods. And that margin existed. At midnight, at \qty{760}{\mega\watt}, the margin was twenty-four rods; in half an hour (the drop occurred at 00:28 a.m.) it could not have fallen further owing to poisoning. And the power coefficient was reported to us as negative. Medvedev's reproachful exclamation, \textit{``Ah, Dyatlov, Dyatlov, you do not know how quickly the reactor is poisoned,''} is out of place. I had long since and firmly mastered this. And although, when I entered the control room, the operators were already raising power, I should have permitted or ordered it even had I been present at the moment of the drop. All in accordance with the Regulations.

Nor did I express dissatisfaction to anyone, nor had I cause to. I do not know operators who have never, for one reason or another, let power fall. Toptunov was a young operator; even had there been an error, I should not have reproached him at the time. Later, in the analysis, of course I should have pointed out the mistakes -- but only then. Over many years of work with reactor operators (and this applies to all panel operators), I had firmly learned a rule: no scolding and no reproaches to an operator at the panel. He is sufficiently distressed by what has occurred, and those indifferent I did not keep. Not a single panel operator at the power plant can say that I berated him at the desk. It would have cost me dearly: in that state he would make further mistakes. In truth, I cannot recall -- and my memory is good -- having scolded any panel operator even afterwards in the last three years. Immediately after the shift there is a short debriefing, at which the participants hand in written explanations and describe their observations. Often no final conclusion can yet be drawn. Only after analysis of the instrument readings and control-system records does one reach a verdict. By then tempers have cooled. Penalties followed, of course, if errors were made, in the form of a \numrange{20}{30}, rarely \num{50} percent reduction of bonus. But if a man is in the least capable of self-criticism, what room is there for resentment?

In all my time at the power plant I did not remove a single operator from duty in the course of a shift. Why, then, should I suddenly have wished to remove someone on 26~April? We had the dispatcher's permission to shut down; practically all the work had been done. The experiment was not carried out -- well, we should have done it after the repair; no deadline constrained us, since the system had been commissioned after completing the rundown scheme in the generator excitation system and its idle tests. Had there arisen a necessity -- for example, had I seen that the operator was demoralized by the power drop -- I would certainly have removed him, and without threats. All the more so, since there were people on hand to replace him.

Medvedev alleges that Dyatlov ran about the premises, wasting precious minutes. Dyatlov ran -- in the forest -- but that is another matter. No one ever saw me running about the control room. Why, then, should I suddenly begin running on 26~April? Such idiotic behaviour by the senior technical person in the control room as Medvedev describes could by itself bring about an accident. But nothing of the sort took place, nor anything remotely resembling it. On 26~April I raised my voice loudly only twice: once with the command, \textit{``Everyone to the standby control panel,''} and once when Kabanov began telling me that the vibration laboratory would remain in the turbine hall; I ordered him to leave the unit immediately. Both incidents were after the accident.

In his tale Medvedev speaks of two operators from the previous shift who remained to observe, Yu.~Tregub and S.~Razin. Here is what they wrote in response to my questions.

\begin{personal}[Letter of Yu.~Tregub to Dyatlov (07 June 1990):]
\textit{``Before the accident there were no conversations in raised tones with members of the operating personnel, nor was any dissatisfaction expressed about the power drop. There were also no attempts to remove L.~Toptunov from control; he performed his duties throughout the shift. After the power drop, ARM was switched on, and, on the order of the unit shift supervisor, A.~Akimov, which I suppose was coordinated with you and the plant shift supervisor, power was raised to \qty{200}{\mega\watt}. I did not notice anything resembling disagreement about raising power''}
\end{personal}

\begin{personal}[Letter of S.~Razin to Dyatlov (07 June 1990):]
\textit{``Before the accident I did not hear any words in raised tones, only orders relating to execution of the experiment according to the programme. During the power drop I went over to the SIUR panel; I saw that, so far as I could understand, tense work was in progress to raise and stabilize reactor power, carried out by Toptunov. I did not see anything that could be regarded as attempts to remove or replace him, nor any pressure on him from your side or from Akimov, allegedly refusing to raise power after the drop, nor any dissatisfaction caused by the drop. I consider that such a conflict situation in the control room could not have gone unnoticed.''}
\end{personal}

There \textit{were} some conversations about removing Toptunov -- but they arose later, after I told Akimov to send him and Kirshenbaum to the Unit~3 control room, since they could do nothing useful in Unit~4 and the radiation situation was extremely unfavourable. This was about an hour after the accident. One further point requires clarification, since it concerns the unjustified irradiation of several people: Medvedev's assertion that it was on Dyatlov's prompting that the version of the reactor's integrity after the explosion began to circulate. How does he know all things? I, for one, told no such thing to anyone, including him. Perevozchenko told no one of any wild \textit{``dance,''} yet Medvedev knows. No one knows of any attempt to remove Toptunov, and Medvedev knows. No one knows that Toptunov and Akimov resisted raising power, and Medvedev knows. It appears Medvedev knows both what happened and what did not. True, he mostly writes of what did \textit{not} happen. Such is our \textit{``documentalist.''}

How it really was I have already described. Bryukhanov asked me nothing, neither by telephone nor in the bunker when I came there. I did not speak with Fomin at all on 26~April. Had I believed the reactor intact, I should certainly have attempted to organize water injection. I knew the unit well, and no one at the station knew the Reactor Department better than I. In case of a shortage of people I would have appealed to Bryukhanov. Yet we made no effort in that direction -- plain testimony to what I thought of the reactor's condition.

Medvedev's remark that Toptunov and Akimov behaved bravely, but to no purpose, is immoral. Leonid, by virtue of his post, indeed could no longer influence events. But by his conduct -- returning of his own accord to the unit -- he set an example of devotion to duty. And all the work of de-energizing mechanisms, removing turbine load, draining turbine oil, displacing hydrogen from the generators -- in short, preventing new fires -- was carried out under Aleksandr Akimov's direct leadership. Nothing more useful could have been done in that situation. What \textit{was} done is much, and necessary. Aleksandr Akimov was a good, exceptionally conscientious worker. And he died with the dignity of a Human Being.

We have examined the technical part of Medvedev's tale. Note the first quotations I cited and their page numbers. You will see that practically everything is incorrect. The remaining words between them serve only as ligatures. I did not continue to take the text sequentially further; but believe me -- or verify it yourself -- beyond that everything is equally wrong. In a row.

As to the non-technical part, I cannot say how \textit{``documentary''} it is, despite the names the author gives. In 1990 I spoke with one of those to whom Medvedev had allegedly talked. Viktor Smagin says that Medvedev distorted his words. The same, according to Smagin, was said by L.~Akimova. It is hard to imagine that the author would dismount from his favourite hobbyhorse in the rest.

The naturalistic scene of pigs devouring a dog is, in principle, possible; yet it is difficult to imagine that a writer, appearing in the town for five minutes, would chance to witness it. One is left to suppose that the pigs specially waited for G.~Medvedev's arrival and deferred their meal. My mere existence refutes his tale of a dose rate of \qtyrange{15e3}{20e3}{\roentgen\per\hour} in the vicinity of Unit~4. I twice walked across the territory near the unit, remained there in aggregate perhaps twenty to thirty minutes, and also entered the unit itself, in various places. My dose was determined as \qty{550}{\rem}. Had it been much higher, I should not be alive. The author's love of punchiness and hyperbole is plain. He assigns to Tolya Sitnikov a dose of \qty{2000}{\rem}, as if death at \qtyrange{500}{600}{\rem} were no tragedy.

Let us consider \textit{The Chernobyl Notebook} as a whole. What, in the end, does the author tell us? Precisely what the commissions told us: the operating personnel, and its leader Dyatlov, are guilty of the explosion. You object: he also speaks of deficiencies of the reactor. Yes -- but merely for form's sake. In our time, one can hardly defend the RBMK and its creators without conceding \textit{something} concerning the reactor. Yet in the tale this is twisted in such fashion that the operating personnel are guilty in any case. Dyatlov is guilty even of not guessing the danger of pressing AZ-5. Not a single commission, not even the prosecutor, accuses me of that, recognizing the absurdity; yet Medvedev knows better. It appears even this can be argued.

We have already spoken enough of Dyatlov. Reactor operator Leonid Toptunov is ``young''; for him, the canons of reactor operation ``had not yet entered his flesh and blood'' -- and therefore\ldots

And therefore what? What violations did Toptunov commit -- in fact, not according to Medvedev? Did he cause the power drop? That occurred because of the faulty regulator to which he switched. Grant even that this arose from inexperience or simple carelessness. Is a criminal case to be opened against an operator for lowering power? Yet such a case was opened. Did he raise reactor power after the drop in accordance with the Regulations, and not contrary to them? Did he indeed overlook the reactivity margin? Probably. But was he provided, as the law requires, with adequate means to monitor this parameter? No. I do not even speak of the absence of protection required by law. The very device for measuring the parameter is wholly unsuitable in a transient process such as that of 26 April, and in many other normal regulated transients.

That, then, is the matter -- according to the law.

I shall not so much as speak of human decency -- it is something foreign to Medvedev. I will merely remind others: in controlling a reactor an operator performs more than a thousand manipulations in the course of an hour, and keeps under supervision more than \num{4000} parameters. And for this man one finds reproach because he overlooked a parameter for which there exists neither measuring instrument nor alarm?

As for Aleksandr Akimov -- allegedly he \textit{``did not work on reactors.''} In fact he did, albeit not long, both during his training for the position and later, already as shift supervisor of the unit, shortly before the accident, when the opportunity presented itself and I gave him a month to work at the reactor.

And what, pray, did he violate? He bypassed one protection and altered the setpoint of another -- in exact conformity with the operating documentation. \textit{``He lacked the strength of character. He obeyed reluctantly.''} No one brought pressure to bear upon him, and he violated nothing.

Observe how G.~Medvedev writes:

\begin{personal}[From G.~Medvedev's \textit{The Chernobyl Notebook}:]
\textit{``The total flow of water through the reactor increased to \qty{60000}{\meter\cubed\per\hour}, with a norm of \num{45000}, which constitutes a gross violation of the Operating Regulations.''}
\end{personal}

Everything is precise in the digits, with a reference to the Regulations. And everything is wrong:

\begin{itemize}
  \item not \num{60000}, but \num{56000}, and not more than that;
  \item not \num{45000}, but \num{48000};
  \item at the moment the button was pressed, the flow was precisely \num{48000};
  \item neither in the Regulations nor in any other document is there so much as a hint of any maximum-flow limitation upon coolant circulation.
\end{itemize}

Why, and at whose behest, does Grigorii Iustinovich Medvedev add his own falsehoods to all the other slanders heaped upon the operating staff?

The quintessence of his story appears in the following phrase: \textit{``And yet, for the sake of fairness, it must be said that the death sentence was predetermined in some degree by the very design of the RBMK.''} One needed only to secure a confluence of circumstances under which an explosion was possible. \textit{``And this was done''}. I have already remarked that the reactor and its creators cannot be defended without an admission of certain \textit{``shortcomings''} in it.

\textit{``For the sake of fairness''} -- this is mere camouflage. There is no fairness in that assertion.

Let us consider more closely the expression \textit{``in some degree.''} This  \textit{``degree''} is acknowledged by everyone, save only Scientist A.\,P.~Aleksandrov. Well, a thrice Hero may permit himself such a luxury.

First. The RBMK-86 reactor failed to satisfy the requirements of thirty-two points of the normative documents; fifteen of them, as stated in the report of the N.\,A.~Shteinberg commission, were directly relevant to the accident of April~26. Normative documents do not contain superfluous requirements, for the fulfillment of each may cost large sums. Yet every requirement is obligatory. In this case, fifteen were not fulfilled.

Second. Owing to its positive power coefficient the reactor was dynamically unstable, and because of the defective design of the control rods, the actuation of AZ-5 introduced \textit{positive} reactivity. Nothing more is required for an explosion. That is precisely why it exploded on April~26.

And if this is \textit{``in some degree,''} what then would constitute \textit{``entirely and completely''}?

To accuse the staff of causing the reactor explosion even \textit{``in some degree''} is unjust. Our actions were in accordance with the operating documentation; the sole possible violation -- overlooking the reactivity reserve -- was itself a consequence of a violation of nuclear-safety requirements in the matter of equipping the reactor with automation, alarms, and instruments. For that reactor, no special \textit{``confluence of circumstances''} was needed for an explosion. It would have exploded under a whole series of other conditions as well.

To criticize the authorities has now become fashionable. Yet in Medvedev's case it so falls out that every criticism he ventures is refuted with elementary ease. So it was with his strictures upon the ministries for approving the rundown program -- they may quite truthfully reply that they never once saw this program.

So it was with the question of the quantity of fission products released from the reactor -- a rigorously documented report, now comprising tens of thousands of measurements, will outweigh by far any purely speculative \textit{``conclusions.''} So it was with the assertion that B.\,E.~Shcherbina and Yu.\,A.~Izrael declared at the press conference of May~6, 1986, that the radiation level near Unit~4 was only \qty{25}{\milli\roentgen\per\hour}. They said something entirely different. There is indeed much that can and should be criticized in Shcherbina, in Izrael, and in the ministries -- criticized justly. But G.~Medvedev never set himself such a task. This is evident from his entire posture in life. Throughout his narrative there runs an undercurrent of his intimacy with the powerful: always with generals, with ministers, always in the role of instructor. And couriers, couriers, couriers! In reality it was not so; but his longing to belong to the circle of the great is too strong.

To step beyond the circle he has traced for himself would be fraught, at the very least, with uncertainty; but to remain within it keeps everything in due order. He supports -- and that to excess -- the government version. A loyal man.

What B.~Kurkin meant when he said that publishing the \textit{Chernobyl Notebook} required courage from its author is incomprehensible. What courage is needed here? He added slander upon the dead and upon those sitting in prison. A brave man indeed! And whom, then, are we to call a boor? A slanderer? Concerning Andrei Dmitrievich Sakharov's preface to the \textit{Chernobyl Notebook} I can only say, with bitterness: evidently a decent man will never learn to recognize the many faces of multifaceted malice.

\section{Y.~N.~Shcherbak, \textit{Chernobyl: A Documentary Story}}

Here everything is simpler. The novel \textit{``Chernobyl: A Documentary Story''} clearly does not rise to the level of an \textit{``outstanding''} work such as \textit{``The Chernobyl Notebook''}. Yurii Nikolaevich Shcherbak, for reasons known only to himself, bound his hands with the vow: ``I cannot allow myself a single imprecise word (though novelists ought not to do so either), I have no right to conjectures or guesses.'' With such an approach, nothing good can come of it. Whereas G.~Medvedev plucks figures out of thin air, cites a Regulation he has never seen, and the whole performance passes for documentary, authoritative, convincing.

And yet not everything in Shcherbak's novel is accurate. Not because he invents or distorts the statements of his respondents; rather, the statements themselves are incorrect, do not reflect the truth of the matter, and are illogical. I shall offer merely a few remarks on the technical side of the catastrophe.

First, as to the assertion that the accident might have occurred during the shift of I.~Kazachkov (08:00-16:00 on 25~April), or during the shift of Yu.~Tregub (16:00-midnight on 25~April), but that the experiment was postponed to the night shift. In this way the author links the accident to the execution of the turbine-rundown experiment. There exists no such connection. The accident happened while the experiment was underway, but it might just as readily have happened during any other operation and, first of all, during shutdown, when the reactor is operating in the regime of stationary fuel-assembly shuffling.

A.~Uskov is correct, of course, that had the experiment been carried out during startup, the accident would not have occurred. Only he forgets that in the first months after startup no such accident could have occurred at all. For the reactor is then an entirely different entity.

When more than two hundred supplemental absorbers are present in the reactor, its steam-void reactivity coefficient is negative, and these same absorbers largely compensate the end-effect of the Control and Protection System rods. A large concentration of absorbers in the lower portion of the core smooths out the effect produced by replacing water columns with displacers in the Control and Protection System channels. And it was precisely due to these reactivity effects that the reactor exploded, and it matters not what operation happened to be underway at the time.

Completely incomprehensible is I.~Kazachkov's remark: \textit{``To the shift supervisor of the unit -- that is, to myself -- I would give eight years. And if this had happened on my shift, I would understand that it was just.''}

His \textit{``that is, to myself''} is pure nonsense. In fact -- to A.~Akimov. In court, Kazachkov quite rightly stated that if modernization were not carried out on Unit~3, he would refuse to work on it. His own words: \textit{``Then the startup will be without me.''} That is, he understands that one cannot operate a reactor in its pre-accident state. Now, having learned what sort of monster the reactor was, he concedes that the catastrophe might have occurred on his shift as well. That is, irrespective of the operator: Akimov, Kazachkov, Tregub. Which means the catastrophe occurred because of the reactor -- its properties -- not because of the operator. And yet he would still give Akimov eight years. And deem that just.

I refuse to understand.

Concerning the power rise after the drop. After G.~P.~Metlenko's testimony in court, I am firmly convinced that I was \textit{not} at the control room at the moment of the drop, but came in a little later. Before this I had begun to doubt my memory, having read his testimony saying I had been there. I have no reason to deny my presence -- whether I was there or not, I bear responsibility for the staff's actions. Whether I or Sasha Akimov gave the order to raise power, there is no violation in that. When I asked Akimov to what level the power had fallen, he named \qty{30}{\mega\watt}; I permitted further increase. I had no grounds to distrust him; he was both competent and honest.

The same \qtyrange{30}{40}{\mega\watt} was given after investigation by Meshkov's commission, and there were enough \textit{``well-wishers''} there who would not have neglected to pounce upon a drop to zero; likewise Shasharin's commission and, finally, the government commission. Only the judicial-technical commission, which had no members who had ever worked on the reactor, asserts entirely without foundation that it dropped to zero. After analyzing the power charts I am convinced it did not fall below \qty{30}{\mega\watt}. And that constitutes a ``partial power reduction'' under the Regulation. There is no violation in raising power. And the reactivity reserve at that moment could not have been below \num{15}~rods. At midnight it was \num{24}~rods; this is in the logbook. Take the strictest scenario -- power plunges from \qty{50}{\percent} to \qty{0}{\percent}. In half an hour the poisoning will by no means reach nine rods -- consult the poisoning curve. And treat the power coefficient as negative, for that is how it was reported to us.

So for what, precisely, must Akimov or Dyatlov be ``severely punished''? They acted according to the operating documentation. Kazachkov ought to have thought before saying: ``But they wanted to complete the test to the end.'' It was in \textit{his} shift, in 1985, that a scram (AZ-3) occurred during a planned shutdown because of faulty instrument behavior. At the very first step of the shutdown program. There was no \num{55}-rod reactivity reserve, and I ordered the unit cooled down, finishing nothing. Here nearly everything had been done, yet for some reason Dyatlov is accused of violating rules.

Some stereotype is triggered. When questions are posed in an unusual situation -- or for an unusual reason -- a man is led by the nose like a bull on a rope, instead of saying: \textit{``Pardon, the question is incorrect.''} And Arkadii Uskov torments himself, wondering whether or not he would have yielded to the pressure of his superior. He concludes that he would have yielded. There was no pressure, and there were no violations.

Gentlemen, when shall we begin to think?

A.~Uskov took offense when I told him all this in person. But why take offense? This is the meanest sort of accusation, when one appears to sympathize with the staff and sincerely strives to understand why they committed violations, and even seems to justify them. People gain the impression: if even a man friendly to the staff acknowledges there were violations, then there must have been.

To this situation applies most aptly: \textit{``Deliver me, O God, from such friends; from my enemies I shall deliver myself.''}

I shall not comment on V.~A.~Zhiltsov's remarks; much has already been said. Only this:

\begin{personal}[From V.~A.~Zhiltsov's remarks]
\textit{``Moreover, the rarest situation arose when the AZ-5 system served as the starting impulse for the runaway of the reactor. Had AZ-5 been normal, the reactor would never have run away, no matter what errors Senior Reactor Operator L.~Toptunov committed. For the brake pedal must brake, not accelerate the automobile.''}
\end{personal}

From this remark of a man who has \textit{``devoted his whole life to nuclear energy''} the following questions arise:

Does V.~A.~Zhiltsov not understand that AZ-5 must \textit{never} -- neither rarely nor very rarely -- serve as an accelerating mechanism? Does V.~A.~Zhiltsov acknowledge the protection system as abnormal, yet accuse the staff? V.~A.~Zhiltsov is driving an automobile, sees a pedestrian step into the road, applies the brakes. The automobile, instead of stopping, accelerates and kills the pedestrian. Will V.~A.~Zhiltsov consider himself guilty?

He, of course, will not consider himself guilty; the staff -- yes.

One marvels at how coldly these gentlemen -- scholars and non-scholars alike -- admit that the emergency protection \textit{blew up} the reactor, and despite all this continue to ferret out the staff's alleged sins.

Madness. Absurd.

\chapter{On the Freedom of the Operator}

The very notion of the operator's ``freedom to make decisions,'' as expressed by I.~Kazachkov and A.~Uskov in Yu.~Shcherbak's novel is, in my view, wholly inapplicable to the Chernobyl Nuclear Power Plant staff in general and certainly has no bearing upon the accident of April~26. Freedom of decision-making, independent of any extraneous considerations except technical ones, is necessary not only for an operator -- for everyone. But complete freedom does not and cannot exist, under any system or in any industrial relations. I speak only of hired workers, to which class I assign all who worked at Soviet state enterprises. In my view, the overwhelming majority of workers had quite sufficient freedom, for in the event of conflict and the impossibility of continuing to work at one place, one could always find another position at similar pay, losing little except perhaps during the initial period. Provided, of course, one was a qualified worker.

Under any system one cannot expect that, when you object to a superior, he will stroke your fur the right way and say: ``How fine, he barks at me.'' Yet if one observes two elementary rules:
\begin{itemize}
\item do not ``get into a lather'' over trifles;
\item be in the right when you object;
\end{itemize}
then the superior will simmer and cool down. To the superior, in positions on which production depends, knowledgeable workers are necessary first, and compliant ones second. True, we also have many useless positions, and those may indeed be filled with the compliant.

It seems to me that the psychological climate at the plant was fully acceptable. The principal credit for this, in my opinion, belongs to the director V.\,P.~Bryukhanov: a man not harsh by nature, self-possessed, not given to hasty conclusions. Of course, all sorts of things occurred, especially amid the nervous, extremely tense atmosphere of construction and installation, with masses of questions arising. But all this was resolved at the level of Department chiefs and deputies.

To possess external freedom, one must possess internal freedom, a sense of one's own dignity. I could object -- and in fact did object -- to anyone at the plant.

Chief engineer V.\,P.~Akinfiev was not, generally speaking, a malicious man, but he possessed a remarkable ability to insult a person for no reason.

On Unit~1, certain pipeline elements -- elbows manufactured by the Baglei Plant of the Ministry of Energy -- had been installed in the ECCS system. They were made to the proper standard, but the plant had authorization to fabricate components for pressures only up to \qty{22}{\atmosphere}.

At that time I was deputy head of the Reactor Department. I went to a plant licensed to manufacture such components, as I had been there earlier while arranging a contract for pipeline supply. I arrived; they said: no, your fitters have already been here. To establish \textit{``contact,''} I had to spend money, originally intended for purchasing a suit, or in the restaurant. I reached an agreement and ordered with surplus; we later used the excess on another unit.

Three days later Akinfiev entered and asked:

\begin{quote}
--- ``Have they made them yet?''\\
--- ``No.'', I said.\\
--- ``Then why were you babbling that they would do it quickly?''\\
--- ``Have you gone daft, chief? It is impossible to do that in such time. Even if the documentation is simple, it still must pass through design, technological, and production departments.''
\end{quote}

To his boorish reproach I replied in kind. A fine conversation ensued. We sent each other to the same three-letter destination (``not to the BAM\footnote{Baikal-Amur Mainline.} but along the \textit{old road}'').

Another case. On Unit~3 we required membranes \qty{600}{\milli\metre} in diameter, but in the entire country stainless strip was rolled no wider than \qty{400}{\milli\metre}. They would have to be welded. The fitters refused: too thin. I established contact with the welding institute in Kyiv. I told Akinfiev that I needed a day for the trip. \textit{``Go, but without blather (!?)''} Naturally, I exploded. One might well have spat and refused -- it was not my duty. But there was no choice: the Department chief would be held responsible anyway. I went, inspected the welded samples, and brought back a labor contract for \num{150} membranes at three rubles apiece.

I did not handle matters for Unit~4. Someone arranged it through VNIIAES; they put it under a ``new technology'' initiative and the station paid forty-six thousand rubles. But they received their bonus for the ``new technology.'' Well, that at least compensated me for the gasoline: I had driven to Kyiv in my own car.

In various such cases, and there were many, I never wondered what I would have to endure for them. I quarreled more than once, unto death, with the deputy chief engineer Yu.\,A.~Kamenev. I do not know about Kamenev -- he is hot-tempered but quick to cool -- but I am certain that Akinfiev went to Briukhanov with the appropriate reports about me. It did not matter. Though sometimes I sensed sidewise glances, I cared little.

I always held: serve the work, not the man. The work will not deceive. True, even this I once doubted. In November 1986, after being discharged from the hospital, we arrived in Kyiv. The books in the new apartment lay in a heap; I began sorting them. Such anger seized me that I threw away all the technical literature. A twentieth-century Luddite\ldots\,They smashed machines; I fell upon books. Sensible. Yet doing everything correctly is rather dull.

Of course, I did not object to superiors thoughtlessly. There were avenues of retreat. First, I believed (however well-founded\ldots) that with my knowledge and attitude toward work I would always find employment. Second, I gave my salary to my wife, and she spent it as she saw fit, but back in Komsomolsk-on-Amur I had told her to keep three thousand rubles on hand. At that time it was enough to relocate if necessary. Within these two conditions lay my freedom. One must be prepared for any reaction from one's superior, up to dismissal. No freedom can exist if you constantly weigh on apothecary scales the consequences of your objections: seven rubles less in a bonus, a sanatorium voucher withheld\ldots

But that is in general. In particular, at the Chernobyl Nuclear Power Plant I do not know of a single case of an operator being dismissed, except for entirely indisputable failings. And even those can be counted on one hand. By ``operators'' I mean everyone from the shift supervisor upward.

Concerning April~26, 1986 specifically. It is entirely in vain that I.~Kazachkov and A.~Uskov torment themselves over whether they could or could not have violated the instructions.

There \textit{were} no violations!

As for the reduction of the reactivity margin, thanks to the designers we did not see it on April~26, and therefore neither Akimov, nor Toptunov, nor Dyatlov tormented themselves over it. I simply did not expect it before 01:30 a.m. And it would not have occurred with a negative power coefficient, as follows from the design documents and from the measurements of the station's Nuclear Safety Department. How were we to know that ``all the calendars lie''?

Again the same thoughtlessness when it does not concern one directly. Uskov is not an operator, though by position he was obliged to know the instructions. But Kazachkov's reply to Shcherbak's question is altogether incomprehensible:
\begin{quote}
--- ``Was the power increase the most fatal decision?''\\
--- ``Yes, it was a fatal decision\ldots''
\end{quote}

And further he says that had he done it, he would have understood and acknowledged the justice of the punishment. And Kazachkov would have done it, without a minute's hesitation, and would have violated nothing -- neither the Regulations nor the instructions. On what, then, is the accusation based? On post-accident knowledge? Yet this same knowledge shows that neither the power drop nor the subsequent increase, on a reactor fulfilling the requirements of the NSR and OPB, would have had any adverse consequences. We were hired to work on a \textit{normal} reactor.

Here it is -- the blinkered mentality. What has operator freedom -- economic or any other -- to do with this? Who is pressuring you here?

One cannot understand freedom as freedom from responsibility, from conscience. When speaking of specific people, base your statements on clear knowledge of circumstances. Otherwise you may voice only suppositions -- but say so plainly.

% --

For a writer, to remain confined within actual events is constricting, and thus the unfettered imagination begins piling ``details'' into a ``documentary'' tale, details that never existed. And Medvedev forgets that, when speaking of real people, one must observe at least elementary decency. He introduces conflicts: how else, for him, could there be a conflict-free tragedy? Yet this gives vividness to the narrative and fodder to critics. About oneself and others one learns from such works both what was and what was not. Perhaps, from a professional or critical point of view, I.~Borisova's analysis (``\textit{Oktiabr},'' No.~10, 1990) of the ``Chernobyl Notebook'' is fine, but it corresponds to reality no better than the tale itself.

Let us see what I.~Borisova writes, and compare it to reality:
\begin{personal}[From I.~Borisova's article in \textit{Oktiabr}; No.~10 of 1990:]
\textit{``But so long as a command pressed upon him (Dyatlov), he acted within it and pressed upon others just as he was pressed upon, transmitting this pressure. Before the explosion Dyatlov insisted on continuing the Turbogenerator rundown experiment, ignoring the emergency reality. After the explosion he fabricated the lie that the reactor was allegedly intact, for this lie was expected of him.''}

\textit{``In the Chronicle pressure bears down on Dyatlov, and Dyatlov bears down. A system of pressure unfolds like a fan. Dyatlov presses, Fomin presses, Briukhanov presses, Shcherbina presses\ldots\,The Chronicle records the `physiology' of pressure, its processes and reactions, observable and recordable. And those who press and those upon whom pressure is exerted -- in the end they are not divided into opposing camps.''}
\end{personal}

No, esteemed I.~Borisova, into ``camps'' they are indeed divided: Bryukhanov, Fomin, Dyatlov -- and Akimov, Toptunov, and Perevozchenko would have been there as well -- while the actual culprits of the tragedy, who hide themselves, including behind works like the ``Chernobyl Notebook,'' are free and smirking. You may speak of other ``camps,'' but my words should not be taken as a joke.

As for the pressure so advantageously described by Medvedev and repeated by the critic: no one pressed upon me, neither visibly nor invisibly. I am not one who yields to pressure. And I pressed upon no one, neither on April~26 nor earlier.

% --

My vocabulary contained no words such as ``Do as I say,'' nor anything similar. Persuasion with reference to instructions and technical facts -- yes; but not naked command. Perhaps there was, on my part, unintentional pressure due to broader knowledge (as Yu.~Tregub and I.~Kazachkov put it -- ``a head above the others''). Well, I was hardly going to pretend to be a fool. And on April~26 I convinced no one of anything, for not a single person had the slightest objection. Nor could they have had.

If upon the foundation of official lies Medvedev builds his own falsehood, then the critic naturally erects a third story of falsehood upon his.

\begin{personal}[From I.~Borisova's article in \textit{Oktiabr}; No.~10 of 1990:]
\textit{``Before the explosion Dyatlov insisted on continuing the turbine rundown experiment, ignoring the emergency reality.''}
\end{personal}

Already in May 1986, the G.\,A.~Shasharin commission established that at the moment the AZ-5 button was pressed, there were no warning or emergency signals. In January 1991 the N.\,A.~Shteinberg commission confirmed the same. Other commissions, for understandable reasons, do not emphasize this, but neither do they name -- nor can they name -- any signals of an ``emergency reality.'' On what grounds, then, would objections have arisen, or Dyatlov have felt any need to exert pressure?

To the third story belongs also:

\begin{personal}[From I.~Borisova's article in \textit{Oktiabr}; No.~10 of 1990:]
\textit{``After the explosion he fabricated the lie that the reactor was allegedly intact, for this lie was expected of him.''}
\end{personal}

Let that lie upon the conscience of the fairytale's author and its critic, that someone allegedly ``expected'' this lie of me. They say that a goose, molting and unable to fly, hides its head in a clump of grass when danger appears: it does not see, therefore it is not seen. Well, if you do not respect the people about whom you write, why reduce them to the level of geese? I am perhaps guilty that, amid the chaos, I did not explain to anyone that the reactor had perished and did not require cooling. I explained nothing even to Sasha Akimov. After the first inspection of the unit, I understood its utter hopelessness and simply told Akimov to stop the pumps that had been started immediately after the explosion at my own command\footnote{During the night, Dyatlov and Yuri Tregub went to survey the plant from the outside. Tregub recalled telling him \textit{``This is Hiroshima.''} to which Dyatlov replied, \textit{``Not in my nightmares have I seen anything like this.''}}. I considered Sasha a competent engineer, and for him my order to stop the pumps was clear. And I think he understood; his participation in attempting to feed water to the reactor is explained by the desire at least to \textit{do} something. As I already wrote, Bryukhanov and I had no conversation on this subject; I did not see N.\,M.~Fomin at all on April~26 and did not speak to him by telephone. Incidentally, Fomin did not forbid Yu.~Bagdasarov to shut down Unit~3, and no one forbade it after my order.

Writers think schematically: every drama must have heroes and villains. In the Chernobyl tragedy there were no villains among the actors present. They remained behind the scenes.

\chapter{The Trial}

The trial was a typical Soviet trial. Everything had been decided in advance. After two sessions in June 1986 of the Interdepartmental Technical Council under the chairmanship of Scientist A.~P.~Aleksandrov, where the employees of the Ministry of Medium Machine Building -- the authors of the reactor design -- predominated, a single, unequivocal version was proclaimed: the operating personnel were guilty. Other considerations, and they already existed at that time, were discarded as unnecessary. The subsequent decision of the Politburo in fact duplicated the conclusion of the Interdepartmental Technical Council, although it did note the shortcomings of the reactor.

After such a Politburo decision one would have to be utterly naive to hope for a favourable outcome. For our People's Court in 1987, the Politburo's decision would have been quite sufficient to condemn Jesus Christ himself for godlessness. And as for an ordinary person, there will always be sins to be found. Whether they are indictable or not -- what difference does that make. There only has to be a person -- the article (in the penal code) will be found\footnote{A saying popularized in the Soviet Union and in Poland in the period of the People's Republic of Poland, attributed to the Stalinist-era Soviet jurist Andrey Vyshinsky, or the Soviet secret police chief Lavrentiy Beria. It refers to the miscarriage of justice in the form of the abuse of power by the jurists, who could find any defendant guilty of ``something,'' if they so desired.}. This cynical saying began to circulate through the country courtesy of the NKVD and our so-called law-enforcement bodies. And the saying is by no means a tribute to eloquence, but an exact reflection of the real state of affairs.

It is appropriate here to speak of the article. I was convicted under Article~220 of the Criminal Code of the Ukrainian SSR for improper operation of explosive industrial facilities. In the list of explosive industrial facilities in the USSR, nuclear power stations are not mentioned. The forensic technical expert commission retroactively classified a nuclear power station as a potentially explosive enterprise. For the court this proved sufficient to apply the article. This is not the place to discuss whether nuclear power plants are explosive or not; to determine that retroactively and apply an article of the Criminal Code is plainly unlawful. But who is going to call the Supreme Court to order? There were those who could, and it acted under their instructions. Anything at all will be \textit{``explosion-hazardous''} if the design rules are not followed.

And then, what does \textit{``potentially explosion-hazardous''} mean? Soviet television sets reliably explode; every year several dozen people are killed. To which category are they to be assigned? Who is to blame?

A lawsuit over the deaths of television viewers would have become a stumbling block for the Soviet court. For all its desire, it cannot accuse the viewers themselves of sitting in front of the television without helmets and flak jackets. Accuse the factory? A state factory? That would mean the state is at fault. The Soviet state? The court could not survive such a perversion of principles. A person's guilt before the state -- yes. And if not, then nobody is guilty. For seventy years our courts have turned the screw in only one direction. For how many years now have we been hearing talk about the independence and autonomy of the courts, about service to the law and only the law. I keep waiting for the precedent when it is not a person who is found guilty in court, but the state. Only that is hardly likely to happen in the coming years. Until the mastodons raised on Vyshinsky's sourdough and that of his kind die out, there will be no change.\footnote{No real change will occur until a certain entrenched type of Soviet official disappears. Andrey Vyshinsky was Stalin's notorious Procurator General, architect of the show trials, famous for ruthless, pseudo-legal repression. ``Raised on his sourdough'' means formed by his methods, mentality, and moral atmosphere -- hardened, dogmatic, accusatory, and punitive.}

While I was in the camp, my wife went from one official and institution to another. There was nowhere she did not go! With great effort she even made it to the Chairman of the Supreme Court of the USSR, Smolentsev. This is how their conversation went:

\begin{personal}[From the conversation between Smolentsev and Dyatlov's wife:]
\textit{--- ``You mean to say you want this -- that others judged him, and I am to release your husband? That I should be the nice one?''}

\textit{--- ``No. I am in no way counting on kindness. I count only on justice. It is now known that the reactor was unfit for operation. And my husband is not guilty of that.''}

\textit{--- ``So you mean to say you want me to lock up Aleksandrov? A man that old?''}
\end{personal}

The natural continuation would have been: Dyatlov is younger, so let him sit. Such is the way the Supreme Judge talks with the wife of a convict, justifying the fairness of the sentence. As if over a cup of tea among acquaintances, to whom it is a matter of complete indifference who is behind bars.

\varthreestars

I spent half a year in the 6th Moscow Hospital and was discharged on 4~November 1986. I was afraid to be parted from the hospital, and not only because I had open, non-healing wounds on my legs, but mainly because for some incomprehensible reason liquid began to ooze in many places from my legs seemingly through intact skin. And how to stop it --- nobody knew. In fact, the doctors did not know either. Still, by trial and error with various medications they managed to stop it. But what was I to do?

As a rule, after discharge from hospital Chernobyl men were sent to a sanatorium for two or three weeks; I asked to be sent as well, so that in case anything happened I could go back to the hospital. They refused. I understood the reason a little later. It turned out that the investigators had already repeatedly sought my arrest.

On 5 November my wife and I arrived in Kiev. But it seems that something changed in the plans of the investigation: they let me live at home for a whole month. That was good. In a month I learned to walk a little. I managed to cut the time needed to walk around the block by ten minutes.

But then they stopped me. On 4~December I was moved, for almost four years, into state accommodation. There is nothing to describe here, it has all been described many times. My own immediate, most vivid impressions were these:

\begin{itemize}
\item The most rightless of all prisoners is the remand prisoner. Everything depends solely on the investigator, or, once the investigation is over, on the judge. There are no rules. And it is all the harder to endure because you have not yet been sentenced; no one has yet formally deprived you of your civil rights. But in fact you have been deprived of all rights; you can only ask. After the trial it becomes easier, both morally and physically. Your range of obligations and your modest rights are set out in the rules of internal order of the colony. And you are no longer confined to four walls; at least you can walk around the area. Each barrack in the zone is additionally fenced off -- that is the ``area.''

Of course I had noticed before that the leaves on the trees are green and rejoiced in them, especially in spring. The trial was held in Chernobyl from the beginning of July. For practically a year I had not seen any greenery. I no longer remember exactly; apparently it had rained and washed the trees. The foliage was such an emerald green, not a single yellow speck. If only one could walk up and touch it -- but there were the guards: a step to the left, a step to the right is counted as an escape\ldots~We do not value what is available to us, and only after losing it do we weep.

\item The \textit{etapy} -- the prisoner transports. For me they are a nightmare. No, I did not have to travel thirty men to a compartment, but fifteen is still too many, and everybody smokes. And the transports were not even long: Kiev-Poltava, Poltava-Kiev. True, one should not imagine that Kiev-Poltava means: you get on in Kiev and arrive in Poltava. In reality, for some reason they unloaded us in Sumy, although the train was going to Kharkov. Then they took us to Kharkov. The transit cells there are excellent -- there is not even one square metre per man. All three of my transports ended in a long illness.

\item And finally. I was deeply oppressed by the awareness that I had been put in prison. The very fact. Apparently I am an old-fashioned sort of person. Under Soviet rule people have been trained not to be ashamed of a criminal record. They grabbed and threw people behind bars for anything and for nothing. I shall not speak of the ``spikelets'' and Article 58\footnote{This refers to two notorious examples of such repression: The ``spikelets'' law (the 1932 law on the protection of socialist property) under which peasants could be sentenced to long prison terms for picking up leftover ears of grain (``spikelets'') from collective fields. Article 58 of the RSFSR Criminal Code, the sweeping clause covering ``counter-revolutionary activities,'' was used to imprison or execute people on the vaguest pretexts.}. Here is a recent example: a veteran of war and labour (his little cottage had collapsed) received an apartment in a multi-apartment building constructed by the collective farm. As always, after the builders there was devastation all around the house. The veteran put the plot in order, went to the kolkhoz, fetched himself a gloveful of clover seed and sowed the plot. Theft. Trial. In the courtroom the veteran flung a handful of orders and medals at the judges. And in the eyes of the people, and in his own -- is he a criminal or a victim of arbitrariness? And here are excerpts from the sentence: ``\ldots~committed theft without intent to appropriate or sell\ldots'', ``\ldots~sold to unidentified persons nine bottles of vodka at \num{15} rubles apiece, as a result of which he obtained unlawful income in the amount of \num{135} rubles\ldots''
\end{itemize}

The court does not know to whom, but he sold them for \num{15} rubles. And the income was \num{135} rubles, as though the vodka had been given to him free of charge in the shop. And to the three years' deprivation of liberty they added confiscation of his \textit{Moskvich} car as a means of transporting the vodka from the shop. The state itself has created vodka-fueled idiocy and has set an army of \textit{oprichniki}\footnote{The Oprichniki were the brutal, personal guard and enforcers of Tsar Ivan the Terrible in 16th-century Russia, created as part of the oprichnina, a special territory and political system he established.} upon the people. Well, we have strayed from the subject.

According to the medical certificate, I was not to be interrogated for more than two hours a day. In fact, the interrogations and familiarisation with the case files lasted six to eight hours. But this was not pressure on the part of the investigators; I myself wanted to bring things to clarity quickly. I was eager for the trial, but it was put off again and again. December, January and February passed for me in interrogations, familiarisation with the materials, and then pondering them. Because of that, detention in the remand prison did not weigh heavily on me. Later on I stopped understanding it. I did not represent any public danger. The investigation was over; I could not influence the witnesses' testimony. They changed their testimony at the trial without that, since by July 1987 many had realised that the indictment of the personnel was unlawful. The witnesses knew what measures were being taken to modernise the remaining reactors, reflected on it and drew conclusions. The modernisation undertaken was not adequate to the official version of the causes of the accident. I am still convinced that there was not the slightest reason to keep me under arrest until the trial. Yes, of course, at liberty I could have come to a clearer understanding of the causes of the accident. But would that really have harmed the truth? Only during the trial did I realise that the judge was in no way striving for the truth. He had no use for it at all.

Judge R.~K.~Brize announced that questions for the expert commission had to be submitted in written form. Under a dim bulb hidden behind a tight lattice, I wrote out twenty-four questions. Most of them were aimed at establishing whether the reactor complied with the nuclear safety documents: the NSR and OPB. The next day the judge, having evidently consulted the experts, rejected all my questions without giving reasons. Why? Very simply. Although the open trial was held in a restricted zone, there were operating personnel present in the courtroom, and they understood the requirements of the NSR and OPB, and that the reactor did not meet them. No, this would not, I am sure, have influenced the verdict in any way, but it could have caused some embarrassment, perhaps a great deal. It was much easier to act as all the commissions did and pretend that these documents did not exist at all, and that they did not know whether the reactor had to comply with them or not.

% --

The forensic technical expert commission answered the question whether the reactor could be operated by listing what the reactor had, what protections it possessed. And it came out to a lot and, at first glance, convincingly. The RBMK-1000 reactor is a complex device, the control and protection system is multi-element, the monitoring system is extensive. List everything -- and the list will look impressive. But the question is something else. What did the reactor \textit{not} have that was prescribed by the NSR? For example, in regard to the parameter of the operational reactivity margin: the reactor had neither automatic scram nor even an alarm. But if this parameter deviated, the reactor would explode when the AZ-5 operated automatically or from the button. That is exactly what happened on 26~April 1986.

The RBMK-1000 also possessed what no reactor at all should possess -- a positive prompt power coefficient of reactivity, which made it dynamically unstable. The commission of experts kept silence about this, and the court, by rejecting my questions, assisted them.

At the very first interrogation I pointed out how unfounded the accusation of the personnel was regarding a violation of the Regulations by blocking the AZ-5. You think that had any effect? Not in the least. At the trial the judge kept insisting on knowing who had ordered the protections to be disabled. And when Elshin and other witnesses answered that, in their opinion, under operational discipline, Akimov could not have disabled protections on his own, the judge drew the conclusion: Dyatlov gave the order -- he is guilty. But, first, since when has ``in one's opinion'' become evidence? Second, from the case files the judge knew that whether Dyatlov had ordered it or Akimov had disabled the protections on his own, there was no violation. And third, for the purpose of blaming Dyatlov this is quite enough, whereas for drawing the conclusion that the personnel had been trained to strict discipline it is not enough.

% --

Any testimony from witnesses in favour of the defendants found no understanding in court. Pressure from the prosecutor or the judge, or even from them in tandem, followed. Some withstood it (G.~A.~Dik, I.~I.~Kazachkov), some did not (Yu.~Yu.~Tregub). The judge was a member of the Supreme Court of the USSR, the prosecutor the head of the department of the USSR Procuracy for supervision of the courts; who could rein in their excessive zeal? What difference does it seem to make what G.~A.~Dik thinks of Dyatlov? Yet the prosecutor openly threatened to ``sort out'' where Dik worked when he expressed a favourable opinion of me. And it is obvious to everyone that a threat from a prosecutor with a general's rank and post is not an innocuous joke. Well, the prosecutor is the accuser; one can understand his zeal. But how is one to understand the judge's conduct?

True, A.~F.~Koni wrote that the prosecutor, in relation to the accused, must use both negative and positive material. But he was a tsarist jurist; he lived little in Soviet times and did not have time to be properly educated. And the Soviet jurist Yu.~N.~Shadrin, from among dozens of witness statements, selects one -- V.~I.~Fazly casts doubt on my professional competence. Shadrin exclaims pathetically that appointing Dyatlov was a strategic (no less!) mistake. Yet from that same testimony Shadrin, had he wished, could have drawn the directly opposite conclusion. The testimony contains the words: the personnel were competent and disciplined. For me that is the highest praise. My main task was to select and train the personnel. I myself did not perform operational actions and personally could have been anything at all.

One must give the head of the investigative group, Yu.~A.~Potyomkin, his due. Of course, he carried out and completed the task of charging the personnel. Had he not done so, someone else would have. At the same time, the case file contained materials not at all suitable for the prosecution, and they need not have been joined to the case. Those documents helped me to understand the causes of the catastrophe. And in fact, from January 1987 my views did not change in any way, and I have drawn practically nothing new from the subsequent reports. At that time I was interested only in the causes that led to the explosion, the first phase of the uncontrolled power surge. The subsequent second explosion and its causes are still of little interest to me. That is important for scientists; for an operating engineer, the task is to prevent even the first phase. From the materials in the case and my knowledge of the actual circumstances of the accident I constructed a picture of the explosion, to which most objective researchers are now coming. I am, moreover, one hundred percent certain that this picture emerged with perfect clarity in the minds of the reactor's creators -- the staff of the Kurchatov Institute and NIKIET -- immediately after the explosion; had they possessed even the slightest good faith, there would have been no fog for five years. Without the blessing of the luminaries of nuclear power -- A.~P.~Aleksandrov along the scientific line, A.~G.~Meshkov along the administrative one -- others would perhaps not have dared to lie so brazenly.

The reports by A.~A.~Yadrikhinsky, B.~G.~Dubovsky and, especially, the detailed report of the Gospromatomenergonadzor commission add nothing to my understanding, but they are important for another reason: it is not I, a former \textit{zek} (camp prisoner), who is speaking, but people whose sole interest is to establish the truth. What can one expect from me? It is only natural that I should seek to justify myself. And my words are met in advance with stored-up scepticism; people do not even bother to think about the meaning of what I say. And I have in fact still scarcely been able to say anything. Complaints (a prisoner does not write letters, he writes complaints) from the prison nobody read. Letters to newspapers and journals nobody printed.

The first complaint after conviction I addressed to M.~S.~Gorbachev as General Secretary. Naturally, I did not expect the letter to reach him, but I did not expect that it would be forwarded from the Central Committee to the Deputy Procurator General O.~V.~Soroka, who had approved the indictment. You could not invent anything better. I think the reply requires no comment.

In general, from behind bars one may write as many complaints and petitions as one likes without any result. Not acknowledging myself or the personnel as guilty of the explosion, I wrote, and my wife wrote and went about. She, unlike me, could go in person. I simply could not reconcile myself to spending the rest of my life behind bars. For someone else's sins. The examples given in the press drove one to despair and at the same time gave rise to hope. Take the case of the state farm director from Krasnodar Krai. In three years he and his wife wrote \num{375} complaints! And finally there was found a man (one must presume a black sheep in the well-drilled ranks of the procuracy) who read the case and lodged a protest. How many people were ``checking'' these complaints? Replies came without fail to them all. And then one wonders whether to write or not to write. It depends on many reasons.

Imprisonment changes a person's psyche. All of prison life says: \textit{``Abandon hope, all ye who enter here.''} I would suddenly catch myself in fright that I had lived several days regarding that life as normal. Not that it was normal, but that I simply had not given a thought to its abnormality. That swamp can quite well suck one down, and then every desire to do anything for one's release is paralysed. And whether you are ten times innocent, nobody in the law-enforcement bodies will remember you.

And this in spite of the fact that not for a minute did I consider myself guilty. While I did not yet understand the causes of the explosion, I did not blame anyone by habit, without first sorting things out, and once I did understand -- all the more so.

This, although I constantly received letters -- besides those from my wife -- from relatives, acquaintances, and fellow students from the institute. They were constantly stirring my mind. For a camp prisoner everything is bad. There are letters and visits -- that is bad; there are none -- even worse. After a visit from my wife it took me several days to recover. And without it, I do not even know how I would have felt. Fortunately, I was not deprived of visits.

In a word, they would not let me fall asleep and forget that the natural state of a person is to be free. But even so, as I understood, my complaints, wherever they were sent, produced no results. Only a person at liberty can by his efforts shift something. My walker on the outside was my wife. She went to the most unimaginable instances, appealed to various people. In the end it had an effect; I was released. Many interceded for me; I will name only one -- the late Andrei Dmitrievich Sakharov. I am deeply grateful to them.

We must state the fact: I would have been imprisoned in 1987 under any ruler. But I was released only under M.~S.~Gorbachev. Under another ruler -- hardly. And yet I cannot bring myself to offer thanks. Not because it is now customary to bay at rulers, but because I cannot imagine how this could be done. Am I to say thank you that they kept me in prison for only four years for nothing, and not for the ten I was given? That would be an obvious equivocation.

But the procuracy and the court are not harsh with everyone. This is what is said in a ruling of the procuracy and, similarly, of the court:

\begin{personal}[From the ruling of the Procuracy of the Ukrainian SSR dated 11 December 1986:]
\textit{``With regard to the officials of the Chernobyl Nuclear Power Plant and of the city of Pripyat, responsible for organising civil defence, labour protection and safety, as well as with regard to officials of the design and construction organisations who failed to take proper measures to improve the control and protection systems of reactor installations with RBMK-1000 reactors, on 11~December 1986 criminal cases were severed out into separate proceedings.''}
\end{personal}

Note how gently this is formulated -- they \textit{``failed to improve''} the reactor protection. For such formulation their monthly bonus will certainly be withheld. What sort of trial can there be -- they failed to improve it? But it could not have been formulated otherwise in relation to the designers. If one were to call things by their proper names: they cobbled together a good-for-nothing reactor and a protection system that defies criticism, then what is left for them to try us for? So they issued a ruling to throw dust in people's eyes.

% --

No trial ever took place, nor was one envisaged from the outset. One cannot try a designer for failing to improve the reactor and its protection. One cannot try him for \textit{``shortcomings,''} for \textit{``peculiarities.''} One can only try for non-compliance with the requirements of the normative documents. And it is that which all the commissions, including the forensic technical commission, the procuracy and the court, carefully avoided. It was not out of na\"{\i}vet\'e that the judge rejected my questions at the trial; at the same time, all the questions of the other defendants, formulated in sufficiently harmless form by their defence lawyers, the judge graciously accepted. And the commission gave written answers. No, one can clearly see that the judge (I do not speak of the commission) understood the essence of the case perfectly well. And just as clearly opposed any attempt to elucidate it.

When I was head of the Department, my deputy was Tolya Sitnikov. Most often he went to meetings of the party committee, the trade union committee, the personnel department; I frankly avoided such gatherings. Tolya would come back and grumble -- they make him draw up some lists. I would ask:

\begin{quote}
--- ``And did you grumble there?'' \\
--- ``What for? I have to do it anyway.'' \\
--- ``Well, so what? At least have a bit of a say. And they will think twice next time before shoving some unnecessary nonsense at you.''
\end{quote}

In virtue of this trait of character, even though the judge rejected my questions, I would have forced him either to ``shut me up'' by depriving me of the floor or to compel the commission to answer: ``How does the reactor comply with the requirements of the NSR?'' For example, such as these:

\begin{itemize}
\item When the operational reactivity margin parameter deviated, the reactor exploded. According to Clause 3.18 there should be an alarm, both emergency and warning. There was none.
\item According to Clause 3.3.21, for the same parameter the reactor must automatically shut down. There was no such signal.
\item According to Clause 3.3.26, the AZ-5 must reliably and rapidly shut down the chain reaction when the button is pressed. And it was precisely from the pressing of the button that the accident began.
\end{itemize}

This would have come in handy when, after my question, the brazenly impudent expert from NIKIET, V.~I.~Mikhan, answered that the RBMK complied with the NSR.

But at that time my physical state was entirely unsatisfactory; I spoke with difficulty, the right side of my head was splitting apart. My repeated requests to be seen by a dentist were ignored. After the court hearings I would rush (as much as I could) from the ``Voronok'' to the remand prison to get some kind of pill.

People familiar with such things will smirk -- yes, rush. Just you try. In my case ``rush'' is not fast. Secondly, to the remand prison -- and I had already been living for eight months with that same warrant officer.

Only after the end of the trial, when I had been brought back to Kiev, to Lukyanovka prison, did the governor send for me, and after our conversation asked what requests I had. I asked to be sent to the dentist. He rang immediately, and in an hour I could already speak; my head had returned to normal. And all that for the loss of one tooth.

On Thursday, 27 September 1990, in the evening I was sitting in the library reading a finished article for \textit{Ogonyok} about the interview with A.~P.~Aleksandrov in that journal (never printed by the free press). In came Vitya Chistyakov, the zone's radio operator and projectionist. He said that the radio had announced my release. Others later said the same. And on Friday the colony governor, V.~P.~Khizhnyak, sent for me and told me the same. Then I believed it.

Naturally, the governor cannot release me on the basis of reports and telephone calls. Every camp prisoner is numbered, entered into the inventory, and stitched into the ledgers. You can be released only on the basis of documents. And the governor, as far as possible, shortened my time in the zone and phoned my wife.

% --

So as not to frighten off capricious fortune, I showed no outward signs, did not pack my things and did not even allow any thoughts inside. As the saying goes: ``We are not superstitious, but why take risks.'' The only thing I did was to hold a farewell tea. Think what you like about it, but observe the ritual. Well, I had enough tea, some food, cigarettes. It was all right. Pity there was nothing to drink. In general, I would advise future camp prisoners: if you land in prison, in the zone, accept the rules of behaviour. And not outwardly, but inwardly. Pretence will not do. Camp prisoners in ordinary-regime colonies often think in rather primitive terms (in the reinforced and strict-regime colonies people are more serious), but in some ways they are subtle. However innocent you may be, if you are in the zone, then you are a prisoner. Nobody cares about your misfortunes. Everyone feels that his own sentence is heavy enough. Do not flaunt your intellectual superiority, even if it is not imaginary. That will help you avoid unnecessary conflicts both with the prisoners and with the administration.

On Monday I went, as usual, to the post office in the settlement for the newspapers and journals. I ran into the dentist, Anatoly Danilovich, and he pointed to his folder and said: ``I've brought your release.'' Here my heart gave a little jump. I brought the papers back and did not distribute them by barrack as usual. I threw my documents, of which I had accumulated a fair number, into my rucksack, and a few books. Then they came: ``With your things -- to the gate!'' But this is not the shout that drives you out on \textit{etap}, neither in tone nor in substance, although the words are the same. Vitya Chistyakov carried my rucksack to the gate, and then, as soon as I stepped out of the zone, there were my wife and Slava Orlov standing there. That is freedom!

I changed my clothes and took with me as a memento a shirt with its tag and the headgear with the romantic name ``piderka.'' Three years and ten months out of my life --- down the drain. They took my health; that seemed too little for them.

\chapter{The Investigation}

Commissions and collectives of authors who have written on the accident in technical journals, however much they may diverge on secondary particulars, have in essence two common features:

\begin{itemize}
\item they unanimously acknowledge the culpability of the operating personnel, and treat that culpability as unquestionable and exclusive;
\item they with equal unanimity and persistence avoid any serious appeal to the principal regulatory documents on reactor nuclear safety -- the NSR and OPB.
\end{itemize}

Since the reactor did in fact explode, certain faults on its part must, of course, be conceded; and so they are, by way of concession, under the modest headings of ``deficiencies'' or ``peculiarities''. Everyone is prepared to admit that nothing in this world is perfect, and that in any reactor one may detect one or another imperfection which, in the natural course of events, will be gradually corrected. They remind us that even on the sun, it seems, spots have been discovered; yet no one, on that account, hauls it down from the sky -- it continues to shine and to warm. To be blond is also, after all, a ``feature'' of a human being; whom, and how, does this hinder? In this fashion public opinion has been soothed and lulled, and even many specialists have been persuaded. The commissions are, nominally, different; yet they all speak in a single register. Their membership is imposing: scientists, doctors and professors; ministers and deputy ministers; institute directors and other officials far from insignificant. All this lends an outward semblance of an objective investigation.

It must, in fairness, be noted that not every collective laid the blame upon the operating personnel.

% --

A group of scientists headed by Ponomarev-Stepnoi concluded that the reactor exploded as the consequence of a large positive steam reactivity coefficient. The commission created by decision of the State Committee for Science and Technology (GKNT) likewise speaks of a large steam effect and adds that with such a steam effect the reactor experienced power excursions even under MPAs (maximum design-basis accidents), which ought not to occur at all. That commission further established the impermissible fact that the Emergency Protection introduced \textit{positive} reactivity and stated that such a protection system does not fulfil its basic function -- to shut down the reactor. Possibly the commission considered (and in that it would be in the right) that what it had already noted sufficed to recognize the reactor as unfit for operation. In its report, however, no such conclusion is formulated. In any case, that unquestionably competent commission did not answer the concrete question of whether the reactor, in the form in which it existed in 1986, complied with the requirements of NSR and OPB. Quite incomprehensibly, it fails to see, in the very properties of the reactor which it itself describes, any contradiction to the normative prescriptions governing reactor design. Having timidly remarked that AZ does not fulfil the requirement of point 3.3.5 of NSR -- to shut the reactor down under any normal and accident conditions -- the commission immediately proceeds to call that requirement itself into question.

These groups express no opinion on the personnel. Nor do they pronounce upon the admissibility or inadmissibility of operating such a reactor. They pose as dispassionate investigators. Even this, however, would be of some use, \textit{were} the investigation truly objective -- which, alas, cannot be said of the GKNT commission's report.

% --

For years specialists have walked around the regulatory documents on reactors. The NSR and OPB are treated as though they simply did not exist. Well, not quite simply. A large group of authors in the journal \textit{Atomic Energy} writes that the NSR and OPB, after the accident, were carefully reviewed and found in the main to satisfy safety criteria. Foreign scholars write the same, for example Professor A.~Birkhofer of the Federal Republic of Germany. Yet all this has no direct bearing upon the RBMK reactor in its 1986 configuration. These authors have not the slightest idea whether that reactor was required to comply with those Rules or with some other; they do not even pose the question.

Why, then, do these gentlemen -- scientific and unscientific alike -- shy away from answering a most natural inquiry? To say plainly that the reactor did not comply with the Rules they cannot bring themselves. Still more obviously, they do not wish to lie outright. That is the entire explanation.

In the main, so far as factual material is concerned, the investigative documents are accurate: crude falsification is comparatively rare. Only the informants to the IAEA permitted themselves undisguised fabrication, and they probably counted on the fact that such liberties would never become known to Soviet citizens. The conclusions drawn from the facts, however, are not the correct ones; the facts are placed in the wrong light. For that there is always the convenient formula: ``I was mistaken,'' if things should, so to speak, come to the crunch.

% --

The admission of ``deficiencies'' and ``peculiarities'' in the reactor itself obliges no one to anything. There have been such things and there will be. We shall eliminate them. The sun has spots and yet continues to shine. But to admit that the reactor design failed to satisfy explicit requirements of the regulatory documents -- that is already a matter for judicial scrutiny. It is therefore judged more prudent to remain silent. Others will be imprisoned? That is their affair. Such is the moral practice.

It may seem that whether this is acknowledged or concealed makes little difference; the technical measures actually implemented on the remaining reactors do not change from this. Yet the difference is very substantial. RBMKs are not the only type of nuclear reactor, and to allow similar errors in other designs would be not merely negligence, but stupidity of the most inexcusable kind. Designers are at work on new reactors. They require clarity. Approaches to the problems of safety in nuclear engineering have their analogues in other branches of the national (or what is it now to be called?) economy. There too clarity is needed -- at least that much. We are not yet ripe for justice, and it is doubtful whether we truly desire it. As S.~Esenin says: \textit{``Calm yourself, wayfarer, and do not demand that Truth which you do not need.''}\footnote{Sergei Alexandrovich Yesenin, sometimes spelled as Esenin, was a Russian lyric poet. He is one of the most popular and well-known Russian poets of the 20th century.}

Some do not need the truth because they are shielding their own hides. Others, it may be, would be glad to speak but, after the Politburo's decision, are afraid. For those decisions are always correct. And even when they are not, even when that is tacitly acknowledged, it must still not be said aloud. Collective wisdom.

% --

Of late, however, everything has become strangely confused. Suddenly, from the rostrum of the XXVIII Congress of the Communist Party there issues a statement that, it appears, the Politburo can also err. They say: \textit{``We failed to sort it out properly.''} This is untrue. Individual persons -- by no means the best informed, and without access to all the materials (of which more below) -- nevertheless contrived to sort things out and to understand the true causes of the catastrophe, while the all-powerful body, in whose hands lay all the scientific resources of the country, allegedly could not. Nor can I accept those assertions now frequently encountered in the press that they were all fools. I am convinced there were no fools among them; they did not inherit Politburo seats by right of birth. Anyone you like -- but not fools. Unlimited power and complete absence of accountability permitted the Politburo never even to reflect upon the justice or correctness of its decisions. Whatever it decided was thereby just; whatever it decided was thereby right. Hence it made such decisions as it found convenient, without the least regard for actual circumstances. They \textit{``failed to sort it out''} for the simple reason that they neither tried nor wished to sort it out. They had no need. The admission made from the Congress rostrum was extorted from them, pressed out under the influence of public opinion in view of the gravity of the accident's consequences. Had it not been for those consequences -- had it been a matter merely of the accident's \textit{causes} -- no such statement would ever have been made; everything would have remained as it always had. Upon this calculation, in essence, the investigators proceeded: the apex and the base would be bound together by a common false version of the catastrophe's origin, and no one would reach the truth. Those who might seek it could easily have their oxygen cut off in some convenient fashion under such an alliance.

Perestroika has not been ruinous in all things; there has been some benefit as well. Even in earlier years there were people who did not fear to go against the current. They were quickly flattened, rolled out and dispatched who knows whither. The same phenomenon is observed now -- as we learn from the press. But the authorities no longer possess the same unlimited possibilities for arbitrary repression. And solitary enthusiasts have not disappeared.

First of all, one must name here an employee of the Kurchatov Institute, V.~P.~Volkov, of whom I have already spoken. From the very beginning he was convinced that the sole cause of the reactor explosion lay in the wholly unsatisfactory properties of the reactor itself. Of course, he was far from alone in thinking so. But others thought and kept silent, whereas V.~P.~Volkov made his way as far as the head of state. Beyond that, there is only recourse to the Almighty.

Next there are two noteworthy reports, again by solitary individuals. Their principal distinction from the mass of other scientific reports is that they attempt to apportion the measure of guilt borne by the personnel and by the reactor's designers. In the others only the causes of the explosion are mentioned: large steam reactivity effect and defective AZ. This is never concretely connected with the physicists and designers. It is as though these properties had somehow arisen in the reactor of themselves. Or, better still, they are silently shifted onto the operators, as if the latter had configured such a core and devised such control rods, and not others. It is said that under such-and-such operating conditions (a nod in the direction of the operators) the negative properties manifest themselves with special force -- as though it were not clear to those saying this that under no operating or accident regime may any property of the reactor culminate in an explosion.

% --

The first of these reports to appear was the study by Professor B.~G.~Dubovsky, ``On the factors of instability of nuclear reactors, using the RBMK reactor as an example''. B.~G.~Dubovsky was head of the national nuclear safety service of the USSR from 1958 to 1973, and he knows the RBMK not at second hand. Already in the 1970s he made proposals to improve the safety systems of precisely these reactors.

In his work the defects of the reactor's control and protection system are examined and explained in detail. They are as follows. The active core is seven metres high; consequently, quasi-independent \textit{``reactors''} can arise in its upper and lower parts. All the Control and Protection System rods participating in Emergency Protection, however, are located above; and when a local reactor forms in the lower part of the core, the neutron absorbers reach it with great delay. The Control and Protection System of the RBMK included, in addition, so-called shortened absorber rods. These are situated permanently in the lower part of the core or withdrawn below it; they can therefore reach the bottom region quickly. But, as Professor Dubovsky writes:

\begin{personal}[From a report by B.~G.~Dubovsky:]
  \textit{``Because of a gross, utterly illogical miscalculation in the safety design, these shortened rods were not connected to the general emergency protection signal AZ-5, which excluded their rapid introduction into the volume where the uncontrolled zonal reactor arose in the lower part of the core -- the most dangerous region from the viewpoint of reactor runaway.''}
\end{personal}

% --

In the lower part of the core a local reactor was created not by technological circumstances; it was created by the Control and Protection System itself. Owing to the inhomogeneous structure of the rods (absorbers, graphite displacers, water columns), when the rod is in its upper position there remains, in the lower part of the channel, a water column \qty{1.25}{\metre} high. Replacing these columns by the graphite displacer, which absorbs neutrons much less strongly, creates the local reactor.

\textit{``The presence of water columns beneath the graphite displacers constituted a second gross miscalculation in the design of the Control and Protection system.''} B.~G.~Dubovsky comments on this phenomenon as follows:

\begin{personal}[From a report by B.~G.~Dubovsky:]
  \textit{``To our great regret, the hazardous pre-accident situation, after pressing the AZ-5 button -- an action performed on the order of the shift supervisor to shut down the reactor -- passed into the first stage of the accident process, caused by the runaway of a zonal uncontrolled reactor formed in the lower part of the core (just think: pressing the emergency protection button AZ-5 -- the salvation button -- causes the reactor to explode).''}
\end{personal}

In this same context a third fundamental design miscalculation manifests itself, both in the AZ rods and in all the other absorber rods: the low speed of their insertion into the core, with an unbelievably long time of full insertion -- \numrange{18}{20} seconds.

% --

While a reactor with high excess reactivity has thus been created in the lower region and its neutron power has begun to increase sharply, the absorbers are still far away. During their descent neutron power has time to transform into heat (for specialists: the thermal time constant of the fuel rods is about \qty{10}{\second}). At this stage the steam reactivity effect begins to operate: the water in the process channels turns into steam, which again increases the reactivity and amplifies the rise in power. The spike in neutron power can lead to boiling of water in the Control and Protection System channels as well, further increasing the reactivity. Such are the characteristics the designers chose for the reactor.

\begin{personal}[From a report by B.~G.~Dubovsky:]
  \textit{``The choice of such extremely unfortunate -- in essence, highly dangerous -- physical characteristics, especially for reactor operation at low power levels, was apparently made with a view to achieving more advantageous economic indicators.''}
\end{personal}

Having convincingly demonstrated the failure of Emergency Protection and of the entire Control and Protection System, the professor is persuaded that it was precisely this system, acting together with the large positive steam reactivity effect, that blew up the reactor of the fourth unit on 26~April 1986.

It is not only a matter of conviction; there are many who are convinced. It is also a matter of B.~G.~Dubovsky's active civic stance. Here is an excerpt from his letter to M.~S.~Gorbachev, written after the formation in the Supreme Soviet of a commission to examine the causes and consequences of the Chernobyl accident:

\begin{personal}[From a letter by B.~G.~Dubovsky to M.~S.~Gorbachev dated 27 November 1989:]
\textit{``The continued unjust shifting of responsibility onto the Chernobyl personnel rules out further development of nuclear power -- it is impossible in the future to exclude human error. The violations committed by the personnel, given even a minimal compliance of the reactor's safety system with its purpose, would have been limited to a one-week outage. Command-administrative pseudo-science has misled the people, the Academy of Sciences, Scientist Sakharov, the Supreme Soviet.}

\textit{I request that I be given the opportunity to explain to the environmentalists of the Supreme Soviet the true causes of the accident of the Chernobyl reactor and the necessary measures to ensure safety.''}
\end{personal}

In speaking of a one-week outage due to personnel violations, the professor is most likely paying homage to the charge still hanging over the personnel. In reality, with a normal protection system, there might at most have occurred an unplanned shutdown, without any destruction. After its modernization the RBMK is, in effect, a new reactor, substantially different in its level of safety from the earlier one. The measures taken do not fit the official version, which lays the principal blame upon the personnel; that blame is clearly exaggerated. A correct appraisal of the errors and miscalculations of both the personnel and the reactor's creators is necessary; only thus can a healthy psychological climate be established in the collectives of the nuclear power plants, in their families, and among the population of the regions adjacent to the stations.

For fourteen years B.~G.~Dubovsky headed the national nuclear safety service; he repeatedly took part in investigating accidents associated with spontaneous chain reactions -- such as occurred at Chernobyl. He knows whereof he speaks:

\begin{personal}[From B.~G.~Dubovsky:]
\textit{``The advisability of conducting a renewed investigation, apart from achieving a clearer understanding of the scientific and technical miscalculations made, is due to the fact that immediately after an accident some participants who have committed errors deliberately distort the circumstances that led to the accident; in some cases a group conspiracy is possible.''}

\textit{``Does it accord with the principles of humanity to mention by name those leaders involved in the emergence of the accident who have already passed away (Feinberg, Kunegin) or have become honored, merited pensioners (Alexandrov, Dollezhal)? It seems that, given the tragic consequences of the accident at the fourth unit of the Chernobyl Nuclear Power Plant, it is precisely considerations of humanity that demand a refusal of anonymity -- in the name of the memory of the dead and of justice toward the victims, and, which is very important, in order to prevent the emergence of new large-scale accidents.''}
\end{personal}

These are just words, but they are just words, founded upon knowledge of the subject under discussion. Nothing stated in B.~G.~Dubovsky's work can be refuted; one can only say: that is exactly how it was. There is merely one clarification connected with a practical circumstance of which B.~G.~Dubovsky was unaware. The professor notes that experiments involving a change of reactivity greater than \qty{0.5}{\betaeff} (equal to five rods) should be carried out only after a stationary xenon regime has been attained and with a large and approximately constant reactivity margin at a power level above \qty{30}{\percent}.

The assertion is debatable, but for the RBMK reactor one can, in the main, agree with it Proceeding from these considerations, he arrives at the conclusion:

\begin{personal}[From B.~G.~Dubovsky:]
\textit{``The main fundamental miscalculation committed by the Nuclear Power Plant personnel was the extremely unfortunate, incompetent choice of the time for carrying out a consciously hazardous experiment -- at a time of strong reduction of the reactivity margin due to rapid radioactive buildup of the strongest neutron absorber, xenon.''}
\end{personal}

Without pretending to any special erudition, I can say that the computations usual for an operator were fully within my powers. The reactivity changes associated with the experiment arose only during the start-up and shutdown of the main circulation pumps as a consequence of changes in the steam content of the coolant. According to a certificate issued to us by the Nuclear Safety Department, the steam reactivity effect amounted to \qty{+1.29}{\betaeff}. From this it follows that, when four of the eight pumps are shut down, one cannot obtain a reactivity change exceeding two control rods. B.~G.~Dubovsky evidently had in mind a steam effect of \qtyrange{5}{6}{\betaeff}, as was assumed in post-accident analyses.

Pump shutdowns -- and shutdowns much faster than in the experiment -- are possible in normal reactor operation, for example when a power bus section trips. Does this mean that under such conditions one could not work at all? In fact, that is how it was; but the experiment has nothing whatever to do with it.

% --

\section{The Report of A.~A.~Yadrikhinsky, Inspector of Gosatomenergonadzor at the Kursk Nuclear Power Plant}

In the work of A.~A.~Yadrikhinsky, for the first time the question is raised explicitly of the compliance of the RBMK reactor, in its 1986 configuration, with the principal regulatory documents on nuclear safety -- the OPB and NSR. True, from the work itself it is clear that this is not the first such document; but the earlier ones are unknown to me.

Regulatory documents contain the necessary and sufficient requirements for the design, construction, and operation of reactors and power units. Each requirement of the Rules is binding; if any requirement is not fulfilled, safety must be justified, confirmed by analysis, and agreed upon with the specified authorities. For the RBMK reactor no deviations were declared, no separate agreements were made. From this the authors draw the entirely natural conclusion that the RBMK was presumed to comply fully with these documents.

Doubts on this point had arisen earlier; but only reflection on the accident and subsequent calculations revealed to the operating personnel the true essence of the RBMK reactor. As A.~A.~Yadrikhinsky notes, a list compiled at the Kursk Nuclear Power Plant identified not fewer than thirty-two deviations of the RBMK from the requirements of the NSR, OPB, and the \textit{``Rules for the Design and Safe Operation of Nuclear Power Plants''}. Obviously, not all of these deviations bore upon the accident of 26~April 1986; but those that were ``active'' on that date are more than sufficient -- more than a dozen -- and this the report demonstrates convincingly.

% --

The isolation of specific clauses of normative documents that the designers failed to fulfil -- documents invested with the force of law -- is crucial. It excludes arbitrariness in interpretation and places upon a legal foundation the question whether the reactor could or could not be operated safely. If the reactor did not comply with legal requirements, then the responsibility lies with its creators, and the forcing of such a reactor into operation is \textit{ipso facto} criminal. This is stated directly in the NSR, as in other norms:

\begin{personal}[From the NSR]
\textit{``Those found guilty of violating the `Rules' shall be held administratively or criminally liable in accordance with applicable legislation.''}
\end{personal}

Doubts concerning the objectivity of the investigation -- an investigation conducted principally by the designers of the reactor, the potential culprits -- rendered an independent inquiry indispensable. On the basis of such an inquiry, A.~A.~Yadrikhinsky arrives at the conclusion that \textit{``the designers must be the defendants, not the plaintiffs, as is the case now.''} He even names the concrete persons:

\begin{personal}[From A.~A.~Yadrikhinsky:]
\textit{``There are few real, presently active culprits of the Chernobyl accident. They are Scientists A.~P.~Alexandrov, I.~A.~Dollezhal -- the leaders of all work on the RBMK reactors; Corresponding Member I.~Ya.~Emelyanov -- leader of the work on the Control and Protection system; and the Chief State Inspector for Nuclear Safety of the USSR N.~I.~Kozlov -- who recognized the RBMK reactor as nuclear-safe, very well knowing that this was not the case.''}
\end{personal}

They must, he maintains, be held liable in accordance with the NSR, which they themselves drafted. Not being a prosecutor or judge, I shall not presume to pronounce upon the degree of responsibility of the persons indicated, but their culpability is, to my mind, beyond doubt.

% --

The point, however, is not criminal liability. More than five years have passed since the accident. And our laws are harsh with some, overly indulgent with others. One recalls, for example, the Minister of Internal Affairs of Uzbekistan, Yakhyaev: while committing crimes he received decorations; afterwards those same decorations became indulgences for his misdeeds.

But the truth must be established, and the unfounded accusations against the personnel must be withdrawn.

This will not be easily achieved. The lie persists with an impudent and unwavering gaze. The Ministry of Atomic Energy has issued a list of deviations, coordinated with the Chief Designer and Scientific Leader, in which the principal violations of NSR are wholly absent. You will not find there points 3.2.2, 3.3.21, 3.3.26, 3.3.28, which I have cited already. It is difficult to conceive how such violations could be overlooked or denied. But as long as the same persons direct affairs, there will be no progress.

At a meeting of the Supreme Soviet commission -- to which I was invited to describe the circumstances of the accident -- the deputy director of NIKIET, Yu.~M.~Cherkashov, began to inquire whether I was physically present in the main control room at the moment of the power drop, as though the behaviour of physical phenomena depended upon my presence. Dolgov, of the court-technical commission, asked:

\begin{personal}[From Dolgov in the court-technical commission:]
\textit{``Why, in violation of the experiment program, was the protection on trip of two turbogenerators disabled?''}
\end{personal}

Even though there is not a word about protections in the program, and he knows this. These participants do everything possible to drown the investigation in superfluous details which have no bearing on the essence. True, the commission chairman quickly grasped the situation. But I was there only twice; they were permanent participants. I cannot and will not anticipate the commission's decision. I have no confidence that the pressure exerted by the official accusation of the personnel will be overcome. Yet I think the ``deficiencies'' of the reactor will at last be called by their proper name. If inadmissible vices in a reactor are ``deficiencies'', then pregnancy discovered in a vestal is likewise a ``deficiency''; and from history we know how vestals with such a ``deficiency'' were treated -- they were immured alive.

However accurate, forceful, and resolute the reports of A.~A.~Yadrikhinsky and B.~G.~Dubovsky may be, they remain self-initiated; no official organization commissioned them. Therefore the official institutes can pretend that they do not exist. In the Soviet Union there is no organization that has an interest in establishing that the RBMK reactor does not comply with the regulatory documents.

Be assured: had the reactor complied with design norms, the Kurchatov Institute and NIKIET would have produced within a week an exhaustive report demonstrating this in the minutest detail. Neither the Ministry of Energy nor its successor, the Ministry of Atomic Energy, needs such a demonstration.

The right and obligation to establish the reactor' compliance with the norms belongs directly to the supervisory body -- Gosatomenergonadzor. Yet having failed to perform its supervisory duties before the accident, this body -- or, more precisely, the persons staffing it -- opposed such an examination for several years after the accident as well. Though styled ``State'', it was not state in essence. Before 1984 it was, indeed, merely a subdivision of the Ministry of Medium Machine Building -- a pocket supervisory authority. Although formally invested with broad powers, in practice it did not dare to exercise them. Not even after the accident. Repeated requests by the supervisory body that the Scientific Leader and Chief Designer furnish a safety justification for the RBMK reactor were simply ignored.

In 1986, when the question arose of restarting Unit~1 of the Chernobyl NPP, which had been shut down after the accident, the matter of providing safety-justification materials arose anew. To this the Scientific Leader, A.~P.~Alexandrov, who was present, replied:

\begin{personal}[From A.~P.~Alexandrov:]
``What further justifications do you need, if here I -- I myself -- say: the reactor is safe, start it up.''
\end{personal}

And by decision of the Government Commission they did start it up. The self-assurance and habit of command on the part of A.~P.~Alexandrov are understandable. Less understandable are the actions of the supervisory body and of the entire commission: A.~P.~Alexandrov had made no shortage of declarations concerning the reactor's ``reliability'' even before the accident, and yet, behold, they worked -- and yet again.

Only after a change of leadership did the supervisory body at last venture to examine the reactor's conformity to the legal norms of design.

In 1990 a meeting of the Scientific-Technical Section was held, to which representatives of various organizations were invited. The report of A.~A.~Yadrikhinsky was examined. I shall not speak here of questions concerning the course of the explosion or the quantity of radioactive material released; no single individual is capable of resolving that problem definitively. Nor do I believe that a precise, complete picture of the explosion will ever be constructed. It is not our present concern. We need only to understand the beginning, and the path leading to that ill-fated beginning.

At the meeting, for the first time, a commission drawn from various organizations recognized that a large number (more than twenty) of articles of the NSR and OPB had been violated. Only the representatives of the Chief Designer dissented. This is already alarming. Such persons must be removed from reactor design, whatever their motives -- whether they cannot perceive these violations, or perceive them yet deny them. Both conditions are intolerable.

\section{The Report of N.~A.~Shteinberg}

A commission under N.~A.~Shteinberg was appointed by Order No.~11 of the Chairman of Gosatomenergonadzor, dated 27~February 1990, and in January 1991 issued its report.

The commission studied dozens of documents on the reactor's design, post-accident analyses and research, and the accident documentation. In my opinion (for I have not seen everything), it is the most objective and complete report available, embracing various aspects of the reactor and the causes that led to the catastrophe. In its reconstruction of the events of 26~April it does not resort to conjecture; its conclusions are drawn almost exclusively from documents.

My own view -- the view of a witness to the events -- coincides with the report's account of the final minutes before the explosion. I consider it necessary to quote this account, preserving all numerical data. To specialists in nuclear power it will be of certain interest; others may pass it over.

\begin{personal}[Quote from the report of the commission under N.~A.~Shteinberg]
``Thus, before the start of the tests, the parameters of the core had determined an increased susceptibility of the reactor to a self-runaway process in the lower part of the core. The commission considers that such a state arose not only because there was an increased, compared to usual, coolant flow through the reactor (under the influence of the operation of eight, instead of the usual six, MCPs, the increased flow impedes steam formation), but primarily because of the low value of the reactor power.

\underline{Similar thermohydraulic parameters can arise during any reactor shutdown for refuelling (underlined - A.D.).}

The initial state of the unit immediately before the tests at 01:23 a.m. was characterized by the following: power -- \qty{220}{\mega\watt}; reactivity margin (value obtained after the accident using the PRIZMA-ANALOG code for the state at 00:22:30 a.m.) -- 8~(RR) rods; the axial power profile was double-humped with a maximum at the top; coolant flow -- \qty{56}{\meter\cubed\per\hour}; feedwater flow -- \qty{200}{\tonne\per\hour}; thermophysical parameters close to stable. The management of the unit shift judged the preparations for the tests to be complete, and after the oscillograph had been switched on, an order followed to close the stop-control valves, which were closed at 01:23:04 a.m.

Both in this period and during the approximately \qty{30}{\second} of rundown of the four Main Circulation Pumps, the unit parameters were confidently controlled, lay within the expected limits for this regime, and did not require any action by the personnel.

However, use of the Emergency Protection of a reactor of this design, under conditions of the allowed reduction of the reactivity margin, either on emergency signals or manually after the tests had been completed, was already impossible without damage to the core, apparently beginning from 00:30 a.m. on 26 April 1986, which needs to be verified by additional studies.

\textbf{4.6.2.\quad Period of the tests according to the program.}

The tests that began at 01:23:04 a.m. caused the following processes in the reactor plant:

The Main Circulation Pumps powered by the slowing turbogenerator No. 8 (Main Circulation Pumps 13, 14, 23, 24) reduced their speed and decreased their capacity. The other Main Circulation Pumps (11, 12, 21, 22) slightly increased theirs. The total coolant flow decreased. Over the \qty{35}{\second} of the transient it fell by \qtyrange{10}{15}{\percent} from the initial value.

The reduction in coolant flow caused a corresponding increase in steam content in the core, which was counteracted to a certain (small) degree by the pressure rise due to closure of Turbogenerator No. 8's stop-control valves.

Mathematical modelling of this stage of the process was carried out by Soviet and American specialists. It showed good agreement between theoretical predictions and the actually recorded data.

Both calculations showed that the released void (steam) reactivity was insignificant and could be compensated by a small insertion of the Automatic Regulator rods into the core (up to \qty{1.4}{\metre}). During the rundown of Turbogenerator No. 8 there was no increase in reactor power. This is confirmed by the DREG program, which from 01:19:39 a.m. to 01:19:44 a.m. and from 01:19:57 a.m. to 01:23:30 a.m., i.e.~before the tests and for a substantial part of the test period, recorded the signal ``PC -- UP'', under which the Automatic Regulator rods cannot move into the core. Their positions, last recorded at 01:22:37 a.m., were \qtylist{1.4; 1.6; 0.2}{\metre} for 1AR, 2AR and 3AR respectively.

Thus neither the reactor power nor other reactor plant parameters -- pressure and level in the drum-separators, coolant and feedwater flows, and others -- required any intervention by either the personnel or the protection systems in the period from the start of the tests until the pressing of the AZ-5 button.

The commission did not identify any events or dynamic processes -- for example, an unnoticed onset of reactor power runaway -- that could have served as the initiating event of the accident. It identified only the existence of a sufficiently long-lasting initial state, under which, if a positive reactivity arose for some reason, a power increase could develop under conditions in which the reactor's Emergency Protection might no longer have acted as a protection.

\textbf{4.6.3.\quad Development of the accident process.}

At 01:23:40 a.m. the AZ-5 manual emergency shutdown button was pressed by the senior control engineer of the reactor.

The commission did not succeed in reliably determining for what reason it was pressed.''
\end{personal}

The AZ-5 Emergency Protection button serves alike for an emergency SCRAM and for the customary, orderly shutdown of the reactor. In the present instance it was pressed for the latter purpose, at the conclusion of the work.

The commission then proceeds, on the basis of post-accident calculations, to set forth its judgment concerning the inception of the catastrophe:

\begin{personal}[Quote from the report of the commission under N.~A.~Shteinberg]
\textit{``Thus, the results of the analytical calculations, performed four years after the accident by those organizations most competent in the physics of reactors -- NIKIET, VNIIAES, Institute of Problems of Energy, and the Institute for Nuclear Research of the Academy of Sciences of the Ukrainian SSR -- demonstrated the possibility of a perilous rise of power in the RBMK-1000 reactor, with a multiple increase in local energy release within the active zone, owing to the insertion of the Emergency Protection rods into the core.''}
\end{personal}

Accordingly, from what has here been adduced, the initiating event of the accident was the pressing of the AZ-5 button under the conditions that had arisen in the RBMK-1000 at low power and with the Control and Protection System manual regulating rods withdrawn from the core beyond the admissible limit.

What the authors of the report emphasize requires clarification. That the AZ button \textit{ought not} to have been pressed (we have lived to see such a pronouncement); that operation \textit{ought not} to have been conducted at low power; that the Control and Protection System rods \textit{ought not} to have been withdrawn beyond the permissible number -- all this is, taken in itself, correct. Yet every one of these ``ought nots'' became known only \textit{after} the catastrophe. One of the commission's most important conclusions, sound in its technical essence, is nevertheless framed in a manner that casts the burden of guilt upon the personnel.

According to the \textit{``General Provisions for Safety''} (OPB):

\begin{itemize}
\item An initiating event is a single failure in systems, an external occurrence, or an erroneous action by personnel that disrupts normal operation and may lead to a breach of the limits and/or conditions of safe operation.
\item An erroneous action by personnel is an unintentional, incorrect, single action committed by personnel in the performance of their duties.
\end{itemize}

First. The actuation of AZ-5 by button can under no circumstances be regarded either as an error or as a violation of nuclear safety. In no textbook on reactor engineering, in no document pertaining to the RBMK, is there even the faintest suggestion that Emergency Protection \textit{must never} be actuated. On the contrary, it is everywhere affirmed that Emergency Protection is required to shut down the reactor under all normal and all accident conditions. For that reason this action cannot be deemed the initiating event.

In the work of A.~A.~Yadrikhinskiy there appears a conclusion, formulated with deliberate paradox:

\begin{personal}[From A.~A.~Yadrikhinsky:]
\textit{``The critical error of the operating personnel in the Chernobyl accident is the pressing of the AZ-5 button (a correct operational action) while operating the reactor with a reactivity margin of less than \num{15} rods.''}
\end{personal}

Yet the introductory phrase -- \textit{``a correct operational action''} -- restores the matter to its proper proportions. And the paradoxical form is justified, given so unnatural a protection system.

Second. The matter of operating at low power has already been explained: there is here no violation of any kind.

Third. From the commission's formulation -- \textit{``withdrawing manually regulated Control and Protection System rods from the reactor in excess of their permissible number''} --- it might be inferred that the personnel violated some mandatory requirement. This is incorrect. Nevertheless, the formulation is already being disseminated. At the A.~D.~Sakharov memorial congress Professor A.~Birkhofer of Germany declared outright that in withdrawing rods \textit{``in excess of the permissible number''} the personnel committed a violation.

I cannot speak to the motives -- different people will have different motives -- yet even specialists unbound by the original falsehood, and who now understand the nature of the reactor, nevertheless cannot wholly relinquish the notion of guilt on the part of the personnel. What is this? The weight of the official accusation still bearing down upon them? Or the simple fact that RBMK reactors did work for some time? In principle such a fact ought not to influence a specialist.

Or perhaps the old proverb is at work: I do not know whether he stole someone's fur coat or someone stole his, but he is mixed up in something unclean\footnote{When a person is surrounded by suspicious circumstances, guilt or innocence almost does not matter -- he is involved in something dubious either way}. One feels the ``spirit'' of the document. Usually, after a close reading, the author's position becomes evident. In the report of the N.~A.~Shteinberg commission, I discerned no desire to accuse the personnel at all costs, although the reactor was defective -- and without acknowledging that, no report today would be taken seriously. Yet something still presses upon them, and I shall offer a few observations on the report.

The commission established the reactor's nonconformity with nine articles of the NSR and six articles of the OPB, yet refrained from drawing the evident conclusion: that operation of the RBMK under such conditions was unlawful. One must recall that the commission consisted of employees of the supervisory authority. If they do not say so, who will? Are they awaiting a confession from the reactor's designers? They will wait in vain; no one doubts that.

It would be instructive to know who is to safeguard the personnel of the plants, and the population of the country, from defective reactors if not the body whose very purpose is to perform this duty.

A sentence in the report's conclusion reads:

\begin{personal}[From the report of the commission under N.~A.~Shteinberg]
\textit{``The RBMK-1000 reactor, with its design characteristics and constructional features as of 26 April 1986, possessed such serious nonconformities with the requirements of safety norms and rules that its operation became possible only under conditions of an insufficient level of safety culture in the country.''}
\end{personal}

A well-turned phrase, yet possessing no legal force whatsoever. The concept of a \textit{safety culture} belongs to the moral sphere and may serve well enough in scientific discourse, whereas the binding norms and rules in force in the country clearly stipulate responsibility for their violation, extending to criminal liability.

Thus the question remains: was the operation of that reactor -- designed in violation of fifteen articles of regulatory documents, the violations concealed by its authors -- lawful or unlawful? The authors of the report do describe a violation by the personnel: operation of the reactor with a reactivity margin of less than \num{15} rods. Yes. But, first, that was not a violation; it was an error, an oversight. Not without reason do the accusers of the personnel insist on the word ``violation.'' The measurement system for this parameter was feeble; yet still, they say, the personnel ``knew,'' because there was a printout. Although all, or almost all, knew that no such printout existed, and that it was impossible to determine the reactivity margin from it. The commission established that no such printout existed. It likewise established a design violation: the absence of an alarm and of automatic emergency shutdown upon deviation of this parameter. The personnel were not furnished with the monitoring and automatic systems required by law. Who, if not the supervisory authority, ought to defend the law? Or is the personnel guilty in every case?

Essentially the same is stated in the abstract of the report:

\begin{personal}[From the report of the commission under N.~A.~Shteinberg]
\textit{``On the basis of an analysis of the results of domestic and foreign studies, design data, and regulatory-technical documentation, it is concluded that the Chernobyl accident, which began due to the actions of the operating personnel, acquired catastrophic dimensions disproportionate to those actions as a result of the unsatisfactory design of the reactor.''}
\end{personal}

Assuredly, the reactor was exploded by the operating personnel --- no one else worked upon it. The commission clearly understands why the reactor exploded. It clearly understands that the personnel's actions, had the reactor complied with design norms, would not have produced \textit{any} accident, let alone a catastrophe. Consequently, the commission has no grounds for asserting that the accident \textit{began} because of the personnel's actions.

Nevertheless, despite a number of imprecise (though not, I think, prejudiced) formulations, the report ought to exert a salutary influence upon the investigation of the accident's causes. Above all, because it officially recognizes that the RBMK reactor, in its 1986 incarnation, failed to conform to the fundamental requirements of the regulatory documents on nuclear safety. Yet it seems that a protracted struggle (I use the word with reluctance) still lies before us in the pursuit of truth. The opposing side is numerous and remains influential.

\chapter{Accident or Inevitability?}

The earliest written judgment on the causes of the Chernobyl catastrophe known to me belongs to V.~P.~Volkov, an employee of the Kurchatov Institute for Atomic Energy. In a letter of 1~May~1986 addressed to the Director of the Institute, A.~P.~Aleksandrov -- and, after a characteristically ``appropriate'' response to that letter, when Volkov's access to the Institute was revoked, in a further letter to M.~S.~Gorbachev -- he advanced the following view:

\begin{personal}[From V.~P.~Volkov's letter to M.~S.~Gorbachev dated 12 May 1986:]
\textit{``The accident was caused not by the actions of the personnel, but by the design of the core and by an incorrect understanding of the neutron-physical processes occurring within it.''}
\end{personal}

The immediate causes he identified were: the introduction of positive reactivity upon actuation of AZ owing to a design defect of the control rods, and the large positive steam-void coefficient of reactivity. This was the first, and entirely correct, diagnosis -- a diagnosis that received open, official recognition only five years later. It was not the product of intuition, but of concrete knowledge of the RBMK's properties -- knowledge of whose unsatisfactory character he had long, and unsuccessfully, attempted to bring to the attention of the leadership, including A.~P.~Aleksandrov. From the Central Committee Volkov's letter was transmitted to the State Atomic Energy Supervision Authority, where a commission under the chairmanship of V.~A.~Sidorenko in effect acknowledged the correctness of the causes he named.

Such a version could not, at that time, be permitted recognition. It was altogether too inconvenient for too many. Soon there appeared the report of the interdepartmental commission headed by the Deputy Minister of Medium Machine Building, A.~G.~Meshkov. Strictly speaking, the commission was ``interdepartmental'' only at the moment of its appointment; thereafter it in effect became a commission of Ministry of Medium Machine Building -- that is, of the reactor's creators -- for the report was refused a signature, owing to its manifest bias, by the Deputy Minister of Energy G.~A.~Shasharin, the Director of VNIIAES S, A.~A.~Abagyan, and the Chief Engineer of Chief Administration of Nuclear Power Units, B.~Ya.~Prushinskiy.

According to the act of that commission, the reactor explosion occurred as a consequence of steam blanketing of the core caused by the rundown of the Main Circulation Pumps. This ``rundown'' -- that is, a supposed cessation of coolant supply to the core -- was said to have arisen from a mismatch between the feedwater flow rate and the coolant flow delivered by each pump.

For many years I dealt professionally with enterprises and organizations of the Ministry of Medium Machine Building. The personnel there are competent; order and discipline are superior to those in many spheres. But their self-regard and self-confidence are also great. Rarely, yet even so, defects in their work did sometimes occur. These were eliminated, to be sure, with all possible speed -- yet always in such a way that no trace remained in the form of acts or protocols. Everything was recorded internally, under strict secrecy. I do not dispute the facts.

It emerges that the ministry was aware of practically all the defects of the RBMK, and that only the wholly inexplicable inaction of its leadership prevented their timely correction. That same V.~P.~Volkov did, after all, reach Aleksandrov. To whom else could he have appealed? Had he attempted to appeal through other channels, still greater obloquy would have fallen upon him. His appeal to M.~S.~Gorbachev was admissible only because the accident had already occurred; yet even so they continue to gnaw at it. Had he appealed before the catastrophe, such filth would have been poured upon him that he would never have washed it off.

The Meshkov commission operated according to fixed stereotypes, wholly ignoring the scale and character of the catastrophe. ``For public consumption'' it drafted an act convenient to itself, while internally, among themselves, it took the measures it judged necessary. This is evident from the list of technical measures produced by the Kurchatov Institute for Atomic Energy and by NIKIET immediately after the accident for the surviving RBMK reactors. A review of these measures shows that the creators of the reactor at once comprehended perfectly why the explosion had occurred, yet concealed -- and continue to conceal -- the truth for reasons readily understood, though wholly alien to both science and ethics.

G.~A.~Shasharin, having refused to sign the act, established within the Ministry of Energy a commission which, with the participation of other specialists -- in particular OKBM (designer of the main circulation pumps; the I. I. Afrikantov Experimental Design Bureau of Mechanical Engineering) and the All-Union Thermal Engineering Institute -- prepared an addendum to the act.

This document, in its essence, was correct; it relied on calculations by VNIIAES, on objective data from the monitoring system, and on the testimony of eyewitnesses. It further demonstrated the groundlessness of the Meshkov commission's principal conclusion regarding pump rundown. In my judgment, its defect lies in its enumeration of all factors influencing the accident's occurrence -- both essential and incidental -- thereby obscuring the central mechanism. Yet this document might have formed the basis for a correct conclusion. It was ignored.

Then, on 2 and 17~June~1986, under the chairmanship of A.~P.~Aleksandrov at the Interdepartmental Technical Council, the reactor was declared sound, the personnel guilty. They were at fault. And thereafter followed the report of the Government Commission and the decision of the Politburo.

For a thrice-decorated Hero of the Soviet Union, for the President of the Academy of Sciences, and so forth, it was no great labour to select the requisite documents and cast aside the inconvenient. A consummate switchman, he shifts the points with unfailing precision: Who is guilty? -- The personnel. Did Aleksandrov participate in the investigation? -- By no means. Who created the RBMK reactor? -- N.~A.~Dollezhal; Aleksandrov, the inventor and Scientific Director, bears no relation to it. Who was dispatched to Chernobyl after the accident? -- Legasov, though in truth he had no connection with it. This is not a question of medals; in that queue Aleksandrov would have been the last.\footnote{Aleksandrov had the authority and prestige to shape the record, highlighting documents that absolved him and suppressing those that implicated him. Dyatlov is not speaking of honours or awards. He means that in the moral line-up of responsibility, Aleksandrov stands at the end, trying to ensure that others take the blame before him.}

It must here be emphasized: at the moment when the AZ-5 Emergency Protection button was pressed, the monitoring system recorded not a single signal -- neither emergency nor warning -- regarding deviations of reactor parameters or of its systems. \textbf{The reactor was blown up by the emergency protection itself -- sheer nonsense!} The reactor's design violated more than thirty articles of the regulatory documents governing reactor construction.

The establishment of any one of these three propositions -- and they are irrefutable -- would suffice, in any normal human society, to remove all accusations from the personnel. But that is in a normal society. And here?

Thereafter, ``pump rundown'' disappears entirely as a causal explanation -- it is far too compromising. The Soviet informants to the IAEA expounded the runaway phase in an incoherent and mutually contradictory manner. A new formula was accordingly devised:
``Thus, the primary cause of the accident was an extremely improbable combination of violations of operating procedures and modes committed by the plant personnel.''

These informants further introduced into circulation the expression ``a non-regulatory state of the reactor.'' I have already explained, in detail, these so-called ``personnel violations'' and this ``non-regulatory state.'' I shall merely cite here the judgment of other specialists who have no desire to falsify.

\begin{personal}[From G.~A.~Shasharin in Novyi Mir; No.~9 (1991):]
\textit{``It has already been indicated in the press that the operators allegedly disabled Emergency Protection altogether. This is untrue.''}
\textit{``Some publications place emphasis on other supposedly erroneous actions of the personnel. None of these plays any role in the occurrence, still less in the development, of the accident. Reducing power to \qty{60}{\mega\watt} is likewise not prohibited by the Regulations, although this circumstance, as noted above, proved extremely adverse''}
\textit{``To this I add: such adversity exists only on such a reactor, under such an emergency configuration.''}
\end{personal}

Yet such tandems are tenacious, especially those pleasing to the ear. Here is an excerpt from a letter of 26~March~1990, signed by the First Deputy Director of the IAE, N.~N.~Ponomarev-Stepnoi, by the Director of NIKIET, E.~O.~Adamov, and by the Director of VNIIAES, A.~A.~Abagyan:

\begin{personal}[From the letter of N.~N.~Ponomarev-Stepnoi and E.~O.~Adamov and A.~A.~Abagyan:]
\textit{``The accident occurred as a result of bringing the reactor into a non-regulatory state caused by a number of factors, the principal among which are: reduction of the operational reactivity margin below the regulatory value; the small degree of subcooling of the coolant at the reactor inlet. Under these conditions there manifested themselves the positive steam-void coefficient of reactivity, the deficiencies in the design of the control-and-protection rods, and the unstable spatial form of the neutron field that arose in consequence of the complex transient regime. The accident concluded with the reactor surging on prompt neutrons.''}
\end{personal}

Let us examine this brief excerpt with due care. It warrants attention, above all because the men who penned it stand at the helm of the nation's nuclear enterprise -- their positions speak for themselves. One must also attend to the date: this is no document of 1986, when confusion might still have been excusable.

\begin{itemize}
\item \textit{Non-regulatory state. Small operational reactivity margin.} Yes: most likely so it was. I say ``most likely'' because the reactor's designers have, to this day, not produced a calculation of the margin at the instant the Emergency Protection button was pressed. Yet that is precisely the essential point. From the authors of the letter one will await in vain any explanation of \textit{why} this margin had grown so small. It grew small because of the absence of automatic Emergency Protection actuation, and of any signalisation of deviations of this parameter -- in direct violation of NSR requirements -- for which the designers, not the operators, bear responsibility.

\item \textit{Non-regulatory state. Small coolant subcooling at the inlet of the core.} And why should it be otherwise? At low power it is always small, and in 1986 \textit{all} power levels employed lay within the Regulations. Should the authors intend to suggest that subcooling was small because coolant flow was excessive, they err. Moreover, at the moment the button was pressed -- the moment from which all else proceeded -- the coolant flow was \textit{nominal}. And in general, for a boiling reactor, the subcooling may assume any value permissible under the Regulations.

Did the positive steam -- void coefficient manifest itself under these conditions? First, it was not merely a positive void coefficient; it was a \textit{wholly unacceptable magnitude} of the positive void coefficient. This had already been discussed in 1976 at the Scientific - Technical Council of Ministry of Medium Machine Building, which resolved that it must not exceed~$\beta$. They forgot this entirely sound decision, returning to it only after the catastrophe. And why should it not have manifested itself under other conditions? Upon rupture of the coolant circuit, rupture of the steam line, sticking of the main safety valves, rundown of the main circulation pumps?

\item \textit{Deficiencies in the design of the control-and-protection rods?} How long are these wholly impermissible defects to be coyly termed ``deficiencies''? These ``deficiencies'' manifested themselves earlier too, as shown by the appearance of overpower and rapid-increase signals when Emergency Protection actuated on other triggers. They were merely not comprehended. Providence alone spared us, until 26~April, from an accident involving a large prompt positive reactivity insertion; with such Control and Protection System rods an explosion would have been inevitable. As though the authors of the letter were unaware that such a design is intolerable.

\item \textit{And what has the power-field shape to do with the matter?} After reducing the power of a xenon-poisoned, steady-state reactor, the field assumes precisely such a form. This follows naturally from the physics. If the field shape is unsuited for shutting down the reactor, shall we refine the field, or shall we contrive a \textit{proper} protection system? If the reactor's ``non-regulatory state'' is not signalled by \textit{any} indication? And none existed. If the positive void coefficient upon dehydration of the core attains \qtyrange[range-phrase=\text{--}]{5}{6}{\beta}, which is unacceptable for many independent reasons? If the Emergency Protection introduces positive reactivity upon actuation, and the conditions under which this occurs are not prevented by \textit{any} technical means?
\end{itemize}

If the neutron-field configuration that naturally arises at every shutdown of the reactor is ``unacceptable'' -- then can such a reactor be operated at all? Manifestly not; and this is tacitly acknowledged by the subsequent reconstruction of the remaining reactors.

Such a stance on the part of the leaders of the nation's principal nuclear institutions is, in truth, alarming. If even after four years they decline to acknowledge the truth forthrightly, striving instead to veil the causes of the catastrophe, can one hope for honest information regarding present operations? For what reason do they refuse to acknowledge the plainly evident miscalculations of the physicists and designers? The formulation adopted by these directors is no slip, no inaccuracy, but a deliberate position.

It was precisely the ignoring of obvious facts that led to the Chernobyl catastrophe. Two examples will suffice.

After the 1975 accident at Unit~1 of the Leningrad Nuclear Power Plant, involving rupture of a technological channel, the staff of the Kurchatov Institute for Atomic Energy issued recommendations for improving the nuclear safety of RBMK reactors:

\begin{enumerate}
\item Reduction of the void coefficient of RBMK reactors by:
\begin{itemize}
  \item increasing fuel enrichment and density;
  \item reducing the quantity of graphite in the core;
  \item retaining displacement pieces (elsewhere termed additional absorbers) in the core;
  \item increasing the reactivity margin.
\end{itemize}

\item Modification of Control and Protection System rod design by lengthening the absorber section and introducing independent control of power release by height and by radius -- that is, ensuring that radial regulation does not distort the axial power field.

\item Creation of a fast-acting Emergency Protection system.
\end{enumerate}

Later, at the Ignalina Nuclear Power Plant and during the physical startup of Unit~4 of the Chernobyl Nuclear Power Plant in 1983, the insertion of positive reactivity at the initiation of Control and Protection System rod motion into the core was discovered. The reactors were nonetheless approved for operation. The languid correspondence between the Scientific Director and the Chief Designer regarding elimination of this impermissible phenomenon accomplished nothing before 1986. Only after the catastrophe were these measures at last implemented. They waited for it.

\varthreestars

Without clear comprehension of what occurred, there can be no hope that such a calamity will not recur. Yet what comprehension is possible when, at the Institute for Atomic Energy, on the very day of the informational briefing, after the proclamation of the official version, even the \textit{asking} of questions was forbidden? It is no accident that from the creators of the reactor -- the IAE and NIKIET -- not a single document exists that analyzes the conformity of the RBMK-1000 reactor to the principal normative documents whose requirements are mandatory for designers. And more than that.

After the publication in 1989 of A.~A.~Yadrikhinsky's independent report, ``The Nuclear Accident at Unit~4 of the Chernobyl Nuclear Power Plant and the Nuclear Safety of RBMK Reactors,'' prepared after extensive analytical work, a meeting was convened at State Committee for Supervision of Nuclear Power Safety (Gosatomenergonadzor). There, representatives of NIKIET persisted in refusing to acknowledge the violations, in the RBMK design, of the requirements of NSR and OPB -- requirements of a fundamental and compulsory character. Thus: all acknowledge them, save they. The requirements are explicit, and in the reactor they were plainly not fulfilled. Such a state of affairs offers little ground for optimism concerning the future, for these same hands continue yet to ``create.''

Yadrikhinsky cites a document compiled at the Kursk Nuclear Power Plant listing thirty-two points of NSR, OPB, and the ``Rules for the Safe Design and Operation of Nuclear Power Plants'' violated by the creators of the RBMK. In January 1991 the commission of State Committee for Industrial and Nuclear Power Oversight, chaired by N.~A.~Shteinberg, published its report, identifying fifteen articles of NSR and OPB whose violation had direct bearing on the 1986 accident. Any half of these would suffice to deem the reactor \textit{unfit for operation}.

Every product must be designed and manufactured in accordance with standards and norms. This is natural for a reactor as well -- only here the norms are incomparably more stringent. NSR and OPB stipulate that any departure must be justified, nuclear safety assured, and the deviation coordinated with the appropriate authorities. \textit{None} of this existed for the RBMK-1000 reactor. An impartial commission of supervisory personnel identified fifteen violations in the RBMK-1000 design directly pertinent to the accident of 26~April~1986. Therefore, there can be no talk of ``accident'' or ``coincidence.''

\textbf{The catastrophe was inevitable.}

The reactor possessed a positive prompt power coefficient of reactivity, rendering it dynamically unstable. The positive reactivity effect upon dehydration of the coolant circuit amounted to \qtyrange[range-phrase=\text{--}]{5}{6}{\beta}. Post-accident calculations demonstrate that the reactor would explode in a broad array of scenarios.

Upon actuation of the Emergency Protection -- whether automatically or manually, as on 26~April~1986 -- the system introduced, over the course of five (!) seconds, a positive reactivity of up to~$\beta$. Such a circumstance defies normal human understanding. Specialists comprehend that, for such a reactor, catastrophe was \textit{inevitable}.

Throughout operation -- beginning with the startup of Unit~1 of the Leningrad Nuclear Power Plant in 1973 -- there were repeated manifestations of unsatisfactory, nuclear-hazardous characteristics of the reactor. Unacceptable properties of the Control and Protection System were likewise identified. These perilous peculiarities were understood by scientific staff; yet over many years, up to the accident itself, all proposals remained unimplemented. Because the Scientific Director and the Chief Designer disregarded the identified dangerous and impermissible properties of the reactor, its catastrophe was inevitable.

The brief survey in this chapter of the versions advanced by the reactor's creators regarding the causes of the Chernobyl catastrophe -- together with the excerpt from the letter of the heads of the Kurchatov Institute and NIKIET -- shows that even after a long interval they are unable to acknowledge their mistakes, striving continually to mislead the scientific community, including the world community. They misinform the public. Making use of their monopoly on the accident materials, they have succeeded in doing so for many years. Such a situation not only offers no guarantee against a future recurrence, but evokes direct apprehension. What is needed are people capable of \textit{creative thought}.

\chapter{On the use of Nuclear Energy}

It is impossible to refrain from offering some judgment on the admissibility or inadmissibility of constructing and utilising nuclear power plants. Once, when asked about my attitude toward the future of nuclear power in light of the Chernobyl catastrophe and my own personal tragedy, I answered that at sixty years of age one does not change one's God. Perhaps. Yet the gods themselves are not immaculate. How much evil has been wrought upon the earth in the name of Jesus and of Allah. True, all this was done by men; but if God is omnipotent, why did He permit ignoble deeds to be committed in His name?

When, after illness, I began gradually to recover, a profound inner disarray set in. It seemed that my entire life had been devoted to a worthless cause; and had I only been indifferent or careless in my work, the loss would not have been so great. Yet for me, work had always stood foremost. Like the overwhelming majority of people (I am convinced of this), I wished simply to live, to work, and to receive enough for my work to live decently. And, as it seemed, I had everything: a family, various interests without excessive passions, and a salary respectable by our standards. I never entertained extravagant notions about some lofty destiny inherent in every human being, nor did I suffer from any sense of inferiority because of its absence. Yet when it seemed that my life had been lived in vain, it became very, very bitter. But is everything truly so hopeless?

Is it possible, at the present stage, to dispense with nuclear power? Yes, it is possible. Will it be better? I am far from certain. Advocates of nuclear power assert that it is indispensable. But why? In 1990 the nuclear power plants of the Soviet Union generated \num{211}~billion~\unit{\kWh} of electricity. To produce the same amount at thermal power stations would have required \numrange{50}{60}~million~tons of coal. With a national output of \num{700}~million tons of coal, such a quantity does not seem unattainable. Nor would the additional burden upon the environment be great, if one considers all forms of fuel consumption. And burning fuel in properly designed thermal plants does far less harm than its combustion in primitive boilers and household stoves. Yes and no.

It may seem that these additional thermal stations would not greatly increase the already existing complex of pollutants (power plants, motor vehicles, diesel locomotives, metallurgy, chemicals, and so forth), yet even the present burden is excessive; we ought not to augment it, but diminish it.

The generation of electricity at nuclear stations may well be one of the means for reducing the environmental load. A good nuclear power plant is ecologically cleaner than any thermal plant, even one burning natural gas. The absence of visible smoke from the latter's chimney should not deceive; it expels gases in enormous quantities and continuously. Even at stations with RBMK reactors, fish were always caught in the cooling pond. Fishing at Chernobyl was forbidden, but not on account of radioactive contamination of the fish; rather, to prevent accidents to the fishermen -- they drowned, and one man cast a wire line onto a power-transmission conductor.

By the 1980s the RBMK reactor appeared, in truth, an archaic structure. Its design corresponded in many respects to plutonium-production reactors. Such reactors have no place in world power engineering. In the literature one sometimes encounters the assertion that the RBMK was driven by cheapness. Not so. The capital cost per installed kilowatt at RBMK stations was one and a half times higher than at stations with pressure-vessel reactors; operating expenses were comparable. Their use at power plants was not dictated by technical considerations.

Once the authorities resolved to develop nuclear power, an unbearable pace was imposed upon industry. In the Soviet Union -- perhaps not quite as in China -- the authorities, too, had a taste for leaps. Thus the decision to employ RBMKs turned into a long jump. Even then it was understood that pressure-vessel reactors would, in the long run, provide higher reliability. But to fulfil the adopted programme, industry could not supply the necessary quantity of large equipment. Therefore, part of the plants were to be built with RBMK reactors. The authority, in nuclear affairs, of Ministry of Medium Machine Building and of Scientist A.~P.~Alexandrov also played its role.

A small anecdote. In 1991 a meeting convened in Kyiv to discuss A.~D.~Sakharov's idea of locating reactors deep underground. And whom did they appoint Scientific Director of the topic? The very same Alexandrov. See how the flywheel had gathered momentum: what inertia! How could they proceed without a ``great luminary'' (as we used to say in our student days)? This is no longer hypnosis; it is psychosis.

The RBMK reactor is not hopelessly bad. Although in some respects it contradicts, fundamentally, the safety concept. For instance, it is impossible to enclose it in a containment vessel at acceptable cost. In 1986 it was hopeless on account of miscalculations by physicists and designers, not because of its inherent structural arrangement. Medvedev writes in his fairytales that the positive steam-void reactivity effect is the most significant in uranium-graphite reactors. No: in uranium-graphite reactors, as in others, it may have any magnitude and either sign -- positive or negative. Everything depends upon the composition of the core's components. Proper calculation and adjustment by experimental data, without significant expenditure, would have allowed the RBMK to be made entirely safe in this respect. But the Control and Protection System rods -- that was sheer blunder. No additional funds whatever were required to make acceptable -- that is, proper -- rods.

Already at the very beginning of operation of Unit~1 of the Leningrad Nuclear Power Plant, clearly negative and dangerous properties were revealed. Specialists of the Kurchatov Institute understood them and issued correct recommendations for their elimination. I have no information why Scientist Alexandrov did not give effect to them. And God was merciful -- He granted a full decade for taking the necessary measures. Indeed, now, knowing the reactor's properties as they were before 1986, one should not marvel at the explosion, but at the fact that it had not occurred earlier. And those specialists ought not to be reproached for not insisting more forcefully upon the modernization required; they had nowhere to turn. To the authorities -- pointless: they listened to Alexandrov. To the press -- not a single newspaper would have published it. At best they would have been expelled from the institute. Even after the catastrophe itself, V.~P.~Volkov was pushed into disability for his statements.

Frankly, it is frightening that the ``masters'' in the institutes are the same individuals who conceived not a reactor, but a monstrosity. It is difficult to trust these people when, in 1991, they collectively state in the Report I have quoted that the reactor's Control and Protection System met all the requirements for such a system.

The future of RBMK reactors? There is none -- unequivocally none. The question can be posed only thus: Is their operation permissible until their mechanical resource is exhausted, or must they be shut down immediately?

I cannot answer that question. Enough irresponsible pronouncements have been made already; I do not intend to add my own. I believe no single person can answer it. True, after modernization the reactor possesses, in many respects, acceptable properties meeting necessary safety criteria. But unfortunately, its inherent and abundant accident potential has not yet been exhausted. Severe accidents -- perhaps not another Chernobyl, but nonetheless severe -- at these reactors cannot be excluded; such events are not merely hypothetical. Their probability must be assessed by competent organizations -- whether ours alone or with participation of international bodies -- but in no case left to the authors of the reactor. And upon that assessment a decision must be made.

The Chernobyl catastrophe represents the most severe of all possible events at a plant. Accidents less severe are likewise unacceptable. In my view a nuclear plant must be designed such that such accidents cannot even be conceived under normal conditions, taking into account the natural cataclysms characteristic of the locale.

There exist no environmentally pure methods of generating electricity, nor are any likely soon to appear. Neither wind nor solar power plants can produce substantial amounts of energy. Hydroelectric plants in mountainous regions may be considered ideal, but their capacity is insufficient. And what do hydro stations on flatland rivers yield? Consider the Kyiv Hydroelectric Power Plant. A vast quantity of excellent floodplain land was inundated; I suspect that in the not very distant future these lands will have to be reclaimed. And the plant's capacity suffices only to open the sluice gate. Not literally, of course, but the energy is in no way worth the land consumed. The influence of hydro plants upon the environment is not confined to flooded land. And the influence is negative. Only ignorant ancestors could reshape the land for their benefit without serious -- indeed, without any -- harm to it. We are otherwise. In the Soviet Union, perhaps, not a single ``transformation'' can be found that did not end ill for nature and for man.

Certainly, within the former Soviet Union there lies an immense reserve for conserving every form of energy, not merely electrical. A mere structural reorganization of industry would yield much. We produce more metal pipes than any other nation, and yet it is still insufficient; indeed, we even import them. In the West, by reducing the output of ferrous metals they increased their gross national product, whereas our economists -- including the journal ``World Economy and International Relations'' -- droned on about a crisis in foreign metallurgy. Scientist B.~Paton, a respected man, spoke of this at the XXVI Party Congress; no one refuted him, yet nothing changed. Zero. And if one observes the empty heavy trucks racing back and forth? Much could be saved; and yet, in the end, it will still be necessary to enlarge electricity production, for mankind will hardly restrain its ambitions. One cannot expect a rational limitation of needs.

I remain convinced that any thermal power plant is ecologically dirtier than a good nuclear one. RBMK reactors have no future, which is hardly astonishing: their design originates in the 1940s. Their use in the power industry became possible because those in authority were people no longer capable of thinking otherwise. But nuclear power did not end with these reactors. There now exist reactors far more reliable; and if we cease persecuting nuclear specialists and allow them to work under normal conditions, there will be reactors whose reliability meets the highest criteria acceptable to humanity.

Under no circumstances should one conceal from the population the existence of problems that still lack satisfactory solutions. Foremost among these is the problem of spent fuel. The second concerns decommissioning. And here the RBMK presents difficulties greater than those of other reactor types. After Chernobyl we have been thoroughly taught what lies can lead to. We have fully achieved absolute distrust -- not only of our own science, but of foreign science as well. Any ignoramus now stands a better chance of being heard than a conscientious scientist. And if such a man says that he ``worked at the plant'' (in what capacity -- it matters not) and caters to popular sentiment, then he is believed without reservation.

Here is how G.~Medvedev writes:

\begin{personal}[From G.~Medvedev's \textit{The Chernobyl Notebook}]
\textit{``The officer and soldiers are not wearing respirators; they hang on their necks. Illiteracy from poorly organized training\ldots After all, future generations will come from these young men. Even \qty{1}{\roentgen\per\hour} gives a \qty{50}{\percent} probability of mutation\ldots''}
\end{personal}

This would indeed be terrible -- if it were true. The entire world proceeds on the assumption of five rem per year for professionals. But the number of such professionals is large. Not all receive the full permissible dose. Yet, for example, the repair personnel of reactor departments practically all receive that dose each year. If at five rem per year the mutation probability were what Medvedev asserts for one roentgen, then where are the anomalies? Why does neither the population nor the medical profession observe them? As with so much else, Medvedev's claim is conjured out of thin air.

Here is what Scientist N.~P.~Dubinin, a geneticist by speciality and, judging from his biography, not inclined to conformism, writes:

\begin{personal}[From N.~P.~Dubinin's biography]
\textit{``According to the UN Scientific Committee on the Effects of Atomic Radiation, with the participation of the world's leading specialists in radiation genetics, it has been established that doubling the mutation rate under acute irradiation occurs upon exposure to \qty{30}{\roentgen}. If a person is subjected to chronic small doses over the reproductive period (30~years), then the total dose capable of doubling the mutation rate amounts to \qty{100}{\roentgen}.}

\textit{Any increase in radiation entails some degree of hereditary damage in humans corresponding to the dose received.''}
\end{personal}

Pay no heed to the differing units -- roentgen, rad, rem: for $\gamma$- and $\beta$-radiation they may be treated as equal. Thus, according to the UN committee, a single dose of \qty{30}{\roentgen} or a chronic dose of \qty{100}{\roentgen} can double the mutation rate over the natural background. Given that the natural background already yields roughly \num{70} congenital defects per \num{1000} newborns, even doubling the dose does not produce anything like a \qty{50}{\percent} mutation rate.

Whom, then, ought one to believe -- Medvedev, or Dubinin, or rather the international committee? The answer is self-evident. Yet some information is broadcast to the public carelessly and irresponsibly by the media, while other information remains the preserve of a narrow circle.

Many kinds of ``information'' may be gathered from newspapers, and all of it presses heavily upon the psyche of a population already anxious.

Here is one ``specialist'' in nuclear matters proclaiming to the whole world that contamination measured in curies was calculated by $\gamma$-radiation, whereas it should have been by $\beta$, and that the true figure would therefore be ten times higher. If you call yourself a specialist, you ought to know that the curie denotes nothing more than a certain number of disintegrations per second. The decay scheme and harmfulness of the isotope are already accounted for by contamination norms, which differ for each nuclide -- one for plutonium, another for cesium\ldots

I have encountered claims that small doses are more harmful than large ones. If that were so, the remedy would be simple: irradiate those who received a small dose up to a large one -- sources could be found. This misunderstanding arises from the fact that the effect of small doses is more significant than a direct extrapolation from high doses would suggest. To illustrate, suppose: reducing the dose by a factor of ten reduces the biological effect not by ten, but only by about five. Once again, I ask that the figures not be taken as literal.

How much has been written about the liquidators of the Chernobyl accident who died or were injured as a result. They died, they fell ill -- and that is all. Before, people did not die, did not fall ill, children were never born with congenital defects. No analysis of age distribution, no comparison with a control group. A bus crosses the Dnieper; the driver loses consciousness and the bus plunges into the river. Why did he lose consciousness? The verdict of the traffic inspector is categorical -- radiation.

Today is 26~April -- a day of mourning. As usual, the dead are remembered, sympathy is expressed to the injured. And, as has become customary, everything is poisoned by lies. On Ostankino television, in the programme ``Vesti,'' the announcer (or whatever their new title is) reports confidentially that the accident occurred on the 26th, and the government announcement came only on the 28th, by which time several hundred had already died. Whence this information? In fact, by the 28th two men had died: Valery Khodemchuk during the explosion itself, and Vladimir Shashenok, who died in the hospital at Pripyat that morning. They lied under the totalitarian regime; the democrats lie as well -- at least that is their chosen name. I listened, and I say plainly: democratic lies are no less vile than totalitarian lies. This year 26~April fell on Easter. They might have refrained from lying on a holy day. One wonders whether such guardians of the people comprehend the enormous harm they cause. It lies as a heavy burden atop the actual harm, multiplying human suffering. There exists the notion of \textit{iatrogeny} -- illness induced by a physician. And this is a punishable act.

I have not been able to discover the term ``radiophobia'' in any dictionary. By it I understand not every fear of radiation, but solely an \textit{inadequate} apprehension of it on the part of an informed person, one who knows the possible consequences as a function of dose. Everything else can scarcely be ascribed to radiophobia; it is the natural concern of a human being for his own health and for that of his family. And the population, for the most part, does not know the serious facts concerning the biological consequences of irradiation and is compelled to trust the media.

It was only the initial, erroneous post-accident policy and practice of the authorities -- and, to my deep regret, of certain scientific workers borne aloft upon a muddy wave -- that elevated to prominence a multitude of incompetent persons and opportunists. Instead of conveying clearly and honestly to the public:
\begin{itemize}
  \item the \textit{actual} levels of territorial contamination;
  \item the harmful effects of radiation in \textit{any} dose, and the magnitude of the attendant risk;
  \item the necessity of relocating from contaminated regions those families with children and youth, where universal resettlement proved impossible. And this is no disdain for the elderly. In lightly contaminated districts, at low doses, the time required for the manifestation of such effects exceeds the natural span of their remaining lives. But the young must live -- and bear children.
\end{itemize}

I fully understand that such reasoning will give occasion to accuse me of cruelty. Very well; hypocrisy is one of the age's most characteristic traits. Is it preferable to keep children for three to five years under unfavourable conditions, only then to relocate them? It is easy to declaim demagogically against the ``immorality'' of \textit{counting money} when human health is at stake; such pronouncements cost nothing and, on the contrary, yield popularity. But this is akin to the ``profound'' assertion: ``It is better to be rich and healthy than poor and ill.''

If we are to cast a bone to the crowd, let it at least be a large one. I shall go further. During the liquidation of the accident, once it became clear that the unit was beyond restoration -- and this was evident immediately -- there was no need for haste. First, the activity of the dust dispersion ought to have been fixed, and preparations made properly. And, above all, \textit{older} generations ought to have been called up for the work, not young soldiers and reservists. These people would have carried out the tasks more skillfully, with lower dose expenditure and fewer genetic consequences. This is unforgivable both for the medical authorities and the highest leadership. You will say -- shameless calculation. But what else should have been done?

If catastrophes of the Chernobyl type cannot be excluded, should nuclear power therefore not be developed -- indeed, should it be forbidden? I think not. Chernobyl is a pathological case; it must be excluded from consideration.

An explosion such as occurred at Chernobyl is practically impossible in a reactor constructed in accordance with accepted design rules. Designers know how to ensure this without exorbitant cost. Each project must meet the standards after passing expert review, including \textit{international} review. In our time, I doubt that any true supporter of sovereignty would consider international expertise an infringement of sovereign rights. It would cost us far more to dispense with it. Accidents cannot be excluded entirely, but methods of containment are known.

In educating the public on the acceptability of nuclear power, all forms of evasion and pressure must be excluded -- ``let them freeze for a winter, then they will agree.'' Every station, by regulation, has its sanitary-protection zone where housing is forbidden, and its supervised zone. The population must be informed of the dosimetric conditions in the supervised zone; whether emergency incidents have occurred; what radioactive releases took place; and what dangers they pose. Or are we once more to ``not know'' the effects of low doses on humans, as Gadjiev maintained? This in a country where tens of thousands of people have worked with ionising radiation for decades. And what shall we say of Chelyabinsk and Semipalatinsk? The doses there ranged from zero to lethal. As for enterprises, the doses received by workers exist for every year, and so do their medical records; and the clean-zone workers -- what are they, if not control groups?

Talk of the high training and high responsibility of all personnel of the stations must be abandoned. Such requirements are unconditional and presupposed, but they cannot persuade anyone of the safety of the plants. Besides, the Chernobyl personnel were, after the accident, recognised as adequately trained -- including by international experts. And if we are to be objective, we must admit that the personnel of the Chernobyl station after 1986 became \textit{less} well trained owing to the constant and heavy turnover. Not because the newcomers were worse, but because time is required for mastery of the work. One cannot raise a worker's qualification by the most stringent order; experience and skill are indispensable.

Mistakes by personnel have existed and will continue to exist. But whether the consequence of such a mistake will be an explosion of the reactor -- as occurred at Chernobyl, where personnel overlooked a parameter deviation for which no measuring device existed -- or merely a harmless shutdown, depends solely on the equipment, on the quality of the design.

Ecological education of the population is needed. People must possess neither the fatal resignation of the indifferent, nor an exaggerated apprehension of radiation or any other technogenic influence. Much has accumulated upon people's shoulders: we poisoned the food because we failed to increase yields; we poisoned the air; we ruined the land by producing goods for which there was no genuine need. Listen to the latest news on radio or television, and one receives the impression that the purpose is not to inform, but to break the human spirit.

Therefore a balanced, non-hysterical approach to the organisation of life is required. If stations are to be shut down, let it be done properly, by government decision, not by disorganising pickets. If stations are to be commissioned, let it be done properly, in a planned manner, and not under the pressure of urgent circumstances. Reports were heard of a possible resumption of operation at the Armenian Nuclear Power Plant. The cause is understandable, but recommissioning it will not be simple; and certainly it is worse than if it had continued in uninterrupted operation.

I harbour no confidence in the capacity of the people to organise social and governmental life rationally. Recent referenda and votes have demonstrated this plainly enough. Still less do I trust in the ability to resolve correctly a specialised technical problem. And yet, under our conditions, the decision on the fate of nuclear power must indeed be entrusted to the people -- but not on the basis of rally passions, where he who shouts loudest prevails, but on the basis of \textit{science}. For this, the former Soviet science (whatever its current state) must regain the people's trust.

\chapter{Closing Remarks}

In my account of the events of the Chernobyl tragedy I have been strictly objective, neither concealing nor embellishing. I have spoken only of that which I know with certainty. By virtue of my, so to speak, peculiar position, I was compelled to dispense with paraphrase and to cite documents verbatim. In so doing, I invite the Reader to think independently and to draw his own conclusions, rather than to take my assertions on trust. On the one hand, because I am a former prisoner and therefore, in the eyes of many, not to be trusted.

Here is a small illustration. After the accident my daughter worked for a time at the station, in the First (security) Department. The deputy director for personnel told her: ``I cannot keep you in the First Department; I can only offer you a place in the cafeteria.'' Naturally so -- her father was expected to be arrested, and meanwhile the daughter held a post in a classified division. Though what secrets remained there is hard to say; the same people are still in place.

On the other hand, so much falsehood has been heaped up around Chernobyl that it is exceedingly difficult for any person to make sense of it. Therefore I set forth the factual events, and the article of the General Provisions for Safety, and I invite the Reader to assess for himself how, for example, AZ-5 fulfilled the requirements of that article. I am virtually certain that I have not allowed inaccuracies in the description of events.

Another matter is interpretation, the evaluation of those events. You understand yourself that here I cannot be altogether impartial, despite every effort. Even now - should someone visit and we talk far into the night about the accident, or I find myself writing about it - a sleepless night is assured.

Yet the situation appears to me so transparent that even a person with biases, with personal involvement, but not inclined to falsehood, cannot go astray. Otherwise I truly understand nothing either of technology or of life. Judge for yourselves.

At 01:23:40 a.m., the operator actuated AZ-5 in order to shut down the reactor, at a moment when the centralised monitoring system registered not a single deviation in the reactor's parameters or those of its systems. This is an indisputable fact. The emergency protection system -- its hardware -- was in working order. The electronic components and the reactivity-control elements (the control rods) functioned according to the algorithm and in full.

The result was the explosion of the reactor. What claim can possibly be made against the operating personnel? In any normal human society -- none. Emergency protection, by its very name and purpose, is designed to shut down the reactor in emergency situations without \textit{any} damage. I say nothing of an explosion. On 26~April the protection system failed to shut down the reactor even from a stationary state.

Even had we violated certain provisions of the Regulations or instructions earlier, that would furnish no basis for accusing the personnel of having caused the explosion. For nothing occurred at those times.

Let us suppose we violated the Regulations (though in fact we did not) when we began raising power after its ``collapse'' and risked an accident analogous to that at the first unit of the Leningrad NPP in 1975. Nothing occurred.

Let us suppose we violated the Regulations by disabling the ECCS. But what has that to do with the explosion? The emergency core cooling system could not possibly have prevented the explosion of the reactor, and after the explosion it was useless, for the reactor had ceased to exist. Only the ``great operational expert'' G.~Medvedev asserts that it could, by some mystical means, have averted the explosion, and the ``naive'' scientist A.~P.~Aleksandrov professed horror at its disconnection. Even the official commissions -- who certainly would not have overlooked such an opportunity -- never suggested that the ECCS could have prevented the accident. Had there been even a ghost of such a possibility, they would undoubtedly have pressed it.

The reactor was blown up by the emergency protection.

The Kurchatov Institute Report lists thirteen hypothesised causes of the reactor's destruction, and on the basis of instrument readings and calculations rejects all but one -- the AZ-5 rods introduced positive reactivity, which served as the initiator of the accident.

At present this version -- the only one free of contradictions -- is accepted by all as the cause of the accident. This has long been evident -- since 1986.

I refuse to dignify the fact of positive reactivity insertion by AZ-5 at its actuation with the term ``design feature.'' It is impossible. As Vitya Tarasenko put it -- it is an oxymoron. It is such a brilliantly absurd idea, on the part of the designers of the control rods, that it could have arisen only in heads whose owners had taken ``seven hundred grams apiece.''\footnote{A caustic, colloquial insult. ``Seven hundred grams'' refers to a very large amount of vodka. He is saying, sarcastically, that only someone drunk out of his wits could conceive such an idea -- that a shutdown mechanism would initially increase reactivity rather than suppress it.}

Once the fact that AZ-5 destroyed the reactor is acknowledged, what further talk of the personnel's guilt is even conceivable?

It appears -- such talk is possible. Where morality is inverted, all things are possible.

And still, dear citizens, gentlemen: whenever, in discussing the Chernobyl tragedy, you feel tempted to cast a stone at the operators, remember -- the reactor was destroyed by AZ-5. This phenomenon defies engineering comprehension. An eternal monument to the bungling of physicists and designers.

The RBMK-1000 reactor in 1986 did \textit{not} meet the fundamental requirements of the safety standards and technical regulations.

On the basis of the very fact of the explosion, and on the basis of post-accident calculations and experiments, the State Committee for Supervision of Industrial Safety established fifteen violations of the General Provisions for Safety in the reactor's design, directly bearing upon the occurrence and scale of the accident. And altogether, taking into account the ``Rules for the Safe Design and Operation of Nuclear Power Plants,'' the reactor's design contains thirty-two violated articles.

A natural question arises: can such a structure even be called a reactor, given so many departures from mandatory requirements? It is something else entirely. Can such a structure operate reliably? In truth, there is no question to ask.

In the Kurchatov Institute (IAE) report mentioned above, under the heading ``Most plausible versions and their analysis,'' the following are listed:

\begin{enumerate}
\item[3.1] Explosion of hydrogen in the bubbler pool.
\item[3.2] Explosion of hydrogen in the lower cooling tank of the Control and Protection System circuit.
\item[3.3] Sabotage.
\item[3.4] Rupture of the Main Coolant Pump pressure header or the distribution group header.
\item[3.5] Rupture of the steam drum-separator or steam-water piping.
\item[3.6] Displacement effect of the Control and Protection System control-rod absorbers.
\item[3.7] Malfunction of automatic regulation (AR).
\item[3.8] Gross operator error in controlling the Control and Protection System manual control rods (RR).
\item[3.9] Cavitation in the Main Coolant Pumps delivering steam-water mixture into the fuel channels.
\item[3.10] Cavitation on the throttle-regulating valves.
\item[3.11] Carryover of steam from the drum-separator into the downcomer lines.
\item[3.12] Steam-zirconium reaction and hydrogen explosion in the core.
\item[3.13] Ingress of compressed gas from the Emergency Core Cooling System, cylinders.
\end{enumerate}

These are not all possible causes that could lead to the explosion of this apparatus, but they suffice for substantive discussion. Version~3.6 has been accepted as the only non-contradictory explanation. Version~3.10 is essentially impossible.

Now the crucial point: the remaining eleven versions, if realised, all blow up the reactor.

And they are fully capable of being realised. They were rejected not because they are intrinsically impossible, but because they contradict instrument readings and the logic of the events of 26~April~1986. This list is especially interesting because it was compiled by the reactor's creators themselves. The creators openly acknowledge that a whole series of scenarios leads their reactor to catastrophe.

Consider, for example, item~3.7, ``Malfunction of the Automatic Regulator.'' The authors write:

\begin{personal}[From the Kurchatov Institute report:]
\textit{``A malfunction related to immobility of all the Automatic Regulator rods can lead to a reactor runaway owing to the large steam reactivity coefficient.''}
\end{personal}

With the coefficient that the RBMK possessed in 1986, a power surge was indeed possible. Failure of the Automatic Regulator is hardly unthinkable; it is entirely plausible. And the authors have taken the mildest case of the Automatic Regulator failure -- immobility of the rods. Reactor automatic-control theory requires the reactor to remain safe under the \textit{more} severe failure mode: rod motion in the direction of increasing reactivity. Under that condition -- an explosion all the more inevitable.

As long as the report's authors remain within technical analysis, their competence does not abandon them. But upon reaching their conclusions, a peculiar transformation of logic ensues:

\begin{personal}[From the Kurchatov Institute report:]
\textit{``It has been determined that the root cause of the accident was an extremely unlikely combination of violations of operating procedures and modes committed by the unit personnel (A.D. -- now that is the unlikely part!), under which shortcomings in the reactor and Control and Protection System rod design manifested themselves.''}
\end{personal}

There is nothing surprising in this, given that the authors are Kurchatov Institute staff -- the reactor's creators. One must at least thank them for acknowledging the \textit{``explosive nature''} of their creation. They have not enumerated every possible scenario, but the ones they do list are sufficient to demonstrate that this reactor required no special conditions to explode.

But there were indeed ``unlikely'' and ``extremely unlikely'' circumstances in the case of the RBMK.

First, the rare unanimity of the commission that slandered the personnel. A company of scholars and non-scholars (the prefix ``non-'' may be written separately or together; the sense is unchanged)\footnote{\textit{неучёный} -- someone who is uneducated. \textit{не учёный} -- not a scientist.}. A commission of venerable elders and those not yet so old, of the highest-ranking administrators. With such a company one may launch any campaign without fear of loss. And they continue to win an obviously unjust case.

Second, it is difficult to imagine how the creators managed to concentrate in one reactor virtually all the defects imaginable for a channel-type design. If not all, then certainly the worst. Now \textit{that} is what one may call ``incredible.''

Three organisations bore responsibility for the RBMK-1000 design:
\begin{itemize}
\item The Kurchatov Institute provided the scientific basis for all reactor matters, including nuclear safety. Its role was not limited to supplying data and recommendations to the designers, as A.~P.~Aleksandrov preferred to portray it (``Dollezhal created the reactor\ldots''); throughout operation it worked on the reactor as well -- hence the positions of Scientific Supervisor (Aleksandrov), two deputies (Kalugin, Kramerov), and a group of specialists, including one mentioned earlier -- V.~P.~Volkov.
\item NIKIET produced the design documentation, exercised design supervision during operation, and calculated the core composition from periodic station data. Both organisations provided methodological supervision to the Nuclear Safety Departments of the stations.
\item The State Nuclear Safety Oversight Committee (\textit{Gosatomenergonadzor}) had as its primary task the prevention of operation of any reactor not meeting the requirements of the safety standards and regulations. It was subordinate to no ministry. It issued startup authorisation and could halt operation at any moment. It was entitled to demand any calculations, including supplemental ones. At least, this was how things were written.
\end{itemize}

Other systems of organisation could be imagined, but it is unclear why this one could not have worked. And yet, for the RBMK-1000, the system failed at every link. Recall the case of the Control and Protection System rods:

\begin{itemize}
\item During the physical startup of Unit~4 in 1983, positive reactivity insertion at the beginning of rod motion was discovered.
\item The State Inspector recorded the impermissibility of this and \textit{still} permitted commissioning.
\item In December 1983 the Scientific Supervisor wrote to NIKIET demanding elimination of the defect.
\item NIKIET accepted the demand and by December 1984 produced a technical assignment for new rods. And that was all. No working drawings, no new rods -- up to the accident.
\end{itemize}

This is the ``extremely unlikely combination'' -- though involving none of the positive qualities of the Heroes and the Committee Chairman.

Thus the Control and Protection System rod design defect was understood long before the accident by the Scientific Supervisor, the Chief Designer, and the Oversight Authority. Replacing the defective rods with proper ones -- though it would not have made the RBMK safe -- would have prevented the accident of 26~April and a whole series of other scenarios.

The same applies to the steam reactivity coefficient. Had it been reduced to an acceptable value, the reactor's reliability would have increased by an order of magnitude.

I cannot say why all those responsible for the reactor's design, knowing full well its intrinsic and wholly inadmissible defects, remained idle. Concerning the steam-reactivity coefficient and the Control and Protection System, I have already cited the words of N.~A.~Dollezhal and I.~Ya.~Emelyanov. All were aware, and yet none acted. They occupied positions of such authority that no one could have hindered them. No system functions of itself; it is always people who make it work. I am not speaking of the political system -- I do not trespass there; I am unqualified. Just as once, for example, the public bath in the town of Torzhok opened ``thanks to the Party and personally to Comrade Stalin,'' so now all blame is laid upon the system, upon stagnation. Even A.~P.~Aleksandrov, over whom no system exerted pressure -- at least not in matters concerning the RBMK. A most convenient loophole for concealing one's own inaction.

In the 1950s, ``scientific organization of labour'' became fashionable. An engineer advises a factory director to move his desk, his bookshelf, his telephone. Aunt Masha, washing the floor, remarks that before the Revolution she worked in a brothel, and when profits fell, they did not move the furniture -- they changed the prostitutes.\footnote{In the 1950s this was a management fad: consultants would recommend rearranging offices, tools, or workflows as if efficiency depended mainly on moving objects around. The consultant tells the factory director to shift his desk, shelves, telephone -- trivial cosmetic adjustments presented as ``science.'' The cleaning woman cuts through the pretence with a coarse but pointed comparison: in the brothel where she once worked, when business declined they didn' t rearrange the furniture -- they replaced the workers. Her line means: the problem isn't the placement of objects; it's the people in charge. Structural or managerial incompetence cannot be solved by moving desks.}

It is clear that long ago the people in leadership positions on the RBMK topic ought to have been replaced. Aleksandrov, Dollezhal, Emelyanov revealed themselves fully in their pronouncements after the accident. I think the leadership of Gosatomenergonadzor was likewise ill-chosen. The head, Kulov, had long been in the shadow of A.~G.~Meshkov, working under him for many years, and could scarcely display independence in his new post. And his deputy, Sidorenko, emerged from under Aleksandrov's wing. It must be said that both Kulov and Sidorenko knew well what the RBMK was. The latter even wrote a letter noting the reactor's negative properties. But he misunderstood something: in his position one does not write letters -- one forbids the operation of unfit equipment. For the epistolary genre there are writers.

One might understand all these leaders if some previously unknown phenomenon had revealed itself during the accident. Certainly such ignorance would be impermissible for the Scientific Supervisor and the Chief Designer -- they are obliged to know. But everything that led to the accident had long been known to all of them.

Such a concentration of irresponsibility! Whether it was merely unlikely or extremely unlikely -- I do not know. Unnatural -- yes. But perhaps explicable. Who among them was punished for their criminal activity or inactivity? No one.

I do not express here my view on the feasibility of using nuclear energy for electricity generation. But in my conviction, when deciding the admissibility or inadmissibility of nuclear power plants, the Chernobyl accident should be excluded as a pathological case.

\emph{The RBMK-1000 reactor in 1986 did not meet a large set of requirements of the regulatory and technical documentation, including fundamental nuclear-safety requirements.}

I waited to see which official organisation in the Soviet Union would dare to declare that this reactor was under no circumstances suitable for operation -- and that the primary crime had been committed against the staffs of the plants operating these reactors. They were deceived into sitting upon an atomic bomb; the true properties of the reactors were never disclosed to them. Now it is too late. The Union is gone; only disagreements remain. Even the Gospromatomenergonadzor commission, which by all rights ought to have determined whether the reactor could be operated at all, did not consider it possible to say so directly, even after identifying fifteen violations. Yet the question admits no doubt. The safety regulations applied to nuclear reactor designs; the RBMK design violated them, and therefore the reactor was not eligible for operation.

Our actions on 26~April, had the reactor been built in accordance with NSR and OPB, would not have led to any accident whatsoever -- let alone an explosion. This is perfectly clear. One might think that the accusations against the staff should therefore be lifted. Not so.

There are always people ready to search for what could have been done to avoid the accident. According to them, any piece of rubbish may be handed to the operators -- and if an accident occurs, the operators are still at fault. We were surrounded by traps and snares, and we fell into them. Einstein said: ``God is subtle, but not malicious.'' The designers proved both subtle and malicious.

Now I know how one could have avoided the explosion on 26~April -- provided only that AZ-5 did not trigger automatically under any signal, for otherwise it was the grave. Here we are: discussing how to save oneself \textit{from the emergency protection system}. But that was only for that day; there was nothing ``emergency'' in the reactor's state. In truth, any one RBMK-1000 reactor was doomed to explode.

I do not know whom to ask by law, so I turn to people -- at least for an answer in the moral sense. The reactor did not meet the design norms adopted in the country and exploded precisely because of these nonconformities. Is the staff guilty, or are the designers guilty before the staff?

As I write these words, everything in me protests: you are writing nonsense. But no. At the very least, my lawyer replied that he knew of no article in the Labour Code or any document of legal force naming someone guilty before a worker because equipment failed to meet standards. Whether this is nonsense or not -- I merely state reality as it stands.

Naturally, the laws in our country always operated in one direction only in relations between the individual and the state. The state could not be guilty. We were hired to work on equipment supposedly built in accordance with the approved norms. The state did not fulfil the terms of that contract. It is inhuman -- but the state cannot be accused under the law. Yet to accuse \textit{us}, in this case, is unjust both by human measure and even by law.

There never existed in the Soviet Union an organisation or society capable of -- or even willing to -- defend the individual. Let us not speak of the state and the Party -- they are gone. Law enforcement agencies have long been oriented in only one direction. The trade union, reduced to absurdity when everyone belongs to one union, could naturally defend no one. And so on.

We appear to be moving toward a society with some kind of ``face.'' One would wish that, upon seeing that face, a person would not cry out in horror.

In conclusion I state the following. The Chernobyl catastrophe is, in pure form, the consequence of the grossest miscalculations of the reactor's physicists and designers.

It is long past time to say: \textit{the properties of the reactor were not the main cause, nor the decisive cause, but the sole cause of the Chernobyl catastrophe.}

\chapter{More on Chernobyl}

\textit{Translator's note:} This chapter was not originally published in Russian or with \textit{``How It Was''}, although it circulates in many editions of the book. It can be traced back to Dyatlov's paper, \textit{``Why INSAG has still got it wrong''} published in \textit{Nuclear Engineering International}, Vol.~40, 1995.

\varthreestars

A group of IAEA experts issued in 1986 the report INSAG-1 on the causes of the Chernobyl catastrophe, and seven years later released their corrected report, INSAG-7. Seven years is ample time for studying many investigations and forming one's own judgment. Upon the publication of INSAG-7, the journal ``Nuclear Engineering'' printed an article by Mr.~D.~Valley, ``Who Is to Blame for the Chernobyl Accident -- Mature Reflections of the International Nuclear Safety Advisory Group.'' Let us try to assess the maturity of the experts' reflections.

\paragraph{1.\ On the subcooling of the coolant}
For eight years an erroneous assertion has persisted: that owing to the high coolant flow rate, its subcooling at the inlet to the core decreased, boiling began at the very bottom of the core, and, as a consequence, thermohydraulic instability arose. The author pointed out this error as early as 1986, and later in a letter addressed to the Director of the IAEA.

\subparagraph{1.1.\ Report, §2.9.}
\textit{``These conditions led to the onset of boiling in the lower part of the core or near it.''}

According to the Operating Manual, subcooling is the temperature difference between the water in the steam-drum separators and at the inlet to the core. It does indeed decrease with increasing flow rate, but at the same time the pressure at the core inlet increases and, accordingly, the boiling temperature increases (Fig.~1, Appendix). At low reactor power, boiling begins outside the core altogether, in the downcomer pipes, gradually descending as power rises. And the higher the flow, the higher the boundary at which boiling starts. Specifically, on 26 April, at a reactor power of \qty{200}{\mega\watt} (channel power in the central part of the core about \qty{160}{\mega\watt}), boiling began at the very top of the core (Table~1 and Fig.~2, Appendix).

\subparagraph{1.2.\ Report, §5.2.3.}
\textit{``The reactor was operated in a mode with coolant boiling in the core and at the same time with small or zero subcooling at the pump suction and at the core inlet. Such a mode by itself could have led to a destructive accident, ... given the positive reactivity feedbacks of the RBMK reactor.''}

The reactor operates exclusively in a boiling mode, and according to the Operating Manual it is permitted to run with small, even zero, subcooling; see Chapter~9 of the Manual. This condition is mandatory, since such regimes cannot be avoided in principle -- they arise at every power increase and when the pressure in the separators decreases.

An interesting position of the experts: to explain to the staff the destructive action of the positive feedback. Very well (the operators will know why they died or were maimed), but it would be better for the reactor to conform to design standards. If a reactor explodes in a regime that cannot be avoided, then there is only one conclusion -- a prohibition on operation. What is there to explain?

On 26 April the subcooling was approximately one degree, and the pressure was slowly rising (Table~2, Appendix).

\begin{figure}[h]
\centering
\includegraphics[width=0.5\textwidth]{fig1.jpg}
\caption{A plot of neutron power (normalized units; left y-axis) to time (seconds; x-axis), giving reactivity $\beta_\text{eff}$ (right y-axis). Dependence of reactivity and neutron power on time in the initial phase of the accident's development: ■ - reactivity; ▲ - neutron power.}
\end{figure}

\begin{figure}[h]
\centering
\includegraphics[width=0.3\textwidth]{fig2.jpg}
\caption{A plot of reactivity (y-axis, \unit{\percent}) to coolant density (x-axis, \unit{\gram\per\centi\meter\cubed}). Dependence of reactivity $\rho$ on coolant density $\gamma$: 1 - design calculations; 2 - actual dependence at the moment of the 26 April 1986 accident; 3 - current state after implementation of corrective measures.}
\end{figure}

\paragraph{2.\ On the operation of the Main Circulation Pumps}

\subparagraph{2.1.}
The experts have resurrected the long-discarded version of pump rundown. There was no loss of circulation:

\begin{itemize}
\item If the pumps did \textit{not} lose circulation when the pressure \textit{decreased}, why should they have lost it when the pressure \textit{increased}?
\item The monitoring system recorded normal pump operation up to the very moment of the sharp power surge.
\item The pumps supplied from the ``coasting'' generator could not have tripped -- there was no cause for it.
\item Yet the first pumps to switch off were precisely these ``coasting'' pumps (see INSAG-7, Appendix~I, Table~I-I), and only thereafter those supplied from the backup. This clearly indicates that the abrupt power surge, and not any rundown, caused the cessation of coolant flow.
\end{itemize}

There are further arguments; but if these do not persuade the experts, then nothing will.

\subparagraph{2.2.}
That reactor indeed \textit{would} explode upon Main Circulation Pump trip. Such an outcome might occur upon rupture of the steam lines, upon the opening and failure to reseat of the main safety valves, or during a Loss-of-Coolant-Accident. But in all such cases the fault lies wholly with the designers.

To conclude the discussion of the Main Circulation Pumps, let us note:

\subparagraph{2.3.\ Report, §2.8.}
\textit{``Moreover, since the coolant temperature on the path from the circulation pumps to the core inlet changes little, with very small subcooling the temperature within the pumps and at their suction approaches the boiling point.''}

This is merely an ornate way of saying something simple: the suction temperature approaches the boiling point at high coolant flow because the coolant is less cooled by the feedwater, and because the greater head loss in the downcomer elevates the temperature (see Fig.~1, Appendix).

\subparagraph{2.4.\ Report, §2.9.}
\textit{``After shutdown of the turbine, the pumps driven by it began to decelerate, since the turbine speed diminished and the voltage of the associated generator fell. The decreasing flow through the core increased the steam content in the core and caused the initial positive reactivity feedback, which was at least part of the cause of the accident.''}

\begin{itemize}
    \item A \qty{10}{\percent} decrease in flow over \qty{36}{\second} of coast-down produces a reactivity increase easily compensated by the Automatic Regulator system. No power increase occurred.
\item One need only consult the power graph submitted to the IAEA in 1986. The same appears in the Appendix, §1-4.6.2 (INSAG-7).
\item If that were insufficient, the experts could have asked Group member E.~Burlakov, who could have provided the 1986 calculation of his colleague A.~Apresov (see Table~2, Appendix).
\end{itemize}

During coast-down the coolant density changed by \qty{6}{\kilogram\per\meter\cubed} (Table~2, Appendix), yielding a reactivity increase of roughly \num{24e-6}. Under actual conditions the rate of reactivity change is often several times greater.

Thus a thought that is, in the abstract, correct, when divorced from data and elementary calculation, leads to unwarranted (indeed false) conclusions.

Accordingly, questions concerning coolant subcooling, Main Circulation Pump rundown or coast-down, and even the turbine coast-down itself, bear no relation to the accident. Had the experiment been abandoned at the last moment, the result would have been the same.

As is now clear, earlier occasions had brought the plant to the brink of catastrophe: following scrams there had been manifestations of (Signal of) Emergency Protection by Power (AZM) and (Signal of) Emergency Protection by Power Surge (AZS) signals dropping out. They \textit{should} not have appeared; they were dismissed as ``spurious'' for want of understanding. In fact, they were genuine power surges caused by the Emergency Protection system itself, unregistered by the SFCRE chart recorder because the silver detectors were too inert. By contrast, the AZM and AZS signals, driven by less inert ionisation chambers, did appear -- but there was no recorder for them. Compare this with 26~April: 01:23:40 a.m., AZ-5 was actuated, and after \qty{3}{\second} the AZM and AZS signals appeared.

It is apt to note that in the Appendix, Chapter~II-2.5.3 (INSAG-7), states that one of the computational models does \textit{not} reproduce a runaway such that, on the third second after AZ-5 actuation, signals appear exceeding the limits on power and its rate of rise. Perhaps; but one must consider not three but nearly four seconds, since the recording interval is one second. Then (see Fig.~16.1) no contradiction remains. For clarity: between two events of 1994 and 1995 the interval may be two hours or two years minus two hours.

\paragraph{3.\ Operational Reactivity Margin}

The designers of the reactor, followed in due course by the IAEA experts, have gradually attributed to the operational reactivity margin (ORM) one function after another:

\subparagraph{3.1.\ Power manoeuvrability}
A natural and universal function for all reactors; described both in standard texts and in the \textit{Rules}.

\subparagraph{3.2.\ Compensation of fuel burnup}
Likewise a standard and universally recognised function.

\subparagraph{3.3.\ Regulation of power distribution throughout the reactor volume}
Also seemingly natural, given the ``continuous'' refuelling regime and the large core dimensions, although the RBMK is hardly the world's sole large reactor.

\subparagraph{3.4.\ A guarantee of the operability of the reactor protection system}
Curiously, the limitation is imposed not on the \textit{maximum} ORM -- which alone would be natural -- but on the \textit{minimum}.

\subparagraph{3.5.\ Operability ensured only for a given ORM and only under a particular configuration of control rods}
Here we cross the boundary of absurdity and enter the realm of a direct breach of design norms. The designers committed an evident error in creating rods that insert reactivity of opposite signs depending on the direction of motion. Immediately after the accident these rods were declared defective by all parties, including the designers; yet, astonishingly, the designers have found defenders among the experts.

\noindent\textbf{INSAG Report, §5.1:} \textit{``A positive reactivity excursion could only occur as a result of a special configuration of the control rods.''}

Such ``special configurations'' are many; the reactivity surge occurred solely because the rods themselves were faulty. With a correctly designed rod there are no ``special configurations'' and cannot be any. The question remains: why did the experts feel obliged to defend a defect long since repudiated?

A final function now attributed to ORM is that of constraining the steam-void coefficient within prescribed limits.

\noindent\textbf{INSAG Report, §4.2:} \textit{``During discussion it appeared that the operators were apparently unaware of another reason for the importance of the ORM, namely that it can strongly influence the steam and power coefficients.''}

Perhaps the experts lacked information? They possessed it.

On p.~45 of Appendix~I to INSAG-7 we read:

\begin{personal}[From INSAG-7 (Appendix I):]
\textit{``The second-generation RBMK units were initially loaded with \qty{2}{\percent} enriched fuel, but even with this enrichment, as burnup increased to \qtyrange[range-phrase=\text{--}]{1100}{1200}{\mega\Wd\per\tonne} and with the regulatory operational margin of \numrange[range-phrase=\text{--}]{26}{30} rods, the steam-void coefficient approached \qty{+5}{\betaeff}. Similar burnup existed at Unit~4 before the accident.''}
\end{personal}

At such a steam-void coefficient the power coefficient reaches \qty{+0.6}{\betaeff\per\mega\watt} above \qty{50}{\percent} power. At lower power it is even more positive.

\subparagraph{4.2.\ As N.~Laletin noted}
For a uniformly burnt-out core the steam-void effect is roughly twice as large as for a core with the same \textit{average} burnup but heterogeneous distribution. Thus the end of the transition period (with virtually all Additional Absorbers removed) is more dangerous than steady-state refuelling. Exactly such a state existed at Unit~4: 1~additional absorber, \num{1}~empty channel, and \num{1659} assemblies with average burnup \qty{1180}{\mega\Wd\per\tonne}; \qty{75}{\percent} of them first-load assemblies burnt to \qtyrange{1150}{1700}{\mega\Wd\per\tonne}.

The steam-void effect exceeded \qty{+5}{\betaeff}, and that alone sufficed for an explosion.

\subparagraph{4.3.\ INSAG Report, §2.1}
\textit{``Therefore, although the steam-void coefficient varied widely from negative to positive depending on core composition and operating mode, the power coefficient under normal operating conditions remained negative. During the accident both the steam-void and the power coefficients became positive.''}

If not to justify the designers, what is the purpose of this phrase?

According to the Regulations, ``normal'' modes included \textit{all} power levels from minimum controllable to nominal, with no time limits. If the experts meant that the power coefficient was negative at nominal power, that is true -- but utterly insufficient. Design rules require it to be negative at \textit{all} operational and accident conditions.

\paragraph{5.\ Additional Remarks on the INSAG-7 Report}

\subparagraph{5.1.\ Report, §4.1:} \textit{``A SCRAM of the reactor before the sharp power excursion that led to the destruction of the reactor could, without doubt, have been a decisive factor contributing to that destruction.''}

Here the experts intimate that a power surge \textit{must have been impending}, and that the personnel either foresaw it, or else -- by a stroke of malign chance -- pressed the shutdown button immediately beforehand, thereby hastening or even predetermining the catastrophe.

This is a novel conjecture. Why did the experts not propose any cause for this alleged imminent surge, at least hypothetically? No commission discerned such a cause.

The present author, an eyewitness, states plainly: the SCRAM button was pressed in a calm situation. The testimonies of G.~Metlenko and A.~Kukhar say the same. Appendix~I, §1-4.9 of INSAG-7 acknowledges that the commission found \textit{no} reason for the AZ-5 actuation. There was but one: the intention to shut down the reactor upon completion of the work.

The scram did not ``contribute'' to the destruction -- it \textit{caused} it.

\subparagraph{5.2.\ Report, §4.1:} \textit{``The rupture of a fuel channel would cause a sharp local increase of steam content due to the flashing of the coolant to steam; this would lead to a local rise in reactivity, which would initiate a propagating reactivity effect.''}

The reactor contained far more ``traps'' for personnel than the experts enumerate. A channel rupture (one or two) is not among them. When a channel fails, water content in the core \textit{increases}; in addition, the water cools the graphite. Both effects reduce reactivity. To claim otherwise is to contradict elementary reactor physics.

\subparagraph{5.3.\ Report, §5.2.1:} \textit{``It had been asserted that prolonged operation of the reactor below \qty{700}{\mega\watt} was prohibited. This assertion was based on incorrect information. Such a prohibition ought to have existed, but at the time it did not.''}

That reactor could ``successfully'' explode even at \qty{700}{\mega\watt}. There existed \textit{no} safe power level -- merely levels more or less perilous. By contrast, a reactor conforming to design rules requires no such artificial limitation.

No technical justification existed for \qty{700}{\mega\watt} as a ``safe'' value. That figure acquired an almost mystical authority solely to create an appearance of operator fault.

The \qty{700}{\mega\watt} level in the programme was set by the present author for collateral reasons. The test originally included functional checks of the main relief valves, requiring substantial steam flow. Since the turbine rundown test was scheduled last, and the reactor was to be shut down immediately afterwards, the level was chosen with preceding high-power work in mind.

After the unplanned power drop, the author limited the rise to \qty{200}{\mega\watt} -- sufficient for the test -- not because \qty{700}{\mega\watt} was unattainable. With a positive prompt power coefficient nothing prevents a rise in power. Moreover, \qty{200}{\mega\watt} was an ordinary, regulation-permitted level.

\subparagraph{5.4.} Accident data and operating experience were not utterly ignored. After the 1975 accident at Leningrad Unit~1, the commission (E.~Kunegin et~al.) proposed in 1976:
\begin{itemize}
  \item reduction of the steam-void coefficient;
  \item modification of the control-rod design;
  \item creation of a fast-acting SCRAM.
\end{itemize}

Likewise, once the positive-SCRAM-reactivity effect was discovered, a technical design for new rods existed by December 1984. Other proposals were on record.

Yet management resolutely ignored all of this -- among them IAEA experts Yu.~Cherkashov, V.~Sidorenko, and A.~Abagyan. Hence the negligible benefit of appointing high-ranking officials: they possessed the data, the computers, the reactor characteristics -- yet ignored the essential facts. By disregarding both design-stage defects and hazardous properties long evident in operation, the RBMK was condemned to destruction.

\paragraph{6.\ Causes of the Accident}

The reactor violated more than three dozen design-norm articles -- ample reason for an explosion.

Put bluntly: before the SCRAM the reactor was in the condition of an atomic bomb, yet not a single warning signal indicated it. By what means, then -- by smell, by touch -- were the personnel expected to discern this?

On 26~April the catastrophe resulted from the combined action of the SCRAM, owing to the defective control-rod design, and the positive prompt power coefficient. In other circumstances, either factor alone could have sufficed.

About the control rods nothing further need be said; the matter is clear. The power coefficient, however, demands attention. Above all, the steam-void effect must be re-evaluated in light of N.~Laletin's analysis (§4.2).

The RBMK was most dangerous at power levels up to roughly \qty{40}{\percent} -- specifics depending on coolant flow -- because of its large positive prompt power coefficient. Changes in coolant density (hence reactivity) per unit power are far greater at low power. Although reactivity is not linearly proportional to density, the qualitative behaviour remains.

Before the accident no document discussed this regime. Only afterwards were studies undertaken at low power; see, for example, Kurchatov Institute Report No.~ZZR/1-1007-90 (Sept.\ 1990).

After measures reducing the steam-void coefficient to \qty{0.8}{\betaeff}, the power coefficient at \qty{200}{\mega\watt} became $-6\text{-}10\times10^{-7}\,\unit{\betaeff\per\mega\watt}$. What was it when the steam-void coefficient was \qty{+5}{\betaeff}? And note: contrary to the experts' assertions, the danger increases as coolant flow \textit{decreases}.

\paragraph{7.\ Actions of the Personnel}

Before discussing the personnel's ``guilt,'' consider the primary fact: the reactor was destroyed by the \textit{emergency protection system}.

If Mr.~Valley concluded -- using the factual material of the appendices -- that the accusations are unjust, then he is correct. But such a conclusion is impossible from the INSAG report alone; indeed, it points in the opposite direction.

In 1986 V.~Legasov and A.~Abagyan suppressed the fact that the scram inserted positive reactivity -- so odious a fact that it could not be admitted. The experts now write that had they known, their opinion would have differed. They learned -- and still concluded by \textit{implicitly blaming} the operators (see §5.1 above). Even during the height of the anti-operator campaign, such equivocation had not been attempted.

\emph{Report, §6.6:} \textit{``Nevertheless, INSAG still holds the view that the operators' critical actions were for the most part erroneous.''}

Imagine a reactor conforming to the design norms. Which actions of the operators would then be ``erroneous'' or ``critical''? Why should the operators have compensated for design errors unknown to them?

Only the lack of legal grounds for prosecuting the staff forced Legasov and Abagyan, in 1986, to resort to outright falsehood. Their motives are intelligible. What is astonishing is how readily the experts adopted this line, assuming the rôle of accusers -- condemning people before the world for violating documents the experts themselves had never seen. Bound by their first report, they were compelled to maintain the same direction in the second.

INSAG-7, like its predecessor, misinterprets processes; and even where its general statements are accurate, its tendentious framing renders them false. It cannot play a constructive rôle.

For the publication of Appendices~I and II the experts deserve thanks: the factual material is scrupulously correct and valuable. But the conclusions and evaluative judgements must be approached with utmost caution. Thus Appendix~II declares: \textit{``These characteristics of the reactor installation ensured reliable and effective RBMK operation in all regulated modes, and safety for all listed design-basis accidents in accordance with the approved design documentation.''}

This is demonstrably untrue:

\begin{itemize}
    \item The list did \textit{not} include a main pipe break.
    \item Under a main pipe break the reactor exploded.
\end{itemize}

\medskip

\noindent\textit{Former Deputy Chief Engineer Anatoly Dyatlov\\
Kyiv, 1995}

\chapter*{Appendix A: About some issues in the operation of atomic power stations in the USSR}
\addcontentsline{toc}{chapter}{Appendix A: About some issues in the operation of atomic power stations in the USSR}
\markboth{Appendix A: About some issues in the operation of atomic power stations in the USSR}{}

\textit{Translator's note:} This document originates from the archives of the Ukrainian KGB. It is a report addressed to the KGB Administration of the USSR for the city of Moscow and Moscow Oblast, dated 20 May 1983, from Lieutenant Colonel A. I. Samoylov, Chief of the 3rd Department of the 6th Service of the KGB Administration of the USSR for the city of Moscow and Moscow Oblast. The report discusses safety issues at Soviet atomic power stations (AES). The original document contains several underlined passages, which are noted in the text. The document discusses weaknesses in the technical designs of nuclear power plants in the USSR and their potential consequences, concluding that the Leningrad, Kursk, and Chernobyl plants are dangerous. Neither this document nor the appendices A through D were part of A. S. Dyatlov's original Russian manuscript. They have been included here for historical context, and made public only in 2019.

\varthreestars

The engineering process at an AES (atomic power station) is connected with the formation and accumulation of radioactive products in the core of the reactor (the primary loop and heat-emitting elements, i.e. fuel rods). In the instance where the radioactive products go beyond established limits, it could lead to radioactive contamination of the territory of the AES and of extensive areas adjoining it. Because of this, the AES is a potential source of radioactive danger for its service personnel and the surrounding population.

For instance, in the case of a breach of the main circulation pipeline as a result of natural aging of the metal and the absence of an emergency inundation of the core and a protective shell around the reactor, the coolant will leak from the [primary] loop in \numrange{10}{25} seconds. In this way, there could be a leak of radioactive products in the coolant, the most dangerous of which are isotopes of iodine- [blank space], which will affect the thyroid gland and cause death. In the epicenter of the accident, radioactivity will be \num{60} times higher than it was in the explosion of the atomic bombs at Hiroshima and Nagasaki.

Specialists have calculated that the explosion, for example, of the main circulation pipeline at the Leningrad AES would lead to the contamination of the city of Leningrad and Leningrad Oblast, as well as a significant territory of Finland.

There are other possible reasons for emergency situations to arise. An example would be the cable fire that took place in 1982 at the Armenian AES, as a result of which all the primary components of the reactor lost power. The pumps providing water to the steam generator stopped working and the danger arose that the heat-emitting elements would fail and, as a consequence, the leak of radioactive products. Panic broke out, and people left their work places. Only a quick connection of a reserve cable to the engines permitted a tragedy to be avoided. This case served as an impetus to create an emergency reactor core inundation system.

These safety measures are lacking in the following units, which are in use at the present time: Units 1 and 2 of type AMB reactors at Beloyarsk AES, AMB Units 1 and 2 and VVER-440 type reactor Units 3 and 4 at Kola AES, VVER-440 Units 1 and 2 at the Armenian AES, RBMK-1000 type reactor Units 1 and 2 at Leningrad AES, RBMK-100 reactor Units 1 and 2 at Kursk AES, and RMBK-1000 reactor Units 1 and 2 at Chernobyl AES. [comment: this passage was underlined]

It is necessary to note that the design of the VVER-440 reactor allows for the possibility of constructing a protective shell around the reactor without stopping the operation of the atomic energy station. The design of the RBMK-1000 reactor was developed so that installation of such a shell is practically impossible (let alone without ceasing the operation of the reactor). For this reason, Leningrad, Kursk, and Chernobyl AES at the present time are the most dangerous with regards to their future use, which could have alarming consequences. [comment: this passage was underlined]

Chief of the 3rd Department of the 6th Service of the KGB Administration of the USSR for the city of Moscow and Moscow Oblast Lieutenant

\begin{flushright}\small{
-- Lieutenant Colonel A.I. Samoylov, [Signature]
}\end{flushright}

20 May 1983

\chapter*{Appendix B: About the emergency situation at the 3rd and 4th energy block of the Chernobyl Nuclear Power Plant}
\addcontentsline{toc}{chapter}{Appendix B: About the emergency situation at the 3rd and 4th energy block of the Chernobyl Nuclear Power Plant}
\markboth{Appendix B: About the emergency situation at the 3rd and 4th energy block of the Chernobyl Nuclear Power Plant}{}

\textit{Translator's note:} This document also originates from the archives of the Ukrainian KGB. It is a report dated 1 May 1984, from the Head of the Pripyat City Department of the Ukr. SSR KGB Administration to the Head of the USSR KGB Administration for the city of Kiev and Kiev Oblast, Lieutenant General M.Z. Banduristiy. This report discusses to violations of reactor design plans and the disintegration of load-bearing concrete due to extreme temperatures and improper wall insulation.

\varthreestars

March 1, 1984 \\

N 89/363 \\

Secret \\

Copy 3 \\

To the Head of the USSR KGB Administration \\

For the City of Kiev and Kiev Oblast \\

Lieutenant General Comrade M.Z. BANDURISTIY \\

According to available intelligence information obtained from agent \textit{"Yuri,"} resident \textit{"Azis,"} trusted individual \textit{"F.V.I.,"} and what is officially available, it has become known that at the vertical markers for \numlist{35.5; 39.0; 43.0} meters of the Chernobyl Nuclear Power Plant's Unit~3, load-bearing and non-load-bearing elements (wall panels) are being damaged in the premises of the reactor area, specifically fractures in the concrete slabs, the removal of spar pieces and concrete slabs, and the removal of reinforced concrete and ceramic curtain panels. Considering the fact that the spar pieces have a design that transfers the load from the slabs and technical equipment on them, including the drum separators, the situation created presents a particular danger for the main building of the third energy block.

We informed the Administration of the Chernobyl Nuclear Power Plant and its Construction Administration about the current issue, after which the departmental commission, composed of Chernobyl's specialists and a work design group from the S.Ia.Zhuk research design and scientific research institute \textit{"Hydroproject,"} noted that the process of destruction of spar pieces is actively occurring, as well as fractures in the protective layer that reach a depth of \qty{5}{\milli\metre} and a height of \qty{200}{\milli\metre} for its entire length. In some places, it was revealed that the spar pieces' protective layer fell up to \qty{50}{\milli\metre}, with a surface area of \qtyproduct{400 x 400}{\milli\metre}. The removal of reinforced concrete and ceramic wall panels occurred from the axis up to a distance of \qty{30}{\milli\metre}.

The observations conducted by the members of the commission show that further decline of concrete slabs with spar pieces has occurred recently (2 or 3 months).

Initial analysis makes it possible to calculate that the cause of this is the significant overheating of the walls of the drum separators because of ineffective thermal insulation (silicate cotton), which is destroyed by high temperatures and constant radiation.

According to technical conditions of the use of an atomic energy station of this type, the temperature in the area of the drum separators is maintained at about \qty{270}{\celsius}, and the inner surface of its reinforced concrete walls, protected by thermal insulation, should have a temperature no higher than \qty{90}{\celsius}. However, at the present time the temperature of these walls is higher than \qty{160}{\celsius}, at which temperature concrete begins to disintegrate.

Currently, with the goal of preventing the destruction of concrete slabs and their collapse, the Administration of the Chernobyl Nuclear Power Plant, with the agreement of the \textit{"Hydroproject"} institute, has taken temporary measures to fortify the load-bearing constructions, but which do not solve the problem that has arisen.

For the reader's information, we also note that a similar situation is taking place in the 4th~Unit of the Chernobyl Nuclear Power Plant.

In consideration of the above, we consider it expedient that high-level authorities and competent specialists re-examine the given information with the goal of determining the true causes of the destruction of load-bearing and other structures, as well as the elimination of circumstances that could lead to a serious accident.

Head of the Pripyat City Department of the Ukr. SSR KGB Administration \\

For the city of Kiev and the Kiev Oblast \\

Lieutenant Colonel, [Signature]

\chapter*{Appendix C: About an interview with trusted individual Zh. V. A.}
\addcontentsline{toc}{chapter}{Appendix C: About an interview with trusted individual Zh. V. A.}
\markboth{Appendix C: About an interview with trusted individual Zh. V. A.}{}

\textit{Translator's note:} This document also originates from the archives of the Ukrainian KGB. It is a report that relays a conversation with a specialist in nuclear energy who explains how gaps at the joints of pipes are causing problems in the blocks at both the Chernobyl and Kursk plants.

\varthreestars

On July 18th, 1984, an interview was held with trusted individual \textit{"Zh. V. A,"} a highly qualified specialist in the area of atomic energy and an employee at the \textit{"Energy"} All-Russian Scientific Research Institute for the Use of Atomic Energy Stations NGO.

He communicated that at the Chernobyl and Kursk atomic energy stations there have been cases of pipeline breaches in the place where the zirconium cover joins with the steel pipes, or the steel-zirconium adapter, as it is called. It occurs in the following way: the zirconium cover is placed in a steel pipe, in which special mushroom-shaped depressions are made, and the steel pipe is compressed at a temperature of \qty{900}{\celsius}. As a result of the pressure and temperature, the diffused zirconium fuses with the steel at a thickness of several microns. The breach occurs in the lower part of the joint, since it is technologically impossible to reach the lack of a small, but nonetheless present, gap between the steel and zirconium in this part of the joint. Since steel and zirconium expand at different rates during use, an increase in the gap occurs, into which corrosion particles and impurities from the water enter, causing further damage. Earlier it was thought that this damage occurs because of an incorrect ratio of titanium to carbon, which should be no less than 5, but the most recent accidents demonstrate that even with a ratio of 8 these gaps occur.

As far as the trusted individual is aware, such problems were not observed at the Leningrad atomic energy station, since bushing is stretched over the joints that heats up first and additionally pulls the steel-zirconium adapter together when it cools. However, it is technologically difficult for the metallurgical factory in the city of Glazove, in the Udmurt ASSR to produce joints like the ones at the Leningrad atomic energy station, because the quantity of such joints that is needed is large, and the time needed to produce them is increasing. The factory will not be able to fulfill the output plan it was given for steel-zirconium joints in accordance with the plan for launching the RBMK type reactor.

\textit{"Zh. V. A."} communicated that such failures are observed in the first and second blocks of the Chernobyl atomic energy station and block II of the Kursk atomic energy station. There are no such failures in block I of the Kursk atomic energy station because this reactor is the first of its type and all of its joints have been thoroughly checked. At the present time the steel-zirconium adapters have been placed online and their quality control, in the opinion of our trusted individual, is weak. To fully abandon the use of such adapters is unthinkable, but it is necessary to better develop the technology of their manufacture, and this, \textit{"Zh. V. A."} says, is realistic.

In connection with this defect in the R.B.M.K. another problem arises, which is the search for all damaged joints, since there are around \num{1700} pieces within the reactors and they are situated in the lower part of the reactor, the most dangerous area in terms of radioactivity. For now, it is possible to determine whether a damaged adapter is present from the appearance of a wet spot in the reactor's graphite moderator. This spot encompasses several adapters and it is difficult to find the damaged one from among them. For this reason, there are times when undamaged adapters are also currently being removed. Work is being conducted in this direction, but so far there are no positive results.

On the whole, according to the opinion of our trusted individual, reactors of such a type still need to be manufactured, since there is not enough power in the country for the production of tank reactors.

Operative Plenipotentiary of Division I, Department 2 of the Sixth Service USSR KGB Administration for Moscow and Moscow Oblast

\begin{flushright}\small{
-- Captain A. E. Nikifiorov, [Signature]
}\end{flushright}

\chapter*{Appendix D: Report from Colonel M. A. Turko}
\addcontentsline{toc}{chapter}{Appendix D: Report from Colonel M. A. Turko}
\markboth{Appendix D: Report from Colonel M. A. Turko}{}

\textit{Translator's note:} This document, dated 1984, also originates from the archives of the Ukrainian KGB. It summarizes the specialists' report on the lack of reliability of the reactors at Chernobyl, citing that the lack of protective layers and other structural flaws in the reactor that could lead to radioactive contamination and accidents.

\varthreestars

\begin{flushright}\small{
Secret

Copy 2

To the Director of the Pripyat City Department of the Ukr. SSR KGB Administration

For the City of Kiev and Kiev Oblast

To Lieutenant Colonel Comrade Iu. V. Nikolaev
}\end{flushright}

According to information from the USSR KGB Administration for Moscow and the Moscow Region, in the process of an investigation of organizations and institutions connected with the development and use of atomic energy stations in our country, information has been obtained about the inadequate reliability of RBMK-1000 type reactors used in the Chernobyl atomic energy station.

According to the opinions of specialists consulted, if there were to be an explosion in the main circulation pipeline as a result of natural corrosion of metal and no emergency extinguishing system in the reactor's active zone or a protective cover surrounding it, a coolant leak will occur. As a result of this, a significant radioactive contamination of the area will occur. An electric power loss in the administrative work system of the reactor could cause a serious accident. According to available information, the RBMK-1000 type of reactor is not equipped with an emergency extinguishing system in the reactor's active zone, nor with emergency power backup, particularly for the pumps. The design of the reactor also does not consider a protective safety cover.

The RBMK-1000 reactors also have a unique design aspect, in that the zirconium fuel rod shield fuses with pipes made of steel alloy. Because of the difference in the temperature coefficients in the joints, microfractures can form and ruptures can occur. Cases of ruptures have occurred in the Chernobyl atomic energy station, however because the station is run by the USSR Ministry of Energy and included in the energy system of the European part of the USSR, the station was not shut down in these instances. According to specialists at the Leningrad atomic energy station, where RBMK-1000 type reactors are also being used, similar ruptures did not occur because an additional bushing is stretched over the steel-zirconium adapter, which increases the reliability of the joint. The presence of a damaged adapter appears as a wet spot in the graphite moderator of the reactor.

The reactor's design flaws, as well as individual violations of the rules of its use could cause serious accidents.

In consideration of the above, we request that a careful analysis of the technical status of the reactors in blocks 1 and 2 be conducted, to determine, together with specialists, the parts of the current plant most vulnerable to create the preconditions for an accident; to conduct a similar analysis of the causes of earlier accidents that have taken place and of the measures that the administration has taken to evaluate their effectiveness and reliability; and to study the violations of the rules of this type of reactor's use that have been uncovered through intelligence operations. At the same time, we request communication on which measures are being taken to increase the reliability of the reactors in the energy blocks that are being newly constructed.

We request that there be no delay in submitting the necessary report to the KGB with a response.

\begin{flushright}\small{
Director of the 6th Department of the KGB Administration

Colonel M.A. Turko

[Handwritten: on 14.08.84, Verfied: (Signature)]
}\end{flushright}

\chapter*{Appendix E: General Provisions for Ensuring the Safety of Nuclear Power Plants in Design, Construction, and Operation (OPB-82)}
\addcontentsline{toc}{chapter}{Appendix E: General Provisions for Ensuring the Safety of Nuclear Power Plants in Design, Construction, and Operation (OPB-82)}
\markboth{Appendix E: General Provisions for Ensuring the Safety of Nuclear Power Plants in Design, Construction, and Operation (OPB-82)}{}

Introduced in place of the ``General Provisions for Ensuring the Safety of Nuclear Power Plants during Design, Construction, and Operation,'' approved by the State Committee for the Utilization of Atomic Energy of the USSR (\textit{GKAE USSR}), the Ministry of Energy and Electrification of the USSR (\textit{Minenergo USSR}), and the Ministry of Health of the USSR (\textit{Minzdrav USSR}) in 1973.

Approved by \textit{GKAE USSR}, \textit{Minenergo USSR}, \textit{Minzdrav USSR}, and the State Mining--Technical Supervision of the USSR (\textit{Gosgortekhnadzor USSR}), and agreed with the State Committee for Construction of the USSR (\textit{Gosstroy USSR}) in July 1982.

\section*{1. General Provisions}
\markright{1. General Provisions}
\subsection*{1.1. Purpose of the Document}
\markright{1.1. Purpose of the Document}

\paragraph{1.1.1.}
The \textit{General Provisions for Ensuring the Safety of Nuclear Power Plants During Design, Construction, and Operation}\footnote{Hereinafter referred to as the ``General Provisions.''} are mandatory for all ministries, enterprises, and organizations engaged in the design, development, and manufacture of equipment, as well as in the construction and operation of nuclear power plants.

The \textit{General Provisions} do not apply to transport reactors or to reactor installations of special purpose.\footnote{The \textit{General Provisions} may be used as a basis for the design and operation of research reactors.}

The \textit{General Provisions} address those aspects of safety arising from the specific nature of a nuclear power plant (NPP) as a source of ionizing radiation and radioactive substances, and do not concern those aspects which are governed by general-purpose documents unrelated to the aforementioned safety considerations.

\paragraph{1.1.2.}
The \textit{General Provisions} constitute the principal regulatory document on the safety of nuclear power plants. They contain organizational and technical requirements, the fulfillment of which is a necessary condition for ensuring the safety of an NPP, and, as a rule, do not specify the methods that must be employed to achieve this goal.

Specific means, whether recommended or obligatory, for attaining safety are set forth in special standards and rules.

\paragraph{1.1.3.}
As experience accumulates in the design, construction, and operation of NPPs with reactors of various types, amendments to the \textit{General Provisions} are introduced by decision of the authorities that have approved the document.

\paragraph{1.1.4.}
The \textit{General Provisions} are mandatory for application at nuclear power plants with reactors of all types for which, at the moment of introduction of the \textit{General Provisions} in the present edition, technical designs have not yet been approved.

The timeframes and scope for bringing existing and under-construction nuclear power plants into conformity with the \textit{General Provisions} are determined in each specific case by the authorities that have approved the \textit{General Provisions}.

For nuclear power plants whose purpose or reactor type is not reflected in the present document, as well as in special cases of NPP siting, the authorities that have approved this document may partially waive or additionally establish the requirements for ensuring safety.

\subsection*{1.2. Fundamental Principles and Safety Criteria}
\markright{1.2. Fundamental Principles and Safety Criteria}

\paragraph{1.2.1.}
A nuclear power plant shall be considered safe if, through technical means and organizational measures, it is ensured that the established limits for internal and external exposure doses to its personnel and to the population, as well as the permissible levels of radioactive substances in the environment, are not exceeded during normal operation and design-basis accidents.

\paragraph{1.2.2.}
The maximum permissible radiation doses for personnel, the dose limits for the population, and the standards for the content of radioactive substances in the environment during normal operation and design-basis accidents shall be established in accordance with the current radiation safety regulations.

\paragraph{1.2.3.}
The safety of an NPP shall primarily be ensured by the following measures:

\begin{itemize}
    \item selection of a favorable site for the NPP and adequate remoteness from large population centers;
    \item establishment of the necessary sanitary protection zone around the NPP;
    \item equipping the NPP with safety systems;
    \item high quality of the designs of systems (elements) important to safety, and of the plant as a whole;
    \item high quality of manufacture, installation, repair, and reconstruction of equipment and pipelines;
    \item high quality of construction and assembly work in accordance with the design documentation;
    \item maintaining in reliable condition the systems important to safety through preventive measures (periodic inspection of equipment condition, verification of operability, repair) and replacement of worn equipment;
    \item operation of the NPP in accordance with the applicable regulatory and technical documentation and operating instructions;
    \item personnel qualification.
\end{itemize}

\paragraph{1.2.4.}
The design of an NPP shall provide for technical means and organizational measures ensuring safety for any design-basis initiating event\footnote{Hereinafter, instead of ``design-basis initiating event,'' the term ``initiating event'' is used.} in combination with one failure, independent of that event, of any of the following elements\footnote{In specific cases, when a high level of reliability of the above-mentioned elements or of the systems including them is demonstrated, or when an element is withdrawn from service for a short time for maintenance, their failures may be disregarded. Reliability is considered high if the reliability indicators of such elements are not lower than those of safety system elements whose failures are not considered. The permissible time for withdrawing an element from service for maintenance shall be determined based on reliability analysis of the system to which it belongs.} of the safety systems: an active element, or a passive element having mechanical moving parts.

In addition to one failure independent of the initiating event of any of the elements listed above, there shall also be considered undetected failures of NPP elements not subject to in-service monitoring\footnote{See previous footnote.} that may lead to violations of the limits of safe operation and influence the development of an accident.

\paragraph{1.2.5.}
The minimum lists of initiating events to be considered for NPPs with various reactor types shall be determined in accordance with the requirements of the present Provisions and shall be given in the \textit{Typical Contents of the Technical Safety Justification of a Nuclear Power Plant}. The complete list of initiating events to be considered shall be agreed upon with the bodies of State Supervision and shall be included in the volume \textit{Technical Justification of the Safety of Construction and Operation of an NPP} (see §2.1.14).

\paragraph{1.2.6.}
The most severe of the initiating events to be considered for each reactor type are given in Section~4.

\paragraph{1.2.7.}
The design shall provide technical means to prevent large-scale melting of fuel elements during initiating events in accordance with §§1.2.5 -- 1.2.6. Requirements for the effectiveness of safety systems with regard to the admissibility of fuel element damage are set forth in Section~4.

\paragraph{1.2.8.}
Protection of personnel and the population in the event of hypothetical accidents at an NPP shall be ensured by developing and implementing, within the plant site and surrounding territory, a plan of measures for protecting the population and personnel in accordance with current regulations. The plan shall be developed by the plant management before commissioning and approved in the prescribed manner. Initial data on radioactive releases into the environment during events exceeding the maximum design-basis accident (MDBA) shall be prepared jointly by the scientific supervisor, chief designer, and general designer.

\subsection*{1.3. State Supervision of NPP Safety}
\markright{1.3. State Supervision of NPP Safety}

\paragraph{1.3.1.}
State supervision of the safety of nuclear power plants is carried out by:

\begin{itemize}
    \item the State Committee for the Utilization of Atomic Energy of the USSR (Gosatomnadzor USSR) -- for nuclear safety;
    \item the State Committee of the USSR for Supervision of Safe Practices in Industry and Mining (Department for Supervision in Nuclear Energy) -- for technical safety;
    \item the Ministry of Health of the USSR (State Sanitary Supervision) -- for radiation safety.
\end{itemize}

\paragraph{1.3.2.}
The bodies of State Supervision over NPP safety perform their functions in accordance with their respective regulations and act in conformity with the rules and standards approved by them and developed in pursuance of the present \textit{General Provisions}.

\paragraph{1.3.3.}
The decisions of the State Supervision bodies on safety matters are binding upon all ministries and organizations involved in the design, construction, and operation of NPPs, as well as in the development, manufacture, and supply of equipment for them.

\paragraph{1.3.4.}
Before approval in the prescribed manner, all technical designs of reactor installations and NPPs shall be coordinated by the design organization with the State Supervision bodies.

\paragraph{1.3.5.}
Organizations and ministries engaged in the design of NPPs, in the development, manufacture, construction, and operation of NPP equipment, shall, upon request of the State Supervision bodies, provide detailed information in the form of design materials, research and calculation results, operating instructions, reports on completed tests and inspections of structures and equipment, materials on quality control of equipment manufacture, data on personnel training, detailed operational data, compliance with limits and conditions of safe operation, accident reports, and analyses thereof.

\paragraph{1.3.6.}
Before the start of operation, the NPP management shall submit to the State Supervision bodies, for approval, data on modifications made jointly with the plant and equipment designers to the solutions approved at the design stage; results of testing and adjustment of systems and equipment important to NPP safety; and limits and conditions of safe operation corrected on the basis of commissioning results.

\paragraph{1.3.7.}
Completed, expanded, or reconstructed NPPs shall be commissioned only after acceptance in accordance with applicable construction standards and regulations. Permission for commissioning shall be issued by the State Supervision bodies on the basis of the review of the necessary materials, corrected according to the results of pre-commissioning work and of the physical and power start-ups of the reactor.

\paragraph{1.3.8.}
The NPP management is obliged to inform the State Supervision bodies of any cases of abnormal operation of equipment important to safety. The list of malfunctions requiring such reporting shall be approved by the State Supervision bodies.

\section*{2. Safety Requirements for the Nuclear Power Plant and Its Systems}
\markright{2. Safety Requirements for the Nuclear Power Plant and Its Systems}
\subsection*{2.1. General Requirements}
\markright{2.1. General Requirements}

\paragraph{2.1.1.}
A nuclear power plant, as a source of ionizing radiation and radioactive substances, shall comply with the requirements of the present \textit{General Provisions}, as well as with the standards and regulations on nuclear, technical, and radiation safety of NPPs.

\paragraph{2.1.2.}
The NPP design shall provide for safety systems intended for:
\begin{itemize}
    \item emergency shutdown of the reactor and maintenance of its subcritical state;
    \item emergency heat removal;
    \item retention of radioactive products within established limits.
\end{itemize}

\paragraph{2.1.3.}
All systems and devices of an NPP important to safety shall comply with the present \textit{General Provisions}, the applicable standards and regulations for the design, manufacture, and installation of nuclear equipment, and other normative documents.

During manufacture, installation, repair, or reconstruction of equipment, and during its commissioning, inspectional supervision and acceptance by the competent authorities shall be provided.

\paragraph{2.1.4.}
Systems and devices of an NPP important to safety shall be designed, manufactured, and installed with consideration of possible mechanical, thermal, chemical, and other effects arising as a result of design-basis accidents.

\paragraph{2.1.5.}
Systems and devices of an NPP important to safety shall be designed, manufactured, and installed with consideration of natural phenomena characteristic of the site region, such as earthquakes, hurricanes, floods, winds, and the like.

\paragraph{2.1.6.}
Safety systems shall be capable of performing their assigned functions under the conditions of effects associated with the aforementioned natural phenomena.

\paragraph{2.1.7.}
Multipurpose use of safety systems and their elements at an NPP is, as a rule, prohibited. In particular cases, multipurpose use may be permitted if it is substantiated in the design that the combination of functions will not result in violation of safety assurance requirements.

\paragraph{2.1.8.}
The use of passive devices in safety systems is preferable.

\paragraph{2.1.9.}
Systems and devices of an NPP important to safety shall be subjected to periodic verification throughout the entire service life of the plant and after repairs. Maintenance and inspections shall not result in any reduction of the safety level.

\paragraph{2.1.10.}
The design of systems and devices of an NPP important to safety, as well as during commissioning, shall provide for:
\begin{itemize}
    \item means and devices for checking the operability of systems and components (including those located inside the reactor);
    \item devices for testing systems for compliance with their design parameters;
    \item devices for verifying the sequence of signal transmission and equipment activation (including switching to emergency power sources);
    \item means and devices for monitoring the condition of metal and welded joints of equipment and pipelines, ruptures of which are initiating events of accidents;
    \item a schedule for maintenance and inspections.
\end{itemize}

\paragraph{2.1.11.}
For systems and devices important to safety, a direct and complete verification of conformity to design characteristics shall, as a rule, be carried out. If such verification is impossible, indirect verification or partial testing shall be performed, providing for the corresponding means and methods.

\paragraph{2.1.12.}
The NPP design shall provide means that, insofar as possible, eliminate or mitigate the consequences of personnel errors that may aggravate the effects of a failure of any device.

\paragraph{2.1.13.}
The NPP design documentation shall contain a quantitative analysis of the reliability indicators of systems important to safety.\footnote{The analysis shall be performed to the extent permitted by the accumulated data on the reliability of equipment and systems.}

\paragraph{2.1.14.}
The NPP design shall include a dedicated volume entitled \textit{Technical Justification of the Safety of Construction and Operation of a Nuclear Power Plant} (TJS), prepared by the general designer, the chief designer, and the scientific supervisor in accordance with the \textit{Typical Contents of the TJS}.  
An analogous TJS shall be prepared for the reactor installation by the chief designer and the scientific supervisor, covering the relevant issues.

\paragraph{2.1.15.}
A quality assurance program for the construction and operation of the NPP shall be developed, defining the responsibilities of design, engineering, construction, and installation organizations, equipment manufacturers, the operating organization, and personnel.\footnote{This requirement shall take effect after the development and approval of a standard content for such a document.}

\subsection*{2.2. Design and Characteristics of the Reactor Core}
\markright{2.2. Design and Characteristics of the Reactor Core}

\paragraph{2.2.1.}
In designing the reactor core, the permissible limits of damage (extent and degree of damage) to fuel elements under normal operating conditions, and the associated levels of radioactivity of the primary coolant, shall be established in advance.

The reactor core and other systems determining its operating conditions shall be designed so as to preclude exceeding the established fuel element damage limits throughout the entire design service life under normal operating conditions.  
Exceeding these limits shall likewise be precluded under any of the following deviations from normal operation (with due account of protective system action):

\begin{itemize}
    \item malfunctions of the reactor control and monitoring system;
    \item loss of power supply to the main circulation pumps;
    \item disconnection of turbogenerators and heat consumers;
    \item total loss of external power sources;
    \item leaks in the primary circuit compensated by standard make-up systems.
\end{itemize}

\paragraph{2.2.2.}
As a rule, the prompt power coefficient of reactivity shall not be positive under any NPP operating mode or any state of the primary circuit heat removal system.

If the prompt power coefficient of reactivity is positive in any operating mode, the design shall ensure and substantiate the safety of the reactor under steady-state, transient, and accident conditions.

\paragraph{2.2.3.}
The characteristics of the nuclear fuel and the design of the reactor and other equipment of the primary circuit (including the coolant purification system), together with other systems, shall preclude the formation of critical masses under any accident conditions, including those resulting in destruction of the reactor core or melting of the fuel.  
If this condition cannot be met, the design shall demonstrate that core destruction and fuel melting leading to criticality cannot occur, taking into account additional failures in safety systems, as specified in §1.2.4, in accordance with supplementary requirements for the reactor type concerned.

\subsection*{2.3. Reactivity Control Systems}
\markright{2.3. Reactivity Control Systems}

\paragraph{2.3.1.}
At least two independent reactivity control systems (two independent control rods or two independent groups of rods) shall be provided, preferably based on different principles.

\paragraph{2.3.2.}
At least two of the independent reactivity control systems provided shall each be capable, independently of the other, of ensuring transition from any state of normal operation to a subcritical state and maintaining that state at the operating temperature of the coolant and moderator.  
The transition to a subcritical state shall occur rapidly enough to prevent fuel element damage beyond permissible limits during any considered initiating event.

\paragraph{2.3.3.}
At least one of the independent reactivity control systems provided shall ensure transition from any state of normal operation to a subcritical state under all temperature conditions and during transient processes of the considered initiating events.  
The transition to a subcritical state shall occur rapidly enough to prevent fuel element damage beyond permissible limits for any considered initiating event, in accordance with the single-failure criterion for the given system, including failure of the most effective reactivity control rod to operate.

Division of the full range of reactivity changes during the specified transients into several temperature or operational ranges is permissible, using for each range part of the system (certain control rods or groups of rods), with the single-failure criterion applied to each part of the system.

\paragraph{2.3.4.}
At least one of the independent reactivity control systems provided shall ensure transition from any state of normal operation to a subcritical state and maintenance of that state, taking into account possible reactivity release during prolonged cooldown under any normal conditions and considered initiating events, in accordance with the single-failure criterion for the given system and in the event of failure of the most effective reactivity control rod.

\paragraph{2.3.5.}
The reactivity control system, in conjunction with the characteristics of the reactor core, shall ensure the absence or prompt suppression, during normal operation, of such power and power-distribution oscillations as could, over the course of a core campaign, cause fuel element damage exceeding the limits for normal operation.

\paragraph{2.3.6.}
In the event of a single malfunction in the control and instrumentation system, the reactivity control system shall ensure suppression of the positive reactivity associated with the withdrawal of control rods (within design speeds) without causing fuel element damage exceeding the limits for normal operation.

\paragraph{2.3.7.}
The maximum effectiveness of reactivity control rods and the maximum possible rate of reactivity increase resulting from personnel error or single malfunction of any NPP device shall be limited so that the resulting power increase will not cause:
\begin{itemize}
    \item the maximum permissible pressure in the primary circuit to be exceeded;
    \item unacceptable degradation of heat removal effectiveness or melting of fuel elements.
\end{itemize}

\subsection*{2.4. Primary Circuit System}
\markright{2.4. Primary Circuit System}

All equipment and pipelines of the primary circuit shall withstand, without damage, static and dynamic loads and temperature effects arising in any of its nodes and components (taking into account the operation of protective devices and their possible failures in accordance with §1.2.4) during inadvertent energy releases into the primary-circuit coolant caused by:
\begin{itemize}
    \item sudden insertion of reactivity due to ejection, at maximum speed, of the reactivity control rod of maximum effectiveness (see §2.3.7), if such ejection is not precluded by design;
    \item introduction of cold coolant into the core (in the case of a negative coolant temperature coefficient of reactivity);
    \item sharp reduction in coolant flow followed by an increase in reactivity, as well as any considered initiating events leading to disturbance of heat removal from the primary circuit.
\end{itemize}

\subsection*{2.5. Control and Instrumentation System}
\markright{2.5. Control and Instrumentation System}

\paragraph{2.5.1.}
The NPP shall be equipped with a main control room from which the reactor and other plant systems are monitored and controlled during normal operation, deviations from normal operation, and accident conditions.

\paragraph{2.5.2.}
The NPP control and instrumentation system shall provide monitoring and recording of parameters characterizing the plant's operation over all possible ranges of variation, as well as automatic or remote control of systems of normal operation under all operating modes.  
The means of monitoring and recording shall be sufficient to enable subsequent determination of the course of accident initiation and development, and of personnel actions.

\paragraph{2.5.3.}
Means shall be provided for monitoring and controlling the fission process of the nuclear fuel under all modes and conditions in the reactor core (including during fuel replacement), when transition to a critical state is possible.

\paragraph{2.5.4.}
Indicators of the position of reactivity control rods, monitoring of soluble absorber concentration, and indicators of the state of other reactivity control means shall be provided.

\paragraph{2.5.5.}
The design documentation shall include an analysis of the reactor and NPP control system circuits for possible hazardous feedback reactions that could violate the limits of safe operation in the event of circuit faults (such as short circuits, insulation degradation, voltage surges, and induced voltages).  
Before reactor start-up, the systems shall be tested to detect hazardous or spurious responses.

\paragraph{2.5.6.}
Means shall be provided for detecting leakage of primary-circuit coolant.

\paragraph{2.5.7.}
Automatic monitoring, where possible, shall be provided for radioactivity levels of the coolant (for principal radionuclides) and of radioactive waste at their organized discharge points, as well as monitoring of the radiation situation in both manned and unmanned rooms and in the surrounding area.

\paragraph{2.5.8.}
Automatic monitoring, where possible, shall be provided for the conditions of fuel and radioactive waste storage, with signaling of any violations of safety conditions.

\subsection*{2.6. Safety Control Systems}
\markright{2.6. Safety Control Systems}

\paragraph{2.6.1.}
Safety control systems shall be provided to perform the automatic actuation of protective, containment, and supporting systems, and to monitor their operation.

\paragraph{2.6.2.}
Automatic control systems (including electrical, hydraulic, mechanical, and other devices and circuits) shall, through protective systems, prevent or eliminate conditions leading to fuel element damage beyond design limits.

\paragraph{2.6.3.}
The actuation of reactivity control rods shall not depend on the availability of external power sources.

\paragraph{2.6.4.}
The reliability of safety control systems shall be ensured through:
\begin{itemize}
    \item compliance with manufacturing quality requirements;
    \item multichannel configuration of systems;
    \item testing and inspection of components and systems during operation;
    \item provision of uninterruptible power supply.
\end{itemize}

\paragraph{2.6.5.}
Failure of control systems shall generate a signal at the control panel and initiate actions aimed at ensuring NPP safety.

\paragraph{2.6.6.}
The multichannel configuration and independence of channels shall be such that any single failure in the control system (including common-cause failures) does not impair its operability.  
Multichannel configuration shall include at least two independent channels. For full independence, it is preferable to employ different principles (e.g., actuation by different parameters or use of different detectors).

\paragraph{2.6.7.}
Safety control systems shall be separated to such an extent from the control and instrumentation system that any failure or shutdown of an element or channel of the latter does not affect the ability of the safety control system to meet safety requirements.

\paragraph{2.6.8.}
Provision shall be made for manual actuation of safety systems.  
Failure in the automatic actuation circuit shall not prevent manual initiation or the performance of safety functions.  
Manual actuation shall require only a single action (turning a key or pressing a button).

\paragraph{2.6.9.}
Safety control systems shall be designed such that any initiated action proceeds to complete performance of the safety function.  
Return to initial states shall require deliberate sequential operator actions.  
The system design shall minimize the possibility of spurious actuations.

\paragraph{2.6.10.}
Means shall be provided for checking the operability of individual channels and entire control systems during reactor operation.  
If any portion of a control system loses functional capability, appropriate information shall be continuously displayed on the control panel.

\paragraph{2.6.11.}
Provision shall be made for actuation of safety systems and retrieval of reactor status information from an auxiliary control panel, if for any reason (fire, etc.) this cannot be done from the main control room.

\subsection*{2.7. Protective Safety Systems}
\markright{2.7. Protective Safety Systems}

\paragraph{2.7.1.}
Protective systems shall perform safety functions under all considered initiating events and under failures independent of the initiating event, in accordance with §1.2.4.

\paragraph{2.7.2.}
The complex of protective systems shall include an emergency reactor heat removal system consisting of several independent channels and ensuring the required effectiveness under a failure, independent of the initiating event, of any single channel of that system.  
The use of cooling systems (channels) intended for normal operation as systems (channels) of emergency heat removal from the reactor is permissible if they meet the requirements imposed on safety systems.

\paragraph{2.7.3.}
Measures shall be provided to prevent the reactor from reaching a critical state upon actuation of the emergency heat removal system.

\paragraph{2.7.4.}
The actuation of protective safety systems shall not cause damage to the equipment of normal operation systems.  
The design shall substantiate the permissible number of actuations of the protective safety systems (including spurious actuations) during the service life of the NPP, with regard to their effect on the service life of the equipment.

\subsection*{2.8. Containment Safety Systems}
\markright{2.8. Containment Safety Systems}

\paragraph{2.8.1.}
Containment systems shall be provided to confine radioactive substances released beyond the reactor installation during an accident within the design-specified boundaries.

\paragraph{2.8.2.}
The primary circuit shall be located either entirely within sealed enclosures or arranged such that, in the event of design-basis accidents, the released radioactive substances are confined within the boundaries of the sealed rooms.  
In particular cases, directed release of radioactive substances to the environment is permissible if the design substantiates that NPP safety is maintained during such release.

\paragraph{2.8.3.}
Containment systems shall perform their full set of functions for the considered initiating events, under failures independent of those events, in accordance with §1.2.4.  
For NPPs with multiple units, individual containment systems shall be provided for each unit.  
The joint use of individual containment system components for several units is permissible if the impossibility of accident propagation from one unit to others is demonstrated.

\paragraph{2.8.4.}
Containment systems shall perform their functions during accident-related leaks of the primary-circuit coolant, accounting for possible mechanical, thermal, and chemical effects.

\paragraph{2.8.5.}
Where active heat removal is provided to prevent pressure increase in sealed enclosures, several independent heat removal channels shall be provided, ensuring the required effectiveness under a failure independent of the initiating event of any one of them.

\paragraph{2.8.6.}
All penetrations through the containment boundary that must be closed at the onset of an accident to prevent the release of radioactive substances beyond the sealed premises shall be equipped with two isolation valves, one located outside and one inside the containment boundary.  
For pipelines not directly connected with the primary circuit or with the space of sealed premises, installation of a single isolation valve outside the containment boundary is permissible.

\paragraph{2.8.7.}
Means shall be provided for individual testing of penetrations at the design pressure, and the corresponding test requirements shall be specified.

\paragraph{2.8.8.}
The design shall substantiate the adopted permissible degree of leakage of the containment boundary of the containment system and shall specify the methods for achieving the required degree of tightness.  
Compliance of the achieved tightness with the design value shall be verified after installation and periodically checked during operation.  
Commissioning tests shall be conducted at the design pressure; subsequent tests may be performed at lower pressures.  
Equipment within containment enclosures shall withstand testing without damage.

\subsection*{2.9. Supporting Safety Systems}
\markright{2.9. Supporting Safety Systems}

\paragraph{2.9.1.}
Supporting systems shall be provided to supply the safety systems with working media and energy, and to create the conditions for their functioning during design-basis accidents.

\paragraph{2.9.2.}
Supporting systems shall perform their functions under failures independent of the initiating event, in accordance with §1.2.4.

\subsection*{2.10. Fuel and Radioactive Waste Storage System at the NPP}
\markright{2.10. Fuel and Radioactive Waste Storage System at the NPP}

\paragraph{2.10.1.}
The possibility of achieving criticality in the storage facilities for fresh or spent fuel during its placement and movement shall be physically excluded by ensuring appropriate characteristics of the storage equipment, primarily through geometric factors incorporated in the design and construction solutions.

\paragraph{2.10.2.}
Reliable systems for residual heat removal and a suitable chemical composition of the cooling medium shall be provided in spent fuel storage facilities to prevent fuel damage that could result in the release of radioactive substances into NPP premises or into the environment.

\paragraph{2.10.3.}
The NPP design documentation shall include an analysis of the composition and quantity of solid, liquid, and gaseous radioactive waste generated both during normal operation and under accident conditions.  
The means of waste processing, locations and methods for temporary and long-term storage, requirements for air and water purification systems prior to discharge into the atmosphere and natural water bodies, and the methods for transporting waste within the NPP site and to disposal locations shall be defined.

\section*{3. Ensuring the Safety of NPP Operation}
\markright{3. Ensuring the Safety of NPP Operation}
\subsection*{3.1. Organizational Requirements and Operating Documentation}
\markright{3.1. Organizational Requirements and Operating Documentation}

\paragraph{3.1.1.}
The overall responsibility for ensuring the safe operation of an NPP lies with the operating organization.  
The direct responsibility for the safe operation of the NPP rests with the plant management.  
The operating organization shall provide the necessary conditions for safe operation and exercise supervision over NPP safety.

\paragraph{3.1.2.}
The principal document defining the safe operation of an NPP shall be the \textit{Operating Regulations}, containing the rules and main methods of safe operation of the plant, the general procedure for performing operations related to NPP safety, and the limits and conditions of safe operation.  
The regulations shall be developed by the NPP management with the participation of the scientific supervisor, chief designer, and general designer, and shall be approved by the operating organization.

\paragraph{3.1.3.}
On the basis of the approved operating regulations, the NPP management shall ensure timely development of operating instructions.  
The instructions for operating systems and equipment shall contain specific directions for operating personnel on how to conduct work under normal operating conditions, as well as define personnel actions in case of deviations and equipment or system failures.

\paragraph{3.1.4.}
Based on the documents referred to in §2.1.14 and the operating regulations, the NPP management shall organize the development and issuance of special instructions defining the actions of operating personnel to ensure safety under all considered initiating events.

\paragraph{3.1.5.}
The NPP management shall organize the development and issuance, for systems important to safety, of:
\begin{itemize}
    \item schedules for preventive and major overhauls of individual devices, components, and systems;
    \item schedules for testing and checking the operability of the plant's safety systems;
    \item instructions for performing such work, prepared in accordance with the requirements of the operating regulations;
    \item other operational documentation.
\end{itemize}

\paragraph{3.1.6.}
Documents related to the design, manufacture, and construction of the NPP, as well as all documents concerning plant safety, shall be stored at the NPP throughout its entire service life and, after decommissioning, until the plant is fully removed from operation.

\paragraph{3.1.7.}
The NPP shall maintain operating technical documentation in accordance with the present \textit{General Provisions}, as well as with the \textit{Rules for Technical Operation of Power Stations and Networks}, the \textit{Rules of Nuclear Safety of Nuclear Power Plants}, the \textit{Rules for the Design and Safe Operation of Equipment and Pipelines of Nuclear Power Installations}, and other applicable regulations.

\paragraph{3.1.8.}
The records of instruments monitoring the limits and conditions of safe operation shall be of sufficient quality and shall be kept for two core campaigns between refuellings or for two years, except for records related to malfunctions and accidents.  
Such records shall be attached to the materials of malfunction or accident investigations and retained with them throughout the entire operating period of the NPP.

\paragraph{3.1.9.}
Destruction of records shall be carried out under an act that shall also include a brief list of deviations from normal operation and references to accident investigation reports.

\subsection*{3.2. Requirements for Operating Personnel}
\markright{3.2. Requirements for Operating Personnel}

\paragraph{3.2.1.}
By the time of commissioning, the NPP shall be fully staffed.  
Recruitment, training, and certification of operating personnel shall be carried out according to a schedule and under programs approved by the operating organization, in accordance with the \textit{Rules for Technical Operation of Power Stations and Networks}, the \textit{Guidelines for Working with Personnel}, and other regulatory documents.  
During training and retraining, special attention shall be paid to personnel actions in accident conditions.

\paragraph{3.2.2.}
The qualification requirements for operating personnel shall be established by the operating organization.

\subsection*{3.3. Commissioning}
\markright{3.3. Commissioning}

\paragraph{3.3.1.}
Before permanent operation, the following shall be completed:
\begin{itemize}
    \item verification of the conformity of NPP structures to the design;
    \item start-up and adjustment work (including testing of individual systems and equipment);
    \item comprehensive testing of the NPP (including the physical and power start-ups of the reactor).
\end{itemize}
Commissioning of the NPP shall be carried out in accordance with the applicable procedures for such facilities and in conformity with the present \textit{General Provisions}.

\paragraph{3.3.2.}
An NPP unit completed and being commissioned shall be isolated from areas where construction work continues, so that ongoing work or potential disturbances at construction sites do not affect the safety of the operating unit, and, conversely, possible disturbances or accidents at the operating unit do not compromise safety at the unit under construction.

\paragraph{3.3.3.}
Documents regulating start-up and adjustment work shall contain a list of operations potentially hazardous from the standpoint of safety (for example, operations that could uncontrollably bring the reactor core to a supercritical state) and a list of measures preventing accidents.

\paragraph{3.3.4.}
Authorization for the physical and power start-ups of the NPP shall be granted upon permission from the State Supervision authorities.

\paragraph{3.3.5.}
During power start-up, prior to achieving power below the nominal value, permission to proceed to nominal power shall be granted by the State Supervision authorities upon request of the operating organization, agreed upon with the scientific supervisor, chief designer, and general designer.

\paragraph{3.3.6.}
The operating organization authorizes NPP operation on the basis of the act of acceptance of the NPP for operation by the State Acceptance Commission, provided that the relevant documentation, as approved by the State Supervision authorities, is available and in conformity with applicable regulatory and technical documentation.

\subsection*{3.4. Operational Safety}
\markright{3.4. Operational Safety}

\paragraph{3.4.1.}
Normal operation and permissible deviations from it shall be established in the operating instructions, based on the design and technical documentation and the operating regulations, corrected according to the results of the physical and power start-ups.

\paragraph{3.4.2.}
If, due to damage in systems important to safety, the established limits and conditions of safe operation cannot be maintained at any reactor power level, the reactor shall be shut down.

\subsection*{3.5. Radiation Safety of Personnel and Population During Operation}
\markright{3.5. Radiation Safety of Personnel and Population During Operation}

\paragraph{3.5.1.}
Radiation safety during NPP operation shall be governed by the present \textit{General Provisions}, the \textit{Sanitary Rules for the Design and Operation of Nuclear Power Plants}, the \textit{Sanitary Rules for Work with Radioactive Substances and Sources of Ionizing Radiation}, the \textit{Sanitary Rules for the Transport of Radioactive Substances and Sources of Ionizing Radiation}, the \textit{Radiation Safety Standards}, and other applicable documents.

\paragraph{3.5.2.}
The principal organizational requirement is strict observance of the zoning regime established in accordance with the \textit{Sanitary Rules for the Design and Operation of Nuclear Power Plants}, and strict control over the movement of personnel and radioactive materials across established zone boundaries.

\paragraph{3.5.3.}
Radiation monitoring shall be provided for releases of radioactive substances into the atmosphere and discharges into water bodies, and for personnel and transport leaving the NPP site.

\paragraph{3.5.4.}
Measurement of wind direction and speed and other meteorological parameters shall be ensured to assess and predict the radiation situation in the surrounding area during normal operation and in case of an accident.

\paragraph{3.5.5.}
Radiation dosimetric monitoring shall be provided at the NPP and in the surrounding area, with its scope determined by the \textit{Sanitary Rules for the Design and Operation of Nuclear Power Plants}.

\paragraph{3.5.6.}
Strict accounting shall be maintained for the quantity, movement, and locations of all fissile and radioactive materials, including fresh and spent fuel, dismantled radioactive equipment, contaminated tools, clothing, industrial waste, and other sources of ionizing radiation.

\subsection*{3.6. Emergency Planning}
\markright{3.6. Emergency Planning}

\paragraph{3.6.1.}
NPP operating personnel shall be trained for actions in the event of accidents.  
Personnel actions during accidents shall be defined by operational and job instructions, as well as by special emergency instructions.

\paragraph{3.6.2.}
Periodic training exercises shall be conducted to prepare personnel for action under accident conditions, in accordance with the instructions.  
It is recommended that special simulators be used.  
Measures shall be taken to exclude the possibility of training exercises leading to actual accidents, including due to personnel errors.  
Following each exercise, an analysis of operator actions shall be carried out.

\paragraph{3.6.3.}
Accidents occurring at the NPP shall be thoroughly investigated by commissions appointed by higher-level organizations in accordance with applicable regulations.  
The results of accident investigations shall be submitted to the State Supervision authorities, together with conclusions and recommendations approved by the operating organization.

\paragraph{3.6.4.}
On the basis of initial data prepared jointly by the scientific supervisor, reactor chief designer, and general designer, the NPP management shall develop, in accordance with the \textit{Typical Contents}, a \textit{Plan of Measures for the Protection of the Population and Personnel in the Event of Hypothetical Accidents}, to be coordinated with the State Supervision authorities and approved in the prescribed manner (see §1.2.8).  
The plan shall provide for coordination of actions of operating personnel and external organizations such as local authorities, fire departments, police, medical institutions, civil defense bodies, and other relevant organizations.

The plan shall clearly specify who, and under what conditions, notifies external organizations of the initiation of the plan, and shall define the necessary equipment and means for its implementation, including their sources and delivery methods.  
The plan shall be periodically reviewed, and exercises shall be conducted in accordance with it.

\subsection*{3.7. Periodic Inspections and Tests}
\markright{3.7. Periodic Inspections and Tests}

\paragraph{3.7.1.}
Before commissioning and periodically thereafter, in accordance with the applicable rules, standards, and instructions, the NPP shall undergo:
\begin{itemize}
    \item verification of normal functioning of safety systems;
    \item inspection of the base metal and welded joints of equipment and pipelines;
    \item verification of measuring instruments used to establish the limits of safe operation.
\end{itemize}

\paragraph{3.7.2.}
The frequency and scope of periodic inspections shall be defined in the technical design documentation.  
They shall comply with current regulatory documents and depend on the role played by the inspected systems or components in ensuring NPP safety, taking into account available reliability data.

\paragraph{3.7.3.}
Extraordinary inspections may be carried out at the request of the State Supervision authorities.

\subsection*{3.8. Repair Work}
\markright{3.8. Repair Work}

\paragraph{3.8.1.}
Repair of equipment activated or contaminated with radioactive substances shall be carried out only after implementation of measures to minimize radiation exposure of repair personnel, in compliance with the requirements of the relevant regulatory documentation.

\paragraph{3.8.2.}
When repairing equipment affecting reactor reactivity, nuclear safety and monitoring of reactor status shall be ensured.

\paragraph{3.8.3.}
During equipment shutdown for maintenance, repair, and subsequent recommissioning, NPP safety shall not be degraded.

\paragraph{3.8.4.}
Upon completion of repair work, equipment and systems affecting NPP safety shall be tested for operability and conformity to design specifications, with proper documentation of the work performed and test results.

\subsection*{3.9. Decommissioning of NPPs}
\markright{3.9. Decommissioning of NPPs}

\paragraph{3.9.1.}
Decommissioning of an NPP shall be considered at the design and construction stages, and taken into account during operation, repair, and modernization.

\paragraph{3.9.2.}
No later than five years before the end of the design service life of the NPP, the operating organization shall ensure the development of a project plan for decommissioning and coordinate it with the State Supervision authorities.

\paragraph{3.9.3.}
Decommissioning shall be preceded by a comprehensive inspection of equipment and pipelines important to safety, conducted by a commission including representatives of the State Supervision authorities.

\paragraph{3.9.4.}
Based on the materials of the comprehensive inspection, a decision on decommissioning shall be made in the prescribed manner.

\section*{4. Additional Safety Requirements for NPPs with Reactors of Various Types and Purposes}
\markright{4. Additional Safety Requirements for NPPs with Reactors of Various Types and Purposes}

\subsection*{4.1. NPPs with VVER Reactors}
\markright{4.1. NPPs with VVER Reactors}

\paragraph{4.1.1.}
The design shall consider as the \textit{maximum design-basis accident (MDA)} involving primary circuit rupture the instantaneous break of a pipeline of the largest diameter with unrestricted bidirectional coolant discharge, when the reactor operates at nominal power, accounting for possible power exceedance due to measurement and control system tolerances and inaccuracies.

\paragraph{4.1.2.}
The design limit of fuel element damage for normal operation, determining the allowable activity level of the primary-circuit coolant, shall be:
\begin{itemize}
    \item \qty{1}{\percent} of fuel rods with defects of the gas leakage type; and
    \item \qty{0.1}{\percent} of fuel rods with direct contact between coolant and nuclear fuel.
\end{itemize}

\paragraph{4.1.3.}
In the event of primary circuit rupture, the emergency core cooling system shall ensure:
\begin{itemize}
    \item fuel cladding temperature not exceeding \qty{1200}{\celsius};
    \item local oxidation depth of fuel cladding not exceeding \qty{18}{\percent} of initial wall thickness;
    \item fraction of reacted zirconium not exceeding \qty{1}{\percent} of its mass in the reactor core.
\end{itemize}

\paragraph{4.1.4.}
The design shall provide and substantiate the possibility of unloading the reactor core after an MDA involving primary circuit rupture.

\subsection*{4.2. NPPs with RBMK Reactors}
\markright{4.2. NPPs with RBMK Reactors}

\paragraph{4.2.1.}
The design shall consider as the MDA involving primary circuit rupture the instantaneous break of a pipeline of the largest diameter with unrestricted bidirectional coolant discharge, when the reactor operates at nominal power, accounting for possible power exceedance due to control and monitoring system inaccuracies and tolerances.

\paragraph{4.2.2.}
The design limit of fuel element damage for normal operation, determining the allowable activity level of the primary-circuit coolant, shall be:
\begin{itemize}
    \item \qty{1}{\percent} of fuel rods with gas leakage defects; and
    \item \qty{0.1}{\percent} of fuel rods with direct contact between coolant and nuclear fuel.
\end{itemize}

\paragraph{4.2.3.}
In the event of primary circuit rupture, the emergency cooling system shall ensure:
\begin{itemize}
    \item fuel cladding temperature not exceeding \qty{1200}{\celsius};
    \item local oxidation depth of fuel cladding not exceeding \qty{18}{\percent} of initial wall thickness;
    \item fraction of reacted zirconium not exceeding \qty{1}{\percent} of the mass of cladding of the fuel rods belonging to one distribution header (RHK).
\end{itemize}

\paragraph{4.2.4.}
The design shall provide and substantiate the possibility of unloading the reactor core and removing the process channel from the core after an MDA involving primary circuit rupture.

\subsection*{4.3. NPPs with BN Reactors}
\markright{4.3. NPPs with BN Reactors}

\paragraph{4.3.1.}
The design shall consider independently one of the following as the MDA:
\begin{itemize}
    \item emergency depressurization of a primary-circuit pipeline without a protective casing;
    \item emergency narrowing or blockage of the flow cross-section in a single fuel assembly due to swelling, deposition of impurities from the coolant, or ingress of foreign objects, resulting in reduced coolant flow through the assembly and damage, destruction, or melting of fuel elements with propagation of damage to one surrounding row of fuel assemblies.
\end{itemize}

\paragraph{4.3.2.}
The design shall substantiate that no melting or destruction of the reactor core occurs that could lead to formation of a critical mass, even if, in addition to the failures stipulated by §1.2.4, one additional active safety element fails.

\paragraph{4.3.3.}
In process rooms containing equipment with radioactive sodium, installation of sealing assemblies of water systems (glands, flanged joints, etc.) shall be excluded.  
As an exception, such sealing assemblies of water systems may be located in sodium rooms if provided with hermetically sealed protective casings.

\paragraph{4.3.4.}
The sensitivity of the system monitoring the subcritical state of the reactor core during normal refueling operations shall be at least sufficient to detect removal of one reactivity control element from the core.

\subsection*{4.4. District Heating and Combined Heat-and-Power NPPs}
\markright{4.4. District Heating and Combined Heat-and-Power NPPs}

This section repeats, without change, the \textit{Requirements for the Siting of Nuclear District Heating Plants and Nuclear Combined Heat-and-Power Plants in Terms of Radiation Safety}, approved on 05.10.1978 (\textit{Atomic Energy}, 1980, Vol.~49, No.~2, p.~150).

\section*{5. Fundamental Definitions}
\markright{5. Fundamental Definitions}

\paragraph{5.1.}
\textbf{Nuclear Power Plant (NPP)} -- a nuclear reactor or reactors with a complex of systems, devices, equipment, and structures designed for the safe production of thermal and/or electrical energy.

\paragraph{5.2.}
\textbf{Nuclear Power Station (NPS)} -- an NPP intended for the production of electrical energy.

\paragraph{5.3.}
\textbf{Nuclear Combined Heat and Power Plant (NCHPP)} -- an NPP intended for the production of both thermal and electrical energy.

\paragraph{5.4.}
\textbf{Nuclear Heating Plant (NHP)} -- an NPP intended for the production of hot water for domestic purposes.

\paragraph{5.5.}
\textbf{Nuclear Industrial Heating Plant (NIHP)} -- an NPP intended for the production of hot water and steam for industrial and domestic use.

\paragraph{5.6.}
\textbf{Safety of an NPP} -- a property of the NPP ensuring, through technical means and organizational measures, that established dose limits for internal and external exposure of personnel and population, as well as standards for radioactive substance content in the environment, are not exceeded.

\paragraph{5.7.}
\textbf{Nuclear Safety of an NPP} -- a property of the NPP ensuring, through technical means and organizational measures, the impossibility of nuclear accidents.

\paragraph{5.8.}
\textbf{System} -- a set of components (devices, equipment, etc.) intended to perform specified functions.

\paragraph{5.9.}
\textbf{Independent Systems (Elements)} -- systems (elements) such that the failure of one does not cause failure of another.

\paragraph{5.10.}
\textbf{System Channel} -- a part of a system that performs the system's function within a defined scope.

\paragraph{5.11.}
\textbf{Normal Operation Systems} -- systems intended to carry out the plant's normal operation.

\paragraph{5.12.}
\textbf{Safety Systems} -- systems intended to prevent accidents and limit their consequences.  
\textit{Note:} Safety systems are divided by function into protective, containment, supporting, and control systems.

\paragraph{5.13.}
\textbf{Systems Important to Safety} -- systems of normal operation whose damage or failure constitutes an initiating event of an accident, and all safety systems.

\paragraph{5.14.}
\textbf{Protective Safety Systems} -- systems intended to prevent or limit damage to nuclear fuel, fuel cladding, the primary circuit, and to prevent nuclear accidents.

\paragraph{5.15.}
\textbf{Containment Safety Systems} -- systems intended to prevent or limit the spread within the NPP and release into the environment of radioactive substances released during accidents.

\paragraph{5.16.}
\textbf{Supporting Safety Systems} -- systems intended to supply safety systems with power and working media and to provide the conditions necessary for their functioning.

\paragraph{5.17.}
\textbf{Control Safety Systems} -- systems intended to actuate safety systems and to monitor and control them during performance of their functions.

\paragraph{5.18.}
\textbf{Control and Instrumentation Systems} -- systems intended for monitoring and controlling systems of normal operation.

\paragraph{5.19.}
\textbf{Active Device (Element)} -- a device (element) whose operation depends on the normal functioning of another device, such as a control unit or power source.

\paragraph{5.20.}
\textbf{Passive Device (Element)} -- a device (element) whose operation does not depend on the normal functioning of another device, such as a control unit or power source.  
\textit{Note:} Passive devices are divided by design into:
\begin{itemize}
    \item passive devices with moving mechanical parts (e.g., check valves);
    \item passive devices without moving mechanical parts (e.g., pipelines, vessels).
\end{itemize}

\paragraph{5.21.}
\textbf{Normal Operation of an NPP} -- all operational states of the NPP consistent with the technology adopted in the design for energy production, including operation at specified power levels, start-up and shutdown processes, maintenance, repairs, and nuclear fuel reloading.

\paragraph{5.22.}
\textbf{Limits of Safe Operation} -- parameter values and characteristics established by regulatory and technical documents, deviation from which may lead to personnel or public radiation exposure and/or environmental contamination with radioactive substances exceeding permissible values for normal operation, and/or to fuel element damage.

\paragraph{5.23.}
\textbf{Conditions of Safe Operation} -- the required number and operability of normal operation and safety systems, and the maintenance and repair schedules necessary to ensure safety.

\paragraph{5.24.}
\textbf{Design Limits} -- quantitative values of parameters and equipment state characteristics established for normal operation, deviations from normal operation, and corresponding initiating events considered in the design.

\paragraph{5.25.}
\textbf{Safety Criteria} -- qualitative characteristics or parameter values established by regulatory and technical documents or adopted in the design, on the basis of which NPP safety is justified.

\paragraph{5.26.}
\textbf{Single Failure Principle} -- the principle whereby a system shall perform its specified functions under any initiating event requiring its operation and under one additional failure independent of the initiating event.

\paragraph{5.27.}
\textbf{Common Cause Failures} -- failures of multiple safety-important systems (elements) resulting from a single internal or external influence.  
\textit{Note:}  
Internal influences include those arising from accident-initiating events (shock waves, reactive jets, flying objects, changes in environmental parameters such as pressure, temperature, chemical activity, fires, etc.).  
External influences include natural phenomena and human activities characteristic of the NPP site (earthquakes, high or low groundwater levels, hurricanes, transport accidents, etc.).

\paragraph{5.28.}
\textbf{Undetected Failure} -- a system (element) failure that does not manifest itself at the moment of occurrence during normal operation and is not detected by existing monitoring means according to the maintenance and testing schedule.

\paragraph{5.29.}
\textbf{Erroneous Human Action} -- an unintended incorrect single action by personnel in the course of performing their duties.

\paragraph{5.30.}
\textbf{Initiating Event} -- a single system failure, external event, or erroneous human action that leads to a deviation from normal operation and may result in exceeding the limits and/or conditions of safe operation.  
The initiating event includes all dependent failures resulting from it.

\paragraph{5.31.}
\textbf{(Radiation) Accident} -- a violation of the limits of safe operation involving the release of radioactive materials or ionizing radiation beyond prescribed boundaries in quantities exceeding values established for normal operation, requiring termination of normal plant operation.

\paragraph{5.32.}
\textbf{Nuclear Accident} -- an accident associated with fuel element damage or potentially hazardous radiation exposure of personnel, caused by:
\begin{itemize}
    \item loss of control of the chain nuclear fission reaction in the reactor core;
    \item formation of a critical mass during refueling, transport, or storage of fuel elements;
    \item loss of heat removal from fuel elements.
\end{itemize}

\paragraph{5.33.}
\textbf{Design-Basis Accident} -- an accident with an initiating event established by regulatory and technical documentation, for which the design ensures NPP safety.

\paragraph{5.34.}
\textbf{Maximum Design-Basis Accident (MDA)} -- the design-basis accident with the most severe initiating event specified for each reactor type.

\paragraph{5.35.}
\textbf{Hypothetical Accident} -- an accident for which the design does not provide technical measures ensuring NPP safety.

\paragraph{5.36.}
\textbf{Maximum Hypothetical Accident (MHA)} -- a hypothetical accident leading to the maximum possible release of radioactive substances into the environment due to fuel melting and failure of containment systems.

\paragraph{5.37.}
\textbf{Accident Consequences} -- damage characterized by the radiation impact on personnel, population, and the environment.

\paragraph{5.38.}
\textbf{Operating Organization} -- the organization to which the NPP is subordinated.

\paragraph{5.39.}
\textbf{Operability, Damage, Failure} -- as defined in GOST~13377-75.

\paragraph{5.40.}
\textbf{Primary Circuit} -- as defined in GOST~20942-75.

\paragraph{5.41.}
\textbf{Technical Safety of an NPP}\footnote{Requirements for equipment, pipelines, and systems covered by this definition are established by rules and standards approved by the USSR State Mining and Technical Supervision Authority (Gosgortekhnadzor).} -- a property of the NPP, achieved through technical means and organizational measures, characterized by the strength of equipment and pipelines whose damage could impair heat removal from the reactor core, and by the ability to retain radioactive substances released under such damage within the sealed area of the NPP.

\chapter*{Appendix F: Rules for Nuclear Safety of Atomic Power Plants (NSR-07-74)}
\addcontentsline{toc}{chapter}{Appendix F: Rules for Nuclear Safety of Atomic Power Plants (NSR-07-74)}
\markboth{Appendix F: Rules for Nuclear Safety of Atomic Power Plants (NSR-07-74)}{}

\begin{center}
\textbf{Rules for Nuclear Safety of Atomic Power Plants}\\
(\textit{ПБЯ-04-74})\\
Approved by the State Committee for Atomic Supervision of the USSR\\
on 31 December 1974
\end{center}

\section*{1. General Provisions.}
\markright{1. General Provisions.}

\paragraph{1.1.} These ``Rules for the Nuclear Safety of Nuclear Power Plants, PBYa-04-74''\footnote{Hereinafter: ``Rules''.} establish the requirements and conditions for ensuring nuclear safety at nuclear power plants of the USSR. The scope of the ``Rules'' extends to all operating, under-construction, and designed nuclear power plants\footnote{Hereinafter: nuclear power plant (NPP).}, as well as nuclear combined heat-and-power plants and nuclear district-heating stations of the USSR, regardless of their type.

\paragraph{\emph{Note.}} Deviations from these ``Rules'' for operating and under-construction NPPs whose technical designs were approved before issuance of the ``Rules'' must be justified by the design organizations and agreed with Gosatomnadzor of the USSR.

\paragraph{1.2.} These ``Rules'' are compiled taking into account the regulatory documents in force in the USSR and the operating experience of nuclear power plants.

\paragraph{1.3.} The ``Rules'' regulate NPP safety issues related to preventing loss of control and management of the fission chain reaction in the reactor core, and excluding the possibility of forming a critical mass during refueling, transportation, and storage of fuel assemblies (FA), as well as during installation and repair work.

\paragraph{1.4.} The ``Rules'' contain the principal technical and organizational requirements for ensuring nuclear safety in the design, construction, and operation of nuclear power plants, and requirements for the training and qualification of NPP personnel.

\paragraph{1.5.} Nuclear safety of NPPs is ensured by the availability of appropriate equipment, its technical excellence and reliability, monitoring of its condition, control of technological processes during operation, proper organization of work, and the professional qualification and discipline of personnel.

\paragraph{1.6.} The ``Rules'' are mandatory for all ministries, enterprises, and organizations engaged in the design, construction, and operation of nuclear power plants. Responsibility for compliance with and oversight of these ``Rules'' during NPP operation is assigned to the plant management, as well as to the leadership and engineering-technical personnel of the ministries under whose authority the nuclear power plant is located.

\paragraph{1.7.} The operating instruction for the reactor installation of an NPP (technological regulations) and other documentation must be prepared in accordance with the nuclear-safety requirements set forth in these ``Rules''.

\paragraph{1.8.} Enterprises and agencies that develop equipment, and those that build and operate NPPs, must, upon request of Gosatomnadzor of the USSR, provide detailed information related to ensuring nuclear safety, in the form of design materials, results of studies and calculations, operating information, test and inspection reports for equipment, and information on equipment operation and personnel training.

\paragraph{1.9.} As experience is accumulated in the design, construction, and operation of NPPs with reactors of various types, the ``Rules'' may be amended and supplemented.

\paragraph{1.10.} Persons found guilty of violating the ``Rules'' are subject to administrative or judicial liability in accordance with applicable legislation.

\section*{2. Basic Concepts, Definitions, and Terminology.}
\markright{2. Basic Concepts, Definitions, and Terminology.}

\paragraph{2.1.} ``Nuclear power plant (NPP)'' -- a complex including a nuclear reactor and associated equipment intended for converting nuclear energy into electrical energy.

\paragraph{2.2.} ``Nuclear cogeneration plant'' and ``nuclear district-heating plant'' -- a complex including a nuclear reactor and associated equipment intended for converting nuclear energy into electrical and thermal energy.

\paragraph{2.3.} ``Control and protection system (CPS)'' -- the reactor technological system of an NPP comprising devices intended for:
-- monitoring power (intensity of the fission chain reaction);
-- controlling the chain reaction;
-- emergency shutdown (rapid quenching) of the chain reaction.

\paragraph{2.4.} ``Instrumentation and control devices (I\&C)'' -- a system of sensors and instruments for monitoring technological parameters of the NPP reactor installation (temperature, pressure, coolant flow, etc.).

\paragraph{2.5.} ``Emergency protection (EP)'' -- a CPS device intended for rapid automatic and manual remote quenching of the chain reaction.

\paragraph{2.6.} ``Automatic regulator (AR)'' -- a CPS device intended for automatic control of reactor power (chain-reaction intensity).

\paragraph{2.7.} ``Manual regulator (MR)'' -- a remotely operated from the control panel CPS device intended to act on reactor reactivity.

\paragraph{2.8.} ``Compensating element (CE)'' -- a CPS device, automatic or remotely operated from the panel, intended to suppress reactivity when regulator effectiveness is insufficient for this purpose.

\paragraph{2.9.} ``Minimum controllable level (MCL)'' -- the minimum reactor power level sufficient for controlling the chain reaction using standard CPS instrumentation.

\paragraph{2.10.} ``Local critical mass'' -- the quantity of nuclear fuel in a part of the core within which an uncontrolled self-sustaining chain reaction can arise.

\paragraph{2.11.} ``Physical startup of an NPP'' -- loading the core with standard fuel assemblies (FA), attaining the critical state of the reactor, and performing necessary experiments at a power level at which coolant heating by fission energy is negligible.

\paragraph{2.12.} ``Power (energy) startup of an NPP'' -- raising reactor power from the physical startup level to the level sufficient for turbine startup and conducting necessary experiments during staged power increase.

\paragraph{2.13.} ``Nuclear accident'' -- loss of control of the chain reaction in the reactor, or formation of a critical mass during refueling, transport, or storage of fuel assemblies, leading to potentially hazardous irradiation of people or damage to fuel elements (fuel rods) beyond permissible limits.

\paragraph{2.14.} ``Nuclear-hazardous condition'' -- deviations from limits and conditions of safe operation of the NPP reactor installation not resulting in a nuclear accident.

\paragraph{2.15.} ``Maximum reactivity margin'' -- the reactivity realized in the reactor upon withdrawal of all CPS actuators, including solutions of liquid absorbers, at the campaign moment and reactor state with the maximum value of the effective multiplication coefficient ($k_{\text{3\phi}}$).

\section*{3. Technical Requirements for the Design of the Reactor Installation and of the Systems Ensuring Nuclear Safety.}
\markright{3. Technical Requirements for the Design of the Reactor Installation and of the Systems Ensuring Nuclear Safety.}

\subsection*{3.1. General requirements.}
\markright{3.1. General requirements.}

\paragraph{3.1.1.} The NPP design shall provide technical and organizational measures aimed at ensuring nuclear safety under any possible single failure of one system (device), which may coincide with an undetected long-term failure in another system (device).

\emph{Note.} The list of system (device) failures considered in the designs and related to ensuring nuclear safety shall be agreed with Gosatomnadzor of the USSR.

\paragraph{3.1.2.} The designs of all NPP systems and components whose damage or malfunction may affect nuclear safety shall include a detailed analysis of all possible failures of constituent elements, identifying hazardous failures and assessing their consequences. These systems and components shall be provided with monitoring means and, when necessary, redundancy.

\paragraph{3.1.3.} Multi-purpose use of NPP systems and assemblies is not permitted unless it is demonstrated that such combination of functions will not lead to violation of nuclear-safety conditions and requirements.

\paragraph{3.1.4.} NPP systems and individual elements that affect nuclear safety shall be subject to inspections and tests during manufacture, installation, and commissioning, as well as to periodic checks during operation. The technical design shall provide fixtures, devices, and methods in order to:
-- check the operability of the most critical parts and units of systems;
-- conduct periodic tests of units for compliance with design parameters;
-- perform periodic checks of the sequence and timing of signal transmission (including emergency protection actuation signals);
-- check equipment shutdowns, including transfer to emergency power supplies.

\emph{Note.} The devices and methods of checking shall not jeopardize the safe operation of the NPP. The technical design of the NPP shall define the list of systems and equipment whose operability is checked with the reactor at power and with the reactor shut down.

\paragraph{3.1.5.} The technical design of NPP systems and equipment affecting nuclear safety shall include:
-- a quantitative reliability analysis;
-- a quantitative analysis of the probability of equipment damage and of various accident situations considered in the design.
The reliability and damage-probability analyses shall be performed taking into account accumulated statistical data on the operation of equipment and systems.

\paragraph{3.1.6.} In the NPP technical design, the materials related to ensuring nuclear safety shall be included as a separate section in the safety justification for the construction and operation of the nuclear power plant.
\emph{Note.} This section shall also indicate all deviations from the requirements of the ``Rules''. Deviations shall be justified and agreed with Gosatomnadzor of the USSR at the technical-design stage.

\paragraph{3.1.7.} The quantity and technical characteristics of CPS and I\&C devices, as well as the scope of recorded information, shall ensure recording of the causes of emergency protection actuation and make it possible to reconstruct the course of an accident process.
\emph{Recommendation.} It is desirable to have devices for analog display of power, reactivity, rate of change of reactivity, and of the positions in the reactor of CPS actuators.

\paragraph{3.1.8.} The NPP reactor installation annunciation system shall provide the following signals:
-- emergency (visual and audible, including an emergency alarm siren) when parameters reach the setpoints for EP actuation and under emergency deviations of the process regime;
-- warning (visual and audible) when parameters approach EP setpoints, when radiation exceeds established limits, and upon malfunctions of equipment;
-- indicating signals registering the positions of CPS actuators, the presence of voltage in power-supply circuits, equipment status, switching on of individual instruments, etc.

\paragraph{3.1.9.} For systems and equipment ensuring nuclear safety, the periodicity of maintenance and preventive measures shall be established for the entire service life of the NPP.

\paragraph{3.1.10.} All structures and equipment operating under pressure and related to ensuring nuclear safety shall conform to the ``Rules for the Design and Safe Operation of Equipment of Nuclear Power Plants, Experimental and Research Nuclear Reactors and Installations'' (Moscow, \emph{Metallurgiya}, 1973).

\paragraph{3.1.11.} NPP systems intended to ensure nuclear safety, as well as to prevent or mitigate the consequences of nuclear accidents, shall be designed, manufactured, and installed taking into account additional loads possible due to climatic effects and natural phenomena characteristic of the site, such as earthquakes, hurricanes, floods, etc.

\paragraph{3.1.12.} The technical design of an NPP, as well as changes affecting nuclear safety to basic design parameters, equipment, and systems (including modernization), shall be agreed with Gosatomnadzor of the USSR.

\subsection*{3.2. Requirements for the design and characteristics of the core.}
\markright{3.2. Requirements for the design and characteristics of the core.}

\paragraph{3.2.1.} In designing the reactor core, permissible limits of fuel-element damage (quantity and degree of damage) under safe NPP operation conditions shall be established in advance and justified. The core shall be designed so that during normal operation throughout the entire design service life the permissible limits of fuel-element damage are not exceeded.

\paragraph{3.2.2.} The reactor shall be designed so that the total power coefficient of reactivity is not positive under any operating modes of the NPP. If under any operating conditions the total power coefficient of reactivity is positive, the design shall ensure and explicitly substantiate nuclear safety of the reactor during steady-state, transient, and accident regimes.

\paragraph{3.2.3.} The core design shall be such that, under normal and accident regimes, CPS actuators are not subject to jamming.

\paragraph{3.2.4.} The reactor core shall be designed so as to preclude any unintended movement of core components that would lead to an increase in reactivity.

\paragraph{3.2.5.} The characteristics of nuclear fuel, the reactor core design, primary-circuit equipment, and auxiliary systems (including the primary-coolant cleanup system, etc.) shall preclude the formation of critical masses of fissile materials in the event of core destruction or fuel melting.

\paragraph{3.2.6.} Actuation of the emergency core cooling system for a shut-down reactor shall not lead to the reactor departing from the subcritical state. This requirement is ensured by the choice of appropriate coolant, absorber concentration in it, the technical characteristics of the emergency core cooling system, etc.

\subsection*{3.3. Requirements for the control and protection system (CPS).}
\markright{3.3. Requirements for the control and protection system (CPS).}

\paragraph{3.3.1.} The CPS shall ensure reliable monitoring of power (intensity of the chain reaction), control, rapid shutdown (quenching) of the chain reaction, and maintenance of the reactor in a subcritical state.

\paragraph{3.3.2.} Any remotely operated devices may be used as CPS actuators: absorbing rods or solutions, movable fuel assemblies, reflector parts, etc.

\paragraph{3.3.3.} The CPS design shall provide at least two independent systems (or independent actuators, or independent groups of actuators) for affecting reactivity (preferably based on different principles of action).

\paragraph{3.3.4.} At least two of the provided independent systems (actuators, groups of actuators) for affecting reactivity shall be capable, independently of each other, of bringing the reactor from any operating state (without exceeding permissible limits of fuel-element damage) to a subcritical state and of maintaining it in that state at the operating temperature of the coolant and moderator.

\paragraph{3.3.5.} At least one of the provided systems for affecting reactivity shall be capable of bringing the reactor to a subcritical state and maintaining it in that state under any normal and accident conditions, even if the single most effective reactivity-affecting actuator fails to actuate.

\paragraph{3.3.6.} During reactor startup, with the EP actuators set and with the remaining CPS actuators inserted, the reactor subcriticality shall be not less than \num{0.01} in the core state with the maximum effective multiplication coefficient.

\paragraph{3.3.7.} Control of the reactor and its systems shall be performed from the control panel and from local stations that have telephone and public-address communication with the control panel. The number of local stations is defined by the design.

\paragraph{3.3.8.} It shall be possible to shut down the reactor from another room if access to the control room is lost (fire, etc.).

\paragraph{3.3.9.} To monitor power (intensity of the chain reaction), the reactor shall be equipped with monitoring channels such that, in the course of startup and at any power level starting from the MCL, monitoring is carried out by at least:
-- three mutually independent channels for measuring power level (neutron-measuring), with indicating instruments;
-- three mutually independent channels for measuring the rate of change of power (or of reactivity), with indicating instruments.
At least two of the three power-monitoring channels shall be equipped with recording instruments. If measuring channels operate in limited ranges, their operating ranges shall overlap by at least one decade.
\emph{Note.} If the above independent power-monitoring channels do not ensure neutron-flux monitoring during core loading and refueling, the reactor shall be equipped with an additional monitoring system. This system may be removable, installed for the period of core loading and refueling, and shall include at least two independent channels for monitoring neutron flux level with indicating and recording instruments.

\paragraph{3.3.10.} All mechanical CPS actuators shall have position indicators and limit switches actuated, where possible, directly by the actuator.

\paragraph{3.3.11.} CPS power supplies shall have redundancy ensuring CPS operation in normal and accident regimes.

\paragraph{3.3.12.} The CPS scheme shall preclude insertion of positive reactivity by AR, MR, and CE actuators if the emergency protection actuators are not armed/reset.

\paragraph{3.3.13.} The rate of insertion of positive reactivity by CPS actuators shall not exceed \qty{0.07}{\betaeff\per\second}\footnote{Here, $\beta_{\text{eff}}$ denotes the effective fraction of delayed neutrons.}. For AR actuators in manual mode, MR, and CE with effectiveness greater than \qty{0.7}{\betaeff}, positive reactivity insertion shall be stepwise, with a step size not exceeding \qty{0.3}{\betaeff} (ensured by technical or organizational measures). Insertion of negative reactivity on an EP signal should be performed by the fastest CPS actuators.

\paragraph{3.3.14.} When liquid control is used in the CPS, there shall be provision for supplying absorber solution into the core via at least two independent trains, and for systematic monitoring of absorber concentration in the reactor and in tanks containing absorber solution. Tanks shall be equipped with at least two level-monitoring systems with issuance of a warning signal to the reactor control panel. During core loading and refueling and during maintenance, filling of the reactor, the primary circuit, and associated systems shall be performed with a homogeneous absorber solution at a concentration not lower than specified. Unintended ingress of unpoisoned water into the reactor, the primary circuit, refueling and storage pools, and other systems that by design shall be filled with absorber solution shall be precluded by special technical measures.

\paragraph{3.3.15.} The CPS shall be able to cope with a single malfunction such as unplanned withdrawal (within design speeds) of simultaneously operating regulating actuators or of the single most effective actuator, without allowing a power increase that could lead to exceeding permissible limits of fuel-element damage.

\paragraph{3.3.16.} CPS technical documentation shall include an analysis of system responses to possible faults: short circuits, loss of insulation quality, voltage dips and interference, etc. Before reactor startup, the CPS shall be checked to identify hazardous and spurious responses.

\paragraph{3.3.17.} The automatic power-control system shall be equipped with at least two independent automatic-control channels with automatic mutual standby.

\paragraph{3.3.18.} If each automatic regulator has its own actuators, the automatic-control system shall ensure automatic transfer from the operating AR to the standby AR when the actuators of the operating AR reach predetermined positions.

\paragraph{3.3.19.} When several identical measuring channels are connected to the AR system input, a device shall be provided to obtain a signal equal to the arithmetic mean of all input-channel signals, so that disconnection of any one of these channels does not cause the AR system to change reactor power.

\paragraph{3.3.20.} During operation of the automatic control system, any increase of reactor power with a period of less than 30 s shall be automatically precluded.

\paragraph{3.3.21.} The CPS shall include fast-acting emergency protection (EP of the first kind)\footnote{Hereinafter, the words ``emergency protection'' mean EP of the first kind. Emergency protection that ensures reduction of power to a specified level at a specified rate is not considered in the ``Rules'', but its use is not precluded.}, ensuring automatic reactor shutdown upon occurrence of an emergency situation. EP signals and actuation setpoints shall be justified in the design.

\paragraph{3.3.22.} Reactor EP shall be designed so that, during startup and at any power level starting from the MCL, protection is ensured at least by:
-- three mutually independent channels by power level;
-- three mutually independent channels by rate of power increase.
If protection channels operate in limited ranges, their operating ranges shall overlap by at least one decade.

\paragraph{3.3.23.} To reduce the number of spurious actuations, it is permissible to actuate EP actuators upon coincidence of signals from any two channels of a given type (``two-out-of-three'' principle). If one of the three channels is faulty or disconnected for checking or repair, the presence of a signal from either of the two operating channels shall lead to EP actuation.

\paragraph{3.3.24.} Any single failure in the EP system shall not impair its protective functions. Multiple failures resulting from any single event, action, or fault shall be considered as a single failure.

\paragraph{3.3.25.} EP shall be sufficiently separated from monitoring and control devices so that failure or shutdown of any element of those devices does not affect the ability of EP to perform its protective functions.

\paragraph{3.3.26.} Reactor EP shall ensure automatic, rapid, and reliable quenching of the chain reaction in the following cases:
-- when the emergency power setpoint is reached;
-- when the emergency setpoint for rate of power increase (or reactivity) is reached;
-- upon loss of voltage on CPS power-supply buses;
-- upon fault or inoperable state of any two of the three protection channels by level or by rate of power increase;
-- upon appearance of emergency process signals requiring reactor shutdown;
-- upon pressing the EP pushbuttons.

\paragraph{3.3.27.} The EP system shall have at least two independent groups of actuators.

\paragraph{3.3.28.} The number, arrangement, effectiveness, and insertion speed of additional EP actuators shall be defined and justified in the reactor design, which shall demonstrate that, for any accident regimes, the EP actuators, excluding the single most effective actuator, provide:
-- an emergency power-reduction rate sufficient to prevent possible fuel-element damage beyond permissible limits;
-- bringing the reactor to a subcritical state and maintaining it in that state, taking into account possible reactivity increase, for a time sufficient to insert other, slower CPS actuators;
-- prevention of local critical-mass formation.

\paragraph{3.3.29.} EP shall be designed so that, as a rule, once a protective action has begun it proceeds to completion. Restoration of the initial operating state of the reactor shall be performed by successive actions of the operating personnel. The permissibility, in certain cases, of terminating protective action upon disappearance of the initiating signal shall be justified in the design.

\paragraph{3.3.30.} Upon appearance of an emergency signal, EP actuators shall be actuated from any intermediate position.

\paragraph{3.3.31.} Checking the passage of emergency signals in the EP system from detectors to actuator drives during power operation shall not lead to reactor shutdown. Any interlocks of EP circuits and units due to fault, adjustment, or removal for repair are permitted only when multiple devices with identical actuation mechanisms are present, with mandatory issuance of corresponding ``channel disconnected'' signals to the control panel.

\section*{4. Commissioning of a Nuclear Power Plant.}

\paragraph{4.1.} Commissioning of an NPP after completion of construction and installation includes:
-- performance of startup and adjustment work, including tests of systems ensuring nuclear safety;
-- preparation of technical and operating documentation;
-- staffing and training of personnel;
-- performance of the physical and power startups (integrated trial of NPP equipment);
-- reactor startup and power operation.

\subsection*{4.2. Physical startup of an NPP.}
\markright{4.2. Physical startup of an NPP.}

\paragraph{4.2.1.} By the start of the physical startup, the following shall be prepared for operation with readiness certificates:
-- the reactor;
-- the CPS (actuators, detectors and electronic equipment, and means for controlling additional actuators, including logic systems and emergency protection);
-- standard startup equipment;
-- the startup neutron source (if required);
-- nonstandard startup equipment (if required), whose EP signals are routed into the reactor emergency protection;
-- devices for transport, loading, and unloading of fresh and spent fuel;
-- spent-fuel storage pools;
-- the dosimetry monitoring system;
-- the system for chemical and special preparation of the coolant, including the heat-up system (if provided by the design);
-- the supply and exhaust ventilation system;
-- the liquid-control system (if provided by the design);
-- the reliable power-supply system;
-- the emergency alarm system for all rooms;
-- the grounding grid;
-- telephone and public-address communication;
-- sanitary access-control points;
-- the fire-extinguishing system.

\paragraph{4.2.2.} During the physical startup, the CPS shall meet the requirements of section 3.3 of these ``Rules''. It is permissible to block EP signals from process systems that are not used during the physical startup.

\paragraph{4.2.3.} The following documentation shall be prepared for the physical startup:

Physical-startup program. The program is developed by representatives of the scientific supervisor, the chief designer of the project, and a representative of the NPP management. The program defines the procedure for loading the reactor with standard FAs and attaining criticality, and gives a description of the experiments and the procedure for conducting them. The physical-startup program shall include expected values of critical loadings, critical positions of CPS actuators, their effectiveness, and assessments of the effects on reactivity of the loaded FAs, coolant, etc. The startup program is agreed with Gosatomnadzor of the USSR.

Experiment procedures during the physical startup. The procedures are prepared by representatives of the scientific supervisor and the chief designer of the project and by NPP personnel.

Operating instruction for the NPP reactor installation (technological regulations). The Instruction shall set forth the rules for operating the reactor installation in various regimes and the limits and conditions of safe NPP operation. The operating instruction is prepared by NPP personnel, agreed with the scientific supervisor and the chief designer of the project, and approved by the NPP management or the ministry supervising the NPP.

Instruction for mitigation of accident consequences, defining the actions of reactor personnel and NPP services in the event of an accident (including a nuclear accident). The instruction is agreed with the scientific supervisor and the chief designer of the project.

Instruction for ensuring nuclear safety during the physical startup.

Instruction for ensuring nuclear safety during transport, refueling, and storage of fresh and spent fuel.

Technical documentation, including descriptions of equipment and systems that ensure nuclear safety.

Operating documentation (operating logs, core-loading maps, etc.).

Acts and protocols of tests of the CPS and I\&C of the reactor installation.

Order appointing the scientific supervisor of the physical startup, their deputies, and the physical-startup group.

Protocols of examinations passed by shift personnel and supervising physicists (taking into account the specifics of work during the physical startup).

Order of the NPP director admitting to work the shift personnel who have passed examinations for their workplaces.

Job descriptions of reactor shift personnel and the regulations for the supervising physicist, approved by the NPP management.

Act of the working commission on the readiness of systems and equipment and on the preparedness of personnel for the physical startup.

Act of the Gosatomnadzor USSR commission on the readiness of the NPP for the physical startup.

Authorization of the State Acceptance Commission to conduct the physical startup.

\paragraph{4.2.4.} Verification of NPP readiness for the physical startup is carried out by:
-- the working commission;
-- the commission of Gosatomnadzor of the USSR.

\paragraph{4.2.5.} The working commission verifies:
-- conformity of completed work to the NPP design;
-- operability of equipment, availability of equipment test protocols and of acts on completion of startup-adjustment work;
-- availability and proper execution of the documentation specified in 4.2.3 (except the last three documents);
-- assignment of shift personnel for the period of the physical startup;
-- availability of protocols of examinations passed by shift personnel and supervising physicists.
The commission draws up an act on readiness of systems and equipment and on personnel preparedness for the physical startup. The act shall be approved in the prescribed manner.

\paragraph{4.2.6.} The Gosatomnadzor USSR commission verifies:
-- the technical readiness of the NPP for the physical startup in accordance with 4.2.1;
-- technical documentation in accordance with 4.2.3 (except the last two documents);
-- personnel preparedness to conduct the physical startup.
The commission formalizes the results in an act, which also reflects deficiencies in ensuring nuclear safety for conducting the power startup. In the absence of remarks concerning the physical startup, the approved act serves as the authorization of Gosatomnadzor of the USSR to conduct the physical startup. If there are remarks preventing its conduct, the authorization is issued after the deficiencies are eliminated and an act on their elimination is prepared by the NPP management.

\paragraph{4.2.7.} The State Acceptance Commission, based on the act of the working commission on readiness of systems and equipment and personnel preparedness for the physical startup and on the authorization of Gosatomnadzor of the USSR, makes a decision to conduct the physical startup.

\paragraph{4.2.8.} The physical startup of the reactor is conducted in accordance with the approved physical-startup program and the schedule developed on its basis.

\paragraph{4.2.9.} Management of the physical startup is carried out by the scientific supervisor of the physical startup or their deputy.

\paragraph{4.2.10.} Responsibility for compliance with nuclear safety during the physical startup lies with:
-- for conformity of assigned operating regimes to the startup program and procedures -- the scientific supervisor of the physical startup; in the shift -- the supervising physicist;
-- for execution of the physical startup -- the NPP chief engineer; in the shift -- the shift supervisor and the shift personnel in accordance with their job descriptions.

\paragraph{4.2.11.} The supervising physicist directs the conduct of experiments during the shift in accordance with the assignment, acting through the shift supervisor. In case of disagreements between the supervising physicist and the shift supervisor, the final decision is made by the scientific supervisor of the startup and the NPP chief engineer.

\paragraph{4.2.12.} If a nuclear-hazardous condition arises, the physical-startup experiments are terminated and the reactor is brought to a subcritical state.

\paragraph{4.2.13.} All orders of the scientific supervisor of the physical startup and of the NPP chief engineer, all operations performed by the shift personnel, and the experiments conducted and their results shall be recorded in the order log and the operating log, which are kept starting from the beginning of core loading.

\paragraph{4.2.14.} During the physical startup, experimental data shall be obtained on the neutron-physics parameters of the core; reactivity effects; characteristics of regulating, compensating, and protection actuators; and the sequence of CPS actuator withdrawal from the core during reactor startup shall be determined. The results of the physical startup are formalized by an act and a report. One copy of the act and report is sent to Gosatomnadzor of the USSR.

\subsection*{4.3. Power startup of an NPP.}
\markright{4.3. Power startup of an NPP.}

\paragraph{4.3.1.} The power startup includes staged and gradual power increase; determination and refinement of reactor parameters; integrated testing of NPP systems and equipment; conduct of planned experiments at each stage; and analysis of the results obtained.

\paragraph{4.3.2.} By the start of the power startup, all standard systems, devices, structures, and installations required for NPP operation shall have been accepted into service, and all documentation listed in 5.18 shall have been prepared (except the first two documents).

\paragraph{4.3.3.} The power startup is conducted in accordance with a program corrected, if necessary, based on the results of the physical startup.

\paragraph{4.3.4.} The power-startup program defines the procedure for its conduct; presents expected power, temperature, and other parameters; the expected reactivity; the expected effectiveness of CPS actuators; etc. To implement the program, representatives of the scientific supervisor and the chief designer of the project and a representative of the NPP management prepare experiment procedures and a power-startup schedule. The power-startup program is agreed with Gosatomnadzor of the USSR.

\paragraph{4.3.5.} Verification of NPP readiness for the power startup is carried out by the working commission. The working commission verifies (per 4.3.2) the readiness of NPP systems and equipment for the power startup, bringing the reactor to power, starting the turbogenerators, and connecting the NPP to the grid; as well as the staffing of the shifts, their training, and admission to work. The commission draws up an act on readiness of the NPP for the power startup, approved in the prescribed manner.

\paragraph{4.3.6.} Gosatomnadzor of the USSR issues authorization, from the standpoint of ensuring nuclear safety, to conduct the power startup on the basis of the following documents:
-- the power-startup program agreed with Gosatomnadzor of the USSR;
-- the act of the working commission on readiness of the NPP for the power startup;
-- the report or act on the results of the physical startup;
-- the act of the NPP management on elimination of deficiencies noted by the Gosatomnadzor commission that prevent conducting the power startup (see 4.2.6).
If necessary, Gosatomnadzor of the USSR dispatches a commission to verify NPP readiness for the power startup.

\paragraph{4.3.7.} The State Acceptance Commission, based on the act of the working commission on readiness of the NPP for the power startup and the authorization of Gosatomnadzor of the USSR, makes a decision to conduct the power startup.

\paragraph{4.3.8.} Management of the power startup is performed by the NPP chief engineer. Plant personnel carry out the work under the power-startup program; if necessary, a special startup group participates. The rights and duties of the startup group members shall be set forth in the regulations on the startup group.

\paragraph{4.3.9.} For the duration of the power startup, responsibility for nuclear safety rests with the NPP chief engineer; in the shift -- with the shift supervisor and the shift personnel in accordance with their job descriptions.

\paragraph{4.3.10.} The results of the power startup are formalized as an act and a report with recommendations on reactor operation. One copy of the act and report is sent to Gosatomnadzor of the USSR.

\section*{5. Operation of a Nuclear Power Plant.}
\markright{5. Operation of a Nuclear Power Plant.}

\paragraph{5.1.} Operation of an NPP shall be carried out in strict accordance with the Operating Instruction for the reactor installation (technological regulations) and with the operating instructions for systems and equipment, which shall reflect the requirements for ensuring nuclear safety.

\paragraph{5.2.} In accordance with the design and technical documentation and based on the physical and power startups, the operating instructions shall establish the normal operating regimes of the NPP reactor installation (limits and conditions for safe operation), agreed with the scientific supervisor and the chief designer of the project.

\paragraph{5.3.} The NPP management shall agree with Gosatomnadzor of the USSR the changes affecting nuclear safety that are introduced into the design schematics and the equipment design of the NPP as a result of the power startup. Operation of the NPP is permitted upon availability of a reactor-installation passport issued by Gosatomnadzor of the USSR (see Appendix).
\emph{Note.} A change of parameters specified in the Gosatomnadzor USSR passport requires issuance of a new passport. These changes shall be agreed in advance with the scientific supervisor and the chief designer of the project.

\paragraph{5.4.} Reactor startup and power operation are permitted subject to at least the following conditions:
-- EP actuators are set/armed;
-- power monitoring meets the requirements of 3.3.9;
-- reactor emergency protection meets the requirements of 3.3.22 and 3.3.26;
-- all CPS actuators are included in the CPS in accordance with the NPP design;
-- the emergency power-supply system is operable;
-- the emergency liquid-absorber injection system is operable and the project-specified stock of absorber solution is provided (for a reactor with liquid reactivity control);
-- the annunciation system meets the requirements of 3.1.8;
-- the emergency cooldown system is in operating condition.

\paragraph{5.5.} If at the initial stage of reactor startup the sensitivity of the standard CPS I\&C is insufficient to monitor the neutron flux, then the rate of positive reactivity insertion shall be such that the power corresponding to the MCL is reached with a doubling period of at least \num{30} seconds.

\paragraph{5.6.} During power operation it is prohibited:
-- to disconnect for checking or replacement (repair) more than one power-monitoring channel and one emergency-protection channel (by level or by rate of power increase);
-- to disconnect for replacement or repair individual CPS actuator mechanisms if the remaining number of CPS actuators does not ensure compliance with 3.3.4, 3.3.5, 3.3.6, and 3.3.28.

\paragraph{5.7.} If a nuclear-hazardous condition arises, measures shall be taken to restore normal operating conditions of the NPP or the reactor EP system shall be actuated.

\paragraph{5.8.} The shift supervisor shall report to the NPP management each occurrence of a nuclear-hazardous condition. Operation of the reactor installation may be continued by written order of the NPP chief engineer after the causes of the condition are determined and eliminated.

\paragraph{5.9.} The reactor operator has the right to shut down the reactor independently if they find that further operation threatens NPP safety.

\paragraph{5.10.} At any moment of the campaign, the following shall be known: the maximum reactivity margin of the core suppressed by CPS actuators; the effectiveness of AR, MR, CE, and EP actuators; and the effectiveness of the absorber of the liquid reactivity-control system.

\paragraph{5.11.} Monitoring of a shut-down reactor with FAs in the core shall be continuous throughout the entire downtime, including during loading and refueling, in accordance with 3.3.9. For reactors with liquid control, the absorber concentration in the coolant is also subject to mandatory monitoring.

\paragraph{5.12.} The NPP management shall compile a list of units, systems, and equipment that ensure NPP nuclear safety, establishing the periodicity of their tests and checks.

\paragraph{5.13.} The procedure for refueling the core is defined by the program, the working schedule, and the refueling maps, drawn up with due regard for nuclear-safety requirements.

\paragraph{5.14.} Refueling in a shut-down reactor shall be performed with EP actuators set. In this case the subcriticality shall be at least 0.01 for the core state with the maximum effective multiplication coefficient.

\paragraph{5.15.} In reactors where refueling is performed with CPS actuators decoupled, refueling is carried out with the actuators immersed in the core. In this case, the minimum subcriticality of the reactor during refueling, taking into account possible errors, shall be at least 0.02.

\paragraph{5.16.} In reactors where refueling is performed with CPS actuators decoupled and reactivity is compensated by absorber solution, refueling is carried out with CPS actuators immersed in the core. In this case, the absorber concentration in the coolant shall be brought to a value that, taking into account possible errors, ensures reactor subcriticality of at least \num{0.02} (without accounting for the immersed mechanical CPS actuators).

\paragraph{5.17.} During refueling, monitoring of the core state shall be ensured in accordance with 5.11.

\paragraph{5.18.} The list of required documentation during NPP operation shall include:
-- Act of acceptance of the NPP into operation by the State Acceptance Commission;
-- Passport of Gosatomnadzor of the USSR for the NPP reactor installation;
-- Operating Instruction (technological regulations) for the NPP reactor installation;
-- Operating instructions for NPP systems and equipment;
-- NPP technical documentation, which shall include descriptions and schematics of equipment, as well as of systems ensuring nuclear safety;
-- Acts and protocols of tests of the control and protection system and other systems related to ensuring nuclear safety (annunciation system, emergency cooldown, etc.);
-- Instruction for ensuring nuclear safety during transport, refueling, and storage of fresh and spent fuel at the NPP;
-- Instruction for mitigation of accident consequences;
-- Job descriptions of NPP shift personnel for each workplace;
-- Protocols of examinations and briefings of shift personnel;
-- Orders of the NPP management on appointment and authorization for independent work of shift personnel;
-- Operating documentation of the shift personnel;
-- Lists approved by the NPP management of effective instructions (general for the NPP and for each workplace), indicating their validity period and the approving official.

\paragraph{5.19.} Responsibility for ensuring nuclear safety during NPP operation rests with:
-- at the NPP -- the NPP management;
-- in the Reactor Department -- the head of the Reactor Department;
-- in the shift -- the shift supervisor and the shift personnel in accordance with their job descriptions.

\section*{6. Transport and Storage of Fresh and Spent Fuel.}
\markright{6. Transport and Storage of Fresh and Spent Fuel.}

\paragraph{6.1.} Transport and storage of fresh and spent fuel shall be carried out in accordance with the Instruction on ensuring nuclear safety during transport, refueling, and storage of fresh and spent fuel at the NPP.

\paragraph{6.2.} The norms and procedure for transport and storage of fresh and spent fuel shall be justified in the NPP design and agreed with the Nuclear Safety Department of the Physics and Power Engineering Institute (Obninsk).

\paragraph{6.3.} When storing fresh fuel in sleeves (on racks), the arrangement of fuel assemblies (FA) therein and the mutual arrangement of sleeves (racks) shall ensure subcriticality of not less than \num{0.05} under possible accident situations (including flooding of the storage with water).

\paragraph{6.4.} During transport and storage of spent fuel in refueling and storage pools, subcriticality of not less than \num{0.05} shall be ensured for all possible accident situations.

\paragraph{6.5.} Each individual transport or process operation involving movement of fresh or spent FA shall be recorded in a special log indicating FA location and safety measures taken.

\paragraph{6.6.} During transport and storage of spent fuel, violation of fuel-element tightness and fuel melting due to decay heat shall be precluded.

\section*{7. Nuclear-Hazardous Maintenance Work.}
\markright{7. Nuclear-Hazardous Maintenance Work.}

\paragraph{7.1.} Work to remove from service for repair, or to return to service after repair, equipment affecting core reactivity is nuclear-hazardous and shall be performed with compliance to nuclear-safety requirements and with reactor status monitoring.

\paragraph{7.2.} Nuclear-hazardous maintenance work is normally performed on a shut down subcritical reactor under a special technical decision approved by NPP management. The decision shall contain: list of nuclear-hazardous operations; procedure (technology) for performing the work; technical and organizational measures ensuring nuclear safety. An individual responsible for conducting the nuclear-hazardous work is appointed by NPP management.

\paragraph{7.3.} Subcriticality of a shut down reactor during nuclear-hazardous maintenance shall be at least \num{0.02} for the core state with maximum effective multiplication coefficient.

\paragraph{7.4.} After completion of repair of equipment and systems affecting core reactivity and NPP nuclear safety, they shall be checked for compliance with approved characteristics.

\section*{8. Measures for Mitigation of Nuclear Accident Consequences.}
\markright{8. Measures for Mitigation of Nuclear Accident Consequences.}

\paragraph{8.1.} Actions of NPP personnel in the event of a nuclear accident are defined by the Instruction for mitigation of accident consequences.

\paragraph{8.2.} The Instruction shall consider possible accident situations and define measures for mitigation of consequences. Duties and actions of shift personnel and coordination of actions of NPP services and external organizations (local authorities, fire brigade, police, medical institutions, civil defense, etc.) shall be specified. Anti-accident drills shall be conducted per the Instruction; their periodicity and procedure are approved by NPP management.

\paragraph{8.3.} From the moment an accident occurs until the start of work of the commission investigating causes, it is strictly prohibited to open instrumentation and devices, or to change setpoints of emergency and warning annunciation and protection.

\paragraph{8.4.} Nuclear accidents are investigated in accordance with the Instruction on investigation of accidents related to violation of nuclear safety at enterprises, organizations, and institutions under the State Nuclear Safety Inspectorate of the USSR.

\section*{9. Training of Reactor Installation Personnel of an NPP.}
\markright{9. Training of Reactor Installation Personnel of an NPP.}

\paragraph{9.1.} Operation of the NPP reactor installation is performed by personnel included in the shift complement (shift personnel).

\paragraph{9.2.} Shift personnel are admitted to independent work after on-the-job training and passing examinations on workplace knowledge and applicable instructions.

\paragraph{9.3.} Admission of shift personnel to independent work is formalized by an order of NPP management.

\paragraph{9.4.} The program of qualification examinations, composition of the examination commission, and procedure for training are approved by NPP management.

\paragraph{9.5.} Shift personnel pass examinations on workplace knowledge at least once per year and receive briefings every six months on current NPP regulations and instructions within their job responsibilities.

\section*{10. Verification and Inspection of Nuclear-Safety Status.}
\markright{10. Verification and Inspection of Nuclear-Safety Status.}

\paragraph{10.1.} The ministry supervising the NPP shall ensure necessary organizational and technical measures aimed at compliance with nuclear-safety requirements at the NPP and control of their fulfillment.

\paragraph{10.2.} Periodically (at least once per year) an internal commission for checking nuclear-safety status is appointed by NPP management order. The commission act is approved by management; one copy is sent to Gosatomnadzor of the USSR.

\paragraph{10.3.} Periodically (at least once every three years) Gosatomnadzor of the USSR dispatches to the NPP a commission to check nuclear-safety status, involving staff of its base organizations (Nuclear Safety Department of the Physics and Power Engineering Institute and the Nuclear Safety Laboratory of the I. V. Kurchatov Institute of Atomic Energy) and specialists of other organizations. The commission prepares an act on results.

\paragraph{10.4.} The act of the Gosatomnadzor commission is approved by the Chief State Inspector for Nuclear Safety of the USSR. Deficiencies noted in the act shall be eliminated within deadlines agreed with Gosatomnadzor. An act on elimination of deficiencies, approved by NPP management, is sent to Gosatomnadzor of the USSR.

\chapter*{Appendix 2: Chronology of the Technological Process at Unit 4 of the Chernobyl Nuclear Power Plant}
\addcontentsline{toc}{chapter}{Appendix 2: Chronology of the Technological Process at Unit 4 of the Chernobyl Nuclear Power Plant}
\markboth{Appendix 2: Chronology of the Technological Process at Unit 4 of the Chernobyl Nuclear Power Plant}{}

\begin{center}
\textbf{Table I}\\
\textbf{Chronology of the Technological Process at Unit 4 of the Chernobyl NPP}
\end{center}

\small\begin{center}
  \begin{longtable}{|p{6em}|p{27em}|}
    \hline
    \textit{Time} & \textit{Events} \\
    \hline
    \multicolumn{2}{|c|}{\textit{25 April 1986 (times per the operating log)}} \\
    \hline
    01:06 a.m. & Start of unloading of the power unit; ORR equal to \num{31} control rods (RR). \\ \hline
    03:45 a.m. & Beginning of replacement of the gas purge composition of the reactor graphite stack from a nitrogen-helium mixture to nitrogen. \\ \hline
    03:47 a.m. & Reactor thermal power \qty{1600}{\mega\watt}. \\ \hline
    04:13:00 a.m. until 12:36:00 p.m. & Successive measurement of characteristics of the control systems and vibration characteristics of Turbogenerator-7 and Turbogenerator-8 at constant reactor thermal power of \qty{1500}{\mega\watt}. \\ \hline
    07:10 a.m. & ORR equal to \num{13.2} RR rods. \\ \hline
    01:05 p.m. & Turbogenerator-7 disconnected from the grid. \\ \hline
    02:00 p.m. & ECCS disconnected from CMPC. \\ \hline
    02:00 p.m. & Postponement of test programme execution at the request of the dispatcher of Kyivenergo. \\ \hline
    03:20 p.m. & ORR equal to \num{16.8} RR rods. \\ \hline
    06:50 p.m. & Load of auxiliary equipment not involved in the tests transferred to electrical supply from working transformer T6. \\ \hline
    11:10 p.m. & Power-unit unloading resumed; ORR equal to \num{26} RR rods. \\ \hline
    \multicolumn{2}{|c|}{\textit{26 April 1986 (times per the DREG printout, operating log marked OL)}} \\ \hline
    00:05 a.m. (OL) & Reactor thermal power reached \qty{720}{\mega\watt}. \\ \hline
    00:28 a.m. (OL) & At reactor thermal power about \qty{500}{\mega\watt}, transition from the local automatic power control system to main-range automatic power control (1AR, 2AR). During the transition an unprogrammed decrease of thermal power occurred down to \qty{30}{\mega\watt} (neutron power to zero). Power rise initiated. \\ \hline
    00:34:03 a.m. & Emergency level deviations in the steam-drum separators. \\ \hline
    00:36:24 a.m. & AZ-5 setpoint for drop in pressure in the steam-drum separators shifted from \num{55} to \qty{50}{kgf\per\centi\metre\squared}.\\ \hline
    00:43:37 a.m. & Emergency level deviations in the steam-drum separators. \\ \hline
    00:39:32 a.m. until 00:43:35 a.m. & DREG program not working. \\ \hline
    00:39:32 a.m. until 00:43:35 a.m. & Personnel blocked the AZ-5 signal for shutdown of two turbines. \\ \hline
    00:41:00 a.m. until 01:16:00 a.m. (OL) & Turbogenerator-8 disconnected from the grid to record vibration characteristics at no-load.\\ \hline
    00:52:35 a.m. until 00:59:54 a.m. & DREG program not working. \\ \hline
    01:03 a.m. (OL) & Reactor thermal power raised to \qty{200}{\mega\watt} and stabilised. \\ \hline
    01:03 a.m. (OL) & Seventh Main Circulation Pump (MCP-12) placed in operation. \\ \hline
    01:07 a.m. (OL) & Eighth Main Circulation Pump (MCP-22) placed in operation. \\ \hline
    01:12:10 a.m. until 01:18:49 a.m. & DREG program not working. \\ \hline
    01:19:39 a.m. until 01:19:44 a.m. & Signal ``1 PK-UP'' registered. \\ \hline
    From 01:19:57 a.m. & Signal ``1 PK-UP''. \\ \hline
    01:22:30 a.m. & Parameters recorded to magnetic tape. (Post-accident calculation at Smolensk Nuclear Power Plant: ORR per PRISMA programme was 8 RR rods.)\\ \hline
    01:23:04 a.m. & Command ``Oscillograph ON'' issued; stop-control valves of the Turbogenerator No.~8 closed. Coastdown of four Main Circulation Pumps begins: MCP-13, MCP-23 (section 8RA), MCP-14, MCP-24 (section 8RB). \\ \hline
    01:23:10 a.m. & MPA button pressed. \\ \hline
    01:23:30 a.m. & Signal ``1 PK-UP'' disappeared (duration 3~min 33~s). \\ \hline
    01:23:40 a.m. (01:23:39 a.m. by teleprinter) & AZ-5 button pressed. AZ and RR rods begin moving into the core. \\ \hline
    01:23:43 a.m. & AZ signals appear for reactor runaway period -- period less than \qty{20}{\second}; also power-limit signals -- power above \qty{530}{\mega\watt}. \\ \hline
    01:23:46 a.m. & First pair of ``coasting'' Main Circulation Pumps disconnected. \\ \hline
    01:23:46.5 a.m. & Second pair of ``coasting'' Main Circulation Pumps disconnected. \\ \hline
    01:23:47 a.m. & Sharp decrease (by \qty{40}{\percent}) of flow in Main Circulation Pumps not involved in coastdown (MCP-11, MCP-12, MCP-21, MCP-22) and unreliable readings of Main Circulation Pump flow for Main Circulation Pumps involved in coastdown (MCP-13, MCP-14, MCP-23, MCP-24); sharp pressure rise in the steam-drum separators (SDS); sharp rise of level in SDS; signals ``malfunction of measuring part'' for both main-range Automatic Regulator systems (1AR, 2AR). \\ \hline
    01:23:48 a.m. & Restoration of Main Circulation Pump flow (non-coasting) to values close to initial; on coasting Main Circulation Pumps of the left side: restoration to \qty{15}{\percent} below initial; on coasting Main Circulation Pumps of the right side: restoration to \qty{10}{\percent} of initial for MCP-24 and unreliability for MCP-23; further pressure rise in SDS (left side \qty{75.2}{kgf\per\centi\metre\squared}, right side \qty{88.2}{kgf\per\centi\metre\squared}) and level rise; actuation of BRU-K1 and BRU-K2.\\ \hline
    01:23:49 a.m. & AZ signal ``pressure rise in RP (channel rupture)''; signal ``no \qty{48}{\volt} supply'' (power removed from Control and Protection System rod servo-clutch coils); signals ``malfunction of actuator part of 1AR, 2AR''. From the operating log of the senior reactor control engineer: ``01:24 a.m.: Strong blows, Control and Protection System rods stopped before reaching the bottom limit switches. Power to the clutch coils removed.'' \\
    \hline
  \end{longtable}
\end{center}\normalsize

\chapter*{Appendix 3: The reactor was doomed to explode.}
\addcontentsline{toc}{chapter}{Appendix 3: The reactor was doomed to explode.}
\markboth{Appendix 3: The reactor was doomed to explode.}{}

\textit{Translator's note:} This article, here enclosed as an appendix, is dated 20~April~1991 and was published by the newspaper \textit{Komsomolskoye~Znamya} by A. S. Dyatlov. 

\varthreestars

Former Deputy Chief Engineer of the Chernobyl Nuclear Power Plant, Anatoly Stepanovich Dyatlov, holds that the immediate culprits of the catastrophe were none other than the Scientific Director and the Chief Designer of the RBMK-1000 reactor -- that very reactor which, on 26~April 1986, was torn asunder.

A new residential quarter in Kyiv; one of those standard, prefabricated buildings. A door upholstered in imitation leather. We ring. A tall, painfully thin man appears. From beneath pale brows peers an intelligent, attentive gaze. He greets us with a kindly smile, extending his hand. Only after the greeting do we, involuntarily, observe upon it the pale pink, but unmistakable, islands of radiation burns. Such is the former Deputy Chief Engineer of the Chernobyl Nuclear Power Plant, A.~S.~Dyatlov.

Convicted and sentenced -- \textit{ergo} guilty of the crime known to all -- Dyatlov was, by law and by public opinion alike, proclaimed one of the principal ``authors'' of the greatest technological calamity of the twentieth century. Only recently has Anatoly Stepanovich returned home.

From the ``lockup'' -- so he names the general-regime correctional-labour colony in which he served his term under Article~220 of the Criminal Code of the Ukrainian SSR.

Let it not be concealed: we went to Dyatlov as to a criminal. We left his dwelling as though departing from the home of a victim. We had intended to denounce him, yet were compelled to sympathise -- and, in part, to assent.

\medskip

\textbf{--- Tell us something about yourself. It is said that, before serving at the Chernobyl plant, you worked with the nuclear power installations of submarines in the Far East?}

\noindent
--- Yes, that is correct. I was born in 1931 near Krasnoyarsk. By training and professional experience I am a specialist in the operation of nuclear power plants. I was fond of my service in the Far East. But once, during a holiday, I visited the nascent Chernobyl Nuclear Power Plant, then still under construction. There I reached agreement with the Director, Viktor
Petrovich Briukhanov, to take a post as Deputy Shop Chief. At Chernobyl I participated in the installation, physical start-up, and operation of all four units. And when the investigation began, the entire course of the inquiry was directed toward establishing that the guilty parties were the operating personnel -- and, first among them, Dyatlov himself. Yet the workers of the station succeeded in grasping that the accident had not arisen through our fault. Therefore, during the trial, the overwhelming majority of witnesses did not question my competence. Moreover, the case materials themselves, in my view, demonstrate convincingly that the operating personnel of the station were not at fault.

\medskip

\textbf{--- But the verdict, as everyone knows, said quite the opposite. How do you explain this?}

\noindent
--- The verdict could not have been otherwise. I am prepared to wager that you will not recall a single case in recent years in which anyone other than dispatchers, operators, captains -- those eternal ``switchmen'' -- has been punished for major accidents. This was expressed very aptly in a letter published in \textit{Molodaia Gvardiia} by mine-rescue
workers of the Donetsk region who were liquidating the consequences of a mine poisoning: ``A system has been created which functions smoothly to divert responsibility from the true authors of outrages. The monopolist itself investigates the accident, itself proposes the measures, and itself supervises their implementation.''

At our Chernobyl Nuclear Power Plant everything proceeded in precisely the same manner. Not one of the commissions -- though there were several -- included representatives of the operating personnel, that is, those whom they were accusing. The commissions consisted exclusively of potential, and at times actual, culprits of the catastrophe. One could not expect an objective investigation from them. Nor was there one. Materials that contradicted the accepted version were disregarded entirely at the interdepartmental scientific-technical councils convened on 2 and 17~June 1986 under the chairmanship of Scientist A.~P.~Aleksandrov. At those meetings the concept of the accident was framed in such a manner as to absolve the designers of the equipment and to lay all blame upon the personnel. These proceedings formed the basis of the Government Commission's report, which was transmitted to the Politburo of the CPSU Central Committee, to the Council of Ministers, and to the IAEA.

Should one then be astonished that in the decisions of the Politburo our guilt was stated directly? To expect impartiality from the court would have been, to put it mildly, naïve.

It is true that, from the podium of the XXVIII Congress of the CPSU, something unprecedented was heard -- that the Politburo and the Government had ``failed to understand'' the causes of the Chernobyl accident. No: it is untrue that they ``failed to understand'' -- they \textit{did not wish} to understand. Nothing prevented the highest authorities from engaging the foremost scientific minds to analyse the causes of the accident thoroughly -- to study, as the phrase goes, the history of the matter.

\medskip

\textbf{--- So the Chernobyl catastrophe had certain preconditions?}

\noindent
--- Preconditions? That is too mild a word! The official explanation of what happened sounds like this: the accident occurred as the result of an unbelievable coincidence of several gross violations by the operating personnel of the norms and rules for operating the power unit. But for me it is very easy to prove that the RBMK-1000 reactor was bound to explode somewhere. The public is practically unaware of such facts.

In 1975 an accident occurred at the Leningrad Nuclear Power Plant: a channel lost tightness in a reactor of the same type as in Chernobyl. A commission from the Kurchatov Institute studied what happened and compiled a list of recommendations for improving the reactor's reliability, including such important measures as reducing the steam-void reactivity coefficient and creating a fast-acting emergency protection system.

They began to implement these recommendations more than ten years later -- already after the Chernobyl catastrophe.

Next. In 1983, when we in Chernobyl were loading the reactor with technological fuel, physical measurements of the characteristics of the core were carried out, and an extremely dangerous phenomenon was discovered: the emergency-protection rods, when moving downwards, introduced not negative but positive reactivity into the reactor during the first five seconds. Yet the physical-startup commission, without any justification, considered it possible to allow the reactor to go into operation. The inspector of Gosatomenergonadzor also concurred with the commission. True, the scientific director, understanding how dangerous this was, wrote a letter to the chief designer about the need to eliminate the defect. The designer, by December 1984, drew up a technical specification -- and\ldots~that is where everything ended.

It took a catastrophe for this matter to be addressed seriously and for the rods in the reactor to begin to be replaced!

And one more astounding fact. The head of the group for reliability and safety of Nuclear Power Plants with RBMK reactors in the laboratory of the Kurchatov Institute, V.~P.~Volkov, repeatedly submitted memoranda to all his superiors substantiating the reactor's danger and offering proposals for its improvement. No one paid any attention to them. In the end Volkov was forced to appeal directly to Scientist A.~P.~Aleksandrov. But alas -- his memorandum lay in the office of the President of the USSR Academy of Sciences until the very accident. And when the disaster occurred, Volkov handed all the materials over to the USSR Procuracy. After that he was no longer admitted to the institute. Then, in search of justice, he wrote to M.~S.~Gorbachev himself. From the apparatus of the CPSU Central Committee, Volkov's materials were forwarded to Gosatomenergonadzor. There a commission was created, which confirmed the specialist's correctness.

There is where the truly unbelievable coincidence of blatant negligence lies!

Nothing prevented the scientific director of the reactor project, Scientist A.~P.~Aleksandrov, and the chief designer, Scientist N.~A.~Dollezhal, from improving the reactor after the accident at the Leningrad Nuclear Power Plant, after the results of the startup tests in Chernobyl, after the serious warnings of V.~P.~Volkov! Had this been done in due time -- the catastrophe would not have occurred. So who are the real criminals -- us, or them?

At the trial, incidentally, the materials concerning the guilt of the reactor designers were placed into a separate proceeding. How it ended -- no one knows. How can that be?

\medskip

\textbf{--- According to rumours (there are no official announcements), the designers' case was closed for lack of judicial prospects. They supposedly fell under the amnesty declared in connection with the 70th anniversary of Soviet power. But let us return to the main issue. As we understand you, the truth about the real causes of the accident has still not been told. Why do you think that is?}

\noindent
--- Because the true culprits of the catastrophe are bound tightly together by a common lie. Sometimes this is so obvious that one is astonished that people do not realise it. How many times, for example, have we been told that the question of the effect of low doses of radiation on health has not been studied. And this is said in a country where for decades thousands of people have worked with radioactive materials. It is enough to take medical records, collect year-by-year data on exposure, and derive elementary mathematical correlations to obtain the answer to that question.

The lies about Chernobyl are disseminated in multi-million print runs. Here is a simple example. A correspondent of \textit{Pravda}, Krivomazov, quoting the chairman of the government commission investigating the tragedy in Ufa, writes: ``Let us recall that in Chernobyl there were four whole systems of fool-proof protection, and all four were somehow disabled.'' Could that really have been the case? Could they not think before they speak! Are we kamikaze, as to disable protections? The truth is that in 1986 the RBMK-1000 reactor had no ``fool-proof'' protections whatsoever. I wrote about this to \textit{Pravda}, but received no reply\ldots~The organ of the CPSU Central Committee, like Caesar's wife, is apparently always above suspicion? And this is not an isolated case.

Scientist L.~A.~Buldakov, in the journal \textit{Smena} (No. 24 of last year), asserts that the evacuation of the population of Pripyat was not delayed but timely. What ``timeliness'' can one speak of, if already by midday on the 26th it was absolutely clear that people in Pripyat must not remain. A \textit{scientist} cannot fail to understand this. Why then does he state an obvious untruth?

I want to emphasise: L.~A.~Buldakov is a medical specialist; he bears no direct responsibility for the accident. Now imagine what people of similar moral fibre who are directly guilty of the catastrophe say -- such as Scientist Anatoly Petrovich Aleksandrov. From the very beginning of events and up to this year (I judge from his latest publication in Ogonyok) he has persistently repeated the responsibility of the operating personnel.

Let us reflect together. If the official version of the causes of the catastrophe were correct, then why was all information about the accident classified? Journalists, granted, may muddle something. But the specialists working at nuclear power plants must know everything about the catastrophe in order not to repeat similar mistakes. Yet nothing was told even to them, because these people would immediately have understood the true reliability of the reactor designed by Aleksandrov.

\medskip

\textbf{--- ``Quite so; yet no one deprived either you or the other specialists of the right to speak at the trial.''}

\noindent
--- Indeed, I had never entertained the intention of remaining silent. Hence the materials of the trial demonstrate, \emph{with a clarity impervious to sophistry}, our complete innocence.

As a purported argument for the reliability of the reactors, the court experts advanced the assertion that, by the moment of the catastrophe, the fleet of RBMK units had accumulated nearly one hundred reactor-years of operation (in point of fact, no more than eighty-seven). At first glance the figure has a certain air of solidity; yet a moment's ciphering dispels the illusion. If one divides eighty-seven reactor-years among the thirteen RBMK-1000 units then in service, one arrives at the grotesque conclusion that every five or six years the country ought to expect a new Chernobyl. Can such a prospect satisfy any rational being? Surely not.

Moreover, the case file contains a most telling confession by Scientist Dollezhal himself. Permit me to cite verbatim:

\begin{personal}[From N.~A.~Dollezhal]
\textit{``With operation on two-percent enriched uranium, the influence of the steam-void effect and reactivity is regulated by installing channels with special absorbers, which is strictly provided for in the operating instructions. Deviation from them is inadmissible, since it renders the reactor uncontrollable.''}
\end{personal}

Yet our reactor possessed \textit{no} such additional absorbers within the core. It follows -- by the chief designer's own admission! -- that, sooner or later, it was destined to explode.

Dollezhal, incidentally, stands alone among the specialists engaged upon the reactor who has uttered the unvarnished truth. He acknowledged, without equivocation, that the Chernobyl unit was \textit{uncontrollable}, and that its emergency protection was faultily conceived. Aleksandrov, by contrast, persists in shifting all responsibility upon the operating personnel and admits nothing.

Let us recall that the catastrophe commenced at the instant the Emergency Protection button was pressed. To avoid misunderstanding, I shall explain: this very button serves equally for the routine shutdown of the reactor under ordinary conditions. On 26~April 1986 we pressed it under wholly normal, fully prescribed circumstances in order to shut down the chain reaction. Instead, we obtained an explosion.

How, indeed, may such a thing be conceived -- that an \textit{Emergency Protection} system does not shut down but rather \textit{detonates} a reactor? There is but one answer: it was \textit{so designed}. In view of all that has been said, I state in the plainest terms: in the Chernobyl catastrophe neither the builders, nor the installers, nor the equipment manufacturers, nor the station personnel bear the slightest guilt. Entirely and exclusively the responsibility rests upon the physicists and the designers.

\medskip

\textbf{--- An accident is a grievous misfortune, and like all misfortunes it has both culprits and rescuers. What can you say of the firefighters? There are those who assert that their deaths were nearly in vain.}

\noindent
--- I do not know whether they infringed any instruction; but I am unshakably convinced that on 26~April the firefighters \textit{saved us from a calamity of global proportions}. Had the flames they subdued blossomed into a vast conflagration and reached the neighbouring units then operating at nominal power, the magnitude of the tragedy would have been immeasurably greater. As for the deaths of those brave men, one reservation must be made: even had they donned special protective gear, it would not have saved them from the gamma radiation. Those who extinguished the fires on the reactor roof, kindled by the incandescent fuel fragments, were heroes without qualification. Only an automatic fire-extinguishing installation could have preserved them -- but none existed there. Before the bright memory of these valiant men we must bow our heads.

\medskip

\textbf{--- We understand your desire to cast light upon the true causes of the catastrophe; yet tell us something further. The reactor lies in ruins; radioactive contamination advances. What, in your view, is the personal responsibility of the station's officials for the irradiation of the populace? In particular -- of the director, Viktor Petrovich Briukhanov?}

\noindent
--- It is not easy for me to pass judgment upon others. I deem Briukhanov culpable in that, on the very first day, he transmitted to Kiev a radiological report whose data were palpably understated. Yet I doubt that this report materially influenced subsequent decisions. Measurements continued uninterrupted, and the authorities were obliged to act according to them. Briukhanov, moreover, bore responsibility only for the civil defence of the station, not of the city. Decisions concerning Pripyat ought to have been taken by the head of its civil defence, Vladimir Pavlovich Voloshko, chairman of the city executive committee. One may indeed argue that at that moment Briukhanov should have -- to speak crudely -- struck his fist upon the table and demanded evacuation.

\medskip

\textbf{--- Indecision in such an hour is, of course, culpable, but to debate this leads us from fact into the domain of morality. Tell us, rather, how the charge against you was formulated in the verdict.}

\noindent
--- The verdict reads thus: ``The principal causes that led to the accident were gross violations of the rules established to ensure nuclear safety at a \textit{potentially explosion-hazardous enterprise}.'' Hence the application of Article~220 of the Criminal Code of the Ukrainian SSR, under which I was consigned for ten years to ``the lockup.'' But nuclear power stations have \textit{never} been classed among explosion-hazardous facilities such as powder mills. If, indeed, nuclear stations are explosion-hazardous, they must be designed and constructed upon entirely different principles. It is sheer absurdity: prior to the trial I believed myself employed at an ordinary power station, and only in the courtroom did I discover that it was ``potentially explosion-hazardous.''

The absurdity of the charge was evident to any impartial observer. The court could easily have established that the reactor would \textit{never} have exploded had it complied with the safety standards for nuclear power stations in force within the country.

\medskip

\textbf{--- Very well; yet in the public imagination you appear as the man who embarked upon some fantastical experiment on an operating industrial reactor. Will you explain the essence of that experiment?}

\noindent
--- The authors of the programme, were Gennady Petrovich Metlenko, a representative of Dontechenergo, and myself. He had previously participated in the testing of numerous electrical systems at the station. The core of the idea was to employ the kinetic energy stored in the Turbogenerator rotor as it coasted down.

Each unit of the station is equipped with an Emergency Core Cooling System, whose purpose is to avert core meltdown in a design-basis accident -- that is, in the maximum credible accident, assumed to be the rupture of a large-diameter pipeline in the primary circuit. When, under such an accident, electrical power is lost in the supply system, the generator, while decelerating, continues to feed the feedwater pumps. Thus it is intended to provide water to the reactor until the long-term emergency cooling system becomes operative. Our aim was to ascertain whether the generator's rundown duration sufficed for this purpose.

A programme of the experiment was drafted and duly approved. After the catastrophe it was subjected to exhaustive analysis by countless specialists, and none discovered an error. They confined themselves to stating that the safety measures had allegedly not been elaborated in our programme. True; yet they had been implemented even before the commencement of the test and recorded in other sections. I stand accused, it seems, of nothing more heinous than failing to re-copy a list of measures from one part of the document into another!

No one wishes to acknowledge the absurdity of such a reproach. The court experts, for instance, asserted that, in accordance with the instructions for starting a main circulation pump, a representative of the Nuclear Safety Department ought to have been summoned. They simply had not read the instruction to its end. It states expressly that such presence was unnecessary ``until special instructions were issued.'' And those instructions had indeed been issued.

\medskip

\textbf{--- You frequently refer to the court experts. As we understand it, the court's judgment depended directly upon them. Who, then, were these people?}

\noindent
--- The overwhelming majority, were representatives of the very same design and construction organisations whose honour was directly at stake. To allow them to participate in a judicial investigation of the causes of the catastrophe was, in my view, profoundly immoral. Their conclusions collapse at the slightest scrutiny. The experimental programme had been examined by many specialists; to declare it incompetent is untenable. That is the first point.

Second: it is said the programme had not been coordinated with the supervisory authorities. This is true; yet the instructions then in force required no such coordination.

Third: it is evident that the catastrophe might have occurred during \textit{any} operation with such a reactor. This is not merely my opinion. I may at least refer to the findings of the commission chaired by Deputy Minister of Energy Gennady Aleksandrovich Shasharin, who appended his conclusions to the investigation protocol as early as May~1986. Why did the court disregard this? The reason is plain: it was necessary to divert public attention from the true causes of the accident, to obfuscate, to conceal the names of the genuinely guilty.

Particularly much has been said about the shutdown of the reactor's emergency cooling system during the test. To laymen this seems incomprehensible: how could an accident occur when the emergency cooling system was turned off! Outrageous! Yet few know that the instructions permitted disabling this system for intervals authorised by the station's chief engineer. And it was he who approved the programme.

Moreover, the emergency system was not designed for such an eventuality. Even had it been engaged, it would not have had time to act. Most importantly, it \textit{could not} have helped. The reactor was annihilated by the explosion: the process channels were shredded, the fuel reduced to powder. There remained nothing to cool.

\medskip

\textbf{--- Do not be offended, but we must ask: who, in truth, are those convicted for the accident---criminals, or victims of the catastrophe?}

\noindent
--- We are, beyond any doubt, victims. The unit's personnel received the first and deadliest blow of radiation. Those who survived were then compelled to endure the ignominy of the trial and the monstrous injustice of universal condemnation. The powerful of our land have ever kept their scapegoats at the ready.

My comrades -- the unit shift supervisor Sasha Akimov, the reactor operator Lenya Toptunov, and the shift supervisor of the reactor shop Valera Perevozchenko -- were spared the disgrace only by death. The cynicism of our bureaucratic machine knows no bounds. The USSR Procuracy contrived to send to the families of Toptunov, Perevozchenko, and Akimov formal notices stating that they were released from criminal liability ``in connection with their deaths.'' As if to proclaim: know that your dead sons, fathers, and husbands were criminals.

It is only fair to note that the truth of what occurred has now begun, however timidly, to break through. There exist -- though almost unknown to the public -- documents of immense significance: the report of State Atomic Supervision inspector A.~A.~Yadrikhinsky; the report of Professor B.~G.~Dubovsky; the conclusions of the commission under N.~A.~Shteinberg; and many other materials. They contain a qualified analysis of the genuine causes of the catastrophe and, for all practical purposes, establish our innocence. These documents are not secret; they may be read in the Commission for the Investigation of the Causes of the Accident at the Chernobyl Nuclear Power Plant of the Supreme Soviet of the USSR. Why, then, does no one write of them? 

In the work of that commission I place all my hopes for justice.

\medskip

\textbf{--- And how do you, Anatoly Stepanovich, intend to live further?}

\noindent
--- My sole task is to secure the publication of the truth concerning the causes of the catastrophe, to save from dishonour at least the memory of my dead comrades. I have no other personal plans, nor can I. I received \qty{550}{\rem} during the accident, and about \qty{100}{\rem} more in the course of my earlier work. My skin is marked by radiation burns. I am now in the second group of disability. My life is kindling out. Therefore, day and night, I think only of one thing; I desire only one thing -- \textit{truth, and nothing but the truth}.''

\medskip

\begin{flushright}
\textit{Interview by A.~Budnitsky and V.~Smaga.}
\end{flushright}

\chapter*{Appendix 4: The Echo of Chernobyl: Radioactive damage to conscience.}
\addcontentsline{toc}{chapter}{Appendix 4: The Echo of Chernobyl: Radioactive damage to conscience.}
\markboth{Appendix 4: The Echo of Chernobyl: Radioactive damage to conscience.}{}

\textit{Translator's note:} This article, here enclosed as an appendix, has been originally published in the \textit{Vecherniy Kyiv} newspaper. June 22, 1992, No. 120 (14484).

\varthreestars

Historians yet to come will, with exactitude, mark the inception of that post-Chernobyl age which has already cast so long a shadow across the destiny of mankind. At 1~hour 23~minutes 40~seconds on 26~April~1986, the operator Aleksandr Akimov pressed the Emergency Protection button of the reactor -- and then there ensued an explosion. Amid the avalanche of grief beneath which we laboured in those early days of mitigating the accident's consequences, this astounding circumstance passed almost unremarked.

The protection system blew up the reactor.

Let the Reader pause and reflect. Conceive, if he can, of setting the safety -- catch upon a hunting -- piece in order to forestall a shot, only to have it discharge both barrels into his very chest.

To this day the world labours to comprehend both the causes and the consequences of the Chernobyl catastrophe, seeking some path of egress from that impasse into which the headlong advance of nuclear power has driven civilization. Abandonment of the entire undertaking is, for many reasons, all but impossible; yet neither may we avert our gaze from the daily accumulation, at the planet's numerous nuclear power plants, of several hundred ``Chernobyls'' worth of radioactive debris -- each day an increment of death to all that lives. Hence the explosion at the Chernobyl Nuclear Power Plant will long remain central to any disputation concerning the future course of scientific and technological progress.

In July~1987 the Criminal Division of the Supreme Court of the former USSR pronounced the director of the station, Viktor Briukhanov; the chief engineer, Nikolai Fomin; the deputy chief engineer, Anatoly Dyatlov; the shift supervisor, Boris Rogozhkin; the head of the reactor shop, Aleksandr Kovalenko; and the State Atomic Supervision inspector, Yuri Laushkin, to be the culprits of the catastrophe. All were sentenced to terms of imprisonment and, in the eyes of the citizenry, to the far heavier burden of moral disgrace for the gravest nuclear disaster of the twentieth century.

Journalists were not admitted to that swift -- and, as is now indisputable, unjust -- Chernobyl trial. The official materials, published both within the country and abroad, declared that the catastrophe had arisen from ``an extremely improbable combination of violations of procedures and operating regime committed by the unit's personnel.'' To the ordinary Soviet reader this needed little elaboration: a handful of careless morons had gathered and blown up a reactor. Specialists, however -- and above all those working upon reactors of the Chernobyl type, the RBMK -- responded otherwise. They required knowledge of the true causes, if only to prevent the recurrence of an analogous calamity.

But no. All materials concerning the accident were sealed, even from those labouring in the nuclear industry. And now, as the truth at length strains toward the light, we discern precisely why. The ``Chernobyl six'' served merely as scapegoats, decoys meant to deflect the public gaze from more exalted personages. These latter, as had long been customary in the empire, did not sit upon the defendants' bench but stood behind it -- dictating their conclusions to the forensic -- technical experts whose judgments furnished the indictment. But let us proceed in order.

Long before the catastrophe, V.\,P.~Volkov, head of the reliability and safety group for RBMK Nuclear Power Plants at the Kurchatov Institute of Atomic Energy, had submitted several memoranda to his chief, Academician A.\,P.~Aleksandrov, exposing the reactor's design defects. They were quietly shelved. When the explosion came, Volkov addressed himself to the nation's leadership, to the Procuracy of the USSR, and, naturally, to his immediate superiors, setting forth his own account of the accident -- an accident which, he affirmed, had been \textit{``caused not by the actions of the operating personnel but by the design of the core and by an incorrect understanding of the neutron physics processes occurring in it.''} Even then the specialist was not heeded. Yet Aleksandrov, though preoccupied by meetings of scientific-technical councils convened to determine the causes of the catastrophe, nevertheless found to find time to have Volkov dismissed.

Thus, brazenly and crudely, the truth concerning the real causes of the Chernobyl catastrophe was suppressed. But, as the proverb says, one cannot conceal an awl in a sack. Several documents now exist that overturn, root and branch, the scientific-technical foundations of the indictment brought against the Chernobyl ``six.'' Of these, the most salient is the report of the commission of the State Industrial Atomic and Energy Supervision of the former USSR, signed by Nikolai Shteinberg -- now Chairman of the State Committee of Ukraine for Nuclear and Radiation Safety -- and published last year.

It must, in fairness, be said that the former State Atomic Supervision itself is not without blame. In 1983, during the physical startup of Unit~4's reactor, an exceedingly dangerous phenomenon was discovered: the introduction of \emph{positive} reactivity during the initial downward movement of the control rods. The state inspector duly recorded this fact and\ldots~permitted startup. That same year the scientific supervisor, Scientist Aleksandrov, wrote to the chief designer, Scientist N.\,A.~Dollezhal, insisting upon elimination of the defect. The designer ordered the preparation of a technical assignment for the correction. But nothing came of it: no drawings were produced -- not up to the very moment of catastrophe.

Here, then, was unveiled that very criminal negligence of which the Chernobyl operators were later accused. It is now definitively established that the insertion of positive reactivity by the rods initiated the explosion. Yet few know that a report of the Institute of Atomic Energy itself enumerated twelve other entirely possible accident scenarios, each of which, in principle, culminated in an explosion. Thus the RBMK emerges as an eternal monument to the bungling of high-ranking physicists and designers.

Immediately after the accident, calculations and experiments established that the reactor design violated thirty two articles of the normative documents governing the safe design and operation of nuclear power plants. On the basis of these findings, the Shteinberg commission concluded:

\begin{personal}[From the conclusions of the Shteinberg commission:]
\textit{``The shortcomings of the RBMK design operating at the fourth power unit predetermined the grave consequences of the Chernobyl catastrophe.''}
\end{personal}

A little earlier in the same document there stood words yet more chilling:

\begin{personal}[From the conclusions of the Shteinberg commission:]
\textit{``\ldots it must be stated that an accident similar to Chernobyl's was inevitable.''}
\end{personal}

Nevertheless, in the public mind -- as already observed -- the principal culprits of the catastrophe were not the aforementioned scientists but the Chernobyl ``six.'' Aleksandrov and Dollezhal were removed from peril by a manoeuvre at once simple and astonishingly formalistic. The Chernobyl court separated the materials concerning the reactor's creators into an independent case file. Soon thereafter the Supreme Court of the former USSR terminated that case ``for lack of judicial prospects.'' To probe into the affairs of venerable grey-haired men was deemed futile, for any accusation brought against them would fall under one amnesty or another.

Let me be plain. I write these lines not to summon the aforementioned scientists before a tribunal, but to see justice prevail. A direct participant in the events of Chernobyl, Ukraine's Minister of Energy Vitalii Skliarov, answered my inquiry concerning the personal responsibility of Aleksandrov and Dollezhal in these terms:

\begin{personal}[From V.~Skliarov]
\textit{``To reduce the essence of the matter to those two surnames is to oversimplify the problem. The whole system is guilty. Dollezhal and Aleksandrov were creating an atomic bomb. Power production became a sort of mass-market offshoot of the military. When it became clear that too many warheads had been manufactured, they sought a `peaceful' use for the powerful technology. The plutonium reactor, colloquially called `Ivan,' was quickly transformed into the RBMK. The reactor's creators were under tremendous pressure from the ruling circles and were in great haste. Thus there emerged a design based on a flawed concept of safety--one that sooner or later was bound to lead the country to catastrophe.''}
\end{personal}

Yet such reflections are cold comfort to the children and grandchildren of Briukhanov, Fomin, Dyatlov, Rogozhkin, Kovalenko, and Laushkin. In the eyes of their neighbours they remain, even now, the relatives of criminals who brought unspeakable misfortune upon their land. The Chernobyl ``six'' were no angels, but human beings subject to human frailty. In the Shteinberg commission's conclusions their errors are set forth with sobriety; but beside the monstrous imputation that they bore \textit{chief} responsibility for the catastrophe, such errors appear trifling. And can six men truly bear the guilt for the radioactive -- if one may employ the term -- tainting of conscience of the leaders of the world's most formidable nuclear monopoly, the late Ministry of Atomic Energy and Industry of the former USSR? It was that ministry which, with the obstinacy of intellectual automatons and with limitless access to state funds, strewed the breadth of a fallen empire with installations as dangerous as the RBMK reactors which continue to operate at many nuclear power plants.

The Soviet Union is no more; its nuclear monopoly has dissolved into several concerns; and the Supreme Court of the USSR -- before which the case of the Chernobyl ``six'' was on the verge of review -- has ceased to exist. The convicted men, now citizens of other states, find themselves in a species of juridical limbo. The case files lie in Moscow, but the Russian Procuracy refuses to address them, for the accused were judged under articles of the Ukrainian criminal code. Scientist Dollezhal remains obstinately silent in the face of every public accusation; Scientist Aleksandrov, for his part, brazenly lays all blame upon the station's operating personnel, as in his interview with \textit{Ogonyok}.

Such a state of affairs cannot endure. The Procuracy of Ukraine is empowered to request from Moscow the case of the Chernobyl ``six'' and to secure its review by the Supreme Court of Ukraine in light of newly revealed circumstances. Abundant material upon this matter is held by the Commission of the Supreme Soviet on Chernobyl Affairs. The highest judicial authority of our state might also profitably evaluate the results of the independent investigation conducted by the Ukrainian environmental association ``Zelenyi Svit.''

The whole truth about Chernobyl has not yet been spoken\ldots

\begin{flushright}
Valentyn Smaga
\end{flushright}

\chapter*{Appendix 5: Hostages of the Reactor.}
\addcontentsline{toc}{chapter}{Appendix 5: Hostages of the Reactor.}
\markboth{Appendix 5: Hostages of the Reactor.}{}

\textit{Translator's note:} This article, here enclosed as an appendix, has been originally published in the \textit{Trud} newspaper. April 3, 1996.

\varthreestars

In the wake of the accident at the Chernobyl Nuclear Power Plant two criminal proceedings were initiated. The first, conducted in the years 1986-1987, culminated in the prosecution and conviction of Director Bryukhanov and several members of the plant staff. The second was opened in 1991, when the inquiry, extending its scope, addressed -- inter alia -- the defects of the reactor's design. Yet two years later that case, too, was brought to an end by a decree of the Senior Investigator for Especially Important Cases attached to the Office of the Procurator General of the Russian Federation, Senior Counsellor of Justice Boris Uvarov.

What follows are excerpts -- here presented in abridged form -- from the \textit{``Decree on the Termination of the Criminal Case,''} a document hitherto unpublished, which, in our judgement, casts additional light upon those dreadful events. At the same time we invited Boris Ivanovich Uvarov to annotate this material, in the hope that he, who had examined that decade-old tragedy with uncommon thoroughness, might shed more light on the present state of the questions raised during the investigation.

\medskip

``\ldots~On 26~April 1986, at approximately 1 hour 24 minutes, at the Fourth Unit of the Chernobyl Nuclear Power Plant, as the consequence of a thermal explosion, there occurred a major accident entailing the destruction of the RBMK-1000 reactor\ldots

\ldots~A criminal case was opened. For abuse of office and violation of safety rules, this case led, in 1987, to the bringing to criminal responsibility and conviction of the director of the Chernobyl Nuclear Power Plant, Bryukhanov, and other employees of the plant. The investigation conducted by the Procuracy of the USSR -- and the Supreme Court of the USSR, which examined this matter as a court of first instance -- concluded that the principal cause of the accident lay in violations of the rules for operating the Fourth Power Unit committed by the operating personnel and the management of the plant.

\ldots~In the press there appeared statements by deputies, journalists, and certain nuclear scientists expressing disagreement with the conclusions of the Supreme Court of the USSR. Many contended that `scapegoats' had been subjected to criminal prosecution in the persons of Bryukhanov and the others, and that the true causes of the Chernobyl accident were rooted in design shortcomings of the RBMK-1000 reactor; and that the development of the accident, together with its escalation into a catastrophe, had been rendered possible because reactors of this type lacked -- contrary to world practice -- systems for localizing such accidents, the so-called containments (concrete domes). The press and the critics named as culpable certain high-ranking scientists and party-economic leaders who, in the 1970s, had determined erroneous principles for the development of nuclear power without adequately ensuring its safety.

In view of this, under the former Supreme Soviet of the USSR (by a resolution of its Presidium of 01.10.1990) there was constituted a Commission `to review the causes of the accident at the Chernobyl Nuclear Power Plant and to assess the actions of officials in the post-accident period.' Attached to this Commission was an expert group which, in 1991, prepared a conclusion, stating -- on the basis of the opinions of individual scientists -- that there existed design shortcomings of the RBMK-1000 reactor which had occasioned the accident\ldots

A paradoxical situation thus arose in the country: on the one hand stood a judgment of the Supreme Court of the USSR, confining the causes of the Chernobyl accident to the errors of the operators; on the other hand, the conclusion of the Commission of the Supreme Soviet of the USSR, which regarded the deficiencies of the reactor's design as the cause of the accident, and in particular of its catastrophic consequences\ldots

\ldots~This contradiction compelled the former party-state leadership of the USSR, at the XXVIII Congress of the CPSU, to instruct the Procuracy of the USSR to verify the soundness and timeliness of the measures undertaken to mitigate the consequences of the accident at the Chernobyl Nuclear Power Plant. In consequence, the Acting Procurator General of the USSR, A.~D.~Vasiliev, by order of 6 August 1990, formed a working group consisting of prosecutors of the USSR, RSFSR, Ukrainian SSR, and Byelorussian SSR procuracies, tasked with conducting an additional review\ldots~On 25~January 1991 a new criminal case was opened\ldots

In December 1991, in connection with the dissolution of the Procuracy of the USSR, the investigative group working on the case was disbanded. The case (41 volumes) was transferred to the Procuracy of the Russian Federation. A portion of the volumes remained in the procuracies of Ukraine and Belarus\ldots~By the present time, decisions to terminate proceedings in these cases have been issued in both states: in Ukraine -- for reason of the expiry of the statute of limitations (with respect to the republican leadership); in Belarus -- for lack of proof of guilt\ldots

In the course of the investigation of the case in Russia, information was obtained concerning design and construction deficiencies of the RBMK-1000 reactor which constituted the principal cause of the accident at the Chernobyl Nuclear Power Plant on 26~April 1986.

\ldots~The report of the governmental commission noted that, under the conditions of the violations permitted by the plant personnel, `the reactor emergency protection system failed to fulfil its functions. The development of the accident, which led to the destruction of the reactor, occurred because of deficiencies in the reactor's design\ldots'

\ldots~When questioned as a witness, the deputy director of the Research and Design Institute of Power Engineering, Scientist V.~V.~Orlov, engaged in reactor-engineering problems, declared that the design deficiencies of the RBMK-1000 project which had manifested themselves in the Chernobyl accident were the result of shortcomings in the physical substantiation of the project and in the design of the reactor itself, which were realized in the accident as a consequence of the personnel's actions that failed to correspond to regulations\ldots

Concerning the design and construction deficiencies of the RBMK-1000 reactor, Doctor of Technical Sciences B.~G.~Dubovsky, possessing extensive experience in the field of nuclear safety, testified that the primary cause of the accident was the designers' failure to comprehend the neutron processes and their errors in the design of the emergency protection\ldots~The actuation time of the Emergency Protection absorber rods exceeded \num{18}~seconds, whereas the development of accident processes in this reactor unfolds within \num{3}{4}~seconds.''

He also testified that when he was still head of the Nuclear Safety Department at the Physics and Power Engineering Institute, his studies -- based on the accident at the Leningrad Nuclear Power Plant in 1975 -- led him to conclude that in the lower part of the RBMK-1000 reactor, because of the peculiarities of the neutron field in the core, a kind of separate explosive reactor arises\ldots~In Professor B. G. Dubovsky's view, the plant personnel were not at fault for the accident, which occurred solely because of design shortcomings in the reactor. According to Professor Dubovsky, the chief designer, Academician Dollezhal, failed to provide in this reactor type any device for venting steam in the event of damage to one or several channels, which led to the development of the accident\ldots

In his testimony, the former chief designer of the RBMK-1000 reactor, Scientist N. A. Dollezhal, stated, in particular, that after the accident at Unit~1 of the Leningrad Nuclear Power Plant it became clear that the system for monitoring power distribution in the reactor was imperfect. But since that was the first RBMK-1000 reactor, they still did not know very much at that time. After the accident, however, the Institute of Power Engineering developed two systems -- LAR (Local Automatic Regulator) and LAZ (Local Automatic Protection) -- which were installed on all RBMK reactors, including those at the Chernobyl Nuclear Power Plant. As for the design of the absorber rods of the reactor, this part of the work had been handled by his deputy, Scientist Emelyanov.

As established during the investigation, the design of the absorber rods had significant deficiencies, which were manifested during the Chernobyl accident. In this connection, the heads of groups at the Power Engineering Institute, I.~N.~Demin and V.~P.~Potapova, testified that after the Chernobyl accident the design of the absorber rods was modified in all RBMK reactors, and a new fast-acting protection system was introduced, enabling the absorber rods to enter the core in \num{2.5}~seconds and completely shut down the reactor\ldots

In his testimony, M.~I.~Miroshnichenko, a former member of the USSR Gosatomnadzor Commission for investigating the causes and circumstances of the Chernobyl accident, stated that virtually all the design deficiencies of the RBMK-1000 reactor and of its Control and Protection System had been known even before the accident, as were the technical measures required to eliminate them\ldots

% --

``\ldots~It should be noted that a criminal case regarding design deficiencies of the reactor had already been separated out in 1987; however, shortly after the main case against Bryukhanov and others was heard in court, it was closed by the USSR Procuracy. Meanwhile, many facts relating to the design shortcomings as a cause of the accident's occurrence were not known to the court at the time of the verdict\ldots~In this connection, these circumstances must be deemed newly discovered, and they are subject to further investigation. If their validity is established, the verdict in the case of Bryukhanov and the other Chernobyl Nuclear Power Plant workers is subject to annulment, since their guilt may prove doubtful in the light of these circumstances, given that previously the main causes of the accident were deemed to be not the reactor's design deficiencies but the personnel's errors during operation\ldots

In the forthcoming investigation it is necessary to examine in depth the causes of the accident at Unit~1 of the same Chernobyl Nuclear Power Plant, which occurred in 1982. Professor Dubovsky referred to this accident, as well as the 1975 accident at the Leningrad Nuclear Power Plant, as a ``rehearsal for the accident at the Chernobyl Nuclear Power Plant on 26~April 1986'' -- because of the identical causes of all three accidents at analogous RBMK-1000 reactors. In Dubovsky's view, only by sheer luck did the accidents of 1975 and 1982 fail to produce grave consequences comparable to those in Chernobyl, but they served as dire warnings, which the industry leadership ignored; indeed, the very fact of those accidents (1975 and 1982) had been unjustifiably classified at the time\ldots

As appears from the case materials, some scientists of the former USSR, including Scientist V.~A.~Legasov, when analysing the causes of the catastrophic consequences of the Chernobyl accident, pointed to the erroneous concept in nuclear power in the 1960s to 1970s, when at the governmental level a decision was made to use RBMK-1000 reactors without containment systems (concrete domes). This decision was driven by economic considerations to reduce the cost of Nuclear Power Plant construction using this reactor type, and by design features (primarily the reactor's large size) that made installation of such domes impossible. Moreover, as experience showed, the scientists -- headed by Scientists Dollezhal and Alexandrov, who carried out the design and scientific leadership -- erred in calculating the physical characteristics of the reactor and not only failed to forecast the possibility of an explosion and destruction of the reactor during operation, but provided an incorrect justification for the impossibility of such accidents\ldots

Taking the above into account\ldots~the investigation has decided to separate out these materials and forward them for resolution to the Office of the Procurator General of Ukraine\ldots''

\medskip

\textbf{--- Boris Ivanovich, you sent all the documents related to the design shortcomings of the reactor to the Procurator General of Ukraine. More than two years have passed. Has the investigation there been continued? If so, what conclusions did it reach?}

\noindent
--- The question of forwarding the materials to Ukraine was coordinated between the Procurators General of Russia and Ukraine. I personally took all the materials to Kyiv and left them there in the hope that the investigation would be continued. However, as far as I know, the Office of the Procurator General of Ukraine did not pursue it. Meanwhile, establishing the final truth is of exceptional importance, since debates continue over what constituted the main cause of the accident: operational errors in managing the reactor, or its design deficiencies. I side with those who consider design deficiencies the main cause. And I proceed here from a simple truth: when it comes to nuclear power, the reliability of the equipment must be such as to automatically preclude any incidents -- even in the case of operator error.

\medskip

\textbf{--- The influence of the reactor's design shortcomings on the occurrence and development of the accident was fully established only after the trial of Bryukhanov and other station employees. Should these newly discovered circumstances therefore lead to their being declared innocent and to an apology to people who were unjustly sentenced to long prison terms?}

\noindent
--- Yes, new circumstances have been revealed. And in light of these, the question of the guilt or innocence of the plant workers can be decided only by a new judicial review. But I would not mix the guilt of the plant director with that of the other managers and operators. The director is guilty of not timely providing truthful information about the level of radioactive contamination of the plant grounds -- of concealing the real scale of the accident. And for this, and only for this, he should bear responsibility. The other workers, in my opinion, are innocent. So the case is indeed subject to review. At the same time, it is necessary, finally, through judicial process to determine the degree of guilt and responsibility borne by the reactor's designers, its developers, and scientific supervisors. 

\medskip

\textbf{--- There are sixteen reactors of the Chernobyl type on the territory of the former Soviet Union. Have the design deficiencies that contributed to the accident been eliminated in them? What is the situation at the Chernobyl plant itself?}

\noindent
--- As far as I know, many of the design shortcomings that contributed to the onset and development of the accident have been eliminated. Thus, the speed of lowering the shutdown rods has been increased, a device for venting excess steam has been installed, and a number of other upgrades have been carried out.

But the principal design flaw -- the absence in this reactor of a device capable of localizing an accident, namely a concrete containment dome -- remains. Such a structure, under any unforeseeable circumstances, simply would not have allowed radiation to escape to the outside. However, installing containment at stations of this type is impossible not only because of its enormous cost (as has been said in the press), but also because the sheer dimensions of the reactor make it technologically impossible. For that reason alone, I am certain that a design of the Chernobyl type should never have been allowed to exist from the very beginning.

One must recognize that at reactors of this type, some measure of risk still remains in the absence of a containment shell. Alas, we shall have to wait until they simply exhaust their service life and will be gradually removed from operation.

As for the Chernobyl plant itself, Ukraine does not have sufficient means to shut it down -- this is an extremely costly undertaking. One can only hope that the modernization measures already adopted, together with more reliable operator performance, will protect us from another catastrophe\ldots

\medskip

\begin{flushright}
Prepared for publication by Nadezhda Nadezhdina.
\end{flushright}

\chapter*{Appendix 6: The Test Program}
\addcontentsline{toc}{chapter}{Appendix 6: The Test Program}
\markboth{Appendix 6: The Test Program}{}

\begin{center}
MINISTRY OF POWER ENGINEERING AND ELECTRIFICATION OF THE USSR \\
All-Union Production Association ``Soyuzatomenergo'' \\
V. I. Lenin Chernobyl Nuclear Power Plant
\end{center}

\begin{flushright}
Approved by: \\
Chief Engineer of the ChNPP \\
$\_\_\_\_\_\_\_\_\_\_\_\_\_\_\_\_$ N. M. Fomin
\end{flushright}

\begin{center}
\textbf{WORKING PROGRAM} \\
for testing Turbogenerator No. 8 of the Chernobyl Nuclear Power Plant \\
in modes of joint coast-down with the load of auxiliary systems
\end{center}

\noindent
Deputy Chief Engineer \hfill A. S. Dyatlov \\
Head of PTO \hfill A. D. Gellerman \\
Head of Electrical Shop \hfill A. T. Zinenko \\
Head of CPSU Labor Protection Dept. of SAEN \hfill I. P. Aleksandrov \\
Head of Reactor Shop No. 2 \hfill A. P. Kovalenko \\
Head of Turbine Shop \hfill L. A. Khoronzhuk \\
Head of Instrumentation and Automation \hfill E. A. Borodavka \\
Deputy Head of Electrical Shop \hfill V. I. Metelev \\
Deputy Head of Electrical Shop for RZAI \hfill S. A. Malafienko \\
Dontekhenergo \hfill G. P. Metlenko

\medskip

\begin{center}
\textbf{WORKING PROGRAM} \\
for testing Turbogenerator No. 8 of the Chernobyl Nuclear Power Plant \\  
in the mode of joint coast-down with the load of auxiliary systems
\end{center}

\section*{General Provisions}

\paragraph{1.1.} The purpose of the tests is an experimental verification of the possibility of using the mechanical coast-down energy to maintain the performance of auxiliary-system mechanisms in conditions of loss of power to auxiliary loads.

\paragraph{1.2.} The tests are conducted prior to placing the unit into scheduled preventive maintenance (SPM), under an authorized work request.

\paragraph{1.3.} Duration of the tests: \num{4}~hours.

\section*{Test Conditions}

\small\begin{center}
  \begin{longtable}{|p{3em}|p{20em}|p{7em}|}
    \hline
    \textit{Number} & \textit{Description} & \textit{Executor} \\
    \hline
    2.1. & Reduce unit load to \qtyrange{700}{1000}{\mega\watt} (thermal). & NSS \\
    \hline
    2.2. & Disconnect and shut down Turbogenerator TG-7 (or place on VPU). Open breaker VTG-7 and disconnect switch RTG-7. & NSB; NS of Electrical Shop \\
    \hline
    2.3. & Supply auxiliary loads of TG-7, sections 7RA and 7RB, from transformer 6T. & NS of Electrical Shop \\
    \hline
    2.4. & Turbogenerator TG-8 operates at normal excitation; coast-down units ARV-SD and ARV-VG are activated. & NS of Electrical Shop \\
    \hline
    \hline
    2.6. & Power supply to sections 8RA, 8RB, 8RNA is provided via operational input feeders. & NS of Electrical Shop \\
    \hline
    2.7. & In section 8RA, switch on the following mechanisms: 4PN-3,4; 4GCN-13,23; 41KN-73,82; 42KN-73,82; 4NGO-81; 2CN-10. & NSB; NS of Electrical Shop \\
    \hline
    2.8. & In section 8RB, switch on the following mechanisms: 2NPRT-5; 41KN-83; 42KN-83; 4GCN-14,24; 4PN-5; 4NGO-82; VK-15; 2CN-11,12. & NSB; NS of Electrical Shop \\
    \hline
    2.9. & Transfer power supply of \qty{0.4}{\kilovolt} sections to reserve supply: section 164N from transformer 24TR, section 74N from transformer 21TR, section 78N from transformer 23TR, section 232N from transformer 231T, section 180N from transformer 179T, section 225N from transformer 226T, section 75N from transformer 22TR, section 167N from transformer 24TR, section 228N from transformer 227T, section 165N from transformer 24TR, section 220N from transformer 221T, section 208N from transformer 208T, section 160N from transformer 159T. & NSB; NS of Electrical Shop \\
    \hline
    2.10. & In section 8RNA, switch on the following connections: 92TNC, 91TIP, 93TIP, 224T, 4NSOS-3, 4NS-3, 4-ZTNPS, 2NA-6. & NSB; NS of Electrical Shop \\
    \hline
    2.11. & Transfer power supply of \qty{0.4}{\kilovolt} reserve busbar for units No. 3 from transformer 16TR by switching on section switch RS-15, 16TR and RS-16, 17TR. Disconnect switches of \qty{0.4}{\kilovolt} transformers 15, 17TR. & NS of Electrical Shop \\
    \hline
    2.12. & Reactor cooling in the test is provided by sections 7RA, 7RB, 7RNA, 7RNB. For this purpose, one PN (4NP-1 or 4NP-2), two GCN each (4GCN-11,21; 4GCN-12,22), and one KN-2 lift mechanism each (42KN-71(72) and 42KN-81), as well as 2NA-4(5) in sections 7NRB, 7NRA must be switched on. & NSB \\
    \hline
    2.13. & In the \qty{6}{\kilovolt} switchgear room and BSHU-4, assemble test circuits for oscillograph recording of parameters: stator voltage and current of TG-8; rotor voltage and current of TG-8; voltages and currents of input feeders of sections 8RA, 8RB, 8RNA; rotation speed of TG-8; currents and rpm of PN and GCN; moment of closure of safety valves. & DTE; SRZA; CTAI; ETL; CTAI \\
    \hline
    2.14. & For recording technological parameters of the unit in tests, the UVS ``Skala'' system and standard recording instruments are used. List in Appendix 1. List of parameters recorded using additionally installed recording instruments is given in Appendix 2. & CTAI; CHPNP \\
    \hline
    2.15. & Close manual gate valves to prevent water surge into KMPZ according to RC-2 for all three SAOR subsystems 4PV-3/2,1,4,5; 4PV-53,54,63,64,73,74; 4PV-25,26,35,36,45,46; 4PV-83,84. Position personnel to monitor opened equipment and activated mechanisms of SAOR. & RC-2; CHPNP \\
    \hline
  \end{longtable}
\end{center}\normalsize

\section*{Experiment Procedure}

\small\begin{center}
  \begin{longtable}{|p{3em}|p{20em}|p{7em}|}
    \hline
    \textit{Number} & \textit{Description} & \textit{Executor} \\
    \hline
    3.1. & Before conducting the test, perform items 2.1\ldots 2.15 of this program. & --- \\
    \hline
    3.2. & Check diesel-generator 2DG-6 at no-load (under load) and place in ``hot'' reserve. & NSB; NS of Electrical Shop \\
    \hline
    3.3. & Disable the AVR circuits of sections 8RA, 8RB using keys PB on panel 2E of BSHU-4. & NS of Electrical Shop \\
    \hline
    3.4. & Station an operator at panel 1E who, when the rotor current of TG-8 exceeds 3,000~A without command, de-excites the field using the field de-excitation key, or after 60 seconds de-excites per command of the test supervisor. & NS of Electrical Shop \\
    \hline
    3.5. & Reduce the load of TG-8 to the auxiliary load level from transformer 28T. & NSB \\
    \hline
    3.6. & Remove the seal labeled ``Closure of safety valve TG-8 at disconnection of switch B2-6T or VTG-8'' from panel 24RG. & --- \\
    \hline
    3.7. & Disconnect the transformer switch B2-6T from panel 1E. & NS of Electrical Shop \\
    \hline
    3.8. & On command of the test supervisor, switch on oscillographs and using the control key of electromagnetic protective devices on panel 6T, trip the protective devices of TG-8 and transmit the MPA signal to the electrical section via the additionally output button on item PB-3. & NSB \\
    \hline
  \end{longtable}
\end{center}\normalsize

\section*{Assessment of Results}

In this case, the spun up generator will coast down together with the electric motors of sections 8RA and 8RB, supplying power to them and maintaining their electromagnetic torque (and consequently their productivity).

In section 8RNA, sectional breakers (1VS-2VS) will open and all mechanisms will be disconnected (except for the non-disconnectable stage), and diesel-generator 2DG-6 will be started.

After 2DG-6 has accelerated and connected to the section, the mechanisms of section 8RNA will be started in stages according to the MPA program.

\small\begin{center}
  \begin{longtable}{|p{3em}|p{20em}|p{7em}|}
    \hline
    \textit{Number} & \textit{Description} & \textit{Executor} \\
    \hline
    3.9. & Review of recordings and oscillograms, inspection of equipment, record of failed indicators and illuminated annunciator lights, readings of monitoring instruments. & CHPNP; NSB; NS of Electrical Shop; DTE \\
    \hline
    3.10. & Restore power supply to sections 8RA, 8RB (sections) \qty{0.4}{\kilovolt}. & NS of Electrical Shop \\
    \hline
    3.10.1 & Disconnect all breakers of auxiliary system mechanisms on de-energized sections 8RA, 8RB. & NS of Electrical Shop \\
    \hline
    3.10.2 & Switch off reserve input feeders of sections 8RA, 8RB. & NS of Electrical Shop \\
    \hline
    3.10.3 & If necessary, switch on auxiliary system mechanisms for this section. & NSB \\
    \hline
    3.10.4 & Close breakers of \qty{6}{\kilovolt} transformers of auxiliary systems \qty{6} and \qty{0.4}{\kilovolt} on sections 8RA, 8RB. & NS of Electrical Shop \\
    \hline
    3.10.5 & Close operational input feeders of \qty{0.4}{\kilovolt} sections. & NS of Electrical Shop \\
    \hline
    3.10.6 & Restore the position of gate valves closed per item 2.15 in RC-2. & CHPNP \\
    \hline
    3.10.7 & Disconnect the 2DG-6 input and close sectional breakers 1VS-2VS at the control system section 8RNA from panel PB-3 of BSHU-4 and necessary mechanisms. & NSB; NS of Electrical Shop \\
    \hline
    3.10.8 & If necessary, transition sections 8RA, 8RB to operational power supply. & --- \\
    \hline
  \end{longtable}
\end{center}\normalsize

\section*{Safety Measures}

\paragraph{4.1.} During the tests, all switching operations in primary circuits shall be performed by duty personnel at the direction of the technical test supervisor, with the permission of the shift supervisor.

\paragraph{4.2.} If equipment malfunction is detected during the tests, further work according to the program shall be suspended until the cause of the malfunction is eliminated. In the event of an emergency situation at the unit, personnel actions shall be determined by local accident mitigation instructions.

\paragraph{4.3.} Before beginning the tests, the test supervisor shall conduct a briefing of the duty shift personnel.

\section*{Responsible Actors}

\paragraph{5.1.} Chief Technical Supervisor of the test: Brigade Engineer of Dontekhenergo Metlenko G.P.

\paragraph{5.2.} Responsible persons for conducting the tests:

\paragraph{5.2.1.} For operative switching in the electrical section, technical safety, and fire safety: Deputy Chief of Electrical Shop for Operations, Lelechenko A. G.

\paragraph{5.2.2.} For serviceability of protective relay circuits of Block No. 4: Chief Engineer, Malafienko S. A.

\paragraph{5.2.3.} For serviceability of MPA startup circuits and reliable power supply for Block No. 4: Chief Engineer, Metelev V.I.

\paragraph{5.2.4.} For ensuring registration of technological parameters in tests: Deputy Chief of Central Technical Measurement and Instrumentation Department, Lapuga N.R.

\paragraph{5.2.5.} For operative switching on equipment of remote service stations in Reactor Circuit, Turbine Circuit, and Central Technical Measurement and Instrumentation Department workshops: Shift Supervisors of respective shops.

\paragraph{5.2.6.} Overall direction during tests is exercised by Deputy Chief Engineer for Operations of the 2nd Sector, Dyatlov A. S.

\section*{Attachment 1 to the Test Program}

\begin{center}
ANALOGUE PARAMETERS CALLED \\
FROM DREG PROGRAMS IN EXPERIMENTS \\
\end{center}

\small
\begin{center}
\begin{longtable}{|p{3em}|p{12em}|p{7em}|p{8em}|}
\hline
\textit{No.} &
\textit{Name of control section} &
\textit{Point designation} &
\textit{Remarks} \\
\hline
\endfirsthead

\hline
\textit{No.} &
\textit{Name of control section} &
\textit{Point designation} &
\textit{Remarks} \\
\hline
\endhead

1 &
Flow rate of feedwater and blowdown &
P1G1111 &
\qtyrange{0}{1600}{\ton\per\hour} \\
\hline

2 &
Same as above &
P2G1111 &
Same as above \\
\hline

3 &
Pressure in blowdown &
P1P2111 &
\qtyrange{0}{100}{\kilogram\force\per\centi\meter\squared} \\
\hline

4 &
Same as above &
P2P2111 &
Same as above \\
\hline

5 &
Level in blowdown &
P1H2511,21 &
--- \\
\hline

6 &
Same as above &
P2H2511,21 &
--- \\
\hline

7 &
Flow rate in discharge nozzle of main circulation pumps (MCP) &
N1G1211,21,31,41 &
\qtyrange{0}{12500}{\meter\cubed\per\hour} \\
\hline

8 &
Water pressure in discharge header of feedwater pumps &
DOR7311,21 &
\qtyrange{0}{160}{\kilogram\force\per\centi\meter\squared} \\
\hline

9 &
Pressure in suction header of condensate pump KN-11 &
T2P4311 &
\qtyrange{0}{15}{\kilogram\force\per\centi\meter\squared} \\
\hline

10 &
Generator power &
T2N7111 &
\qtyrange{0}{600}{\mega\watt} \\
\hline

11 &
Voltage on 6 kV auxiliary buses &
T2H7412,13 &
\qtyrange{0}{7.8}{\kilovolt} \\
\hline

12 &
Pressure in discharge header of control rod drive system &
AOR3211 &
\qtyrange{0}{6}{\kilogram\force\per\centi\meter\squared} \\
\hline

13 &
Stop valve closed &
T2A1411 &
--- \\
\hline

\end{longtable}
\end{center}

\medskip
\textbf{Notes:}
\begin{enumerate}
  \item All parameters are polled with a cycle of 2 seconds.
  \item Printout of parameters via the DREG system is provided by SDIVT; responsible party: head of CTAI.
\end{enumerate}

\section*{Attachment 2 to the Test Program}

\begin{center}
  LIST OF PARAMETERS RECORDED ON THE EIK CONTROL PANEL DURING EXPERIMENTS
\end{center}

\small
\begin{center}
\begin{longtable}{|p{3em}|p{16em}|p{7em}|}
\hline
\textit{No.} &
\textit{Name} &
\textit{Designation} \\
\hline
\endfirsthead

\hline
\textit{No.} &
\textit{Name} &
\textit{Designation} \\
\hline
\endhead

1 &
Feedwater flow rate through four loops &
P2G1311,12 \\
\hline

2 &
Water pressure at suction of main circulation pumps (MCP) &
N2P1111,13 \\
\hline

3 &
Water pressure at discharge of main circulation pumps (MCP) &
N2P1211,13 \\
\hline

4 &
Water flow rate in discharge header of MCP-13 and MCP-23 &
H1G1231, H2G1231 \\
\hline

5 &
Water pressure in discharge header of feedwater pumps &
DOR7311,12 \\
\hline

6 &
Pressure downstream of first-stage condensate pump TG-8 &
T2P4111 \\
\hline

7 &
Pressure at suction of second-stage condensate pump TG-8 &
T2P4311 \\
\hline

8 &
Pressure at discharge of second-stage condensate pump TG-8 &
T2P4411 \\
\hline

9 &
Condensate flow rate after low-pressure heater LPH-5 &
T2G4611 \\
\hline

10 &
Water pressure in suction header of feedwater pumps (two points) &
DOR4311 \\
\hline

\end{longtable}
\end{center}\normalsize

\chapter*{Appendix 7: Who Was Anatoly Dyatlov?}
\addcontentsline{toc}{chapter}{Appendix 7: Who Was Anatoly Dyatlov?}
\markboth{Appendix 7: Who Was Anatoly Dyatlov?}{}

\begin{center}
  \textit{Translator's note: The contents of this chapter have been compiled by Anatoly Dyatlov's wife, Isabella, who has also published her husband's memoirs in the form of this book.}
\end{center}

\varthreestars

\section*{P.~V.~Vyrodov}

I wish to speak of a small period of our shared life with Dyatlov -- our studies and our work in the city of Norilsk.

There, in the late 1940s, the Mining--Metallurgical Technical College of the USSR Ministry of Internal Affairs was opened. It was on its electrical engineering faculty that, in 1946, I first made the acquaintance of Tolya. During a break between classes he was the first to approach me and introduce himself. This surprised me greatly, for I considered myself already a grown man (I was nineteen years old, though only \qty{165}{\centi\metre} tall), and four years of work in a locomotive depot among prisoners had cured me of any excess sociability.

We made each other's acquaintance. My appearance was somewhat \textit{priblatnennyi}\footnote{Colloquial slang denoting an affected or criminal-adjacent manner, acquired through close contact with the penal milieu.}, and this drew the interest of the other lads, who also came over to be introduced. The proceedings were directed, as I later learned, by the group's monitor, a front-line soldier. Among these young men there were solid people indeed -- participants in the war. In comparison with them my first acquaintance, Tolya Dyatlov, appeared quite a stripling at his fifteen years, though he was tall and broad-shouldered. Nevertheless, among the other grown men he did not lose himself, nor did he show embarrassment. He appealed to me by his openness and by the quiet confidence of a Siberian.

What it was that attracted him to me, and how our friendship began, I do not know. At first we strove to outdo one another in our studies, frequently becoming top students (the stipend was higher by 25\%) -- now for half a year, now for a full year. A full year he achieved more often; I was usually undone by the spring examinations.

We lived together in the dormitories. The last two years of the technical college are especially vivid in my memory, when we occupied the same room. As in any student environment, there were disputes, quarrels, and fights. But between the two of us there arose neither quarrel nor fight, though disputes there were. I think the reason was this. Untouched by fortune (since 1944 I had lived independently, having left my stepmother when my father was mobilized), I suddenly sensed that he, as it were, took care of me, watched that I should not be wronged. Although I was older and more experienced than he in the affairs of life, here I yielded to such solicitude, and in my thoughts dubbed him \textit{``my protector.''}

In general I was no meek lamb; at times I flared up like powder, and a scandal seemed inevitable. Yet Tolya somehow sensed the necessary moment and would appear as if from nowhere and, laying a hand upon the shoulder of my adversary, would ask: \textit{``Why the noise, lads?''} His height and build allowed him to be indulgent; they produced the requisite impression, and the quarrel would cease.

After graduating from the technical college, Tolya and I began working in a post office box\footnote{Colloquial term for a closed enterprise of the defense sector, designated only by a postal number.}, producing ``heavy water.'' Later he departed to study at the Moscow Engineering Physics Institute, yet our friendship continued. After the institute Anatoly found himself in the Far East, while I was in the west, in the city of Obninsk, where -- after completing the evening division of MEPhI -- I worked for many years at a test stand for reactors intended for nuclear submarines (APL). There we prepared the first crews for those submarines. Anatoly visited our test stands on several occasions.

I wish to note one important circumstance. Over the course of nearly fifty years of friendship -- \textit{``unto the last days of Don'' -- }\footnote{An idiomatic expression meaning ``to the very end of life.''} I did not observe in Tolya a single dishonest act toward me, my relatives, or my acquaintances. I valued him for his straightforwardness, his purposefulness, his courage, his intellect, and his kindness toward people. By his entire life he confirmed these qualities.

\begin{flushright}
Former Senior Engineer\\
Directorate of Reactor Test Stands\\
for Nuclear Submarines\\[0.5\baselineskip]
P.\,V.~Vyrodov
\end{flushright}

\section*{T.~Pokrovskaya}

\begin{flushright}
\textit{``For nothing is secret, that shall not be made manifest;\\
neither any thing hid, that shall not be known and come abroad.''}\\
(Luke~8{:}27)
\end{flushright}

The years 1953--59 were the years of our youth, of our student days. Tolya was older than most of us by some five years. Yet even then he seemed to us a man seasoned by life's experience. He was always an authority among us, and we loved him. He was the mentor of our group.

After graduating from the institute in 1959, some of the lads were assigned to facilities connected with atomic energy -- with the production and operation of nuclear installations -- to the nuclear-powered icebreaker \textit{Lenin}, to the cities of Dubna and Sverdlovsk; while Tolya was sent to the Far East.

Our group proved to be remarkable. For forty years now we have regularly held reunions. Beyond these, the ``lads'' (and they remain lads for us, always and without exception) have maintained their own ties, as has our little circle of ``girls.''

I remember all the meetings of our group. One in particular stands out -- the reunion marking the twentieth anniversary of our graduation from MEPhI. Tolya was then so cheerful, so energetic; he scarcely spoke in prose at all -- he so loved poetry.

It was the year 1979. There remained a little more than six years until the explosion.

Our ranks began to thin. Many departed for the next world. And only one was killed. It was Tolya Dyatlov.

How and by whom Anatoly was killed -- and the young men, the staff of Unit~4 of the Chernobyl Nuclear Power Plant on 26~April~1986; how the firefighters who extinguished the blaze at Unit~4 after the explosion were destroyed; and how many, many others in the past, the present, and the future; how and by whom -- all this became clear upon reading this book.

It is a cry from the soul of a life insulted and destroyed.

The book is written with such detail and conviction, so intelligibly even for the non-specialist, that after reading it no doubt remains.

Tolya writes:

\textit{``26~April~1986. An accursed day. It divided the lives of many people into before and after. What then can one say of my own life -- it was cleft by a deep chasm into two wholly dissimilar parts. I had been practically healthy and in the previous years had spent only three or four days at a time on sick leave -- I became disabled. I had been a reliable, law-abiding person -- I became a criminal. And finally, I had been a free citizen -- I became a convicted citizen.''}

And not merely a criminal, but one to whom was ascribed guilt for the most monstrous catastrophe of the twentieth century -- the explosion of a nuclear reactor with all the consequences that followed.

On 26~April~2001, fifteen years will have elapsed since the explosion. Among the living there is no longer either the inventor of this reactor, the scientific director of its development, at that time President of the Academy of Sciences of the USSR, Academician A.~P.~Aleksandrov; nor the chief designer N.~A.~Dollezhal -- men who in their time were so carefully shielded by our system of justice.

It is said truly: of the dead, either good or nothing. Therefore, of them, as the principal developers -- regarding the fact that they designed and permitted for operation an unreliable reactor (a reactor!); that they once ignored the warnings of their own colleagues V.~P.~Volkov and V.~L.~Ivanov concerning the danger of its use and undertook nothing; that they led matters to an explosion and the deaths of people (this is why I say: Tolya was killed) -- of them one may say only this: God is their judge!

And of the court, and of commissions just as biased as the court, it is written:
\textit{``Judge not, that ye be not judged. For with what judgment ye judge, ye shall be judged: and with what measure ye mete, it shall be measured to you again.''}
(Matthew~7{:}2)

Almost ten years of life were granted to mortally ill Tolya, who had received a dose of radiation beyond all bounds. Of these, four he spent in confinement. Many of his friends, pupils, colleagues -- those who were with him there at the station on 26~April~1986 -- had long since departed this life. And Tolya still lived. He lived by the book, by the hope that it would be published, that people would learn the truth about the guiltless who were made guilty, and about those who were truly culpable.

I thank God that He granted Tolya those years of life, and that they sufficed for him to complete his work.

He lived sixty-four years. Now he would have been seventy. He was a very courageous man, strong both physically and spiritually. He could have continued to work and to rejoice in life.

And, most importantly, Tolya must be rehabilitated. He must be!

\begin{flushright}
T.~Pokrovskaya
\end{flushright}

\section*{V.~A.~Orlov}

I first became acquainted with Anatoly Stepanovich Dyatlov in the city of Komsomolsk-on-Amur, when in 1967 I arrived there as a young specialist, assigned after graduating from TEM, to the Lenin Komsomol Plant (ZLK). In truth, I did not so much meet him in person as learn of him indirectly, from those around me with whom I had to work in ``Service~22'' -- such was the name of the division to which the plant's personnel department assigned me.

At that time ZLK was a closed plant, working for the defense sector. It is not surprising that everything there was shrouded in secrecy. Any questions not directly related to one's immediate duties could arouse a certain interest on the part of the relevant authorities. And yet, after some time, and without excessive curiosity, I learned of the existence of ``Laboratory~23,'' which was part of ``Service~22.'' The head of that laboratory was of A.~S.~Dyatlov.

Later, when in the line of duty I had to take part in many months of trials of ``orders,'' I became better acquainted with the lads from ``Laboratory~23'' and with their chief. It was a group of specialists in the control of power installations of the orders. While studying the systems of the orders and working in acceptance teams, I repeatedly became convinced of the very highest qualifications of these ``controllers.'' One could obtain from them an answer to virtually any question connected with power installations -- of course, within the bounds of their competence. One of the principal reasons for such an attitude toward the work was the high exactingness of the head of the group of controllers.

Anatoly Stepanovich enjoyed unquestioned authority among his subordinates, for he himself was fanatically devoted to the task entrusted to him, knew it perfectly, and demanded the same of those under him. There was in him no trace of posturing, and he himself accepted nothing false or contrived. Under conditions of total secrecy, ``Laboratory~23'' lived its own life, closed to outsiders. We who worked in other divisions of the testing department, in essence, knew nothing of the internal relations or inner life of the laboratory. One episode comes to mind.

All who lived in those years remember well the regular spring subbotniks, timed to Lenin's birthday, to the First of May, and the like. Under the guidance of the plant's Party committee and the Party organizations of the divisions, comprehensive preparations for the next subbotnik would begin many days in advance. The list of participants was determined; the work was planned beforehand. It was desirable that the work be conspicuous; therefore engineering and technical personnel, as a rule, were assigned to cleaning the grounds.

At one such subbotnik the entire testing department, as always, worked in the plant's park of culture. We had our own corner there, where each spring we raked leaves and rubbish into piles. ``Laboratory~23,'' however, was to work on the plant's territory: it was to spread out a pile of earth that had been delivered several days before the subbotnik. After the subbotnik it emerged that ``Laboratory~23'' had disrupted it. It turned out that the day before, Anatoly Stepanovich, without much concern for the ritual aspect of the work planned for the subbotnik, had asked a bulldozer operator who was working nearby to level, at the same time, the ``subbotnik'' pile of earth.

When the program of large-scale construction of nuclear power plants in the European part of the USSR was launched, Dyatlov moved to the settlement of Pripyat to work in the directorate of the Chernobyl Nuclear Power Plant then under construction. Following him came the Komsomol members who had worked together with Anatoly Stepanovich in ``Laboratory~23'' of the ZLK testing department. It should be noted that, as far as I know, A.~S.~Dyatlov invited no one himself. Each of the Pripyat Komsomol members in due time approached Anatoly Stepanovich with a request to accept their application documents and, if possible, to send an official call. So it turned out with me as well.

At the Chernobyl Nuclear Power Plant, A.~S.~Dyatlov, working first as deputy head of the reactor-turbine shop for the reactor section, and later as deputy head of the reactor shop (RC) for operations, did not depart from his principles -- to know the entrusted work thoroughly. During the period of installation of the equipment and systems of the RC, he studied the equipment and systems of the reactor installation ``down to the last hanger.'' Then began the staffing of shifts. It was necessary to prepare the workplaces of the operating personnel and to equip them with operational documentation. Several ``creative brigades'' were formed from among the operators to create a set of operational diagrams of the reactor section. Anatoly Stepanovich set the task: the diagram must be maximally clear and visual.

Time and again he returned for revision diagrams that did not meet this principle, explaining nothing in particular, simply saying: \textit{``The diagram is bad -- think!''} As a result, an excellent set of operational diagrams was created in the RC, into which later only current changes were introduced, without altering their structure.

After the start-up of the first, and then the second, power units of the Chernobyl Nuclear Power Plant, the routine of operation began. A.~S.~Dyatlov was a demanding -- one might say, severe -- manager. Recalling that time (I was then working as a senior mechanical engineer, and later as a shift supervisor of the RC), I can state with confidence that there were no problems with A.~S. for those operators who conscientiously, with full dedication, approached their work. At times it was even necessary to employ ingenuity in order to fulfill the shift task -- to pump out water without a pump, to thaw frozen pipes without heaters\ldots~Whoever worked on the RBMK-1000 knows what kind of project it was.

Those, however, who sought to dissemble, to ``crawl away'' from performing an assignment, to hide behind contrived reasons, and especially to conceal a violation of instructions, Dyatlov would ``calculate'' instantly. And then one received one's due. Many took offense, many were indignant, while inwardly understanding the justice of the assessment.

As the construction of the third unit of the Chernobyl Nuclear Power Plant progressed, the formation of the operating divisions of the second stage began. A.~S.~Dyatlov was appointed head of RC-2 and, in turn, set about selecting personnel for the future RC-2. Naturally, the backbone of the new shop consisted of specialists from RC-1, who already had experience both in commissioning work and in operational work on an operating power unit.

It so happened that Anatoly Stepanovich also proposed that I transfer to RC-2 as deputy for operations. I agreed and had already begun studying the systems of the second stage and delving into the problems of the unit under construction. But after some time A.~S. called me aside (we were still all in RC-1 then) and informed me that a hitch had arisen with my transfer to RC-2. As he put it at the time: \textit{``I cannot persuade the station Party committee. They are stubbornly pushing their own candidate. The main complaint is that you are non-party.''} Then Stepanovich said: \textit{``Forgive me, it is not working out as planned. Then let it be neither their way nor mine.''} And he proposed another candidate who met the formal requirements of the station Party committee.

After that our paths diverged. Anatoly Stepanovich worked on the second stage of the Chernobyl Nuclear Power Plant as head of RC-2, and later as deputy chief engineer of the station for operation of the second stage. We ``crossed paths'' again only on the day of the accident, 26~April~1986. True, directly at the station that day I did not meet Dyatlov. When we -- that is, the assistance group from RC-1, consisting of senior mechanical engineer of shift~No.~5 A.~A.~Nekhaev, senior operations engineer of RC-1 A.~G.~Uskov, and myself -- arrived at the emergency unit on 26~April, Anatoly Stepanovich was no longer there -- his health had given out. We met later that same evening in the medical unit of the city of Pripyat.

After that came the road to the 6th Clinical Hospital of Moscow and prolonged treatment. The Chernobyl victims were housed on all floors of the hospital. Anatoly Stepanovich was placed in a ward on the fourth floor, while I ended up on the sixth. This did not prevent all of us, the Chernobyl men, from sometimes -- usually in the evenings -- meeting on the stair landing between floors and hotly discussing what had happened: who had done what, who had seen what, what had occurred, what the possible causes might be.

At that time everyone was still alive -- there was, as the physicians said, a period of imaginary well-being. We did not yet know the true causes of the accident; there were many different versions. There was simply no necessary information. And that the cause had already been embedded in the design of the station, that we had, it turns out, been working at an explosion-hazardous enterprise(!) -- such a thing, of course, none of us could even imagine.

Then came the investigation. The trial. I did not attend a single court session, although I received an official invitation as a victim. I simply did not wish to be present at that farce, the final outcome of which had been predetermined in advance by the official version of the causes of the accident.

A large-scale campaign was under way to divert responsibility from the true culprits of the accident. The mass media purposefully shaped public opinion about the guilt of the station personnel, who allegedly had disabled all protections and interlocks almost with the sole aim of blowing up the reactor. Many nimble-of-pen ``writers and poets'' gained popularity from this, on the wave of which they ultimately improved their material well-being substantially. And behind all this hid the true culprits of the accident -- those who, in violation of all norms and rules, designed an explosion-hazardous reactor; those who, after the accident, investigated its causes and, naturally, did everything possible to shift the blame onto the station personnel.

As became known later, the real causes of the accident were already known to the designers in May, but all this was for internal use only.

The real technical cause of the accident for specialists at nuclear power plants with RBMK reactors, including the Chernobyl Nuclear Power Plant, became clear from the set of measures that began to be urgently implemented on RBMK units. Of course, we did not know all the details of the events on the main control room of Unit~4 and on the unit itself. But it was somehow hard to believe that Dyatlov -- whom we knew as a manager, as a specialist who always rigidly and punctiliously demanded compliance with instructions -- would suddenly, in his presence, allow the violations attributed to the personnel, let alone order the wanton violation of instructions. Nor did any of the lads -- operators of Unit~4, eyewitnesses of the events (many of whom were still alive at that time) -- say anything of the sort when we spoke in the Sixth Hospital.

Of course, in the course of discussions there were remarks that one should not have performed this or that operation -- for example, raising power after the drop, and the like. But these are all reasonings of the kind: \textit{``Had I known where I would fall, I would have spread straw.''} The essence is that the personnel did only what they had the right to do under the instructions in force at that time -- and that is no violation at all.

During the investigation, under conditions of isolation and ill health, Anatoly Stepanovich conducted his own inquiry into the causes of the accident. We -- those of us who were at liberty -- were often struck by the questions he sent from the detention facility through his wife, Izabella Ivanovna. In his notes he would ask, for example, to be informed of the exact wording of a specific clause of the Rules on Nuclear Safety. At the same time he would quote almost verbatim the first two paragraphs of that clause and convey in essence the last, the full wording of which he asked to be sent to him. Whoever does not understand what this means should try to read at least one page of the Rules and then quote it. That was Dyatlov in his entirety.

We visited Stepanovich several times while he was serving his sentence in the ``lockup,'' as he himself called it. And in those meetings the conversation about the details of the accident continued. Once Izabella Ivanovna reported that Stepanovich was finally being released. It was necessary to pick him up no later than 16~hours. Time was short, and the road was not near. We arrived at 15{:}30. Stepanovich appeared at the gate of the checkpoint after 16~hours. In his hands he carried some belongings. At that moment a column of prisoners was entering the colony gates -- apparently returning to the zone after work. Several voices from the column shouted farewell words to Stepanovich. He raised his hand and replied something. One felt that here, too, he was respected. We returned to Kyiv already at night.

\textit{P.S.} After his release, Anatoly Stepanovich did not cease to dismantle the official version of the causes of the accident; he continued to appeal to various agencies, up to and including the IAEA. And this began to yield results, although high organizations, including foreign ones, which in their time had replicated the version of personnel violations, cannot acknowledge their error even now and, already defending the ``honor of their uniform,'' stubbornly continue to multiply falsehoods.

\begin{flushright}
V.~A.~Orlov
\end{flushright}

\section*{A.~V.~Kryat}

Esteemed reader, you are offered a book written by a man who passed through the first, most tragic hours of the accident together with the personnel of the Chernobyl Nuclear Power Plant, and who managed -- under an unceasing hail of accusations and lies -- to hold his ground. Mortally ill, and understanding that with him the information would depart, he found within himself the strength to fulfill his duty toward the dead and to relate the events that occurred on that dreadful night of 26~April~1986.

Fate first brought me together with Anatoly Stepanovich in 1969 in the Far East, when I, a young specialist, came to the plant after graduating from the institute; the second time our paths crossed was in 1974, when I arrived to work at the Chernobyl Nuclear Power Plant.

Principle, honesty, personal responsibility and devotion to the work one serves; impeccable knowledge of the technical side; simple human decency; and, in addition, complete self-dedication -- these were the criteria to which everyone had to correspond who intended to work with Dyatlov.

At the beginning it was not merely difficult for us, the young specialists -- it seemed impossible to master the entire volume of technical material in order to learn the reactor installation as A.~S. himself knew it. Not only twelve hours of work were insufficient, but Saturdays and Sundays as well; and only time confirmed that he was right, and that we became like-minded.

He could understand mistakes made by the personnel if they were argued and substantiated; but he absolutely could not accept slackness, incompetence, or a negligent attitude toward one's duties. A.~S. was, as a rule, distinguished by the directness, clarity, and brevity with which he stated his position -- and this did not always serve him well.

A.~S. permitted neither himself nor others, in his presence, to ``settle accounts'' with personnel who had made errors while they were at their workplace. I recall a case when I (I was at the console) triggered the reactor protection and the ship was left without way while casting off from the pier. The delivery mechanic began to ``educate'' me, but A.~S. simply turned him out of the control room. After I had handed over the shift, at the review of the cause of the actuation of AZ -- an error had been committed by me -- I received my due in full, from A to Z, but not from the mechanic: from A.~S.

A characteristic feature of his disposition was a pathological rejection of any untruth and lie. If anyone was caught even in an untruthful account of events, not to speak of deception, he never again trusted that person's word.

A.~S. was a man who had his own point of view on all questions (and often one that did not coincide with the generally accepted). He was strict in his demands, but not cruel in his relations; he did not fear to assume responsibility within the bounds of his competence and to answer for it; but he had no intention of bearing responsibility for the ignorance, inability, and unprofessionalism of others.

He knew how to listen and to hear his interlocutor, to argue his position -- something the ``tribunal'' of 1987 confirmed brilliantly. A characteristic episode is this. When it was said in court that the RBMK does not even today (and this was already 1987, that is, a year after the accident) meet the requirements of norms, rules, and standards of nuclear safety, the judge replied that this did not pertain to the case. It turns out that what did not pertain to the case was precisely that for which the Chernobyl men were being judged. Such was that court. Therefore, formally A.~S. was convicted, but the sentence was passed upon the system. An acute sense of responsibility was joined in this man with a mighty intellect. His phenomenal memory simply astonished: he would recite Yesenin or Blok by heart.

Outwardly A.~S. did not especially care for his image; from the side he seemed abrupt, categorical, indeed a man of difficult character. But one had to know Dyatlov; one had to see how he loved children, nature, the forest; to see his eyes, his face -- therefore, having believed him in 1969, I believe him today as well.

\begin{flushright}
State Inspector\\
for Nuclear Safety\\
of Ukraine\\[0.5\baselineskip]
A.~V.~Kryat
\end{flushright}

\section*{V.~V.~Grishchenko}

Izabella Ivanovna asked me to write some brief reminiscences of Dyatlov, and I do not know how this can be done briefly, for it is an entire life. It is impossible to separate Dyatlov from our common acquaintances, from our work; and, in fact, reminiscences of Dyatlov are reminiscences of work, of the city of Komsomolsk-on-Amur, of Pripyat, and, of course, of the accident.

Many years of joint work bind me to Anatoly Stepanovich Dyatlov, and perhaps to an even greater extent the period after the accident at the ChNPP -- although after the accident we met with him not as often as would have been necessary.

First, a few words about myself. I am a professional power engineer in the nuclear field, both by education and by the work of my entire life. In 1970 I first met Anatoly Stepanovich in the city of Komsomolsk-on-Amur, where I had arrived to work at the Lenin Komsomol Plant after graduating from the institute. At that time nuclear submarines were being built at that plant, and it truly was the flagship of the Soviet defense-industrial complex. Enormous workshops, in which modern nuclear missile submarines -- by the standards of that time -- were being completed in various stages of readiness. Shop~No.~22, in which I began work, was an auxiliary one and united several laboratories and sections that ensured radiation safety, assembly of reactors and measurements of neutron-physical characteristics, installation and adjustment of special electronic navigation equipment and reactor control systems.

The laboratory in which I worked was called the physical laboratory, and its principal tasks were the oversight of the assembly of the ship's main power installation, including the reactor equipment, the adjustment of that equipment, and participation in acceptance trials of the submarine. The head of this laboratory was Dyatlov. The laboratory was small, about twenty people, all men. All were young; only three -- Rusakov, Dyatlov, and Fochkin -- were old-timers, about forty years of age. People did not remain there long in those days: the term of the young specialist ended, and then one returned to the mainland.

In our laboratory only Borya Rusakov was a local; he had completed an evening institute and, with great difficulty, had been accepted into the laboratory -- his education, you see, was not that of an engineer-physicist. In the laboratory Dyatlov was the unquestioned leader, and rather not by virtue of official position; it was simply something no one doubted. Of course, one might say that in such a collective, where more than half the staff were young specialists, it is easy to be a leader -- but I would not say so. At the plant Dyatlov was an absolute authority in matters of physics and the safety of nuclear power installations; with his opinion, I believe, our scientific supervisors reckoned -- the Kurchatov Institute and the Obninsk Institute of Physics and Power Engineering.

My subsequent fate, as that of many other laboratory staff members, was determined by Dyatlov. He moved to work in Chernobyl, and others from Komsomolsk followed him (Kryat, Padenok, Sitnikov, Chugunov, Shulgin). In Pripyat we maintained a sense of fellowship among compatriots and socialized as families. I cannot say that I was a close friend of Dyatlov; probably only Sitnikov maintained very close friendly relations with him.

Dyatlov was a difficult person in communication: direct, possessed of his own point of view, and he never changed it to suit a superior. He argued, he disagreed, in the end he complied -- but he remained of his own opinion. In the same way, he paid little heed to the opinions of subordinates. As you understand, not everyone loves such a person. Once we had a conversation about money, and he said that he had a little more than five thousand rubles in his savings account. He explained that he needed this account for independence: \textit{``If they try to break me, I will leave the job and somehow manage for a few months.''} He had some inner core, convictions over which he could never step.

Back in Komsomolsk-on-Amur, Volodya Vlasov called him a \textit{kerzhak}\footnote{Colloquial designation, originally referring to Old Believers of Siberia, here implying a man of inflexible principles.}, and not because Dyatlov had been born and raised in Siberia, but because forcing him to act against his convictions was practically impossible. And when it is said that, to please the station director or the chief engineer, Dyatlov could have ignored safety principles, issued instructions to disable reactor protections, or violate regulations, I will never believe it.

\begin{flushright}
Chairman\\
State Committee\\
for Nuclear Safety of Ukraine\\[0.5\baselineskip]
V.~V.~Grishchenko
\end{flushright}

\section*{V.~V.~Lomakin}

In the life of any person there are people whose personal qualities evoke a certain sympathy and respect. But only a few of them leave a vivid and indelible mark that remains in the soul for a lifetime. One of those few whom I had the fortune to encounter in my life was A.~S.~Dyatlov.

I became acquainted with A.~S.~Dyatlov in 1973 at the Chernobyl Nuclear Power Plant, where I began working after graduating from the institute in 1972. Until Anatoly Stepanovich transferred to the second stage of the Chernobyl Nuclear Power Plant in 1979, my entire professional path was bound up with him. From the moment the reactor shop collective was formed -- where A.~S.~Dyatlov served as deputy head of the shop for operations -- there were only a few young engineers such as myself. The start-up of the unit was still far off, and all shop personnel carried out the work typically performed during the construction and installation phase of a unit.

In addition, Anatoly Stepanovich set the future operating personnel the task of preparing for qualification examinations and, accordingly, of studying the equipment, operational documentation, and the like. There were the appropriate orders and schedules for the stages of examinations, but the question arose of a lack of time for study during the working day -- and this question was put to A.~S.~Dyatlov. His reply was brief: \textit{``Study after work and on weekends,''} which, in fact, he himself did. And everyone had to study: the RBMK-1000 was, at that time, new technology for all, and for us young people all the more so.

Above all, Stepanich -- as we called him among ourselves -- was demanding of himself and gave himself to the work to the fullest extent; but he also demanded the full measure of fulfillment of assigned tasks. He recognized no trifles in our work, whether during installation or during operation. Those who did not understand this he would explain to directly; he knew not only how to persuade, but also how to listen to a worker.

Anatoly Stepanovich could not tolerate slackness in work, and especially attempts to get around him -- to deceive, to tell an untruth. In such cases, in conversation he would immediately switch to addressing the person by name and patronymic, and the culprit often knew that he would have to report on the completion of work ten times more often, that he would be questioned more strictly, and that trust would be lost for a long time.

Anatoly Stepanovich was sparing in praise; he accepted the proper execution of work as something due. Yet he could stand up before higher management for a shop worker if they proposed to punish him undeservedly. For a heart-to-heart conversation, when necessary, he was open both to a simple shop worker and to an engineer; he never flaunted his position.

He was not lacking in a sense of humor -- you will read the book and see for yourselves.

It should be said that A.~S.~Dyatlov astonished many of us with his knowledge not only of his own field. I will give one example. Once, late in the evening, during a shift accepting reactor schemes, A.~S.~Dyatlov entered the trailer where several people from the shift were located, to check on the progress of the work. Two young specialists were solving a differential equation for some correspondence students; something was not working out, they were making noise and did not notice him. Dyatlov asked them to make room on the bench, sat down, and within a short time pointed out the error and solved the problem. Then he said: \textit{``At work one must work. If I see this again, you will be writing explanations.''}

It is quite understandable that not everyone found his insistence on work being done at least to a ``good'' standard to their liking; some realized this only later. Naturally, there were also those who seized the opportunity during the investigation to cast a stone at him -- but did not appear in court\ldots

Personally, I am grateful to Anatoly Stepanovich for the fact that his ``school'' helped me fully integrate into the ranks of the best operating specialists at the Chernobyl Nuclear Power Plant.

For the second time, our tragedy of 26~April~1986 drew me closer to A.~S.~Dyatlov. I wish, esteemed readers, to share with you how I came to know yet another facet of Anatoly Stepanovich's courageous character. Judge for yourselves: here is an excerpt from letters, beginning in 1989, written on behalf of the Kyiv Council of the \textit{Society of Veterans of the Liquidation of the Accident and Its Consequences at the Chernobyl Nuclear Power Plant}, which were essentially unchanged in substance and were sent successively to the Supreme Court of the USSR, to the Chairman of the Supreme Soviet of the USSR, and finally to the President of the USSR, M.~S.~Gorbachev:

\begin{personal}[From the letter of the Kyiv Council:]
\textit{``Of the fourteen illnesses of the convicted Dyatlov A.~S., five illnesses (ischemic heart disease, diffuse pneumosclerosis, atrial fibrillation, chronic obstructive bronchitis, pulmonary emphysema) are included in the list of diseases constituting grounds for granting convicts release from punishment in accordance with Order of the USSR Ministry of Internal Affairs No.~213 of 30~October~1987. At the same time, the said List does not include such a disease as acute radiation sickness (ARS) of the second and third degree, which, in our view, is inhumane, and this issue requires immediate resolution.}

\textit{We wish to emphasize that for Dyatlov A.~S., a disabled person of Group~II, suffering from ARS of the third degree at the age of fifty-nine and serving a sentence in places of deprivation of liberty, his state of health cannot possibly improve, since ARS constantly produces indeterminate side effects that worsen the patient's condition.}

\textit{We consider unlawful the response of the Medical Directorate of the Ministry of Internal Affairs of the Ukrainian SSR, outgoing No.~4/1-69 of 02~February~1990, addressed to us, in which, in particular, it is stated: `Dyatlov A.~S. is not subject to release from places of deprivation of liberty due to illness; he does not fall under the requirements of the Order of the USSR Ministry of Internal Affairs.' Of the six persons convicted in this case, only two -- Dyatlov A.~S. and Fomin N.~M. -- remain in places of deprivation of liberty; they, due to their state of health, cannot, like the other convicts, be conditionally released after serving one third of their sentence with mandatory involvement in labor.''}
\end{personal}

After the mendacious trial, while in a camp from which, by law and in view of the above-mentioned illnesses, the authorities ought to have released him—but did not wish to, resisting with formal replies of ``No'' -- A.~S.~Dyatlov found within himself the strength to overcome moral and physical suffering and to set forth the truth about the events of 26~April~1986, to analyze and assess various documents concerning the accident, hoping that he would yet one day be heard and understood.

The machine of falsehood concerning the causes of the Chernobyl accident operated without interruption until 1988; therefore, few of his former pupils and colleagues maintained contact with A.~S.~Dyatlov while he was serving his sentence.

Some even now torment themselves with the question: was Stepanich right in everything? Falsehood is strong -- but it is not eternal.

I believe I will not be mistaken in saying that, having read the book, you, esteemed readers, will know the true truth about the causes of the Chernobyl catastrophe and will cast aside all the doubts that have been imposed upon you over the past fifteen years.

A.~S.~Dyatlov did not live long after his release. Very often we -- his comrades-in-arms in the struggle for the truth about Chernobyl -- gathered together and saw how hard it was for him, how illness tormented him, and how courageously he bore it all.

May his memory be bright.

\begin{flushright}
Deputy Director\\
State Scientific and Technical Center\\
for Nuclear and Radiation Safety\\
of Ukraine\\[0.5\baselineskip]
V.~V.~Lomakin
\end{flushright}

\section*{N.~F.~Gorbachenko}

On the night of 25 to 26~April~1986, at 24{:}00, together with N.~V.~Navalnyi, I took over the shift at the radiation-monitoring control panel of the second stage. Our duties included radiation monitoring of the central halls of Reactors~3 and~4 and of all adjacent premises with heightened radiation hazard. After making a round of Unit~3, I returned to the control panel; I did not go to Unit~4, since it was being shut down.

Only a few minutes passed when a crash was heard from the direction of the turbine hall. Five to ten seconds later -- a dull blow of tremendous force. The lights and the illuminated signaling on the panel of Unit~4 went out; the panel of Unit~3 lit up red and the audible alarm shrieked; from the intake-ventilation hatch there poured a black-and-ginger dust\ldots

Thus I found myself at the epicenter of a nuclear catastrophe which, within a matter of minutes, could have become fatal for me. That night I came into close contact with the shift supervisor of Unit~4, A.~F.~Akimov; with the Deputy Chief Engineer for Operations of the second stage, A.~S.~Dyatlov; with L.~Telyatnikov; and with many other participants in the liquidation of this catastrophe -- many of whom, to great regret, are no longer among the living.

On the fifteenth anniversary of that dreadful night, one wishes to remember all our fallen comrades:
Akimov~A.~F., Baranov~A.~I., Brazhnik~V.~S., Vershinin~Yu.~A., Degtyarenko~V.~M., Konovalov~Yu.~I., Kudryavtsev~A.~G., Kurguz~A.~Kh., Lelechenko~A.~G., Lopatyuk~V.~I., Novik~A.~V., Perevozchenko~V.~I., Perchuk~K.~G., Proskuryakov~V.~V., Sitnikov~A.~A., Toptunov~L.~F., Khodemchuk~V.~I.\ (grave—Unit~4), Shapovalov~A.~I., Shashenok~V.~F., Busygin~G.~V., Kovalenko~A.~P., Dyatlov~A.~S., Gashimov~M.~U., and others\ldots

There was no panic that night -- no negligence or slackness; there was no fear and no unqualified action, as has been written so much over these fifteen years. In everyone's eyes there was a single, mute question: \textit{why?} And each did what he must, what he was obliged to do, what he could -- saving equipment, extinguishing fires, searching for and carrying out the injured.

On 27~April I, together with A.~S.~Dyatlov, was taken to the Sixth Hospital in Moscow. In the first days the same conversations were conducted: we conjectured, argued, thought -- why the explosion? Anatoly Stepanovich was lean, trim, always calm and composed; he spoke convincingly: \textit{``We did everything correctly.''} At times he would fall silent in thought\ldots~Investigators often pestered us; there was an opinion that it was sabotage. Versions appeared regarding the guilt of the personnel\ldots~Almost every day someone died\ldots~In the soul -- emptiness; the future unknown -- life or death; where is the family?

On the sixtieth day a second wave of burns began for me: the first had not yet healed when new dark cherry-red spots began to appear, more and more with each passing day\ldots~Long months of treatment. A low bow to all the medical staff -- both doctors and nurses.

On 27~October I was discharged home as a cripple, with bandages on wounds that trouble me to this day. On 31~December~1986 they took us to our home station and secretly handed out government awards: orders and medals -- to the living; orders posthumously -- to the widows. Could no place be found in Kyiv, or were heroes awarded only there?! They found ``scapegoats'' -- they were tried. After several years they were released. And again -- silence?! How is one to look into the eyes of the children of the fallen?! For what -- or for whom -- did their fathers die?!

With Anatoly Stepanovich Dyatlov we lived in the same building; we often met in the courtyard, in hospitals. The last time, a few months before his death, we were together in Pushcha-Vodytsia. Evening; we were gathering for supper. Anatoly Stepanovich had severe headaches, and he said:
\textit{``How one longs, lads, to hear the truth -- when, at last, will the people learn how it was?! How one longs to live a bit longer, but my head gives me no peace -- neither by day nor by night; and how one longs to pull back a little glass and have a bite of black bread with lard!''}
Stepanich ate the black bread with lard with pleasure.

He was a courageous man. May the Kingdom of Heaven be his; may the earth be light upon them all; and may there be eternal memory among the living -- not mute silence and oblivion.

\begin{flushright}
Former Duty Dosimetrist,\\
5th Shift\\
Department of Labor Protection\\
and Industrial Safety,\\
Chernobyl Nuclear Power Plant\\[0.5\baselineskip]
N.~F.~Gorbachenko
\end{flushright}

\end{document}
